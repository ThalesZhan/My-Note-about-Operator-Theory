\documentclass[a4paper,11pt]{report}
\usepackage{Mydef}
\title{My Report about Operator Theory}
\author{Z.~Zhan\\ $<$\href{mailto:thaleszhan@gmail.com}%
            {thaleszhan@gmail.com}$>$}

\begin{document}
\maketitle
\tableofcontents
\chapter{Topological Vector Spaces and Banach Spaces}

\section{Topological Vector Spaces}
Linear operations, i.e addition and scalar multiplication, provide an algebraic structure on a set, therefore constructing a vector space. In the course, linear algebra, we have learn the algebraic structure of finite dimensional vector spaces. But how to deal with infinite dimensional vector spaces? By learning the topological spaces, we know that the topological structure can give us a method to research properties of infinity. Thus, we need to equip a vector space with an additional topological structure, which should coincide with the algebraic structure. This is the reason why we define the topological vector space.

\subsection{Topological Spaces}
\rule{1mm}{1mm} \textbf{Definition:} First, we define the topological structure on a general set.
\begin{defn}
A topological space $X$~=~($X$, $\mathscr{T}_X$) consists of a set $X$, called the underlying space of $X$, and a family $\mathscr{T}_X$ of subsets of $X$ s.t.
	\begin{enumerate}[label=\arabic*)]
		\item $X, \varnothing \in \mathscr{T}_X$.
		\item if $U_\alpha \in \mathscr{T}_X \ \text{for} \ \alpha \in A$, then $\bigcup_{\alpha \in A}U_\alpha \in \mathscr{T}_X$.
		\item if $U_1, U_2 \in \mathscr{T}_X$, then $U_1 \bigcap U_2 \in \mathscr{T}_X$.
	\end{enumerate}
And, $\mathscr{T}_X$ is called a topology on $X$. The element in $\mathscr{T}_X$ is called open set.
\end{defn}

Thus, the topological structure on a set $X$ is totally determined by the family $\mathscr{T}_X$. In particular, from $2)$, we can simplify $\mathscr{T}_X$. In other words, like the basis of a vector space, there is a "basis" of $\mathscr{T}_X$.

\begin{defn}
If $X$ is a set, a basis for a topology on $X$ is a family $\mathscr{B}$ of subsets of $X$, s.t.
	\begin{enumerate}[label=\arabic*)]
		\item $\forall ~ x \in X,~ \exists ~ B \in \mathscr{B},~\text{s.t.}~ x \in B$.
		\item if $x \in B_1 \bigcap B_2$, where $B_1,~B_2 \in \mathscr{B}$, then there is a $B_3 \in \mathscr{B}$, s.t. $B_3 \subset B_1 \bigcap B_2$.
	\end{enumerate}
\end{defn}

$\mathscr{B}$ can generate a topology $\mathscr{T}_X$ on $X$ by doing infinite times union of elements in $\mathscr{B}$. In fact,
\begin{equation*}
	\mathscr{T}_X~ = ~\{~U \subset X \colon U = \bigcup_{~\alpha \in A}U_\alpha,~U_\alpha \in \mathscr{B}~\},~ \text{where A is any index set.}
\end{equation*}

We can do more on the basis by $3)$.
\begin{defn}
A subbasis of a topology on $X$ is a collection $\mathscr{S}$ of subsets of $X$, whose union is $X$. The topology generated by $\mathscr{S}$ is noted by $\mathscr{T}(\mathscr{S})$.
\end{defn}
In fact, we can generate the basis of $\mathscr{T}(\mathscr{S})$ by $\mathscr{S}$, that is,
\begin{equation*}
	\mathscr{B}~ = ~\{~U \subset X \colon U = \bigcap_{\alpha=1}^{n}U_\alpha,~U_\alpha \in \mathscr{S},~n \in \N~\}
\end{equation*}
Therefore, this basis can generate the conincided topology on $X$.\vspace{0.2in}

\rule{1mm}{1mm} \textbf{Continuous maps:} Next, we need to endow the general maps with the topological structure.
\begin{defn}
Let $X$, $Y$ be topological spaces and $f \colon X \sto Y$ be a map. We say $f$ is continous if
	\begin{equation*}
		\forall ~V \in \mathscr{T}_Y,~f^{-1}(V) \in \mathscr{T}_X.
	\end{equation*}
\end{defn} 
In other words, if $f \colon X \sto Y$ is a continuous map, $\mathscr{T}_X$ has "more" elements than $\mathscr{T}_Y$ has. To get more rigorous discription, we define the following concept.
\begin{defn}
	Let $X$ be a set and $\mathscr{T}$, $\mathscr{T}^{'}$ be two topologies on $X$ we say that $\mathscr{T}$ is coaser than $mathscr{T}^{'}$ if $\mathscr{T} \subset \mathscr{T}^{'}$.
\end{defn}
Therefore, if $f \colon X \sto Y$ is a continuous map, the topology $\mathscr{T}(f^{-1}(\mathscr{T}_Y))$ is coaser than $\mathscr{T}_X$. \vspace{0.1in}

\rule{1mm}{1mm} \textbf{Generating topologies:} We use above methods to generate some interested topologies on a set.
\begin{enumerate}[label=\arabic*)]
	\item Inital topology: Given maps $f_\alpha \colon X \sto Y_\alpha ~( \alpha \in A )$ from a set $X$ to a family of topological spaces $\{Y_\alpha \colon \alpha \in A \}$. Let
	\begin{equation*}
		\mathscr{S}~=~\{~f_{\alpha}^{-1}(V) \colon V \in \mathscr{T}_{Y_\alpha},~ \alpha \in A ~\}
	\end{equation*}
	Then $\mathscr{T}(\mathscr{S})$ is the coarsest topology on $X$ such that each $f_\alpha$ is continuous, called the initial topology induced by the family of maps $\{ f_\alpha\ \colon \alpha \in  A \}$.

	\item Final topology: Given maps $f_\alpha \colon X_\alpha \sto Y ~ (\alpha \in A)$ from a family of topological spaces $\{X_\alpha \colon \alpha \in A\}$ to a set $Y$. Let
	\begin{equation*}
		\mathscr{S}~=~\{~ V \colon f_{\alpha}^{-1}(V) \in \mathscr{T}_{X_\alpha},~ \alpha \in A~\}
	\end{equation*}
	Then $\mathscr{T}(\mathscr{S})$ is the finest topology on $Y$ such that each $f_\alpha$ is continuous, called the final topology induced by the family of maps $\{f_\alpha \colon \alpha \in A \}$.
\end{enumerate}
Here is some important examples using above way to generate topologies.
\begin{exam}
	(Initial topology)
	\begin{enumerate}[label=\arabic*)]
		\item Subspace topology: Let $Y$ be a topological space and $X \subset Y$ be a subset of $Y$. The inclusion map $i \colon X \sto Y$ can generate the initial topology $\mathscr{T}_X$ on $X$, then $X$ become a subspace of $Y$. In fact,
		\begin{equation*}
			\mathscr{T}_X ~=~ \{~X \bigcap U \colon U \in \mathscr{T}_Y~\}
		\end{equation*}
		\item Product topology: Let $\{Y_\alpha \colon \alpha \in A\}$ be a family of topological spaces. The product set is 
		\begin{equation*}
			\prod_{\alpha \in A} Y_\alpha ~=~ \{~A \stackrel{f}{\sto} \bigcup_{\alpha \in A} Y_\alpha \colon \forall \alpha \in A,~ f(\alpha) \in Y_\alpha ~\}
		\end{equation*}
		There is a family of maps $\{ p_\beta \colon \prod_{\alpha \in A} Y_\alpha \sto Y_\beta (\beta \in A) \}$. Therefore, these maps can generate the initial topology $\mathscr{T}$ on $\prod_{\alpha \in A} Y_\alpha$ and let $\prod_{\alpha \in A} Y_\alpha$ be the product topological space. In fact,
		\begin{equation*}
			\mathscr{T} ~=~ \{~\prod_{\alpha \in A} V_\alpha \colon V_\alpha \in \mathscr{T}_{Y_\alpha} ~ \& ~ \#\{\alpha \in A \colon V_\alpha \neq Y_\alpha \} < \infty ~\}
		\end{equation*}
		\begin{rem}
			The condition, $\#\{ \alpha \in A \colon V_\alpha \neq Y_\alpha \} < \infty$, is because that when the subbasis generates the basis, only finite many elements can do intersection.
		\end{rem}
	\end{enumerate}
\end{exam}
\begin{exam}
	(Final topology)\\
	Quotient topology: For a topological space $X$ on which an equivalent relation R is fixed, $\pi \colon X \sto X/R$ is the quotient map, then the quotient set can be equiped with the final topology $\mathscr{T}$ generated by the quotient map. Therefore, $X/R$ become a quotient topological space. In fact,
	\begin{equation*}
		\mathscr{T} ~=~ \{~ U \subset X/R \colon \pi^{-1}(U) \in \mathscr{T}_X ~\}
	\end{equation*}
\end{exam}

\rule{1mm}{1mm} \textbf{Countability and metrizability:} When learning the analysis of real functions, we usually use the sequence to discribe the topological properties. But in some general topology, we cannot just use sequence since some properties of "uncountabibily". In this case, the concept of net can be applied to some "uncountable" topologies. Furthermore, there is a class of more special topology, metrizable topology, which has some better properties.

\begin{defn}
	(Net)
	\begin{enumerate}[label=\arabic*)]
		\item Direct set: A direct set $(D,~\geqslant)$ consists of a nonempty set $D$ and a relation $\geqslant$ on D, satisfies:
		\begin{enumerate}[label=\roman*)]
			\item $\forall ~ d \in D,~ d \geqslant d$
			\item $\forall ~ d_1,~d_2,~d_3 \in D, if ~ d_3 \geqslant d_2 ~ \& ~ d_2 \geqslant d_1,~ then ~ d_3 \geqslant d_1$
			\item $\forall ~ d,~ d^{'} \in D,~ \exists ~ d^{''}  ~s.t.~ d^{''} \geqslant d ~ \& ~ d^{''} \geqslant d^{'}$.
		\end{enumerate}
		\item if $X$ is a set, a net is a map $x_. \colon D \sto X$ from a direct set $D$ to $X$
	\end{enumerate}
\end{defn}
\begin{exam}
	If $X$ is a topological space and $x \in X$, then let 
	\begin{equation*}
		D ~=~ \{~ \text{all open neighbourhoods of x} ~\},~ U \geqslant V \Leftrightarrow U \subset V
	\end{equation*}
	Then $D$ is a direct set and $x_\alpha(\alpha \in D)$ is a net.
	And we say $ x_\alpha \sto x$ if and only if \\
	$\forall$ open neighbourhood $U$ of $x$ in $X$, $\exists  \delta \in D,~ \forall ~ \alpha \in D \text{ with } \alpha \geqslant \delta \Rightarrow x_\alpha \in U$
\end{exam}
Nets can be used as sequences in topological spaces. Like, 
\begin{prop}
	If $X$ is a topological space and the net $x_\alpha(\alpha \in D)$ defined above and $A \subset X$, then
	\begin{enumerate}[label=\arabic*)]
		\item $\clo{A} ~=~ \{~ x \in X \colon \exists ~ x_\alpha \text{ in } A,~ x_\alpha \sto x ~\}$
		\item $f \colon X \sto Y$ is continuous between two topological spaces, $x_0 \in X$, $f$ is continuous at $x_0$, if and only if\\
		$\forall$ net $x_\alpha(\alpha \in D)$, s.t. $x_\alpha \sto x_0 \Rightarrow f(x_\alpha) \sto f(x_0)$
	\end{enumerate}
\end{prop}

\begin{defn}
	(Countability)
	\begin{enumerate}[label=\arabic*)]
		\item First countability: For a topological space $X$, $X$ is called first countable if for each point $x \in X$, $x$ has a countable neighbourhood basis.
		\item Second countability: A topological space $X$ is second countable if it has the countable topological basis. 
	\end{enumerate}
\end{defn}
\begin{rem}
	Clearly, the second countable topological space is first countable, but the converse is not true.
\end{rem}
In particular, if $X$ is first countable, sequences can be used to illuminate tpological properties rather than nets. Like,
\begin{prop}
	If $X$ is first countable, then
	\begin{enumerate}[label=\arabic*)]
		\item $U \subset X$ is closed $\Leftrightarrow$ $\forall ~ x \in U,~ \exists ~\text{a sequence} \{x_n\} \subset U,~ s.t.~ x_n \sto x$.
		\item sequential compactness is equivalent to compactness.
	\end{enumerate}
\end{prop}

And for the second countability, it is about the separability.
\begin{defn}
	(Separability)
	\begin{enumerate}[label=\arabic*)]
		\item A subset $A$ of a topological space $X$ is called dense if $\clo{A} ~=~ X$.
		\item A topological space is called separable if it has a countable dense subset.
	\end{enumerate}
\end{defn}

By the definition, we can clearly know that:
\begin{prop}
	If $X$ is a second countable topological space, then it is separable and every open covering of $X$ has a finite subcollection covering $X$.
\end{prop}

We can classify topological spaces into some classes.
\begin{defn}
$X$ is a topological space, then we call $X$ is:
\begin{enumerate}
	\item [($T_0$)] $\forall ~ x,~y \in X,~ \exists ~ \text{open } U \subset X,~ s.t.~ x \in U \text{ but } y \notin U \text{ or } y \in U \text{ but } x \notin U$ (Kolmogorov space)
	\item [($T_1$)] $\forall ~ x,~y \in X,~ \exists ~ \text{open } U,~ V \subset X, ~s.t.~ x \in U \text{ but } y \notin U \text{ and } y \in V \text{ but } x \notin V$\\
	($\Leftrightarrow \forall~ x \in X,~ \{x\} \text{ is closed} $)
	\item [($T_2$)] $\forall ~ x,~y \in X,~ \exists ~ \text{open } U,~ V \subset X, ~s.t.~ x \in U ~\&~ y \in V \text { and } U \bigcap V = \varnothing$ (Hausdorff space)
	\item [($T_3$)] $T_1$ holds and $\forall ~ x \in X$ and closed $C \subset X$, if $x \notin C$, then $\exists ~ \text{open } U,~ V \subset X, ~s.t.~ x \in U ~\&~ C \subset V \text{ and } U \bigcap V = \varnothing$ (regular space)
	\item [($T_4$)] $T_1$ holds and $\forall ~ \text{closed } C_1,~ C_2 \subset X$, if $C_1 \bigcap C_2 = \varnothing$, \\ then $\exists ~ \text{open } U,~ V \subset X, ~s.t.~ C_1 \subset U ~\&~ C_2 \subset V \text{ and } U \bigcap V = \varnothing$ (normal space)
\end{enumerate}
\end{defn}

Then we can specify a class of more powerful topological space.
\begin{defn}
	If $X$ is a topological space, then $X$ is said to be metrizable if there exists a metric $d$ on the set $X$ that induces the topology of $X$.
\end{defn}
\begin{rem}
	Clearly, if $X$ is metrizable, $X$ is second countable and normal.
\end{rem}

Here is two metrization theorems provides the essence of metric spaces.
\begin{thm}
(Metrization theorems)
	\begin{enumerate}
		\item[Urysohn] A topological space is separable and metrizable if and only if it is regular, Hausdorff and second countable.
		\item[Nagata–Smirnov] A topological space is metrizable if and only if regular, Hausdorff and has a $\sigma$-locally finite basis.
	\end{enumerate}
\end{thm}
\vspace{0.2in}
\rule{1mm}{1mm} \textbf{Complete metic space:} For a metric space, we know it is first countable, so the concept of net is unnecessary. And thus sequences are enough to determine the topological structures, like that sequential compactness is equvilent to compactness. 

\begin{prop}
	A compact subset of a metric space is closed, bounded and separable.
\end{prop}
\begin{rem}
	it is clearly, since compactness is also about finity.
\end{rem}

For any metric space, we can use the following theorem to get a completion of that and this completion is unique. Thus, we can always assume a metric space is complete.
\begin{thm}
	Let ($X,~ d$) be a metric space. Then, there exists a metric space ($\hat{X},~ \hat{d}$) with the following properties:
	\begin{enumerate}[label=\arabic*)]
		\item ($\hat{X},~ \hat{d}$) is complete.
		\item There is an embedding $\sigma$ from $X$ to $\hat{X}$.
		\item $\sigma(X)$ is dense in $\hat{X}$.
	\end{enumerate}
	And this ($\hat{X},~ \hat{d}$) is unique with respect to isomorphism.
\end{thm}

Complete metric space is imporatant since it is "sufficiently large". Rigorously, we can the following definition to describe it.

\begin{defn}
	(Baire Category)
	A metric space is said to be of the first category if it can be written as a countable union of sets that are nowhere dense. Otherwise, it is of the second category.
\end{defn}

\begin{prop}
	A complete metric space is a space of the second category.
\end{prop}

\vspace{0.2in}
\rule{1mm}{1mm} \textbf{Filters:} For convenience, we define some terminologies.
\begin{defn}
	A filter on a set $X$ is a family $\mathscr{F} $ of subsets of $X$ satisfying the following conditions:
	\begin{enumerate}[label=\arabic*)]
		\item $\varnothing \notin \mathscr{F}$
		\item $\mathscr{F}$ is closed under finite many intersections
		\item Any subset of $X$ containing a set in $\mathscr{F}$ belongs to $\mathscr{F}$.
	\end{enumerate}
\end{defn}
\begin{exam}\label{exam1}
	For a topological space $X$ and $x \in X$, and let
	\begin{equation*}
		\mathscr{F}(x) ~=~ \{~\text{all neighbourhoods of x}~\}	
	\end{equation*}
	Then $\mathscr{F}(x)$ is a filter and $\mathscr{F}(x)$ satisfies the following properties:
	\begin{enumerate}[label=\arabic*)]
		\item $\forall~ U \in \mathscr{F}(x),~ x \in U$
		\item $\forall~ U \in \mathscr{F}(x),~ \exists ~ V \in \mathscr{F}(x),~ s.t.~ \forall~ y \in V,~ U \in \mathscr{F}(y)$
	\end{enumerate}
	And conversely, if we can find $\mathscr{F}(x)$ for any $x \in X$ with above two properties, these can define a unique topology $\mathscr{T}$ s.t. $\mathscr{F}(x)$ is the filter of neighbourhoods of $x$ for any $x \in X$. In fact,
	\begin{equation*}
		\mathscr{T} = \{~ U \subset X \colon x \in U \Rightarrow U \in \mathscr{F}(x) ~\}
	\end{equation*}
	Also, we can define the basis of $\mathscr{F}(x)$, noted by $\mathscr{B}(x)$. That is $\mathscr{B}(x) \subset \mathscr{F}(x)$ with the following properties:
	\begin{enumerate}[label=\arabic*)]
		\item $\forall~ U \in \mathscr{B}(x),~ x \in U$
		\item $\forall~ U_1 ~\&~ U_2 \in \mathscr{B}(x),~ \exists ~ U_3 \in \mathscr{B}(x),~ s.t.~ U_3 \subset U_1 \bigcap U_2$
		\item If $y \in U \in \mathscr{B}(x),~ \exists ~ W \in \mathscr{B}(y),~ W \subset U$
	\end{enumerate}
\end{exam}

\subsection{Definition and Properties}

\rule{1mm}{1mm} \textbf{Definition:} Now, we need to endow the topological structure on a vector spaces. And the most important thing is that the topological structure should coincide with the algebraic structure.
\begin{defn}
	A vector space $X$ over a field $\K$ (where $\K = \C or \R$) is called a topological vector space if $X$ is equiped with a topology $\mathscr{T}$ s.t. the addition and the scalar multiplication, i.e. 
	\begin{center}
		\begin{tabular}{r @{$\mapsto$} l}
			$(x,~y)~$ & $~x+y$ \\
			$(\lambda,~ x)~$ & $~ \lambda x$
		\end{tabular}
	\end{center}
	are continuous with respect to the topology $\mathscr{T}$.
\end{defn}

In this definition, the most important part is that the addition and the scalar multiplication are continuous. This condition provides some additional properties for the topology and also for the linear operations. First, it can simply the topology.

\begin{prop}
	Given a t.v.s. $X$, 
	\begin{enumerate}[label=\arabic*)]
		\item For any $x_0 \in X$, the map $x \mapsto x+x_0$ is a homeomorphism.
		\item For any $\lambda \in \K$, then map $x \mapsto \lambda x$ is a homeomorphism.
	\end{enumerate}
\end{prop}
\begin{proof}
	It is clearly, since by the definition, $x \mapsto x-x_0$ and $x \mapsto \frac{1}{\lambda}x$ are continuous.
\end{proof}

Therefore, the topology of a t.v.s is completely determined by the filter of neighbourhoods of any point. Or, more rigorously,
\begin{cor}
	For a t.v.s $X$, the filter $\mathscr{F}(x)$ of neighbourhoods of $x \in X$ is as same as $\{~ U+x \colon U \in \mathscr{F}(e) ~\}$, where $e$ is the unit element in $X$.
\end{cor}

Thus, to research the topology of a t.v.s. $X$, we just need to research the filter $\mathscr{F}(e)$ of neighbourhoods of $e$. First, there are two special properties of some subsets of a t.v.s. $X$.
\begin{defn}
	For a subset $U$ of a t.v.s. $X$, 
	\begin{enumerate}[label=\arabic*)]
		\item $U$ is absorbing if $\forall ~ x \in X,~ \exists ~ \rho > 0 ~s.t.~ \forall~ \lambda \in \K$ with $ \abs{\lambda} \leqslant \rho$, we have $\lambda x \in U$.
		\item $U$ is balanced if $\forall ~ x \in U,~ \forall \lambda \in \K$ with $\abs{\lambda} \leqslant 1$, we have $\lambda x \in U$.
	\end{enumerate}
\end{defn}

Then, the following theorem reveals the essence of $\mathscr{F}(e)$.
\begin{thm}
	A filter $\mathscr{F}$ of a vector space $X$ over $\K$ is the filter of neighbourhoods of the unit element $e$ w.r.t. some topology compatible with the algebraic structure of $X$ if and only if 
	\begin{enumerate}[label=\arabic*)]
		\item $\forall ~ U \in \mathscr{F},~ e \in U$
		\item $\forall ~ U \in \mathscr{F},~ \exists ~ V \in \mathscr{F} ~s.t.~ V+V \subset U$
		\item $\forall ~ U \in \mathscr{F},~ \forall ~ \lambda \in \K$ with $\lambda \neq 0$, $\lambda U \in \mathscr{F}$
		\item $\forall ~ U \in \mathscr{F},~ U$ is absorbing
		\item $\forall ~ U \in \mathscr{F},~ \exists ~ V \in \mathscr{F} ~s.t.~ V \subset U$ is balanced
	\end{enumerate}
\end{thm}
\begin{proof}
	If $\mathscr{F} = \mathscr{F}(e)$, these statements clearly hold.\\
	$1)$ is trivial.\\
	$2)$ is true since the addition is continuous.\\
	$3)$ and $4)$ hold since the scalar multiplication is continuous.\\
	For $5)$, because the scalar multiplication is continuous, we can find a $W \in \mathscr{F} ~s.t.~ \lambda W \subset U$ for any $\abs{\lambda} \leqslant \rho$, then let $V = \bigcup_{\abs{\lambda} \leqslant \rho} \lambda W$. Clearly, $V \in \mathscr{F}$ and $V$ is balanced.\\
	Conversely, We can define
	\begin{equation*}
		\mathscr{F}(x) = \{~ U+x \colon U \in \mathscr{F} ~\}
	\end{equation*}
	for any $x \in X$. It can be easily checked that $\mathscr{F}(x)$ satisfies the conditions in Example \ref{exam1} in last subsection. Therefore, these $\mathscr{F}(x)$ can determine a unique topology $\mathscr{T}$ on $X$.\\
	Now, we just need to check the continuity of the addition and the scalar multiplication. The addition is continuous, since $\mathscr{F}$ satisfies $2)$.  Using conditions $2)$ and $4)$ and $5)$ to get a balanced absorbing open neighbourhood in $\mathscr{F}$, and this neighbourhood prove the continuity of the scalar multiplication.
\end{proof}

Here is some simple properties of a t.v.s. $X$. These properties are directly obtained by definition and above theorem.
\begin{prop}
	For a t.v.s. $X$,
	\begin{enumerate}[label=\arabic*)]
		\item proper subspaces of $X$ are never absorbing. In particular, if $M \subset X$ is a open subspace, then $M = X$.
		\item each linear subspace of $X$, endowed with subspace topology, is also a t.v.s.
		\item if $H$ is a linear subspace of $X$, then $\clo{H}$ is also a linear subspace of $X$.
		\item if $Y$ is also a t.v.s. and $f \colon X \sto Y$ is a linear map, then $f$ is continuous if and only if $f$ is continuous at the unit element $e$.
	\end{enumerate}
\end{prop}

\vspace{0.2in}
\rule{1mm}{1mm} \textbf{Hausdorff t.v.s.~:} The Hausdorff Space is important since it can let the concept of limit make sense. And the topology of a t.v.s. can be simplified and has some additional properties, we can get a easier condition that make a t.v.s. become Hausdorff.

\begin{prop}
	A t.v.s $X$ is a Hausdorff space if and only if for any $x \in X$ with $x \neq e$ there exists a $U \in \mathscr{F}(e)$ s.t. $x \notin U$.
\end{prop}
\begin{proof}
	Since the open neighbourhoods of any point in $X$ is completely determined by the open neighbourhoods of $e$, this proposition is equivalent to the statement that ($T_1$) implies Hausdorff.\\
	The proof can be accomplished by obtaining a contradiction to the given condition that $x \neq e$, $\exists ~U \in \mathscr{F}(e)$ s.t. $x \notin U$. For that $U$, there is a balanced $V \in \mathscr{F}(e) ~s.t.~ V+V \subset U$ and the balance implies that $V-V \subset U$. Therefore, $(x+V) \bigcap V = \varnothing$. If not, $x+v_1 = v_2$ for $v_1,~ v_2 \in V$. This implies that $x = v_1 - v_2 \in V-V \subset U$. Thus it is a contradiction.
\end{proof}
The following theorem is more explicit.
\begin{thm} \label{thm1}
	For t.v.s. $X$ the following statements are equivalent.
	\begin{enumerate}[label=\arabic*)]
		\item $X$ is Hausdorff.
		\item the intersection of all neighbourhoods of the unit element $e$ is \{e\}.
		\item \{e\} is closed.
	\end{enumerate}
\end{thm}
\begin{proof}
	Before the rigorously proving, the intuition is clearly. Since in a t.v.s. ($T_1$) is equivalent to Hausdorff, the equivalence of $1)$ and $3)$ is clearly true.
	\begin{enumerate}
		\item[$1) \Rightarrow 2)$] It is because that elements in $\mathscr{F}(e)$ can separate $e$ and other points.
		\item[$2) \Rightarrow 3)$] If $x \in \clo{\{e\}}$, i.e. $\forall V_x \in \mathscr{F}(x),~ V_x \bigcap \clo{\{e\}} \neq \varnothing \Rightarrow e \in V_x$, and $V_x = U + x$ for some $U \in \mathscr{F}(e)$, then $u + x = e$ for some $u \in U$. Thus, $x = -u \in -U$ for all $U \in \mathscr{F}(e)$. That implies $x = e$.
		\item[$3) \Rightarrow 1)$] By above mentioned, it just needs to check that if for any topology space $Y$, $\{y\}$ is closed $\forall y \in Y$, $Y$ is ($T_1$). \\
		Since $\{y_1\}$ is closed, $Y \backslash \{y_1\}$ is open. That means if $y_2 \neq y_1$, there exists a open neighbourhood $U$ of $y_2$ s.t. $y_1 \notin U$. Similarly, we can find a open neighbourhood $V$ of $y_1$ s.t. $y_2 \notin V$. Therefore, $Y$ is ($T_1$). \qedhere
	\end{enumerate}	
\end{proof}

\vspace{0.2in}
\rule{1mm}{1mm} \textbf{Quotient t.v.s.~:} For a linear subspace $M$ of a t.v.s. $X$, the quotient topology on $X/M$ can be obtained by the quotient map $\pi \colon X \sto X/M$. But because of the algebraic structure, it has more properties.
\begin{prop}
	For a linear subspace $M$ of a t.v.s. $X$, the quotient map $\pi \colon X \sto X/M$ is open.
\end{prop}
\begin{proof}
	Let $V \subset X$ be open, then we have
	\begin{equation*}
		\pi^{-1}(\pi(V)) = V + M = \bigcup_{m \in M}(V+m)
	\end{equation*}
	Since $V$ is open, $V+m$ is open. Thus $\pi^{-1}(\pi(V))$ is open. And by the definition of the topology on $X/M$, $\pi(V)$ is open.
\end{proof}
\begin{cor}
	For a linear subspace $M$ of a t.v.s. $X$, the quotient space $X/M$ endowed with the quotient topology is a t.v.s..
\end{cor}
\begin{proof}
	We have the following commutative graph, where $f$ and $g$ are corresponding addition maps or scalar multiplication maps on $X$ and $X/M$.
	\begin{center}
		\begin{tikzcd}
			X \times X \arrow[r, "f"] \arrow[d, "\pi \times \pi"]
				& X \arrow[d, "\pi"] \\
			X/M \times X/M \arrow[r, "g"]
				& X/M
		\end{tikzcd}
	\end{center}
	Then for an open set $V \subset X/M$, since $f$ and $\pi$ are continuous, and $\pi$ is open, $(\pi \times \pi) \circ f^{-1} \circ \pi^{-1}(V)$ is open. By above commutative graph, we have $g \circ (\pi \times \pi) = \pi \circ f$. Therefore, $g^{-1}(V)$ is open, i.e. $g$ is continuous.
\end{proof}

Also, we can find the condition that lets the quotient topological vector space be Hausdorff.
\begin{prop}
	Let $X$ be a t.v.s..
	\begin{enumerate}[label=\arabic*)]
		\item $M$ be a linear subspace of $X$. Then $X/M$ is Hausdorff if and only if $M$ is closed.
		\item $X/\clo{\{e\}}$ is Hausdorff.
	\end{enumerate}	 
\end{prop}
\begin{proof}
	$2)$ is true because $1)$. And $1)$ clearly holds since $M$ is the unit element in $X/M$ and Theorem \ref{thm1} in this subsection.
\end{proof}
\begin{rem}
	By this method, for any t.v.s., we can find a Hausdorff space w.r.t it.
\end{rem}

\subsection{Continuous Linear Maps}

The interesting maps between two topological vector spaces not only preserve the algebraic structure, but also the topological structure, thus these are continuous linear maps.\\
First, for a linear map $f \colon X \sto Y$ between vector spaces $X$ and $Y$, we have the commutative graph, where $\tilde{f}(x+\ker{f}) = f(x)$ is well-defined.
\begin{center}
	\begin{tikzcd}
		X \arrow[r, "f"] \arrow[d, "\pi"]
			& \Img{f} \arrow[r, "i"]
			& Y \\
		X/\ker{f} \arrow[ru, "\tilde{f}"]
	\end{tikzcd}
\end{center}

\begin{prop}
	Let $f \colon X \sto Y$ be a linear map between two t.v.s.'s $X$ and $Y$.
	\begin{enumerate}[label=\arabic*)]
		\item If $Y$ is Hausdorff and $f$ is continuous, then $\ker{f}$ is closed.
		\item By above notation, $f$ is continuous if and only if $\tilde{f}$ is continous.
	\end{enumerate}
\end{prop}
\begin{proof} 
	$1)$ is because that $\ker{f} = f^{-1}(\{e\})$ and $Y$ is Hausdorff.\\
	For $2)$, if $\tilde{f}$ is continuous, it is clearly that $f = i \circ \tilde{f} \circ \pi$ is continuous. Conversely, it is because of the universal property of quotient maps. And in this case, let $U \subset \Img{f}$ be open, then $f^{-1}(U)$ is open and $\tilde{f}^{-1}(U) = \pi(f^{-1}(U))$. Since $\pi$ is open, $\tilde{f}^{-1}(U)$ is open. Thus, $\tilde{f}$ is continuous.
\end{proof}

\subsection{Complete Topological Vector Spaces}

We have just defined the completeness on a metric space by using sequence, but in metric spaces, we know the topology is so powerful that sequences can do any thing, but in general topology, or the topology in a t.v.s., we need an equivalent concept to describe the completeness.

\begin{defn}
	(Completeness)
	\begin{enumerate}[label=\arabic*)]
		\item A filter $\mathscr{F}$ on a subset $A$ of a t.v.s. $X$ is said to be a Cauchy filter if
			\begin{equation*}
				\forall~ U \in \mathscr{F}(0) \textbf{ in } X,~ \exists ~ M \subset A ~s.t.~ M \in \mathscr{F} ~\&~ M-M \subset U
			\end{equation*}
		\item A subset A of a t.v.s. X is said to be complete if every Cauchy filter on A converges to a point $x \in A$.
	\end{enumerate}
\end{defn}
\begin{rem}
	Said "the filter converges to a point" means that we can define a net on this filter, and this net converge a point. And this definition is also valid without the algebraic structure.
\end{rem}

By this definition, and using the factor that Hausdorff spaces let the limit point of a net uniquely exist, we have similar results comparing with the metric spaces.
\begin{prop}
	Let $X$ be a t.v.s..
	\begin{enumerate}[label=\arabic*)]
		\item If $X$ is Hausdorff, any complete set is closed.
		\item If $X$ is complete, any closed set is complete.
	\end{enumerate}
\end{prop}

We known any metric space can be completion. Similarly, the same result can obtained in any t.v.s..
\begin{thm}
	Let $X$ be a Hausdorff t.v.s., then there exists a complete Hausdorff t.v.s. $\hat{X}$ and a map $i \colon X \sto \hat{X}$ with the following properties.
	\begin{enumerate}[label=\arabic*)]
		\item $i$ is a topological monomorphism.
		\item $\clo{i(X)} = \hat{X}$.
		\item For any complete Hausdorff t.v.s. $Y$ and for every continuous linear map $f \colon X \sto Y$, there exists a continuous map $\hat{f} \colon \hat{X} \sto Y$, s.t. the following graph is commutative
		\begin{center}
			\begin{tikzcd}
				X \arrow[r, "f"] \arrow[d, "i"]
					& Y \\
				\hat{X} \arrow[ru, "\hat{f}"]
			\end{tikzcd}
		\end{center}
	\end{enumerate}
	And ($\hat{X}$, $\hat{f}$) is unique with respect to the isomorphism
\end{thm}
\begin{proof}
	The proof is similar as the proof of the completion of metric spaces, which contructs the $\hat{X}$ as a set of equivalent classes of Cauchy sequences. In a t.v.s., we just need to replace Cauchy sequences by Cauchy filters (in fact, Cauchy nets). Let 
	\begin{center}
		\begin{tabular}{r c l}
			$\tilde{X}$ & $=$ & $\{~ \text{all Cauchy filters in } X ~\}$\\
			$R $ & $\colon$ & $ \mathscr{F} ~R~ \mathscr{G} \Leftrightarrow \forall ~ U \in \mathscr{F}(e),~ \exists ~ A \in \mathscr{F} ~\&~ B \in \mathscr{G} ~s.t.~ A-B \subset U$\\
			$\hat{X}$& $=$ & $\tilde{X} / R$
		\end{tabular}
	\end{center}
	We can easily define linear operations and topology, s.t. $\hat{X}$ become a complete t.v.s.. Then we just need to check the statements in above theorem hold.
\end{proof}

\subsection{Finite Dimensional Topological Vector Spaces}

For a finite dimensional topological vector space, the topology compatible with the algebraic structure has some properties coincided with the "finity". First, continuous linear functionals on a t.v.s. have some properties.

\begin{lem}
	Let $X$ be a t.v.s. over $\K$. Fixed $v \in X$, then the $\phi_{v} \colon \K \sto X$ by $\xi \mapsto \xi v$ is continuous,
\end{lem}
\begin{proof}
	It is because that $\phi_{v} = f \circ \psi_{v}$ where $f$ is the multiplication map.
	\begin{center}
		\begin{tabular}{r l c l r}
			$\K$ & $\stackrel{\psi_{v}}{\rightarrow}$ & $\K \times X$ & $\stackrel{f}{\rightarrow}$ & $X$ \\
			$\xi$ & $\mapsto$ & ($\xi$, $v$) & $\mapsto$ & $\xi v$ 
		\end{tabular} 
	\end{center}
\end{proof}

\begin{lem}
	For a non-zero linear functional $L \colon X \sto \K$, where $X$ is a t.v.s. over $\K$, the following statements are equivalent.
	\begin{enumerate}[label=\arabic*)]
		\item $L$ is continuous,
		\item $\ker{L}$ is closed,
		\item $\ker{L}$ is not dense in $X$,
		\item $L$ is bounded in some neighbourhood of the origin in $X$.
	\end{enumerate}
\end{lem}
\begin{proof}
	The equivalence of $1)$ and $2)$ and $4)$ is clearly.
	\begin{enumerate}
		\item[$1) \Rightarrow 2)$] It is because that $\ker{L} = L^{-1}(\{0\})$.
		\item[$2) \Rightarrow 3)$] Since $L$ is non-zero, it clearly holds.
		\item[$3) \Rightarrow 4)$] By the assumption, there exists a balanced set $V \in \mathscr{F}(e)$ and a point $x \notin \clo{\ker{L}}$ s.t. $(x+V) \bigcap \ker{L} = \varnothing$. $L(V)$ is balanced on $\K$, therefore $L(V)$ is bounded or $L(V) = \K$. But since $(x+V) \bigcap \ker{L} = \varnothing$, $L(V) \neq \K$.
		\item[$4) \Rightarrow 1)$] This implies that $L$ is continuous at $e$. But since $X$ is a t.v.s., $L$ is continuous at every point. \qedhere
	\end{enumerate}
\end{proof}

\begin{thm}
	Let $X$ be a finite dimensional Hausdorff t.v.s. over $\K$ (endowed with the standard topology), and $\dim{X}$ = $d$. Then we have: 
	\begin{enumerate}[label=\arabic*)]
		\item $X$ is topologically isomorphic to $\K^{d}$,
		\item every linear  functional on $X$ is continuous,
		\item every linear map from $X$ to any t.v.s. $Y$ is continuous
	\end{enumerate}
\end{thm}
\begin{proof}
	For $1)$, we just need to find a homeomorphic isomorphism from $\K^{d}$ to $X$, like the following map, where $\{e_i\}_{i=1}^{d}$ is the basis of $X$.
	\begin{center}
		\begin{tabular}{c c c}
			$\K^{d}$ & $\stackrel{\phi}{\longrightarrow}$ & $X$ \\
			$(\lambda_1,~ \lambda_2,~ \cdots,~ \lambda_d)$  & $\longmapsto$ & $\lambda_1 e_1 + \lambda_2 e_2 + \cdots + \lambda_d e_d$
		\end{tabular} 
	\end{center}
	$\phi$ is clearly an algebraic isomorphism. Thus we just need to check $\phi$ is both continuous and open.\\
	Check: $\phi$ is continuous.\\
	When $d = 1$, it is continuous by above lemma. For the general case, since $d$ is finite, $\phi$ is continuous.\\
	Check: $2)$ holds and $\phi$ is open.\\
	When $d = 1$, it is clearly $1)$ and $2)$ are trued. And suppose $1)$ and $2)$ hold for $\dim{X} \leqslant d-1$, then when $\dim{X} = d$, let $L$ be a non-zero linear function on $X$. Then since $\iso{X/\ker{L}}{\Img{L}} \subset \K$, $\dim{\ker{L}} = d-1$. Therefore, $\iso{\ker{L}}{\K^{d-1}} \Rightarrow \ker{L}$ is complete $\Rightarrow \ker{L}$ is closed $\Rightarrow L$ is continuous by above lemma. And,  
	\begin{center}
		\begin{tabular}{c c c}
			$X$ & $\stackrel{\phi^{-1}}{\longrightarrow}$ & $\K^{d}$ \\
			$\lambda_1 e_1 + \lambda_2 e_2 + \cdots + \lambda_d e_d$  & $\longmapsto$ & $(\lambda_1,~ \lambda_2,~ \cdots,~ \lambda_d)$
		\end{tabular} 
	\end{center}
	is continuous since each 
	\begin{equation*}	
			\lambda_1 e_1 + \lambda_2 e_2 + \cdots + \lambda_d e_d \longmapsto \lambda_i
	\end{equation*}
	is continuous.\\
	Then for $3)$, it is clealy since $\dim{\Img{L}} < \infty$. 
\end{proof}
 
\begin{cor}
	~
	\begin{enumerate}[label=\arabic*)]
		\item Every finite dimensional Hausdorff t.v.s. is complete.
		\item Every finite dimensional subspace of a Hausdorff t.v.s. is closed.
		\item For a finite dimensional vector space, there is only one topology w.r.t. homeomorphism that can make it be a Hausdorff t.v.s..
		\item Every bounded subset on a finite dimensional Hausdorff t.v.s. is compact.
	\end{enumerate}
\end{cor}
\begin{proof}
	These properties can be easily obtained by regarding the t.v.s. as $\K^{d}$ endowed with the standard topology.
\end{proof}

Finally, the most important theorem in this subsection is that the converse of $4)$ in above corollary is also true.
\begin{thm}
	A Hausforff t.v.s. is locally compact if and only if it is finite dimensional.
\end{thm}
\begin{proof}
	Let $X$ be a locally compact Hausdorff t.v.s. and K be a compact heighbourhood of $e$ in $X$, i.e.
	\begin{equation*}
		\exists ~ x_1,~ \cdots,~ x_r \in X ~s.t.~ K \subset \bigcup_{i=1}^{r}(x_i+\frac{1}{2}K)
	\end{equation*}
	Let $M = \spn{\{x_1,~ \cdots,~ x_r\}}$, and $M$ is closed. Therefore, $X/M$ is a Hausdorff t.v.s.. Let $\pi \colon X \sto X/M$ be the quotient map.\\
	Since $K \subset M+\frac{1}{2}K$, $\pi(K) \subset \pi(\frac{1}{2}K)$. Thus, by iterating $\pi(2^{n}K) \subset \pi(K)$.\\
	As $K$ is absorbing, $X = \bigcup_{n=1}^{\infty}2^{n}K$,
	\begin{equation*}
		X/M = \pi(X) = \bigcup_{n=1}^{\infty}\pi(2^{n}K) \subset \pi(K) \subset X/M
	\end{equation*}
	And since $\pi$ is continuous, $\pi(K)$ is compact, i.e. $X/M$ is compact.\\
	claim: $\dim{X/M} = 0$ \\
	Suppose $\dim{X/M} > 0$, then for some $\clo{x_0} \in X$ with $\clo{x_0} \neq \clo{e}$, $\R \clo{x_0} \subset X/M$. And since $X/M$ is Hausdorff compact and $\R \clo{x_0}$ is closed, $\R \clo{x_0}$ is compact, which is a contradiction.
\end{proof}

\section{Locally Convex Topological Vector Spaces}

The locally convex topological vector space is a topological vector spaces whose topology is generated by a family of seminorms, thus it can provide more properties.

\subsection{Definition by Convex Sets}

Firstly, the conception of locally convex space can be obtained by convex sets. So, we need to research some elementary traits of convex sets.

\begin{defn}
	~
	\begin{enumerate}[label=\arabic*)]
		\item Let $S$ be any subset of a vector space $X$ over $\K$. The convex hull of $S$, $conv(S)$, is the smallest convex subset containing $S$. In fact, 
		\begin{equation*}
			conv(S) = \{~ \sum_{i=1}^{n}\lambda_i x_i \colon x_i \in S,~ \lambda_i \in [0,1],~ \sum_{i=1}^{n}\lambda_i = 1,~ n \in \N ~\}
		\end{equation*}
		\item A subset $S$ of a vector space $X$ over $\K$ is absolutely convex, if $S$ is convex and balanced
		\item A subset $S$ of a vector space $X$ over $\K$ is called a barrel if $S$ is closed, absorbing and absolutely convex.
	\end{enumerate}
\end{defn}

\begin{prop}
	~
	\begin{enumerate}[label=\arabic*)]
		\item Arbitrary intersections of convex sets are convex sets, and the sum of two convex sets is convex, and linear maps preserve convex.
		\item The convex hull of a balanced set is balanced.
		\item The closure and the interior of a convex set in a t.v.s. is convex.
		\item Every neighbourhood of the origin of a t.v.s. is contained in a neighbourhood of the origin which is a barrel.
	\end{enumerate}
\end{prop}
\begin{proof}
	$1)$ and $2)$ are easily obtained by the definition. \\
	To prove $3)$, for any $\lambda \in [0,1]$, we define the map
	\begin{center}
		\begin{tabular}{l c c l}
			$\phi ~ \colon$ & $X \times X$ & $\longrightarrow$ & $X$ \\
			~ & $(x,y)$ & $\mapsto$ & $\lambda x + (1-\lambda) y$
		\end{tabular}
	\end{center}
	Using the fact that $\phi(S \times S) \subset S$, $S \times S \subset \phi^{-1}(S) \phi^{-1}(\clo{S})$. And since $X$ is a t.v.s., $\phi$ is continuous. Thus $\phi^{-1}(\clo{S})$ is closed, i.e. $\clo{S} \times \clo{S} \subset \phi^{-1}(\clo{S}) \Rightarrow \phi(\clo{S} \times \clo{S}) \subset \clo{S}$. $\clo{S}$ is convex.\\
	Since $z+u = \lambda x + (1-\lambda) y + \lambda u + (1-\lambda) u$ if $z = \lambda x + (1-\lambda) y$, $x + U ~\&~ y + U \subset S \Rightarrow z + U \subset S$. That means $\inte{S}$ is convex.\\
	For $4)$, $\forall ~ U \in \mathscr{F}(e)$, the set
	\begin{equation*}
		\clo{conv\left(\bigcup_{\abs{\lambda} \leqslant 1}\lambda U\right)}
	\end{equation*}
	is a barrel.
\end{proof}
\begin{cor}
	Every neighbourhood of the origin in a t.v.s. $X$ is contained in a neighbourhood of the origin which is absolutely convex.
\end{cor}

Now, we can get the definition of a locally convex t.v.s..
\begin{defn}
	A t.v.s. $X$ is said to be locally convex if there is a basis of neighbourhoods of the origin in $X$ consisting of convex sets.
\end{defn}

By this definition, the structure of the neighbourhoods of the origin in a locally convex t.v.s. can be more explicit by using above proposition.

\begin{prop}
	Let $X$ be a locally convex t.v.s..
	\begin{enumerate}[label=\arabic*)]
		\item $X$ has a basis of neighbourhoods of origin consisting with oepn absorbing absolutely convex sets.
		\item  $X$ has a basis of neighbourhoods of origin consisting with barrels.
	\end{enumerate}
\end{prop}

\begin{thm}
	If $X$ is a locally convex t.v.s., then there exists a basis $\mathscr{B}$ of neighbourhoods of origin consisting of absorbing absolutely convex set s.t.
	\begin{equation*}
		\forall ~ U \in \mathscr{B},~ \forall ~ \rho > 0,~ \exists ~ W \in \mathscr{B} ~s.t.~ W \subset \rho U
	\end{equation*}
	Conversely, if $\mathscr{B}$ is a collection of absorbing absolutely convex subsets of a vector space satisfying above condition, it can generate a unique locally convex t.v.s..
\end{thm}

\subsection{Definition by Seminorms}

\begin{defn}
	Let $X$ be a vector space over $\K$. A map $p \colon X \sto \R^{+}$ is called a seminorm if it satisfies:
	\begin{enumerate}[label=\arabic*)]
		\item $p(x+y) \leqslant p(x)+p(y),~ \forall ~ x,~ y \in X$,
		\item $p(\lambda x) = \abs{\lambda} p(x),~ \forall ~ x \in X,~ \forall ~ \lambda \in \K$.
	\end{enumerate}
\end{defn}
\begin{rem}
	In fact, $\ker{p}$ is a linear sbuspace and if $\ker{p} = \{0\}$, $p$ is called a norm.
\end{rem}

By the intuition, the seminorm could construct the continuity of the addition and multiplication since it satisfies above properties. Now, we can build this rigorously.

\begin{defn}
	Let $X$ be a vector space and $A \subset X$ be a nonempty subset. The Minkowski functional of $A$ is the map
	\begin{center}
		\begin{tabular}{l c c l}
			$p_A ~ \colon$ & $X$ & $\longrightarrow$ & $\R$ \\
			~ & $x$ & $\mapsto$ & $\inff\{~ \lambda > 0 \colon x \in \lambda A ~\}$
		\end{tabular}
	\end{center}
\end{defn}

Let $X$ be a vector space and $p$ is a seminorm on $X$, then let $\inte{U}_p = \{~ x \in X \colon p(x) < 1 ~\}$, $U_p = \{~ x \in X \colon p(x) \leqslant 1 ~\}$. Thus, $U$ may be the basis generating the topology. To see it, the following proposition is helpful.

\begin{prop}
	Let $A \subset X$ be a nonempty subset of a vector space, which is absorbing and absolutely convex, then $p_A$ is a seminorm and $\inte{U}_{p_A} \subset A \subset U_{p_A}$. Conversely, if $q$ is a norm on $X$ then $\inte{U}_q$ is an absorbing absolutely convex set and $q = p_{\inte{U}_q}$.
\end{prop}
\begin{proof}
	Since $A$ is balanced, $\xi A \in \lambda A \Leftrightarrow x \in \frac{\lambda}{\abs{\xi}} A$. Thus,
	\begin{equation*}
		p_A(x) = \abs{\xi} \inff{\{~ \frac{\lambda}{\abs{\xi}} \colon x \in \frac{\lambda}{\abs{\xi}} A~\}} = \abs{\xi} p_A(x)
	\end{equation*}
	And $p_A(x) < \infty  ~( \forall ~ x \in X)$ since $A$ is absorbing.\\
	Fixed $x,~ y \in X$, $\forall \varepsilon > 0,~ \exists ~ \lambda,~ \mu > 0$, s.t. $x \in \lambda A ~\&~ y \in \mu A$ and
	\begin{equation*}
		\lambda \leqslant p_A(x) + \varepsilon ~,~ \mu \leqslant p_A(y) + \varepsilon,
	\end{equation*}
	By convexity of $A$, $\lambda A + \mu A \subset (\lambda + \mu) A$. Thus,
	\begin{equation*}
		p_A(x) = \inff{\{~ \delta > 0 \colon x+y \in \delta A ~\}} \leqslant \lambda + \mu \leqslant p_A(x) + p_A(y) + 2\varepsilon
	\end{equation*}
	$\Rightarrow ~ p_A(x)$ is a seminorm.
	\begin{center}
		\begin{tabular}{r c l}
			$x \in \inte{U}_{p_A}$ & $\Rightarrow$ & $\exists ~ \lambda \in [0,1] ~s.t.~ x \in \lambda A \subset A$ \\
			$x \in A$ &  $\Rightarrow$  & $1 \in \inff\{~ \lambda > 0 \colon x \in \lambda A ~\} ~ \Rightarrow ~ p_A(x) \leqslant 1 \Rightarrow x \in U_{p_A}$
		\end{tabular}
	\end{center}
	That means $\inte{U}_{p_A} \subset A \subset U_{p_A}$.\\
	Finally, the statements for $q$ can be obtained easily by the definition.
\end{proof}

Now, we can give the definition of a locally convex t.v.s. by seminorms coinciding with the definition by convex sets.

\begin{thm}
	Let $X$ be a vector space and $\mathscr{P} = \{p_i\}_{i \in I}$ be a family of seminorms. Then the initial topology $\mathscr{T}_P$ generated by $\mathscr{P}$ makes $X$ be a locally convex t.v.s. In fact, the basis of neighbouhoods of the origin in $X$ is like
	\begin{equation*}
		\mathscr{B} = \left\{~ \{x \in X \colon p_{i_1}(x)<\varepsilon,\cdots,p_{i_n}(x)<\varepsilon\} \colon i_1,\cdots,i_n \in I,n \in \N, \varepsilon > 0 ~\right\}
	\end{equation*}
	Conversely, the topology of any locally convex t.v.s. can be generated by a family of seminorms.
\end{thm}
\begin{proof}
	Each element in the subbasis of $\mathscr{T}_P$ is like $\{ x \in X \colon p_i(x) < \varepsilon \} = \varepsilon \inte{U}_{p_i}$, which is clearly absorbing and absolutely convex. Therefore, every element in $\mathscr{B}$ is convex. $(X,\mathscr{T}_P)$ is a locally convex t.v.s..\\
	Convesrly, if $(X, \mathscr{T})$ is a locally convex t.v.s., the basis of neighbourhoods of the origin in $X$ consists of absorbing and absolutely convex stes, which can generate a family of seminorms by above proposition. Therefore, these seminorms can generate a locolly convex topology $\mathscr{T}_P$. In fact, $\mathscr{T}_P = \mathscr{T}$, since $\inte{U}_{p_A} \subset A \subset U_{p_A}$.
\end{proof}
\begin{rem}
	By this theorem, in a vector space $X$, the seminorms on $X$ can coincide with a locally convex topology making $X$ be a t.v.s..
\end{rem}

There some extra properties for the seminorms on a vector space.
\begin{prop} \label{prop1}
	Let $X$ be a vector space and $p$ be a seminorm on $X$. Then,
	\begin{enumerate}[label=\arabic*)]
		\item $\forall~ r>0,~ r \inte{U}_p=\{x \in X \colon p(x) <r\} = \inte{U}_{\frac{1}{r}p}$.
		\item $\forall x \in X,~ x+\inte{U}_p = \{y \in X \colon p(y-x) < 1\}$.
		\item if $q$ is a seminorm on $X$, $p \leqslant q \Leftrightarrow \inte{U}_q \subset \inte{U}_p$.
		\item if $\{s_i\}_{i=1}^{n}$ are seminorms on $X$, then $s(x) = \max_{i=1,\cdots,n}{s_i(x)}$ is also a seminorm and $\inte{U}_s = \bigcap_{i=1}^{n} \inte{U}_{s_i}$
	\end{enumerate}
\end{prop}

\begin{thm}
	Let $\mathscr{P} = \{p_i\}_{i \in I}$ and $\mathscr{Q} = \{q_j\}_{j \in J}$ be two families of seminorms on a vector space $X$ inducing $\mathscr{T}_P$ and $\mathscr{T}_Q$, then
	\begin{equation*}
		\mathscr{T}_Q \subset \mathscr{T}_P \Leftrightarrow \forall~ q \in \mathscr{Q},~ \exists~ \{i_k\}_{k=1}^{n} \subset I,~ \exists~ C>0,~s.t.~ Cq(x) \leqslant \max_{k=1,\cdots,n}{p_{i_k}(x)}
	\end{equation*}
\end{thm}
\begin{proof}
	This right side of above statement is equivalent to that
	\begin{equation*}
		\forall~ q \in \mathscr{Q},~ \exists~ \{i_k\}_{k=1}^{n} \subset I,~ \exists~ C>0,~s.t.~ C \bigcap_{k=1}^{n} \inte{U}_{p_{i_k}} \subset \inte{U}_q
	\end{equation*}
	So, it is clearly equivalent to $\mathscr{T}_Q \subset \mathscr{T}_P$.
\end{proof}
\begin{rem}
	Because of this, we have the definition of equivalent norms. In fact, two norms $p$ and $q$ are said to be equivalent if and only if there exists $C_1,~ C_2 > 0$, s.t. $C_1p(x) \leqslant q(x) \leqslant C_2 p(x)$, for all $x \in X$. This definition means that two equivalent norms can generate one same topology. 
\end{rem}
\begin{cor} \label{cor1}
	The family $\mathscr{P} = \{p_i\}_{i \in I}$ of seminorms and \\$\mathscr{Q} = \{~ \max_{i \in B} p_i \colon \varnothing \neq B \subset I \text{ with } B \text{ finite} ~\}$ can generate one same topology on $X$
\end{cor}

\subsection{Separability and Metrizability}

For a t.v.s., $(T_1)$ is equivalent to Hausdorff and $(T_1)$ is associated with the ability of a topology separating points. Therefore, wheather a t.v.s. is a Hausdorff space or not is completely determined by wheather the topology of it can separate points or not. Then for a locally convex t.v.s., whose topology is induced by a family of seminorms, this separability is related to these seminorms.

\begin{defn}
	A family of seminorms $\mathscr{P} = \{~p_i~\}_{i \in I}$ on a vector space $X$ is said to be speparating, if
	\begin{equation*}
		\forall~ x \in X \backslash \{0\},~ \exists~ i \in I ~s.t.~ p_i(x) \neq 0
	\end{equation*}
\end{defn}
\begin{rem}
	In fact, above condition is equivalent to 
	\begin{equation*}
		\text{if } p_i(x) = 0, ~\forall~ i \in I \Rightarrow x = 0
	\end{equation*}
\end{rem}	

Now, we can give the condition that makes a locally convex t.v.s. be Hausdorff.

\begin{thm}
	A locally convex t.v.s. $X$ is Hausdorff if and only if its topology can be induced by a separating family of seminorms $\mathscr{P} = \{~p_i~\}_{i \in I}$ .
\end{thm}
\begin{proof}
	If $\mathscr{P} = \{~p_i~\}_{i \in I}$ is separating, the fact that $\mathscr{T}_P$ is Hausdorff can be obtained easily by the definition.\\
	Conversely, if $X$ is Hausdorff, for $x \neq 0$, we can find a $U \in \mathscr{F}(0)$, s.t. $U$ can separate $x$ and $0$. But since $X$ is locally convex, $U$ can be chosen as $\inte{U}_p$ for a seminorm $p$. Thus, for this $p$, $p(x) \neq 0$.
\end{proof}

For the metrizability of a locally convex t.v.s., the consequece is also easier than general case.
\begin{thm}
	A locally convex t.v.s. $X$ is metrizable if and only if its topology is determined by a countable separating family of seminorms.
\end{thm}
\begin{proof}
	It can be directly obtained by the Nagata–Smirnov's Metrization Theorem. Also, there is a more explicit proof. If the topology of $X$ is generated by a countable separating family of seminorms $\mathscr{P} = \{~p_n~\}_{n=1}^{\infty}$, we can define the metric $d$ on $X$ by
	\begin{equation*}
		d(x,y) = \sum_{n=1}^{\infty} 2^{-n} \frac{p_n(x-y)}{1+p_n(x-y)}
	\end{equation*} 
	Conversely, if $(X,d)$ is the metric space, the subbasis of the topology generated by this metric is like $U_n = \{x \colon d(x,0) < 1/n\}$. And these $U_n$ can provide a countable separating family of seminorms. In fact, if $\mathscr{Q} = \{~q_i~\}_{i \in I}$ generates the topology of $X$, for each $U_n$, there are $q_1,\cdots,q_k \in \mathscr{Q}$ and $\varepsilon_1,\cdots,\varepsilon_k > 0$, s.t. $\bigcap_{i=1}^{k} \{x \colon q_i(x) < \varepsilon_i\} \subset U_n$. Then let $p_n = \sum_{i=1}^{k} \varepsilon_i^{-1}q_i$. It can check that the family $\{p_n\}_{n=1}^{\infty}$ generate the coincided topology on $X$.
\end{proof}

\subsection{Continuous Linear Maps on LCTVS}

To give the special property of continuous linear maps on LCTVS, we need firstly refine the family of seminorms.
\begin{defn}
	A family $\mathscr{Q} = \{~q_j~\}_{j \in J}$ of seminorms on a vector space $X$ is said to be directed if
	\begin{equation*}
		\forall ~ j_1, \cdots, j_n \in \mathscr{Q},~ \exists ~ j \in J ~\&~ C>0,~ s.t.~ C q_j(x) \geqslant \max_{k=1,\cdots,n} q_{j_k}(x),~ \forall~ x \in X
	\end{equation*}
\end{defn}
\begin{rem}
	By \textbf{Proposition} \ref{prop1} in \textbf{1.2.2}, this definition is equivalent to that
	\begin{equation*}
		\forall~ \inte{U}_{q_{j_1}}, \cdots, \inte{U}_{q_{j_n}},~ \exists~ \inte{U}_{q_j} ~s.t.~ \inte{U}_{q_j} \subset \bigcap_{k=1}^{n} \inte{U}_{j_i}
	\end{equation*}
	And thus the basis of this directed family of seminorms should be like
	\begin{equation*}
		\mathscr{B}_d = \{~ r\inte{U}_q \colon q \in \mathscr{Q},r>0 ~\}
	\end{equation*}
\end{rem}

By this special topology, we can find the condition making linear functional continuous.

\begin{prop}
	Let $\mathscr{T}$ be a locally convex topology on a vector space $X$ generated by a directed family $\mathscr{Q}$ of seminorms on $X$. Then
	\begin{equation*}
		L \colon ~ X \longrightarrow \K
	\end{equation*}
	is a $\mathscr{T}$-continuous linear functional if and only if $\exists ~ q \in \mathscr{Q}$ s.t. $L$ is $q$-continuous, i.e. $\abs{L(x)} \leqslant Cq(x)$ for some $C > 0$.
\end{prop}
\begin{proof}
	In fact, this property of continuous linear functional is because the element in a directed locally convex topology is like $r\inte{U}_q$. In fact, we just need to check the origin point.\\
	If $L$ is continuous, there exists a $r\inte{U}_q$ s.t. $r\inte{U}_q \subset L^{-1}(B_1(0))$, where $B_1(0)$ is the unit ball centered at $0$. This is equivalent to the $q$-continuity of $L$.\\
	Conversely, it is clearly by the fact $\mathscr{T}_q \subset \mathscr{T}$.
\end{proof}

We can easily see that the topology of a locally convex t.v.s. can be always induced by a directed family of seminorms by the \textbf{Corollary} \ref{cor1} in \textbf{1.2.2}. Thus, we have the corollary.

\begin{cor}
	$(X,\mathscr{T})$ is a locally convex t.v.s. and $\mathscr{T}$ is generated by the family $\mathscr{P} = \{p_i\}_{i \in I}$. Then $L \colon ~ X \longrightarrow \K$ is a continuous linear functional if and only if 
	\begin{equation*}
		\exists ~ i_1,\cdots,i_n \in I,~ \exists ~ C>0 ~s.t.~ \abs{L(x)} \leqslant C \max_{k=1,\cdots,n} p_{i_k}(x),~ \forall ~ x \in X
	\end{equation*}
\end{cor}

And this corollary can be easily extended to linear maps. And the proof is similar as above statement In fact, we just need to replace $B_1(0)$ by $\inte{U}_q$
\begin{thm}
	Let $X$ and $Y$ be two locally convex t.v.s.'s generated by $\mathscr{P}$ and $\mathscr{Q}$. Then linear map $f \colon X \sto Y$ is continuous if and only if 
	\begin{equation*}
		\forall ~ q \in \mathscr{Q},~ \exists ~ p_1,\cdots,p_n \in \mathscr{P},~ \exists ~ C>0 ~s.t.~ q(f(x)) \leqslant C \max_{i=1,\cdots,n} p_i(x)
	\end{equation*}
\end{thm}










\end{document}
