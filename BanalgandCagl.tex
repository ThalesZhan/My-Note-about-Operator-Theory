\chapter{Banach Algebras and \texorpdfstring{$C^{*}$-Algebras}{C*-Algebras}}
\section{Banach Algebras}

Let $\fml{H}$ be a Hilbert space. Then $\oper$ is indeed a Banach space. But we have more sructure on it. Any two element $S,T \in \oper$ can do multiplication, defined as $ST = S \circ T$, then $ST \in \oper$ by the definition of norm, and moreover $\norm{ST} \leqslant \norm{S}\norm{T}$. Therefore, $\oper$ is an algebra with the extra property of the multiplication, called a Banach algebra.

\subsection{Elementary Properties}

\begin{defn}
	A Banach algebra $\alg{A}$ is an algebra over $\C$ with a norm $\norm{\cdot}$ relative to which $\alg{A}$ is a Banach space and s.t. for all $a,b \in \alg{A}$,
	\begin{equation*}
		\norm{ab} \leqslant \norm{a}\norm{b}.
	\end{equation*}
\end{defn}
\begin{rem}
	The extra condition guarantees the multiplication is norm continuous. In fact, $\alg{A}$ with the multiplication and the norm is a topological semigroup.
\end{rem}

If $\alg{A}$ has an identity $1$, it assumes $\norm{1}=1$. But if $\alg{A}$ does not have an identity, we can add an identity to it.

\begin{prop}
	If $\alg{A}$ is a Banach algebra without the identity. The let $\tilde{\alg{A}}= \alg{A} \oplus \C$ be an induced vector space. Then we define a norm on $\tilde{\alg{A}}$ as
	\begin{equation*}
		\norm{(a,\lambda)} = \norm{a} + \abs{\lambda}
	\end{equation*}
	and define a multiplication on $\tilde{\alg{A}}$ as
	\begin{equation*}
		(a,\alpha)(b,\beta) = (ab+\alpha b + \beta a, \alpha \beta)
	\end{equation*}
	Then $\alg{A}$ is a Banach algebra with the identity $(0,1)$.
\end{prop}

Then $\alg{A}$ can be isometrically imbedded into $\tilde{\alg{A}}$. Therefore, we can always assume $\alg{A}$ has an identity. For a unit algebra, the invertibility of a element is important. 

\begin{thm}
	Let $\alg{A}$ be a Banach algebra and $a \in \alg{A}$. If $\norm{a-1} < 1$, then $a$ is invertible.
\end{thm}
\begin{proof}
	For a nonzero real number $\lambda \in \R$ with $\abs{\lambda} < 1$, we know 
	\begin{equation*}
		(1-\lambda)^{-1} = \sum_{n=0}^{\infty} \lambda^{n}
	\end{equation*}
	Similar, for $a \in \alg{A}$ with $\norm{a-1} < 1$, set
	\begin{equation*}
		b= \sum_{n=0}^{\infty} (1-a)^{n}
	\end{equation*}
	Firstly, since $\norm{(1-a)^{n}} \leqslant \norm{(1-a)}^{n}$, $b$ is well-defined. Then we can prove $b = (1-(1-a))^{-1} = a^{-1}$.
\end{proof}
\begin{rem}
	This result is important. It says a small perturbation of an invertible element is also invertible. It is because that the multiplication is norm continuous. And by the continuity of multiplication, this result can be true at any point other than 1.
\end{rem}

\begin{cor}
	Let $\alg{A}$ be a Banach algebra and 
	\begin{eqnarray*}
		G_l &=& \{~a \in \alg{A} \colon a \text{ is left invertible}~\}\\
		G_r &=& \{~a \in \alg{A} \colon a \text{ is right invertible}~\}\\
		G &=& G_l \bigcap G_r = \{~a \in \alg{A} \colon a \text{ is invertible}~\}
	\end{eqnarray*}
	Then $G_l$ and $G_r$ and $G$ are open. Moreover, the map $a \sto a^{-1}$ from $G$ to $G$ is continuous.
\end{cor}
\begin{proof}
	$G_l$ and $G_r$ and $G$ are open by above theorem. \\
	We just need to check this map is continuous at $1$ because of the continuity of multiplication. For $\{a_n\} \subset G$ with $a_n \sto 1$, thus $\norm{1-a_n}<\delta<1$. Since
	\begin{equation*}
		a_n^{-1} = (1-(1-a_n))^{-1} = \sum_{k=0}^{\infty} (1-a_n)^{k} = 1+ \sum_{k=1}^{\infty} (1-a_n)^{k}
	\end{equation*}
	Therefore,we have
	\begin{eqnarray*}
		\norm{1-a_n^{-1}} &=& \norm{\sum_{k=1}^{\infty} (1-a_n)^{k}}\\
		&\leqslant&  \sum_{k=1}^{\infty} \norm{1-a_n}^{k}\\
		&<& \frac{\delta}{1+\delta} < \delta = \norm{1-a_n}
	\end{eqnarray*}
	i.e. $\lim a_n = 1$.
\end{proof}

\begin{cor}
	Let $\alg{A}$ be a Banach algebra.
	\begin{enumerate}[label=\arabic*)]
		\item The closure of a proper ideal is a proper ideal.
		\item A maximal ideal is closed.
		\item every ideal contained in a maximal ideal.
	\end{enumerate}
\end{cor}

If $\alg{B}$ is a closed ideal of a Banach algebra $\alg{A}$, then then quotient algbra $\alg{A}/\alg{B}$ with the induced norm is also a Banach algebra since 
\begin{equation*}
	\norm{(a+\alg{B})(b+\alg{B})} = \norm{ab+\alg{B}} \leqslant \norm{(a+b_1)(b+b_2)} \leqslant \norm{(a+b_1)}\norm{(b+b_2)}
\end{equation*}
for any $b_1, b_2 \in \alg{B}$.

\subsection{Spectrum}


\begin{defn}
	Let $\alg{A}$ be a Banach algebra and $a \in \alg{A}$. The spectrum of $a$, denoted by $\sigma(a)$ defined as
	\begin{equation*}
		\sigma(a) = \{~ \lambda \in \C \colon a-\lambda \text{ is invertible}~\}
	\end{equation*}	
	And the resolvents of $a$, $\rho(a) = \C \backslash \sigma(a)$.\\
	Moreover, we can define the spectral radius of $a$ as
	\begin{equation*}
		r(a) = \sup{\{~\abs{\lambda} \colon \lambda \in \sigma(a)~\}}
	\end{equation*}
\end{defn}

Firstly, there are some elementary properties of the spectrum.

\begin{thm}
	Let $\alg{A}$ be a Banach algebra and $a \in \alg{A}$.
	\begin{enumerate}[label=\arabic*)]
		\item If $\abs{\lambda} > \norm{a}$, then $\lambda \notin \sigma(a)$.
		\item $\sigma(a)$ is a compact subset of $\C$.
		\item the map $\lambda \mapsto (a-\lambda)^{-1}$ from $\rho(a)$ to $\alg{A}$ is analytic and $\sigma(a)$ is nonempty.
		\item $r(a)=\lim_{n \sto \infty} \norm{a^{n}}^{\frac{1}{n}}$.
	\end{enumerate}
\end{thm}
\begin{proof}
	$1)$ holds by above theorem. \\
	For $2)$, since $\lambda \sto a-\lambda$ is continuous from $\C$ to $\alg{A}$ and $G$ is open, $\rho(a)$ is open i.e. $\sigma(a)=\C \backslash \rho(a)$ is closed. Then by $1)$, $\sigma(a)$ is compact.\\
	For $3)$, by the identity $a^{-1} - b^{-1} = a^{-1}(b-a)b^{-1}$ and the continuity of $a \sto a^{-1}$, we can compute the derivative of $F(\lambda) = (a-\lambda)^{-1}$,
	\begin{equation*}
		F^{'}(\lambda) = (a-\lambda)^{-2}
	\end{equation*}
	And clearly, $F^{'}(\lambda)$ is continuous. Thus it is analytic and it vanishes at $\infty$. By the Liouville's Theorem, if $\rho(a) = \C$, $F$ is constant. Therefore, $\rho(a) \neq \C$ i.e. $\sigma(a) \neq \varnothing$.\\
	For $4)$, let $U=\{\lambda \in \C \colon \lambda = 0 \text{ or } \lambda^{-1} \in \rho(a)\}$ and 
	\begin{equation*}
		f(\lambda) = 
		\begin{cases}
			(\lambda^{-1}-a)^{-1} & x \neq 0,\\
			0,& x = 0
		\end{cases}
	\end{equation*}
	Then $f$ is analytic on $U$, i.e $f(\lambda)=\lambda\sum_{n=0}^{\infty} \lambda^{n} a^{n}$ is well-defined. Therefore, the convergent radius $R = r(a)^{-1}$
	\begin{equation*}
		R^{-1} = \limsup_{n \sto \infty} \norm{a^{n}}^{\frac{1}{n}} = r(a)
	\end{equation*}
	Conversely, by the identity $(a^{n}-\lambda^{n}) = (a-\lambda)(a^{n-1}+\lambda a^{n-2}+\lambda^{2} a^{n-3} + \cdots + \lambda^{n-1})$. Then, if $(a^{n}-\lambda^{n})$ is invertible, then $(a-\lambda)$ is invertible, i.e. $\sigma(a) \subset \sigma(a^{n})$. Thus $\abs{\lambda}^{n} \leqslant \norm{a^{n}}$ for any $\lambda \in \sigma{a}$. $r(a)=\liminf_{n \sto \infty}\norm{a^{n}}^{\frac{1}{n}}$. Therefore, $r(a)=\lim_{n \sto \infty} \norm{a^{n}}^{\frac{1}{n}}$.
\end{proof}

If $\alg{B} \subset \alg{A}$ is a subalgebra with the same identity of a Banach algebra $\alg{A}$, then we know for any element $b \in \alg{B}$, $\sigma_{\alg{A}}(b) \subset \sigma_{\alg{B}}(b)$. Then we can have more results other than it. Since the spectrum is a subset of $\C$, we need some topological properties results of $\C$.

\begin{lem}
	If $K$ is any compact subset of $\C$, then $\C \backslash K$ has a countable components, only one of which is unbounded. And the boundary of each component is in $K$.
\end{lem}
\begin{proof}
	Let $\tilde{K} = \C \backslash K$, then $\tilde{K}$ is open.
	\item Firstly, the connected component of open set in $\C$ is open. \\
		  Let $U$ be an connected component in $\tilde{K}$ and $x \in U$. For any point $x \in U$, Since any open neighbourhood of $x$ is connected, and $K$ is open, there is a open neighbourhood $V$ of $x$ s.t. $V \subset U$.
	\item Secondly, $\C$ has just at most countable many open sets, which are pairwise disjoint.\\
		  This result is because any open set in $\C$ contains a rational point.
	\item For any two disjoint open sets $A$ and $B$ in $\C$, $\partial A \bigcap B = \varnothing$. Thus the boundary of some compoment of $\tilde{K}$ can not be contained in any component of $\tilde{K}$, i.e. it is contained $K$.
	\item Finally, since $K$ is bounded, there is a closed ball $B$ containing $K$. But the complement of $B$ is connected, thus there is only one component of $\tilde{K}$ containing $B$. Thus the other components of $\tilde{K}$ are bounded.
\end{proof}
\begin{rem}
	The bounded component of $\C \backslash K$ is called a hole of $K$.
\end{rem}

\begin{defn}
	If $f \colon A \sto \C$, where $A$ is a set, then the norm of $f$ on $A$ is defined as
	\begin{equation*}
		\norm{f}_A = \sup{\{\abs{f(x)} \colon x \in A\}}
	\end{equation*}
	For a compact set $K \in \C$, the polynomially convex hull of $K$ is defined as
	\begin{equation*}
		\hat{K} = \{~z \in \C \colon \abs{p(z)} \leqslant \norm{p}_K \text{ for any polynomial } p ~\}
	\end{equation*}
	If $K = \hat{K}$, $K$ is called polynomially convex.
\end{defn}

\begin{prop}
	Let $K$ be a compact subset of $\C$. Then $\C \backslash \hat{K}$ is the unbounded component of $\C \backslash K$. Therefore, $K$ is polynomially convex if and only if $\C \backslash K$ is connected.
\end{prop}
\begin{proof}
	Let $L$ be the set containing $K$ and all bounded component of $\C \backslash K$. Then by the Maximal Principle, $L \subset \hat{K}$. Conversely, if $\alpha \notin L$, then $(z-\alpha)^{-1}$ is analytic in a neighbourhood of $L$. Therefore, there is a sequence of polynomials $\{p_n\}$ s.t. $p_n \sto (z-\alpha)^{-1}$. Let $q_n=(z-\alpha)p_n$. Then $q_n \sto 1$, i.e. $\norm{q_n-1} < \frac{1}{2}$ for some $n$. But $\abs{q_n(\alpha)-1}=1$, this implies $\alpha \notin \hat{K}$, i.e. $\hat{K} \subset L$.
\end{proof}

By above results, now we can provide the relationships betweem $\sigma_{\alg{A}}(b)$ and $\sigma_{\alg{B}}(b)$.

\begin{thm}
	If $\alg{A}$ and $\alg{B}$ are Banach algebras with same identity s.t. $\alg{B} \subset \alg{A}$ and $b \in \alg{B}$, then
	\begin{enumerate}[label=\arabic*)]
		\item $\sigma_{\alg{A}}(b) \subset \sigma_{\alg{B}}(b)$ and $\partial\sigma_{\alg{B}}(b) \subset \partial\sigma_{\alg{A}}(b)$
		\item $\hat{\sigma_{\alg{A}}(b)} = \hat{\sigma_{\alg{B}}(b)}$
		\item if $G$ is a hole of $\sigma_{\alg{A}}(b)$, then $G \subset \sigma_{\alg{B}}(b)$ or $G \bigcap \sigma_{\alg{B}}(b) = \varnothing$
	\end{enumerate}
\end{thm}
\begin{proof}
	\item For $1)$, let $\lambda \in \partial\sigma_{\alg{B}}(b)$. Since $\inte{\sigma_{\alg{A}}(b)} \subset \inte{\sigma_{\alg{B}}(b)}$, it is sufficient to show $\lambda \in \sigma_{\alg{A}}(b)$. Suppose $\lambda \notin \sigma_{\alg{A}}(b)$, i.e. $(b-\lambda)$ is invertible in $\alg{A}$. But since $\lambda \in \partial\sigma_{\alg{B}}(b)$, there are $\lambda_n \in \C \backslash \alg{B}$ with $\lambda_n \sto \lambda$. Thus $(b-\lambda_n)^{-1} \in \alg{B}$. But $(b-\lambda_n)^{-1} \sto (b-\lambda)^{-1} \in \sigma_{\alg{B}}(b)$, contradicting to $\lambda \in \sigma_{\alg{A}}(b)$.
	\item $2)$ holds because of the result of $1)$ and the Maxiamal Principle.
	\item For $3)$, let $G_1 = G \bigcap \sigma_{\alg{B}}(b)$ and $G_2 = G \backslash \sigma_{\alg{B}}(b)$. Since $\partial\sigma_{\alg{B}}(b) \subset \sigma_{\alg{A}}(b)$ and $G \bigcap \sigma_{\alg{A}}(b) = \varnothing$, $G_1 = G \bigcap \inte{\sigma_{\alg{B}}(b)}$ is open. By the facts that $G_2$ is clearly open and $G = G_1 \bigcup G_2$ and $G_1 \bigcap G_2 = \varnothing$, either $G_1$ or $G_2$ is empty.
\end{proof}

Then we can have some useful corollaries.

\begin{cor} \label{cor5}
	Let $\alg{A}$ and $\alg{B}$ be Banach algebras with same identity s.t. $\alg{B} \subset \alg{A}$ and $b \in \alg{B}$.
	\begin{enumerate}[label=\arabic*)]
		\item If $\sigma_{\alg{A}}(b)$ has no holes, then $\sigma_{\alg{A}}(b)=\sigma_{\alg{B}}(b)$.
		\item If $\sigma_{\alg{B}}(b) \subset \R$, then $\sigma_{\alg{A}}(b)=\sigma_{\alg{B}}(b)$.
		\item $\sigma_{\alg{A}}(b)=\sigma_{\alg{B}}(b)$ if and only if $\rho_{\alg{A}}(b)$ is connected.
	\end{enumerate}
\end{cor}
\begin{proof}
	$1)$ is clearly true since ubbouded component does not intersect $\sigma_{\alg{B}}(b)$. $2)$ is because $\C \backslash \sigma_{\alg{A}}(b)$ has no holes. $3)$ is similar as $2)$.
\end{proof}

\subsection{Riesz Functional Calculus}

For any polynomial $p$ with complex coefficients,  
\begin{equation*}
	p(z) = \sum_{k=0}^{n} \alpha_k z^{k}
\end{equation*}
we can define $p(a)$ for some $a \in \alg{A}$, where $\alg{A}$ is a Banach algebra
\begin{equation*}
	p(a) = \sum_{k=0}^{n} \alpha_k a^{k}
\end{equation*}
Clearly, $p(a)$ is well-defined. But we can do more. If $f$ is an analytic funcion on $A \subset \C$, then $f$ can be approximated by a sequence of polynomials
\begin{equation*}
	f(z) = \sum_{n=0}^{\infty} \alpha_n z^{n}
\end{equation*}
Similarly, we can define $f(a)$ for $a \in \alg{A}$ as
\begin{equation*}
	f(a) = \sum_{n=0}^{\infty} \alpha_n a^{n}
\end{equation*}
If the radius of convergence of this sequence is $R$, then it can be well-defined for $\norm{a} \leqslant R$. By the fact that $r(a) \leqslant \norm{a}$, for the analytic function $f$, if $\sigma(a) \subset A$, $f(a)$ can be well-defined.\\
Let $\hol{(a)}$ denote all functions that are analytic in a neighbourhood of $\sigma(a)$. Then there is a map from $\hol{(a)}$ to $\alg{A}$ defined as $f \mapsto f(a)$. Now, we can find more properties of this map. Firstly, we can give another formula of $f(a)$.\\
If $f \in \hol{(a)}$, then for any closed curve $\gamma$ which encloses $\hol{(a)}$ and any point $z_0 \in \hol{(a)}$, 
\begin{equation*}
	f(z_0) = \frac{1}{2 \pi i}\int_{\gamma} f(z) (z-z_0)^{-1} dz
\end{equation*}
Therefore, replacing $z_0$ by $a \in \alg{A}$, then we have 
\begin{equation*}
	f(a) = \frac{1}{2 \pi i}\int_{\gamma} f(z) (z-a)^{-1} dz
\end{equation*}
Clearly, by the Cauchy's Integral Formula, this definition is well-defined and is coincided with above definition. But this definition can provide us a conivient method to research the map $f \mapsto f(a)$.

\begin{thm}[Riesz Functional Calculus]
	Let $\alg{A}$ be a Banach algebra and $a \in \alg{A}$ and the map
	\begin{center}
		\begin{tabular}{l c c l}
			$\rho \colon$ & $\hol{(a)}$ & $\longrightarrow$ & $\alg{A}$ \\
			~ & $f$ & $\longmapsto$ & $f(a)=\frac{1}{2 \pi i}\int_{\gamma} f(z) (z-a)^{-1} dz$
		\end{tabular}
	\end{center}
	has the following properties.
	\begin{enumerate}[label=\arabic*)]
		\item $\rho$ is an algebra homomorphism.
		\item $\rho(1) = 1$ and $\rho(z)=a$.
		\item If $\{f_n\} \subset \hol{(a)}$ and $f \in \alg{A}$ with $f_n \sto f$ uniformly on a compact set of $\hol{(a)}$, then $\rho(f_n) \sto \rho(f)$ in norm.
	\end{enumerate}
	Moreover, if any map $\tau \colon \hol{(a)} \sto \alg{A}$ satifies above conditions, then $\tau = \rho$. 
\end{thm}
\begin{proof}
	For $1)$, $\rho$ is clearly linear. And
		\begin{eqnarray*}
			f(a)g(a)&=& -\frac{1}{4 \pi^{2}}\int_{\gamma_1} f(z) (z-a)^{-1} dz\int_{\gamma_2} g(\zeta) (\zeta-a)^{-1} d\zeta \\
			&=& -\frac{1}{4 \pi^{2}}\int_{\gamma_1}\int_{\gamma_2} f(z)g(\zeta) \frac{(z-a)^{-1}-(\zeta-a)^{-1}}{\zeta-z} d\zeta dz \\
			&=& -\frac{1}{4 \pi^{2}}\int_{\gamma_1}f(z) \int_{\gamma_2} \frac{g(\zeta)}{\zeta-z} d\zeta (z-a)^{-1} dz \\
		    && \negmedspace{} + \frac{1}{4 \pi^{2}}\int_{\gamma_2}g(\zeta) \int_{\gamma_1} \frac{f(z)}{\zeta-z} dz (\zeta-a)^{-1} d\zeta
		\end{eqnarray*}
		We can choose $\gamma_2$ to enclose $\gamma_1$, thus 
		\begin{equation*}
			\int_{\gamma_1} \frac{f(z)}{\zeta-z} dz =0,~ \int_{\gamma_2} \frac{g(\zeta)}{\zeta-z} d\zeta] = 2\pi i g(z)
		\end{equation*}
		Therefore, 
		\begin{equation*}
			f(a)g(a) = {2 \pi i}\int_{\gamma_1} f(z)g(z)(z-a)^{-1} dz = (fg)(a)
		\end{equation*}
	\item For $2)$, let $f(z) = z^k$ and $\gamma = R e^{2 \pi i t}$, where $R>\norm{a}$ and $t \in [0,1]$, then
	\begin{eqnarray*}
		f(a) &=& \frac{1}{2 \pi i}\int_{\gamma} z^k (z-a)^{-1} dz \\
		&=& \frac{1}{2 \pi i}\int_{\gamma} z^{k-1} (1-\frac{a}{z})^{-1} dz \\
		&=& \frac{1}{2 \pi i}\int_{\gamma} z^{k-1} \sum_{n=0}^{\infty} \frac{a^{n}}{z^{n}}dz \\
		&=& \sum_{n=0}^{\infty} (\frac{1}{2 \pi i}\int_{\gamma} \frac{1}{z^{n-k+1}}) a^{n} \\
		&=& a^{k}
	\end{eqnarray*}
	\item For $3)$, 
	\begin{eqnarray*}
		\lefteqn { \norm{\int_{\gamma} f_n(z) (z-a)^{-1} dz - \int_{\gamma} f(z) (z-a)^{-1} dz} }\\
		&=& \norm{\int_{0}^{1}(f_n(\gamma(t))-f(\gamma(t)))(\gamma(t)-a)^{-1}d\gamma(t)}\\
		&\leqslant& \int_{0}^{1} \abs{f_n(\gamma(t))-f(\gamma(t))}\norm{(\gamma(t)-a)^{-1}}d\abs{\gamma}(t) \\
		&\leqslant& M \norm{\gamma} \sup{\{\abs{f_n(z)-f(z)} \colon z \in \gamma(t)\}}
	\end{eqnarray*}
	where $M$ is the bound of $\norm{(\gamma(t)-a)^{-1}}$ since $t \mapsto \norm{(\gamma(t)-a)^{-1}}$ is continuous on $\gamma(t)$. Therefore, by the fact that $f_n \sto f$ uniformly,
	\begin{equation*}
		\norm{f_n(a)-f(a)} \sto 0
	\end{equation*}
	\item Finally, the uniquness is because any $f \in \hol{(a)}$ can be approximated uniformly by a sequence of polynomials. Thus, $1)$ and $2)$ means $\tau(p) = \rho(p)$ for any polynomial $p$, and $3)$ provides the fact that $\tau(f) = \rho(f)$ for any $f \in \hol{(a)}$.
\end{proof}
\begin{rem}
	we have mentioned that the integral definition is coincided with the convergent difinition. In fact, by $2)$, this statement can be proved rigorously.
\end{rem}

\begin{thm}[Spectral Mapping Theorem]
	If $a \in \alg{A}$ and $f \in \hol{(a)}$, then
	\begin{equation*}
		\sigma(f(a)) = f(\sigma(a))
	\end{equation*}
\end{thm}
\begin{proof}
	Firstly, there is a $g \in \hol{(a)}$ s.t. for $\alpha \in \sigma(a)$, $f(z)-f(\alpha) = (z-\alpha)g(z)$, that means $f(\sigma(a)) \subset \sigma(f(a))$.
	\item Conversely, if $\alpha \notin f(\sigma(a))$, $g(z)=(f(z)-\alpha)^{-1} \in \hol{(a)}$. Thus, $g(a)(f(a)-\alpha) = 1$. Therefore, $\alpha \notin \sigma(f(a))$.
\end{proof}

\begin{prop} \label{prop10}
	Let $\alg{A}$ be a Banach algebra and $a \in \alg{A}$. $\sigma(a) = F_1 \bigcup F_2$, where $F_1$ and $F_2$ are disjoint nonempty closed sets. Then there is a nontrial idempotent $e$, i.e. $e^{2}=e$, s.t.
	\begin{enumerate}[label=\arabic*)]
		\item if $ab=ba$, then $eb=be$.
		\item if $a_1=ae$ and $a_2=a(1-e)$, then $a_1a_2=a_2a_1=0$.
		\item $\sigma(a_1)=F_1 \bigcup \{0\}$ and $\sigma(a_2)=F_2 \bigcup \{0\}$.
	\end{enumerate}
\end{prop}
\begin{proof}
	Since $F_1$ and $F_2$ are disjoint closed set, there are two disjoint open sets $G_1$ and $G_2$ separating $F_1$ and $F_2$. Let $f$ be the characteristic function of $G_1$ and $e=f(a)$. Thus $e^{2}=e$ by $f^{2}=f$.
	\item For $1)$, there is a more genera result, $f(a)b=bf(a)$ for any $f \in \hol{(a)}$. It is because by extending the fact $p(a)b=bp(a)$ for any polynomial $p$.
	\item $2)$ is clearly true.
	\item Let $f_1(z)=zf(z)$ and $f_2(z)=z(1-f(z))$. Then $a_j = f_j(a)$ for $j=1,2$. Then by the Spectral Mapping Theorem $\sigma(a_j) = f_j(\sigma(a_j)) = F_j \bigcup \{0\}$.
\end{proof}

\begin{cor}
	Let $\Hs$ be a Hilbert space and $T \in \oper$. If $\sigma(T) = F_1 \cup F_2$, where $F_1$ and $F_2$ are disjoint nonempty closed sets, then there are closed subspaces $\Hs_1$ and $\Hs_2$ with $\Hs = \Hs_1 \oplus \Hs_2$, s.t.
	\begin{enumerate}[label=\arabic*)]
		\item $B\Hs_j \subset \Hs_j$ for $j=1,2$, whenever $BT=TB$.
		\item If $T_j = T|_{\Hs_j} \colon \Hs_j \sto \Hs_j$, then $\sigma(T_j) = F_j$, for $j=1,2$.
	\end{enumerate}
\end{cor}
\begin{proof}
	From above proposition, $e$ is actually the propjection from $\Hs$ to $\Hs_1$. Therefore by the result in above propostion, $1)$ clearly holds.\\
	For $2)$, since $e$ is in fact the indentity from $\Hs_1$ to $\Hs_1$, similarly, by the Spectral Mapping Theorem, $\sigma(T_j) = F_j$.
\end{proof}

\subsection{Abelian Banach Algebras} \label{sec2}
 
 \begin{thm}[Gelfand-Mazur Theorem]
	If $\alg{A}$ is a Banach algebra and a division ring, then $\alg{A} = \C$.
\end{thm}
\begin{proof}
	It is because that for any $a \in \alg{A}$, $\sigma(a) \neq \varnothing$.
\end{proof}

Next, we reach the structure of an abelian Banach algebra. The structure of abelian Banach algebras can be explicit by constructing a map from an abelian Banach algebra to a continuous function space on a compact space. Firstly, we can find this compact space. Let 
\begin{eqnarray*}
	\Sigma(\alg{A}) &=& \{ \text{all algebra homomorphism } h \colon \alg{A} \sto \C\}\\
	\fml{M} &=& \{\text{all maximal ideals of } \alg{A}\}
\end{eqnarray*}
for an abelian Banach algebra $\alg{A}$. Then we can find the relationship between $\Sigma(\alg{A})$ and $\fml{M}$.

\begin{thm}
	Let $\alg{A}$ be an abelian Banach algebra. Define a map
	\begin{center}
		\begin{tabular}{l c c l}
			$\gamma \colon$ & $\Sigma(\alg{A})$ & $\longrightarrow$ & $\fml{M}$ \\
			~ & $h$ & $\longmapsto$ & $\ker{h}$
		\end{tabular}
	\end{center}
	Then $\gamma$ is a bijection.
\end{thm}
\begin{proof}
	Since $\alg{A} / \ker{h} \cong \C$, $\ker{h} \in \fml{M}$, i.e. $\gamma$ is well-defined.
	\item Check: $\alg{A} / M \cong \C$ for any $M \in \fml{M}$\\
		Let $\pi \colon \alg{A} \sto \alg{A} / M$. If $\pi(a)$ is not invertible, then $\pi(a\alg{A})$ is a proper ideal in $\alg{A} / M$. Thus $I = \pi^{-1}(\pi(a\alg{A}))$ is a proper ideal in $\alg{A}$ and $M \subset I$. Then by the maximality of $M$, $I=M$, i.e. $\pi(a)=0$. In fact, for any commutative ring, this result is true. Therefore, by Gelfand-Mazur Theorem, $\alg{A} / M \cong \C$.
	\item Check: $\gamma$ is surjective.
		Let $M \in \fml{M}$. Define $\tilde{h} \colon \alg{A} / M \sto \C$ as the algebraic isomorphism. Then $h=\pi \circ \tilde{h} \in \Sigma(\alg{A})$ with $\ker{h} = M$.
	\item Check: $\gamma$ is injective.
		If $\ker{h} = \ker{h^{'}}$ for $h,h^{'} \in \Sigma(\alg{A})$, then by the \textbf{Propostion} \ref{prop2} in the subsection \textbf{1.4.2}, $h = \alpha h^{'}$. And since $h(1) = h^{'}(1)=1$, $h=h^{'}$.
\end{proof}

Then we have some properties of $h \in \Sigma(\alg{A})$.

\begin{prop} \label{prop5}
	Let $\alg{A}$ be an abelian Banach algebra and $h \in \Sigma(\alg{A})$.
	\begin{enumerate}[label=\arabic*)]
		\item $h$ is continuous.
		\item $\norm{h}=1$ for $h \neq 0$.
	\end{enumerate}
\end{prop}
\begin{proof}
	$1)$ holds since $\ker{h}$ is maximal, thus it is closed.
	\item Let $\lambda = h(a)$. Suppose $\abs{\lambda} > \norm{a}$. Then $1-\frac{1}{\lambda}$ is invertible. Set $b = (1-\frac{1}{\lambda})^{-1}$, then 
	\begin{equation*}
		1=h(b(1-\frac{1}{\lambda})) = h(b) - \frac{h(b)h(a)}{\lambda} =0
	\end{equation*}
	Therefore, $\abs{h(a)} \leqslant \norm{a}$ i.e $\norm{n} \leqslant 1$. Since $h(1) = 1$, $\norm{h}=1$.
\end{proof}

\begin{defn}
	Let $\alg{A}$ be an abelian Banach algebra. Then $\Sigma(\alg{A}) \subset \alg{A}^{*}$ endowed with the induced $wk^{*}$-topology, is called the maximal ideal space of $\alg{A}$.
\end{defn}

\begin{prop} \label{prop6}
	If $\alg{A}$ is an abelian Banach algebra, then $\Sigma(\alg{A})$ is a compact Hausdorff space. Moreover, if $a \in \alg{A}$, then
	\begin{equation*}
		\sigma(a) = \Sigma(a) = \{~ h(a) \colon h \in \Sigma(\alg{A}) ~\}
	\end{equation*}
\end{prop}
\begin{proof}
	Since $\Sigma(\alg{A}) \subset \alg{A}^{*}$, we just need to show $\Sigma(\alg{A})$ is $wk^{*}$-closed. Let $\{h_i\}$ be a net in $\Sigma(\alg{A})$ s.t. $h_i \sto h$ $wk^{*}$ for some $h$ in the unit closed ball of $\alg{A}^{*}$. Then for $a,b \in \alg{A}$,
	\begin{equation*}
		h(ab) = \lim_{i} h_i(ab) = \lim_{i} h_i(a)h_i(b) = h(a)h(b)
	\end{equation*}
	and $h(1)=\lim_{i}h_i(1)=1$, thus $h \in \Sigma(\alg{A})$. $\Sigma(\alg{A})$ is compact.
	\item If $h\in \Sigma(\alg{A})$ and $h-h(a) \in \ker{h} \in \fml{M}$, then $h-h(a)$ is not invertible, i.e. $\Sigma(a) \subset \sigma(a)$. Conversely, if $a-\lambda$ is not invertible, $(a-\lambda)\alg{A}$ is a proper ideal, which can be contained in a maximal ideal. Then $(a-\lambda) \in \ker{h}$ with some $h \in \Sigma(\alg{A})$, $\lambda = h(a) \in \Sigma(a)$.
\end{proof}

Therefore, $\Sigma(\alg{A})$ is the compact space we need. Then we define the map from $\alg{A}$ to $C(\Sigma(\alg{A}))$.

\begin{thm} \label{thm5}
	If $\alg{A}$ is an abelian Banach algebra, the Gelfand transform is defined as
	\begin{center}
		\begin{tabular}{l c c l}
			$\Gamma \colon$ & $\alg{A}$ & $\longrightarrow$ & $C(\Sigma(\alg{A}))$ \\
			~ & $a$ & $\longmapsto$ & $\hat{a} = \Gamma(a)$
		\end{tabular}
	\end{center}
	where $\hat{a}(h)=h(a)$.
	\begin{enumerate}[label=\arabic*)]
		\item $\Gamma$ is a continuous homomorphsim.
		\item $\norm{\Gamma}=1$.
		\item $\ker{\Gamma}=\bigcap\{M \colon M \in \fml{M}\}$.
		\item $\norm{\hat{a}}_{\infty}=r(a)$.
	\end{enumerate}
\end{thm}
\begin{proof}
	Firstly, for $1)$ if $h_i \sto h$ in $wk^{*}$, $\hat{a}(h_i) = h_i(a) \sto h(a) = \hat{a}(h)$. Thus $\Gamma$ is well-defined. And
	\begin{equation*}
		\Gamma(ab)(h) = h(ab) = h(a)h(b) = \Gamma(a)(h)\Gamma(b)(h)
	\end{equation*}
	Thus $\Gamma$ is a homormophsim.
	\item For $2)$, since $\abs{\hat{a}(h)} = \abs{h(a)} \leqslant \norm{a}$, $\norm{\hat{a}}_{\infty} \leqslant \norm{a}$. Then $\norm{\Gamma} \leqslant 1$. By $\Gamma(1) = 1$, $\norm{\Gamma} = 1$.
	\item For $3)$, $a \in \ker{\Gamma}$ if and only if $h(a) = 0$ for any $h \in \Sigma(\alg{A})$, i.e. $a \in \bigcap\{M \colon M \in \fml{M}\}$. 
	\item For $4)$, it holds since $\sigma(a) = \{~ h(a) \colon h \in \Sigma(\alg{A}) ~\}$.
\end{proof}

If $a \in \alg{A}$ s.t. $\clo{\{p(a) \colon p \text{ is any polynomial}\}} = \alg{A}$, then $a$ is called a generator of $\alg{A}$. Clearly, this $\alg{A}$ is commutative. Then we can find an extral property of this special algebra.

\begin{prop} \label{prop7}
	If $\alg{A}$ is an abelian Banach algebra with a generator $a$, then there is a homeomorphism $\tau \colon \Sigma(\alg{A}) \sto \sigma(a)$ s.t. $\Gamma(p(a)) = p \circ \tau$.
\end{prop}
\begin{proof}
	In fact, $\tau$ can be defined as $\tau(h)=h(a)$. By above mention, $\tau$ is continuous and surjective. If $\tau{h_1}=\tau{h_2}$, then $ h_1(a) = h_2(a)$. By the fact that $h_1,h_2 \in \alg{A}$ and $a$ is a generator of $\alg{A}$, $h_1=h_2$. Thus $\tau$ is a bijection. And because $\Sigma(\alg{A})$ is compact, $f$ is a closed map, i.e. $f$ is a homeomorphism.
	\begin{equation*}
		\Gamma(p(a))(h) = p(\Gamma(a))(h) = p(\Gamma(a)(h)) = p(\tau(h))
	\end{equation*}
\end{proof}
\begin{rem}
	If $\alg{A}$ is generate by $a$, then $\Gamma \colon \alg{A} \sto C(\sigma(a))$ can be defined as $\Gamma(p(a)) = p$. 
\end{rem}

In fact, above proposition can extends to $n$ generators. If $\{a_i\}_{i=1}^{n}$ are generators of $\alg{A}$, i.e $\clo{\{p(a_1,\cdots,a_n) \colon p \text{ is any } n \text{ variables polynomial}\}} = \alg{A}$, then we have similar results as above proposition.


\section{\texorpdfstring{$C^{*}$-Algebras}{C*-Algebras}}

Now, we have known $\oper$ is a Banach algebra. But there is another algebraic operation on $\oper$, which let $\oper$ be more interesting than the general Banach algebra. This operation is a map $T \sto T^{*}$ on $\oper$, called an involution, and moreover, it satisfies the condition $\norm{T} = \norm{T^{*}} = \norm{\st{T}T}^{\frac{1}{2}}$. This identity provides a strong relation between the topological structure and the algebraic structure on $\oper$. In fact, the topology is completely determined by the algebraic structure on $\oper$. In order to research this structure, we firstly define an general algebra satisfying above condition, called a $C^{*}$-algebra. By digging its topological structures and algebraic structures, we can embed it into $\oper$ for some Hilbert space $\fml{H}$. Therefore, any $C^{*}$-algebra can be regarded as a subalgebra of $\oper$.

\subsection{Elementary Properties}
\begin{defn}
	If $\alg{A}$ is a Banach algebra, an involution is a map $a \sto a^{*}$ from $\alg{A}$ to $\alg{A}$ satisfying for any $a,b \in \alg{A}$ and any $\alpha \in \C$, 
	\begin{enumerate}[label=\arabic*)]
		\item $(a^{*})^{*}=a$,
		\item $(ab)^{*}=b^{*}a^{*}$,
		\item $(\alpha a+b)^{*} = \clo{\alpha} a^{*}+b^{*}$.
	\end{enumerate}
\end{defn}

\begin{defn}
	A $C^{*}$-algebra is a Banach algebra $\alg{A}$ with an involution s.t. for every $a \in \alg{A}$,
	\begin{equation*}
		\norm{\st{a}a} = \norm{a}^{2}
	\end{equation*}
\end{defn}

Then we can get some easy properties for the norm and the involution.

\begin{prop} \label{prop14}
	Let $\A$ be a \Cs and $a \in \A$.
	\begin{enumerate}[label=\arabic*)]
		\item $\norm{\st{a}} = \norm{a}$.
		\item $\norm{a\st{a}}=\norm{a}^{2}$.
		\item $\norm{a}=\sup{\{\norm{ax} \colon \norm{x} \leqslant 1\}} = \sup{\{\norm{xa} \colon \norm{x} \leqslant 1\}}$.
	\end{enumerate}
\end{prop}
\begin{proof}
	For $1)$, $\norm{a}^{2} = \norm{\st{a}a} \leqslant \norm{\st{a}}\norm{a}$, thus $\norm{a} \leqslant \norm{\st{a}}$. Taking the involution, $\norm{\st{a}} \leqslant \norm{a}$.
	\item $2)$, $\norm{a\st{a}} = \norm{\st{(\st{a})}} = \norm{\st{a}}^{2} = \norm{a}^{2}$.
	\item For $3)$, let $\alpha$ be the supremum, then $\alpha \leqslant \norm{a}$. $a = 0$ is clearly true. For $a \neq 0$, let $x = \st{a} / \norm{a}$. Thus $\alpha \geqslant \norm{a}$.
\end{proof}

If a \Cs $\A$ is without the identity, we can use same method of the Banach algebra to extend it to a unit \Cs $\tilde{\A}$. The only thing we need to prove is the identity. And this can be obtained the result in above proposition. Therefore, we always assume a \Cs is with the identity.

\begin{defn}
	Let $\A$ be a \Cs and $a \in \A$.
	\begin{enumerate}[label=\arabic*)]
		\item $a$ is self-adjoint if $a = \st{a}$.
		\item $a$ is normal if $a\st{a} = \st{a}a$.
		\item $a$ is unitary if $a\st{a}=\st{a}a=1$.
		\item $a$ is a projection if $a$ is self-adjoint and $a=a^{2}$.
	\end{enumerate}
\end{defn}

Then we can see the algebraic strucuture on a \Cs completely determine its norm topology.

\begin{thm}
	Let $\A$ be a \Cs and $a \in \A$. If $a$ is self-adjoint, then
	\begin{equation*}
		r(a) = \norm{a}
	\end{equation*}
\end{thm}
\begin{proof}
	Since $a$ is sef-sdjoint,
	\begin{equation*}
		\norm{a}^{2} = \norm{\st{a}a} = \norm{a^{2}}
	\end{equation*}
	Thus by induction, we have $\norm{a}^{2n} = \norm{a^{2n}}$. Then
	\begin{equation*}
		r(a) = \lim_{n \sto \infty} \norm{a^{n}}^{\frac{1}{n}} = \lim_{n \sto \infty} \norm{a^{2n}}^{\frac{1}{2n}} = \norm{a} \qedhere
	\end{equation*}
\end{proof}
\begin{rem}
	For any $b \in \A$, we know $\st{b}b$ is self-adjoint,
	\begin{equation*}
		r(\st{b}b) = \norm{\st{b}b} = \norm{b}^{2}
	\end{equation*}
	Thus, the norm in a \Cs is completely determined by the spectral radius, which is totally an algebraic trait.
\end{rem}

Now, we can see how the algebraic property influences the topological structure.

\begin{prop} \label{prop8}
	Let $\rho \colon \A \sto \B$ be a $*$-homomorphsim between two $C^{*}$-algebras.
	\begin{enumerate}[label=\arabic*)]
		\item $\rho$ is continuous, and moreover $\norm{\rho(a)} \leqslant \norm{a}$.
		\item If $\rho$ is a $*$-isomorphism, then $\rho$ is an isometry.
	\end{enumerate}
\end{prop}
\begin{proof}
	$2)$ is the direct corollary from $1)$. For $1)$, clearly $\sigma(\rho(a)) \subset \sigma(a)$, thus
	\begin{equation*}
		\norm{\rho(a)}^{2} = r(\rho(\st{a}a)) \leqslant r(\st{a}a) = \norm{a}^{2} \qedhere
	\end{equation*}
\end{proof}

Let $\Rea{\A}$ denote the set of all self-adjoint elements in a \Cs $\A$. Then, for any $a \in \A$, there are $x,y \in \Rea{\A}$, s.t.
\begin{equation*}
	a = x + iy, \text{ where } x = \frac{a + \st{a}}{2}, y = \frac{a-\st{a}}{2i}
\end{equation*}
Therefore, any element in a \Cs $\A$ can be combined by two self-adjoint elements. And the self-adjoint element play a important role in the algebraic structure of a \Cs. 

\begin{prop}
	If $h \colon \A \sto \C$ is an algebraic homomorphis on a \Cs $\A$. 
	\begin{enumerate}[label=\arabic*)]
		\item If $a \in \Rea{\A}$, $h(a) \in \R$.
		\item For any $a \in \A$, $h(\st{a})=\clo{h(a)}$.
		\item $h(\st{a}a) \geqslant 0$ $\forall~ a \in \A$.
		\item If $u \in \A$ is a unitary, then $\abs{u} = 1$.
	\end{enumerate}
\end{prop}
\begin{proof}
	For $1)$, let $h(a) = \alpha +i \beta$ for $\alpha,\beta \in \R$ and $\Cg{a+it}$ be the \Cs generated by $a+it$ and $1$, which is abelian. Therefore, $\norm{h}_{\Cg{a+it}} = 1$ by \textbf{Proposition} \ref{prop5} in the subsection \textbf{2.1.4}. Then we have
	\begin{eqnarray*}
		\abs{h(a+it)} &\leqslant& \norm{a+it}^{2} \\
		&=& \norm{\st{(a+it)}(a+it)} \\
		&=& \norm{a^{2}+t^{2}} \\
		&\leqslant& \norm{a}^{2}+t^2
	\end{eqnarray*}
	i.e.
	\begin{eqnarray*}
		\norm{a}^{2}+t^2 &\geqslant& \abs{\alpha +i(t+\beta)}^2 \\
		&=& \alpha^2 + (\beta+t)^2 \\
		&=& \alpha^2 + \beta^2 +2 \beta t + t^2
	\end{eqnarray*}
	Therefore, for any $t \in \R$, $\norm{a}^2 \geqslant \alpha^2 + \beta^2 +2 \beta t$. Thus, $\beta = 0$. \\
	$2)$ and $3)$ and $4)$ is the direct results from $1)$.
\end{proof}

\begin{cor}
	If $a \in \Rea{A}$, then $\sigma(a) \subset \R$.
\end{cor}
\begin{proof}
	Let $\Cg{a}$ be the \Cs generated by $1$ and $a$. Thus $\Cg{a}$ is abelian. Then by \textbf{Proposition} \ref{prop6} in the subsection \textbf{2.1.4}, 
	\begin{equation*}
		\sigma_{\Cg{a}}(a) = \{~ h(a) \colon h \in \Sigma(\alg{A}) ~\} \subset \R
	\end{equation*} 
	And by the $Corollary$ \ref{cor5} in the subsection \textbf{2.1.2},
	\begin{equation*}
		\sigma(a) = \sigma_{\Cg{a}}(a) \subset \R
	\end{equation*}
\end{proof}

And the spectrum of a element in a \Cs has better property.

\begin{thm}
	If $\B$ is a $C^{*}$-subalgebra of a \Cs $\A$ and $b \in \B$, then
	\begin{equation*}
		\sigma_{\B}(b) = \sigma_{\A}(b)
	\end{equation*}
\end{thm} 
\begin{proof}
	It suffices to show that if $b$ is invertible in $\A$ with the inverse $x$, then $x \in \B$. Then $(a^{*}a)(xx^{*})=1$. Since $a^{*}a \in \B$ and by above corollary, we know $xx^{*} \in \B$. But $x = (x\st{x})\st{a}$, thus $x \in \B$.
\end{proof}

\subsection{Abelian \texorpdfstring{$C^{*}$-Algebras}{C*-Algebras}}

The abelian \Cs is firstly an ablian Banach algebra, thus all results in the subsection \ref{sec2} can be applied to it. But since the involution and the related norm provide more information, we can have better results than the general abelian Banach algebra has. Firsly, we strengthen the \textbf{Theorem} \ref{thm5} in subsection \textbf{2.1.4}.

\begin{thm}
	If $\A$ is an abelian \Cs, then the Gelfand transform $\Gamma \colon \A \sto C(\Sigma(\A))$ is an isometric $*$-isomorphism.
\end{thm}
\begin{proof}
	Firstly, $\Gamma$ is a $*$-homormophism, since the result in \textbf{Theorem} \ref{thm5} and
	\begin{equation*}
		\Gamma(\st{a})(h) = h(\st{a}) = \clo{h(a)} = \clo{\Gamma(a)}(h)
	\end{equation*}
	Now, we can easily see $\Gamma$ is an isometry by $4)$ in the \textbf{Theorem} \ref{thm5},
	\begin{equation*}
		\norm{\hat{a}}_{\infty}^{2} = \norm{\hat{\st{a}a}}_{\infty} = r(\st{a}a) = \norm{a}^2
	\end{equation*}
	Finally, we need to check $\Gamma$ is surjective. It is because $\Gamma(\A)$ is a closed subalgebra of $C(\Sigma(\A))$, which is closed under the complex conjugate and separates points in $\Sigma(\A)$. Then by the Stone-Weierstrass Theorem $\Gamma(\A)=C(\Sigma(\A))$
\end{proof}

We know if $\A = \Cg{a}$ for some normal element $a$, then $\A$ is an abelian \Cs. In fact, 
\begin{equation*}
	\A = \clo{\{~p(a,\st{a}) \colon p(z,\clo{z}) \text{ is a polynomial} ~\}}
\end{equation*}

Then, we can modify the result in the \textbf{Propostion} \ref{prop7} in subsection \textbf{2.1.4}.

\begin{thm}
	Let $\A = \Cg{a}$ for some normal element $a$. Then there is a unique isometric $*$-isomorphism $\rho \colon \A \sto C(\sigma(a))$.
\end{thm}
\begin{proof}
	Firstly, we have a similar homeomorphism
	\begin{center}
		\begin{tabular}{l c c l}
			$\tau \colon$ & $\Sigma(\A)$ & $\longrightarrow$ & $\sigma(a)$ \\
			~ & $h$ & $\longmapsto$ & $h(a)$
		\end{tabular}
	\end{center}
	Then $\rho(x) = \Gamma(x) \circ \tau^{-1}$ is indeed an isometric $*$-isomorphism by the property of Gelfand transform. And moreover, by the result of \textbf{Propostion} \ref{prop7}, for any $z \in \sigma(a)$, $z = h(a)$ for some $h \in \Sigma(\A)$
	\begin{eqnarray*}
		\rho(p(a,\st{a}))(z) &=& \rho(p(a,\st{a}))(h(a)) = \Gamma(p(a,\st{a})) \circ \tau^{-1} (h(a)) \\
		&=& \Gamma(p(a,\st{a}))(h) = h(p(a,\st{a})) \\
		&=& p(h(a), h(\st{a})) = p(h(a), \clo{h(a)}) \\
		&=& p(z,\clo(z))
	\end{eqnarray*}
	That means that $\rho$ maps the polynomials in $\A$ to polynomails in $\sigma(a)$. Therefore, $\rho$ is unique.
\end{proof}

By the Riesz Functional Calculus, we have the map $f \mapsto f(a)$ from $\hol{(a)}$ to $\A$ for a \Cs $\A$ and any element $a \in \A$. Now, we can extend this definition a $C(\sigma(a))$ by above $\rho$, but we need $a$ is normal, then
\begin{center}
	\begin{tabular}{l c c l}
		$\rho^{-1} \colon$ & $C(\sigma(a))$ & $\longrightarrow$ & $\Cg{a}$ \\
		~ & $f$ & $\longmapsto$ & $f(a)$
	\end{tabular}
\end{center}
defined above is an isometric isomorphism and $\rho^{-1}$ maps
\begin{center}
	\begin{tabular}{r @{$~\longmapsto$~} l}
		$1$ & $1$ \\
		$z$ & $a$ \\
		$\clo{z}$ & $\st{a}$ \\
		$z^{-1}$ & $a^{-1}$\\
		$p(z,\clo{z})$ & $p(a,\st{a})$ 
	\end{tabular}
\end{center}

Therefore, this map is unique and it is clearly the extension of the Riesz Functional Calculus, called Continuous Functional Calculus. Like the Riesz Functional Calculus, there is also a Spectral Theorem.
\begin{thm}[Spectral Theorem]
	Let $\A$ be a \Cs and $a \in \A$ be a normal element, then for $f \in C(\sigma(a))$, 
	\begin{equation*}
		\sigma(f(a)) = f(\sigma(a))
	\end{equation*}
\end{thm}
\begin{proof}
	For some compact space $X$ and $f \in C(X)$, then $C(X)$ with the supremum norm is a \Cs and $\sigma(f) = \ran{f}$. Then since $f \mapsto f(a)$ is a $*$-isomorphism, 
	\begin{equation*}
		\sigma(f(a)) = \sigma_{C(\sigma(a))}(f) = \ran{f} = f(\sigma(a)) \qedhere
	\end{equation*}
\end{proof}


There is an important example.
\begin{exam} \label{exam2}
	Let $\mu$ be a compactly supported, regular Borel measure on $\C$ and ($X,\Omega,\mu$) be the measure space. For each $\pi \in \lfs{\infty}(\mu)$, we define the map
	\begin{center}
		\begin{tabular}{l c c l}
			$M_{\phi} \colon$ & $\lfs{2}(\mu)$ & $\longrightarrow$ & $\lfs{2}(\mu)$ \\
			~ & $f(z)$ & $\longmapsto$ & $\phi(z)f(z)$
		\end{tabular} 
	\end{center}
	Then clearly $M_{\phi}$ is in $\fml{B}(\lfs{2}(\mu))$.
	\begin{enumerate}[label=\arabic*)]
		\item $\st{(M_{\phi})} = M_{\clo{\phi}}$ and $M_{\phi}$ is a normal element in $\fml{B}(\lfs{2}(\mu))$.
		\item $\phi \mapsto M_{\phi}$ is a $*$-homomorphism from $\lfs{\infty}(\mu)$ to $\lfs{2}(\mu)$.
		\item $\norm{M_{\phi}} = \norm{\phi}_{\infty}$.
		\item $\sigma(M_{\phi})= \bigcap\{\clo{\phi(U)} \colon U \in \Omega ~\&~ \mu(X \backslash U) = 0\}$.
		\item If $f \in C(\sigma(M_{\phi}))$, then $f(M_{\phi}) = M_{f \circ \phi}$.
	\end{enumerate}
	If $\phi(z) = z$, we set denote $N_{\mu} = M_{\phi}$ and in fact, $\sigma(N_{\mu}) = \supp{\mu}$.
\end{exam}


\subsection{Positive Elements}

We have known that self-adjoint elements play a important role in a \Cs $\A$. The self-adjoint element $a$ in $\A$ is like the real number in $\C$, and the relationship between them can be revealed by the fact $a \in \Rea{\A}$ if and only if $\sigma(a) \subset \R$. The converse is obtained by the Continuous Functional Calculus. In fact, Continuous functional calculus can provide more relation between the element in $\A$ and the elment in $\C$, like positivity.

\begin{defn}
	Let $\A$ be a \Cs and $a \in \Rea{\A}$. Then $a$ is called a positive element if and only if $\sigma(a) \subset \R^{+}$, denoted by $a \geqslant 0$. And let $\A_{+}$ be the set of all positive elements.
\end{defn}

This definition is nature by above mention, but it may not be very explicit. So we need to show more direct equivalent definitions of positive elements.

\begin{thm}
	Let $\A$ be a \Cs. Then the following statements are equivalent.
	\begin{enumerate}[label=\arabic*)]
		\item $a \geqslant 0$.
		\item $a = b^2$ for some $b \geqslant 0$.
		\item $a \in \Rea{\A}$ and $\norm{t - a} \leqslant 0$ for all $t \geqslant \norm{a}$.
		\item $a \in \Rea{\A}$ and $\norm{t - a} \leqslant 0$ for some $t \geqslant \norm{a}$
	\end{enumerate}
\end{thm}
\begin{proof}
	All of these can be done by the functional calculus. And our goal is to find some vilid functions in $C(\sigma(a))$ to complete these.
	\item $1) \Rightarrow 2)$ Let $f(x) = \sqrt{x}$ in $C(\sigma(a))$ and since $\sigma(a) \subset \R^{+}$, $f$ is well-defined. Let $b = f(a)$. Then, we have $a = b^{2}$. And by Spectral Theorem, $\sigma(b) = \sigma(f(a)) \subset \R{+}$.
	\item $2) \Rightarrow 3)$ Let $f(x) = x^2$ defined on $\sigma(b)$, then $a = f(b)$ and $\norm{a} = \norm{f}_{\infty}$. By this condition, $f(x) \geqslant 0$. Thus $\sigma(a) = f(\sigma(b)) \subset \R^{+}$.
	\item $3) \Rightarrow 4)$ It is trivial.
	\item $4) \Rightarrow 1)$ Let $f(x) = x$ defined on $\sigma(a) \subset \R$. Then this condition means 
	\begin{equation*}
		\norm{t-f}_{\infty} = \norm{t-f(a)} =\norm{t-a} \leqslant t
	\end{equation*}
	for some $t \geqslant \norm{a} = \norm{f}_{\infty}$. Therefore, $f(x) \geqslant 0$ for all $x \in \sigma(a)$. Thus $\sigma(a) = f(\sigma(a)) \subset \R^{+}$.
\end{proof}

Like the fact that any element in a \Cs can be combined by two self-ajoint elements, any self-adjoint element can be combined by two positive elements.

\begin{prop}
	Let $\A$ be a \Cs. If $a \in \Rea{\A}$, then there are unique $u,v \in \R^{+}$, s.t.
	\begin{equation*}
		a = u - v ~~\&~~ uv = vu = 0
	\end{equation*}
\end{prop}
\begin{proof}
	Let $f(x) = \max{x,0}$ and $g(x) = - \min{x,0}$. Then $f,g \in C(\sigma(a))$ and $f(x)-g(x)=x$ and $f(x)g(x)=0$. Then $u = f(a)$ and $v = g(a)$ satisfy above conditions.\\
	If $a=u_1-v_1$, then we can know $\Cg{a,u,v,u_1,v_1}$ is an abelian \Cs, thus for some compact space $X$, $\Cg{a,u,v,u_1,v_1} \cong C(X)$. And this uniqueness can be proved in a continuous function space.
\end{proof}

\begin{cor}
	Let $\A$ be a \Cs. Then $\A_{+}$ is a cone.
\end{cor}
\begin{proof}
	Let $\{a_n\} \subset \A_{+}$ be a sequence s.t. $a_n \sto a$. Then by above proposition, $\norm{a_n-\norm{a_n}} \leqslant \norm{a_n}$. Taking norm limit, 
	$\norm{a-\norm{a}} \leqslant \norm{a}$, thus $a \in \A_{+}$.\\
	Clearly, $\alpha \A_{+} \subset \A_{+}$ for any $\alpha >0$. For $a, b \in \A$, we can assume that $\norm{a} \leqslant 1$ and $\norm{b} \leqslant 1$, then
	\begin{equation*}
		\norm{1-\frac{1}{2}(a+b)} = \frac{1}{2}\norm{(1-a)+(1-b)} \leqslant 1
	\end{equation*}
	Thus $\frac{1}{2}(a+b) \in \A_{+}$, i.e. $a+b \in \A_{+}$.
\end{proof}

Then, we can build an order on $\Rea{\A}$ by defining $a \leqslant b \Leftrightarrow b-a \in \A_{+}$. And moreover, let $\A_{-} = - \A_{+}$, then $\A_{-} \bigcap \A_{+} = \{0\}$. There are other properties of positivity.

\begin{prop}
	Let $\A$ be a \Cs.
	\begin{enumerate}[label=\arabic*)]
		\item If $a \geqslant 0$, then there is a unique $b \geqslant 0$ s.t. $a = b^n$.
		\item If $a \in \A$, then $\st{a}a \in \A_{+}$.
		\item If $a \leqslant b$ in $\Rea{\A}$, then $\st{c}ac \leqslant \st{c}bc$ for any $c \in \A$.
		\item For any $a \in \Rea{\A}$, $-\norm{a} \leqslant a \leqslant \norm{a}$ and if $a \in \A$, $0 \leqslant \st{a}a \leqslant \norm{a}^2$.
		\item If $0 \leqslant a \leqslant b$, then $b^{-1} \leqslant a^{-1}$.
		\item For any $a \in \A$, we define $\abs{a} = \sqrt{\st{a}a}$, then $\abs{a} = a_{+}+a_{-}$.
		\item If $0 \leqslant a \leqslant b \in \A$, then $\norm{a} \leqslant \norm{b}$.
	\end{enumerate}
\end{prop}
\begin{proof}
	For $1)$, let $f(x)= \sqrt[n]{x}$ defined on $\sigma(a) \subset \R$, then $b=f(a)$ sastisfying $ a = b^{n}$.
	\item For $2)$, let $b = \st{a}a = b_{+} - b_{-}$ and $c = \sqrt{b_{+}}$  and $d=ac$. Since $\st{d}d = b_{-}^2 \in \A_{+}$, $\st{d}d \in \A_{-}$. Let $d = x + iy$, then $d\st{d} + \st{d}d = 2(x^2+y^2) \in \A_{+}$. Thus
	\begin{equation*}
		d\st{d} = d\st{d} + \st{d}d -\st{d}d \in \A_{+}
	\end{equation*}
	By the fact that $\sigma(d\st{d})\bigcup\{0\} = \sigma(\st{d}d)\bigcup\{0\}$, $b_{-}^2 = -\st{d}d =0$. Therefore, $b = \st{a}a \in \A_{+}$.
	\item For $3)$, let $d = \sqrt{b-a}$, then 
	\begin{equation*}
		\st{c}bc - \st{c}ac = \st{c}(b-a)c = \st{c}\st{d}dc = \st{(dc)}dc \in \A_{+}
 	\end{equation*}
 	\item For $4)$, if $x \in \sigma(a)$, $\abs{x} \leqslant \norm{a}$. Therefore, $f(x) = \norm{a}-x$ and $g(x) = \norm{a}+x$ are positive on $\sigma(a)$.
 	\item For $5)$, since $0 \leqslant a \leqslant b$,
 	\begin{equation*}
 		1-b^{-\frac{1}{2}}ab^{-\frac{1}{2}} = b^{-\frac{1}{2}}(b-a)b^{-\frac{1}{2}} \geqslant 0
 	\end{equation*}
 	i.e. $\st{(a^{\frac{1}{2}} b^{-\frac{1}{2}})}(a^{\frac{1}{2}} b^{-\frac{1}{2}}) \leqslant 1$, therefore $\norm{a^{\frac{1}{2}} b^{-\frac{1}{2}}} \leqslant 1$ by functional calculus as similar as $4)$. And thus $1 \geqslant (a^{\frac{1}{2}} b^{-\frac{1}{2}})\st{(a^{\frac{1}{2}} b^{-\frac{1}{2}})} = a^{\frac{1}{2}}b^{-1}a^{\frac{1}{2}}$. Therefore, $a^{-1}=a^{-\frac{1}{2}}1a^{-\frac{1}{2}} \geqslant b^{-1}$.
 	\item $6)$ holds by the functional calculus and the uniqueness is by $1)$.
 	\item For $7)$, $\sigma(a) = \{h(a) \colon h \in \Sigma(\A)\} \subset \R^{+}$ and $\sigma(b) = \{h(b) \colon h \in \Sigma(\A)\} \subset \R^{+}$ and $\sigma(b-a) = \{h(b-a) \colon h \in \Sigma(\A)\} \subset \R^{+}$, therefore $h(b) \geqslant h(a) \geqslant 0$ for any $h \in \Sigma(\A)$. That means $r(b) \geqslant r(a)$, i.e. $\norm{b} \geqslant \norm{a}$.
\end{proof}

By using $2)$ on above proposition, we can easily see that
\begin{cor} \label{cor7}
	Let $\Hs$ be a Hilbert space and $T \in \oper$. If $T$ is positive, then for any $h \in \Hs$
	\begin{equation*}
		\langle Th,h \rangle \geqslant 0
	\end{equation*}
\end{cor}
\begin{rem}
	There is a $A \in \oper$ s.t. $T=\st{A}A$, thus above statement is true. In fact, the converse is also true.
\end{rem}

\subsection{Approximate Identities}

When we research the proper ideal of an algebra, this ideal does not contain the identity. So for the ideal of a $\st{C}$-algebra, we want to find some element has similar property as the identity has in the ideal.

\begin{defn}
	Let $\A$ be a \Cs and $\{e_i\}$ be a net in $\A$ s.t.
	\begin{enumerate}[label=\arabic*)]
		\item $0 \leqslant e_i \leqslant 1$ for all $i$,
		\item $e_i \leqslant e_j$ for $i \leqslant j$,
		\item $\lim_{i} ae_i = \lim_{i} e_ia = a$ for any $a \in \A$,
	\end{enumerate}
	Then $\{e_i\}$ is called an approximate identity for $\A$.
\end{defn}

\begin{thm}
	Every \Cs $\A$ has an approximate identity.
\end{thm}
\begin{proof}
	Firstly, let $\Lambda = \{e \in \A_{+} \colon e < 1\}$. We can check $\Lambda$ is indeed a direct set with respect to $\leqslant$. Define two functions as
	\begin{eqnarray*}
		f(t) &=& \frac{t}{1-t},~ \forall~ t \in [0,1),\\
		g(t) &=& \frac{t}{1+t} = 1 - \frac{1}{1+t},~ \forall~ t \in [0,\infty).
	\end{eqnarray*}
	In fact, $g(f(t))=t$. Then for any $a,b \in \Lambda$, let $y=f(a)+f(b)$ and $c = g(y)$. And since $\norm{g}_{\infty} < 1$, $c \in \Lambda$. The fact that $x=f(a) \leqslant y$ implies $1+x \leqslant 1+y$. Then $(1+x)^{-1} \geqslant (1+y)^{-1}$.
	\begin{equation*}
		a = g(f(a)) = g(x) = 1 - (1+x)^{-1} \leqslant 1-(1+y)^{-1} =c
	\end{equation*}
	Similarly, $b \leqslant c$. Therefore, $\Lambda$ is direct.
	\item If $a \in \A_{+}$, let $e_n=g(na) \in \Lambda$. Define
	\begin{equation*}
		h(t) = t^2(1-g(nt)) = \frac{t^2}{1+nt} \leqslant \frac{t}{n}
	\end{equation*}
	Thus $h(a)= a^2(1-g(na)) = a(1-e_n)a$, that means
	\begin{equation*}
		\norm{a(1-e_n)a} = \norm{h}_{\infty} \leqslant \frac{\norm{a}}{n}
	\end{equation*}
	For any $\varepsilon > 0$, we can choose a $N$, s.t. for $n > N$, $\norm{a(1-e_n)a} < \varepsilon$. Moreover, since for $0 \leqslant d \leqslant b \leqslant 1 \in \A$, $\st{c}(1-b)c \leqslant \st{c}(1-d)c$ for any $c \in \A$. Therefore, 
	\begin{equation*}
		\norm{\st{c}(1-b)c} \leqslant \norm{\st{c}(1-d)c}
	\end{equation*}
	And combining above mention and the fact for $0 \leqslant d \leqslant b \leqslant 1 \in \A$,
	\begin{eqnarray*}
		\norm{c-dc}^2 &\leqslant& \norm{\st{c}(1-d)c} \\
		\norm{c-cd}^2 &\leqslant& \norm{\st{c}(1-d)c}
	\end{eqnarray*}
	implies for $e \geqslant e_N$, 
	\begin{eqnarray*}
		\norm{a-ea}^2 &<& \varepsilon \\
		\norm{a-ae}^2 &<& \varepsilon
	\end{eqnarray*}
	Therefore, 
	\begin{equation*}
		\lim_{i} ae_i = \lim_{i} e_ia = a,~~\forall~~a \in \A 
	\end{equation*}
	For arbitrary $a \in \A$, $a$ can be write as the linear combination of four positive elements.
\end{proof}
\begin{rem}
	The result that $\Lambda$ is direct is also true for $\A$ without the identity, since $g(0)=f(0)=0$, which means $x, y, c \in \A$. And if $\A$ is separable, we can find a sequential approximate identity in its countable dense subset. Moreover, this sequential approximate identity can apply in whole $\A$.
\end{rem}

Then we can use the approximate identity to get some interesting results. First, we need a lemma.

\begin{lem}
	If $\A$ is a \Cs and $x,y \in \A$, $a \in \A_{+}$ s.t. $\st{x}x \leqslant a^{\alpha}$ and $\st{y}y \leqslant a^{\beta}$ for some positive scalars $\alpha$ and $\beta$ with $\alpha + \beta > 1$, then the sequence $u_n = x(n^{-1}+a)^{-\frac{1}{2}}$y converges to a $u \in \A$ with $\norm{u} \leqslant \norm{a^{\frac{1}{2}(\alpha+\beta-1)}}$.
\end{lem}
\begin{proof}
	Let $d_nm = (n^{-1}+a)^{-\frac{1}{2}}-(m^{-1}+a)^{-\frac{1}{2}}$.
	\begin{eqnarray*}
		\norm{u_n-u_m}^2 &=& \norm{xd_nmy}^2 = \norm{\st{y}d_nm\st{x}xd_nmy} \\
		&\leqslant& \norm{a^{\frac{\alpha}{2}}d_nmy}^2 = \norm{a^{\frac{\alpha}{2}}d_nm\st{y}yd_nma^{\frac{\alpha}{2}}} \\
		&\leqslant& \norm{a^{\frac{\alpha}{2}}d_nm a^{\beta} d_nma^{\frac{\alpha}{2}}} \\
		&=& \norm{d_nm a^{\frac{\alpha+\beta}{2}}}^2
	\end{eqnarray*}
	Since $f_n(t) = (n^{-1}+t)t^{\frac{\alpha+\beta}{2}}$ is an increasing positive sequence, $d_n=(n^{-1}+a)a^{\frac{\alpha+\beta}{2}}$ is an increasing positive sequence in $\A_{+}$. By Dini's Theorem, since $\sigma(a)$ is compact, there is a continuous fuction $f$ s.t. $f_n \sto f$ uniformly. Let $d = f(a)$, thus $d_n \sto d$ in norm. Thus $\{u_n\}$ is Cauchy and there exists $u = \lim_{n \sto \infty} u_n$.
	\begin{equation*}
		\norm{u_n} = \norm{x(n^{-1}+a)^{-\frac{1}{2}}} \leqslant \norm{(n^{-1}+a)a^{\frac{\alpha+\beta}{2}}} \leqslant \norm{a^{\frac{1}{2}(\alpha+\beta-1)}} \qedhere
	\end{equation*}
\end{proof}

\begin{prop}
	If $\A$ is a \Cs and $a \in \A{+}$ and $x \in \A$ with $\st{x}x \leqslant a$, and $0<\alpha<\frac{1}{2}$, then there exists $u \in \A$ with $\norm{u} \leqslant \norm{a^{\frac{1}{2}-\alpha}}$ and $x=ua^{\alpha}$.
\end{prop}
\begin{proof}
	Let $u_n=x(n^{-1}+a)^{-\frac{1}{2}}a^{\frac{1}{2}-\alpha}$. By above lemma, $u_n \sto u$ with $\norm{u} \leqslant \norm{a^{\frac{1}{2}-\alpha}}$. We use similar prove as above lemma, 
	\begin{eqnarray*}
		\norm{x-u_na^{\alpha}}^2 &=& \norm{x(1-(n^{-1}+a)^{-\frac{1}{2}}a^{\frac{1}{2}})}^2 \\
		&\leqslant& \norm{(1-(n^{-1}+a)^{-\frac{1}{2}}a^{\frac{1}{2}})a(1-(n^{-1}+a)^{-\frac{1}{2}}a^{\frac{1}{2}})} \\ 
		&=& \norm{a^{\frac{1}{2}}(1-(n^{-1}+a)^{-\frac{1}{2}}a^{\frac{1}{2}})}^2 \\
		&=& \norm{a^{\frac{1}{2}}-(n^{-1}+a)^{-\frac{1}{2}}a}
	\end{eqnarray*}
	By the Dini's Theorem, $(n^{-1}+a)^{-\frac{1}{2}}a \sto a$ in norm. Therefore, $u_na^{\alpha} \sto a$ i.e. $a = ua^{\alpha}$.
\end{proof}
\begin{cor}
	If $\A$ is a \Cs and $x \in \A$ and $0 < \beta <1$, then there is a $u \in \A$ s.t.
	\begin{equation*}
		x = u\abs{x}^{\beta}	
	\end{equation*}
\end{cor}

\subsection{Ideals and Quotients}

Firstly, there are two easy results of closed ideal in a $\st{C}$-algebra.

\begin{prop}
	Let $\A$ be a $\st{C}$-algebra.
	\begin{enumerate}[label=\arabic*)]
		\item If $\I$ is a closed left or right ideal of $\A$, $a \in \I$ with $a=\st{a}$, then for $f \in C(\sigma(a))$ with $f(0)=0$, $f(a) \in \A$.
		\item If $\I$ is a closed ideal, then $a \in \I$ implies $\st{a} \in \A$.
	\end{enumerate}
\end{prop}
\begin{proof}
	For $1)$, if $\I$ is proper, $0 \in \sigma(a)$. Then $f(0) = 0 and \sigma(a) \subset \R$, $f$ can be approximated by a sequence of polynomials $p_n$ with $p_n(0) = 0$. Therefore, $p_n(a) \in \I$ and by the fact that $\I$ is closed, $f(a) \in \I$.
	\item For $a \in \I$, by the corollary in above subsection, we know there is a $u \in \A$ s.t. $a=u\abs{a}^{\frac{1}{2}}$. By $1)$, $\abs{a}^{\frac{1}{2}} \in \I$. Therefor,
	\begin{equation*}
		\st{a} = \abs{a}^{\frac{1}{2}} u \in \I \qedhere
	\end{equation*}
\end{proof}

\begin{defn}
	If $\A$ is a \Cs and $\B$ is a $*$-subalgebra of $\A$, then $\B$ is called hereditary if for any $b \in \B_{+}$ and $x \in \A$ with $0 \leqslant x \leqslant b$, $x \in \B$.
\end{defn}

Now, we can give more profound properties of the closed left ideals.

\begin{thm} \label{thm6}
	Let $\A$ be a $\st{C}$-algebra.
	\begin{enumerate}[label=\arabic*)]
		\item If $\I$ is a closed left ideal of $\A$ and $\B = \I \bigcap \st{\I}$, then $\B$ is a hereditary subalgebra of $\A$.
		\item If $\B$ is a hereditary subalgebra of $\A$ and $\I = \{x\in \A \colon \st{x}x \in \B\}$, then $\I$ is a closed left ideal of $\A$.
		\item If $\I$ is a closed left ideal of $\A$ and $\B = \I \bigcap \st{\I}$, then $\I = \{x\in \A \colon \st{x}x \in \B\}$.
		\item If $\B$ is a hereditary subalgebra of $\A$ and $\I = \{x\in \A \colon \st{x}x \in \B\}$, then $\B = \I \bigcap \st{\I}$.
	\end{enumerate}
\end{thm}
\begin{proof}
	For $1)$, clearly, $\B$ is a $\st{C}$-algebra. Let $0 \leqslant x \leqslant b$ for some $b \in \B_{+}$. Since $x^{\frac{1}{2}}x^{\frac{1}{2}} \leqslant b$ there is a $u \in \A$, s.t. $x^{\frac{1}{2}} = u b^{\frac{1}{3}}$. But $b^{\frac{1}{3}} \in \B_{+} \subset \I$. Therefore, $x^{\frac{1}{2}} \in \I$ and thus $x \in \I$. Since $x$ is self-adjoint, $x \in \B$.
	\item For $2)$, if $x \in \I$ and $a \in \A$, then 
	\begin{equation*}
		\st{(ax)}ax = \st{x}\st{a}ax \leqslant \norm{a}^2 \st{x}x \in \B.
	\end{equation*}
	Therefore, $\st{(ax)}ax \in \B$, i.e. $ax \in \I$. If $x,y \in I$, then 
	\begin{equation*}
		\st{(x+y)}x+y \leqslant \st{(x+y)}(x+y)+\st{(x-y)}(x-y) = 2 (\st{x}x+\st{y}y) \in \B
	\end{equation*}
	Thus, $\I$ is a closed left ideal.
	\item For $3)$, if $x \in \A$ and $\st{x}x \in \B$, $\st{x}x \in \I$. Thus $\abs{x}^{\frac{1}{2}} \in \I$ since $\I$ is a closed left ideal. Therefore, $x = u \abs{x}^{\frac{1}{2}} \in \I$. The converse is clearly true.
	\item For $4)$, if $x \in \I_{+}$, then $x^2 \in \B$. So $x = \sqrt{x^2} \in \B_{+}$. Conversely, $x \in \B_{+}$ means $\st{(x^{\frac{1}{2}})}(x^{\frac{1}{2}}) = x \in \B$, $x^{\frac{1}{2}} \in \I_{+}$. Therefore, $\B_{+} = \I_{+}$. Thus, $\B = \I \bigcap \st{\I}$. 
\end{proof}

\begin{thm}
	If $\I$ is a closed ideal of a \Cs $\A$, then the quotient algebra $\A / \I$ with the induced norm and the induced involution, i.e. $\st{a+\I} = \st{a}+\I$ is also a \Cs and the norm can be
	\begin{equation*}
		\norm{a+\I} = \inf{\{~\norm{a-ax} \colon x \in \I_{+}, \norm{x} \leqslant 1~\}}
	\end{equation*}
\end{thm}
\begin{proof}
	Firstly, let $\{e_i\}$ be an approximate identity for $\I$. Since $0 \leqslant e_i \leqslant 1$, if $a \in \A$ and $y \in \I$, $\norm{(a+y)(1-e_i)} \leqslant \norm{a+y}$.
	\begin{eqnarray*}
		\norm{a+y} &\geqslant& \liminf_{i} \norm{(a+y)(1-e_i)} \\
		&=& \liminf_{i} \norm{(a-ae_i)+(y-ye_i)} \\
		&=& \liminf_{i} \norm{(a-ae_i)}
	\end{eqnarray*}
	The converse is clearly true.
	\item The involution defined on the quotient algebra is well-defined and satifies the conditions since $\I$ is self-adjoint. We just need to show the induced norm satisfies the $\st{C}$-identity. For $a \in \A$, we have $\norm{\st{a}+\I} = \norm{a+\I}$, and 
	\begin{eqnarray*}
		\norm{a+\I}^2 &=& \norm{(\st{a}+\I)(a+\I)} \\
		&\leqslant& \norm{\st{a}+\I}\norm{a+\I} \\
		&=& \norm{a+\I}^2
	\end{eqnarray*}
	Conversely,
	\begin{eqnarray*}
		\norm{a+\I}^2 &=& \inf{\{~\norm{a-ax}^2 \colon x \in \I_{+}, \norm{x} \leqslant 1~\}}\\
		&=& \inf{\{~\norm{(1-x)\st{a}a(1-x)} \colon x \in \I_{+}, \norm{x} \leqslant 1~\}}\\
		&\leqslant& \inf{\{~\norm{\st{a}a(1-x)} \colon x \in \I_{+}, \norm{x} \leqslant 1~\}}\\
		&=& \norm{\st{a}a+\I}
	\end{eqnarray*}
	Therefore, $\A / \I$ is indeed a $\st{C}$-algebra.
\end{proof}

Then it can provide a \Cs with some similar results as general rings have. These results can also show us how the algebraic structure in a \Cs affect the topological structure. In my oppion, for researching a \Cs, researching the algebraic structure may be more important.

\begin{cor}
	Let $\A$ and $\mathfrak{C}$ be two $\st{C}$-algebras.
	\begin{enumerate}[label=\arabic*)]
		\item If $\rho \colon \A \sto \mathfrak{C}$ is a $*$-homomorphism, then $\ran{\rho}$ is a \Cs and the induced map $\tilde{\rho} \colon \A / \ker{\rho} \sto \ran{\rho}$ is an $*$-isomorphism.
		\item If $\I$ is a closed ideal of $\A$ and $\B$ is a subalgebra of $\A$, then there is a $*$-isomorphism
		\begin{equation*}
			 \B / (\B \cap \I) \cong (\B+\I) / \I
		\end{equation*}
	\end{enumerate}
\end{cor}
\begin{proof}
	For $1)$, we just need to show $\ran{\rho}$ is closed.\\ Since $\tilde{\rho}$ is a $*$-monomorphism from $\A / \ker{\rho}$ to $\C$, by the result in \textbf{Proposition} \ref{prop8} in the subsection \textbf{2.1.1}, $\rho$ is an isometry. Thus $\ran{\rho}=\rho(\A)$ is closed.
	\item For $2)$, there is a commutative graph like
	\begin{center}
		\begin{tikzcd}
			\B \arrow[r, "i"] \arrow[d, "Q"]
				& \A \arrow[ld, "\pi"] \\
			\A / \I
		\end{tikzcd}
	\end{center}
	$Q=\pi \circ i \colon \B \sto \B / \I$ and thus $\pi^{-1}(Q(\B)) = \B + \I$. By restricting $\pi$ on $\B + \I$, then we have
	\begin{equation*}
		(\B+\I) / \I \cong Q(\B) \cong \B / (\B \cap \I) \qedhere
	\end{equation*}
\end{proof}

\subsection{Positive Functionals and GNS Construction}

We have known that $\oper$ is a $\st{C}$-algebra. In fact, defining the general \Cs is to research the operator algebra. Moreover, by the properties, we can see any \Cs is a $*$-subalgebra of $\oper$ for some $\Hs$. To prove that, we just need to find a faithful representation of the fixed $\st{C}$-algebra. Therefore, the main idea is to construct a representation $(\pi,\Hs)$. How to construct it is a question. But, fortunately, some special functionals can provide us a method.

\begin{defn}
	Let $\A$ be \Cs and $\phi$ is a linear functional on $\A$. $\phi$ is called positive if for any $a \in \A_{+}$, $\phi(a) \leqslant 0$. A positive linear functional $\phi$ is called a state if $\phi(1) = 1$.
\end{defn}
\begin{rem}
	Let $\St_{\A}$ denote the set of all states on $\A$.
\end{rem}

Then there are some properties of positive functionals.

\begin{prop}
	Let $\A$ be a \Cs and $\phi$ be a positive functional.
	\begin{enumerate}[label=\arabic*)]
		\item For any $x,y \in \A$, then
		\begin{equation*}
			\abs{\phi(\st{y}x)} \leqslant \phi(\st{x}x)\phi(\st{y}y)
		\end{equation*}
		\item $\phi$ is bounded and if $\{e_i\}$ is an approximate identity of $\A$, then
		\begin{equation*}
			\norm{\phi} = \lim_{i} \phi(e_i)
		\end{equation*}
	\end{enumerate}
\end{prop}
\begin{proof}
	For $1)$, it can easily check that $<x,y> = \phi(\st{y}x)$ is a semi-inner product, thus by the CBS inequality, above inequality is true.
	\item Assume $\A$ without identity.  If $\phi$ is unbounded, then there is a sequence $\{a_k\} \in \A_{+}$ with $\norm{a_k} \leqslant 1$ s.t. $\phi(a_k) > 2^k$. Let $a = \sum_{k=1}^{\infty} 2^{-k} a_k$. Then
	\begin{equation*}
		\phi(a) \geqslant \phi(\sum_{k=1}^{n} 2^{-k} a_k) > n
	\end{equation*}
	which is a contradiction.
	\begin{equation*}
		\alpha = \sup{\{~\phi(a) \colon a \in \A_{+}, \norm{a} \leqslant 1~\}} < \infty
	\end{equation*} 
	Since any element in $\A$ can be a linear combination of four positive elements, $\norm{\phi} \leqslant 4\alpha$. Therefore, $\phi$ is bounded.\\
	Let $\beta = \lim_{i}\phi(e_i)$. Clearly, $\beta \leqslant \norm{\phi}$. And since for $a \in \A$ with $\norm{a} \leqslant 1$ then $0 \leqslant \st{a}a \leqslant 1$
	\begin{equation*}
		\abs{\phi(a)}^2 = \lim_{i} \abs{\phi(e_ia)}^2 \leqslant \lim_{i}\phi(e_i)\phi(\st{a}a) \leqslant \beta \norm{\phi}
	\end{equation*}	
	\begin{equation*}
		\norm{\phi}^2 \leqslant \beta \norm{\phi}
	\end{equation*}
	i.e. $\norm{\phi} \leqslant \beta$.
\end{proof}
\begin{rem}
	For $2)$, if $\A$ has an identity, then $\norm{\phi} = \phi(1)$. It is because for $a \in \A$ with $\norm{a} \leqslant 1$
	\begin{equation*}
		\abs{\phi(a)}^2 \leqslant \phi(\st{a}a)\phi(1) \leqslant \phi(1)^2
	\end{equation*}
	i.e. $\abs{\phi(a)} \leqslant \phi(1)$.
\end{rem}

In fact, the converse of $2)$ in above proposition is also true.
\begin{prop}
	Let $\A$ be \Cs and $\phi$ is a bounded linear functional with $\norm{\phi} = \phi(1)$, then $\phi \leqslant 0$.
\end{prop}
\begin{proof}
	If $\A=C(X)$ for some compact space $X$, $\phi$ is a measure $\mu$ with $\mu(X)=\norm{\mu}$, then $\mu \geqslant 0$, i.e. $\phi$ is positive. Then for any $\A$, if $a \in \A_{+}$, $\B=\Cg{a} \cong C(\sigma(a))$. Then $\phi|_\B(1) \leqslant \norm{\phi} = \phi(1) = \phi|_\B(1)$, thus $\phi|_\B(a) = \phi(a) \geqslant 0$.
\end{proof}
Using above proposition and the Hahn-Banach Theorem, we can get the corollary.
\begin{cor} \label{cor6}
	If $\A$ is a \Cs and $\B$ is a $\st{C}$-subalgebra of $\A$, then every state on $\B$ can extend to $\A$.
\end{cor}	

Now, we can construct the a representation of a $\st{C}$-algebra. A representation $(\pi,\Hs)$ of a \Cs $\A$ is called cyclic if there is a unit cyclic vector $e$ s.t. $\clo{\pi(\A)e} = \Hs$.

\begin{thm}[Gelfand-Naimark-Segal Construction]
	Let $\A$ be a \Cs and $\St_{\A}$ be the coincided state space.
	\begin{enumerate}[label=\arabic*)]
		\item If $\phi \in \St_{\A}$, then there is a cyclic representation $(\pi_{\phi},\Hs_{\phi})$ with the unit cyclic vector $e_{\phi}$ s.t.
		\begin{equation*}
			\phi(a) = \langle \pi_{\phi}(a)e_{\phi},e_{\phi} \rangle ,~~\forall~ a \in \A
		\end{equation*}
		\item If $(\pi,\Hs)$ is a cyclic representation with the unit cyclic vector $e$, then there is a $\phi \in \St_{\A}$ defined as 
		\begin{equation*}
			\phi(a) = \langle \pi(a)e, e \rangle,~~ \text{for}~ a \in \A
		\end{equation*}
		And for the $(\pi_{\phi},\Hs_{\phi})$ defined as above mention, $\pi_{\phi} \cong \pi$.
	\end{enumerate}
\end{thm}
\begin{proof}
	For $1)$, the prove can be completed by several nature steps.
	\begin{enumerate}[label=\arabic*)]
		\item Constructing semi-inner product: Define a semi-inner product $\langle \cdot, \cdot \rangle$ on $\A$ like
		\begin{equation*}
			\langle x,y \rangle = \phi(\st{y}x),~~ \text{for}~ x,y \in \A
		\end{equation*}
		\item Constructing $\Hs_{\phi}$: $\langle \cdot, \cdot \rangle$ is just a semi-inner product on $\A$, thus we need to make it be nondegenerate. The nature method is constructing a equilaten relation or a closed ideal, and then inducing a quotient space. Naturally, we define
		\begin{equation*}
			\B = \{~x \in \A \colon \phi(x) = 0~\}
		\end{equation*}
		Clearly, $\B$ is a hereditary subalgebra. Then by the \textbf{Theorem} \ref{thm6} in the subsection \textbf{2.2.5}, $\B$ can induced a closed left ideal
		\begin{equation*}
			\I = \{~x \in \A \colon \st{x}x \in \B~\} = \{~x \in \A \colon \phi(\st{x}x) = 0~\}
		\end{equation*}
		And define the inner product on $\A / \I$ as
		\begin{equation*}
			\langle x+\I,y+\I \rangle = \phi(\st{y}x)
		\end{equation*}
		Therefore, we can easily check that $\A / \I$ is a inner product space. Then let $\Hs_{\phi}$ be the completion of $\A / \I$.
		\item Constructing $\pi_{\phi}$: Firstly, let $\pi_{\phi}(a)$ be defined on $\A / \I$ for $a \in \A$.
		\begin{center}
		\begin{tabular}{l c c l}
			$\pi_{\phi}(a) \colon$ & $\A / \I$ & $\longrightarrow$ & $\A / \I$ \\
			~ & $x+\I$ & $\longmapsto$ & $ax+\I$
		\end{tabular}
		\end{center}
		But since
		\begin{eqnarray*}
			\norm{ax+\I}^2 &=& \langle ax+\I,ax+\I \rangle = \phi(\st{ax}ax) \\
		 	&\leqslant& \norm{a}^2 \phi(\st{x}x) = \norm{a}^2\norm{x+\I}^2
		\end{eqnarray*}
		$\norm{\pi_{\phi}(a)} \leqslant \norm{a}$. Therefore, $\pi_{\phi}(a)$ can extend to $\Hs_{\phi}$ for any $a \in \A$. $\pi_{\phi} \colon \A \sto \fml{B}(\Hs_{\phi})$ is a representation.
		\item Check the conditions: Let $e_{\phi} = 1+\I$. Then
		\begin{equation*}
			\pi_{\phi}(\A)e_{\phi} = \{~a+\I \colon a \in \A~\} = \A / \I
		\end{equation*}
		Therefore, $\pi_{\phi}$ with $e_{\phi}$ is indeed a cyclic representation of $\A$. And clearly,
		\begin{equation*}
			\langle \pi_{\phi}(a)e_{\phi}, e_{\phi} \rangle = \phi(a)
		\end{equation*}
	\end{enumerate}
	For $2)$, we just need to construct a unitary from $\Hs_{\phi}$ to $\Hs$. By observation, 
	\begin{equation*}
		\langle \pi_{\phi}(a)e_{\phi}, e_{\phi} \rangle =\langle \pi(a)e, e \rangle
	\end{equation*}
	and the facts that $\Hs_{\phi}=\clo{\pi_{\phi}(\A)e_{\phi}}$ and $\Hs=\clo{\pi(\A)e}$, we can find the unitary $U$. Firstly, let $U$ be defined on $\pi_{\phi}(\A)e_{\phi}$ 
	\begin{center}
		\begin{tabular}{l c c l}
			$U \colon$ & $\pi_{\phi}(\A)e_{\phi}$ & $\longrightarrow$ & $\pi(\A)e$ \\
			~ & $\pi_{\phi}(a)e_{\phi}$ & $\longmapsto$ & $\pi(a)e$
		\end{tabular}
	\end{center}
	Since
	\begin{equation*}
		\norm{\pi(a)e}^2 = \langle \pi(a)e, \pi(a)e \rangle = \phi(\st{a}a) = \norm{\pi_{\phi}(a)e_{\phi}}^2
	\end{equation*}
	$U$ can extend to an unitary from $\Hs_{\phi}$ to $\Hs$. And
	\begin{equation*}
		U\pi_{\phi}(a)(\pi_{\phi}(x)e_{\phi}) = U\pi_{\phi}(ax)e_{\phi}=\pi(ax)e = \pi(a)(\pi(x)e) = \pi(a)U(\pi_{\phi}(x)e_{\phi})
	\end{equation*}
	Then $U\pi_{\phi}\st{U} = \pi$, $\pi_{\phi} \cong \pi$.
\end{proof}
\begin{rem}
	In fact, in above theorem, if $\phi$ is a positive functional, the results is also true.
\end{rem}

By above theorem, our faithful representation of a \Cs can not be constructed by a state, since $\pi_{\phi}$ is not injective. But we may choose enough many states to construct a faithful representation. Therefore, we need more properties of states.

\begin{prop}
	If $\A$ is \Cs and $a \in \A$ is self-adjoint, let $\alpha = \min{\sigma(a)}$ and $\alpha = \max{\sigma(a)}$, then
	\begin{equation*}
		[\alpha,\beta]= \{~\phi(a) \colon \phi \in \St_{\A}~\}
	\end{equation*}
\end{prop}
\begin{proof}
	Let $\B = \Cg{a}$. Then $\B = \{f(a) \colon f \in C(\sigma(a))\}$. For $\phi \in \St_{\A}$ and $\phi_0 =\phi|_{\B}$, then there is a measure $\mu$ s.t.
	\begin{equation*}
		\phi(f(a))=\phi_0(f(a)) = \int_{\sigma(a)} f d \mu,~~ \forall~ f \in C(\sigma)
	\end{equation*}
	In particular, $\phi(a) = \int_{\sigma(a)} t d \mu \in [\alpha, \beta]$.\\
	Conversely, if $\alpha \leqslant t_0 \leqslant \beta$, define $\phi_0 \in \St_{\B}$ as
	\begin{equation*}
		\phi_0(f(a)) = \frac{t_0 - \alpha}{\beta-\alpha}f(\alpha)+\frac{\beta - t_0}{\beta-\alpha}f(\alpha)
	\end{equation*}
	Then $\phi_0$ can extend to $\A$, and $t_0 = \phi(\alpha)$.
\end{proof}

By this proposition, we can get an important corollary.

\begin{cor}
	If $\A$ is a \Cs and $a \in \A$ and $\St_{}$ is $wk^{*}$-dense subset of $\St_{\A}$, then
	\begin{equation*}
		\norm{a}^2 = \sup{\{~\phi(\st{a}a) \colon \phi \in \St_{}~\}}
	\end{equation*}
\end{cor}
\begin{rem}
	In fact, we can easily check that $\St_{\A} \subset \st{\A}$ is a $wk^{*}$-compact convex subset. And this corollary is equivalent to saying that $\St_{}$ can seperate the points in $\A_{+}$.
\end{rem}

Now, using this corollary, we can finally construct a faithful representation.

\begin{thm}[Gelfand-Naimark Theorem]
	Every \Cs $\A$ has a faithful representation $(\pi,\Hs)$. Moreover, $\Hs$ is separable if and only if there are countable number of states on $\A$ that can separates points in $\A_{+}$, each of which defines a separable representation. In particular, each separable \Cs has a faithful, separable representation.
\end{thm}
\begin{proof}
	Let $\St_{}$ be $wk^{*}$-dense subset of $\St_{\A}$. Define
	\begin{equation*}
		\Hs = \oplus \{~\Hs_{\phi} \colon \phi \in \St_{}~\}
	\end{equation*} 
	\begin{equation*}
		\pi = \oplus \{~\pi_{\phi} \colon \phi \in \St_{}~\}
	\end{equation*}
	Then we can see
	\begin{eqnarray*}
		\norm{a}^2 &=& \sup \{~\phi(\st{a}a) \colon \phi \in \St_{}~\}\\ 
		&=& \sup \{~\langle \pi_{\phi}(a)e_{\phi},\pi_{\phi}(a)e_{\phi} \rangle \colon \phi \in \St_{}~\} \\
		&=& \sup \{~\norm{\pi_{\phi}(a)e_{\phi}}^2 \colon \phi \in \St_{}~\} \\
		&=&	\norm{\pi(a)e}^2\\
		&\leqslant& \norm{\pi(a)}^2
	\end{eqnarray*}
	And in the GNS Construction, we have seen $\norm{\pi_{\phi}(a)} \leqslant \norm{a}$ for any $\phi \in \St_{\A}$. Therefore,
	\begin{equation*}
		\norm{a} \geqslant \norm{\pi(a)}
	\end{equation*}
	Thus $\norm{\pi(a)} = \norm{a}$ i.e. $(\pi,\Hs)$ is indeed a faithful representation.
	\item If $\Hs$ is separable, let $\{e_n\}$ be the dense subset of $\{h \in \Hs \colon \norm{h} =1\}$. Define $\Hs_n = \clo{\pi(\A)e_n}$ and $\pi_n = \pi|_{\Hs_n}$. Therefore, in above construction,
	\begin{equation*}
		\{(\pi_{\phi},\Hs_{\phi},e_{\phi}) \colon \phi \in \St_{}\}
	\end{equation*}
	can be replaced by
	\begin{equation*}
		\{(\pi_n,\Hs_n,e_n) \colon n \in \N\}
	\end{equation*}
	Then for $a \in \A_{+}$, there exists $b \in \A$ s.t. $a = \st{b}b$, if for $n \in \N$
	\begin{equation}
		0=\phi_n(a) = \langle \pi_n(\st{b}b)e_n,e_n \rangle = \norm{\pi(b)e_n}^2 \tag{$*$}
	\end{equation}
	But $\norm{b} = \sup_{n} \norm{\pi(b)e_n}$, that implies $a = 0$.\\
	Conversely, if $\{\phi_n\}$ is these states, we can use the GNS Construction to get $\{(\pi_n,\Hs_n,e_n) \colon n \in \N\}$. Similarly, we can use direct sum to get the representation $(\pi,\Hs)$. Then by using the $(*)$, we can know this representation is definitely faithful.
	\item Finally, if $\A$ is separable, then the closed unit ball in $\A^{*}$ is $wk^{*}$-compactly metrizable. Therefore, there is a countable $wk^{*}$-dense subset of $\St_{\A}$. Thus by above corollary, this subset can separate the points in $\A_{+}$
\end{proof}
\begin{rem}
	Therefore, any \Cs can be isometrically imbedded in a operator algebra on some Hilbert space. Researching a abstract \Cs is actually researching the $\st{C}$-subalgebra of an operator algebra on some Hilbert space.
\end{rem}

Now, we can see the importance of states or positive funtionals. Thus, we want to find the relation between general linear functionals and the positive linear functionals. Firstly, we observe that for a positive functional $\phi$, $\phi(\st{a})=\clo{\phi(a)}$ for all $a$. 
\begin{defn}
	If $\A$ is a \Cs and $L$ is a linear functional, $L$ is called self-adjoint if 
	\begin{equation*}
		L(\st{a})=\clo{L(a)},~~ \forall~ a \in \A
	\end{equation*}
	Or equivalently, $L(a) \in \R$ for all $a \in \Rea{\A}$.
\end{defn}

The bounded self-adjoint linear functional in $\st{\A}$ is like the self-adjoint element in $\A$. So by the fact that any self-adjoint element can be a linear combination of two positive element, we want to find similar result of bounded linear functionals. There is a clearly lemma of bounded self-adjoint linear functional.

\begin{lem}
	If $\A$ is a \Cs and $L$ is a bounded self-adjoint linear functional, then
	\begin{equation*}
		\norm{L} = \sup{\{~L(a) \colon a \in \Rea{\A},~\norm{a} \leqslant 1~\}}
	\end{equation*}
\end{lem}

\begin{thm} [Jordan Decomposition]
	If $\A$ is a \Cs and $L$ is a bounded self-adjoint linear functional, then there are positive linear functionals $\phi_{+}$ and $\phi_{-}$ s.t. 
	\begin{equation*}
		L = \phi_{+} - \phi_{-} ~~\text{and}~~ \norm{L} = \norm{\phi_{+}}+\norm{\phi_{-}}
	\end{equation*}
\end{thm}
\begin{proof}
	Let $\Omega$ be the set of all bounded self-adjoint linear functional in the unit ball of $\st{\A}$ and $\tilde{\Omega}$ be the $wk^{*}$-closed convex hull of $\St_{\A}\cup (-\St_{\A})$.
	\begin{enumerate}[label=\arabic*)]
		\item Claim: $\Omega = \tilde{\Omega}$ \\
		Clearly, $\tilde{\Omega} \subset \Omega$. If there is a $L_0 \in \Omega \backslash \tilde{\Omega}$, then there exists a $x_0 \in \A$ separating $L_0$ and $\tilde{\Omega}$, and let $a_0 = \Rea{x_0}$, we have
		\begin{equation*}
			L(a_0) \leqslant \alpha < L_0(a_0),~~\forall~L \in \tilde{\Omega}
		\end{equation*}
		That means 
		\begin{equation*}
			\norm{a_0}=\sup{\{\abs{\phi(a_0)} \colon \phi \in \St_{\A}\}} \leqslant \alpha < L_0(a_0)
		\end{equation*}
		but $\norm{L_0} \leqslant 1$, which is a contradiction.
		\item Claim: $\Omega = \{ s\phi-t\psi \colon \phi,\psi \in \St_{\A},~s+t = 1 ~\&~ s,t \geqslant 0\}$ \\
		By the fact that $(s,t,\phi,\psi) \sto s\phi-t\psi$ is continuous and $\St_{\A}$ is $\st{wk}$-compact, above claim is clearly true.
		\item Existence: Therefore, for any self-adjoint linear functional $L$ with $\norm{L} = 1$, there are two $\phi,\psi \in \St_{\A}$, and positive real numbers $s,t$ with $s+t=1$ s.t. $L=s\phi-t\psi$. Then let $\phi_{+} = s\phi$ and $\phi_{-}=t\phi$, then $L = \phi_{+} - \phi_{-}$ and
		\begin{equation*}
			\norm{\phi_{+}}+\norm{\phi_{-}}= s+t = 1 = \norm{L}
		\end{equation*}
		For general $L$, just let $L / \norm{L}$ for $L \neq 0$, the statements are true. \qedhere
	\end{enumerate}
\end{proof}

In fact, for any $f \in \st{\A}$, $f = \phi + i \psi$ for some self-adjoint linear funtional $\phi, \psi$, where
\begin{equation*}
	\phi(a) = \frac{f(a) + \clo{f(\st{a})}}{2} ~\&~ \psi(a) =\frac{f(a) - \clo{f(\st{a})}}{2i}
\end{equation*}
By combining this and the Jordan Decomposition Theorem, we can have the following corollaries.

\begin{cor}
	Let $\A$ be a \Cs and $\pi \colon \A \sto \oper$ be the faithful representation constructed in the Gelfand-Naimark Theorem.
	\begin{enumerate}[label=\arabic*)]
		\item Any bounded linear functional can be a linear combination of four positive linear functionals.
		\item If $L \in \st{\A}$, then there are $g,h \in \Hs$, s.t. for any $a \in \A$
		\begin{equation*}
			L(a) = \langle \pi(a)g,h \rangle
		\end{equation*}
	\end{enumerate}
\end{cor}

\subsection{Representations}

\begin{defn}
	Let $\pi \colon \A \sto \oper$ be a representation of a \Cs $\A$.
	\begin{enumerate}[label=\arabic*)]
		\item If there is no invariant subspace for $\pi(\A)$, $\pi$ is called algebraically irreducible. If there is no invariant closed subspace, $\pi$ is called topologically irreducible.
		\item If $\clo{\pi(\A)e} = \Hs$ for some $e \in \Hs$, $\pi$ is called cyclic and $e$ is called a cyclic vector for $\pi$.
		\item If $\clo{\pi(\A)\Hs} = \Hs$, $\pi$ is called non-degenerate.
	\end{enumerate}
\end{defn}
\begin{rem}
	Since $\pi$ is $*$-homomorphism and $\A$ is $*$-closed, the definition of irreducibily of $\pi$ is valid. In fact, if $\pi$ is algebraically irreducible, then $\pi$ is clearly topologically irreducible. But the converse is also true. Then we have,
	\begin{center}
		$\pi$ irreducible $\Rightarrow$ $\pi$ cyclic $\Rightarrow$ $\pi$ is non-degenerate
	\end{center}
\end{rem}

\begin{thm}
	Every representation of a \Cs is equivalent to the direct sum of cyclic representations.
\end{thm}
\begin{proof}
	Let $\pi \colon \A \sto \oper$ be a representation and $e \in \Hs$ be any unit element. Then $\M = \clo{\pi(\A)e}$ is clearly a invariant space for $\pi$. If $\M \neq \Hs$, then $\Hs = \M \oplus \M^{\bot}$. We have another unit vector $e^{'} \in \M^{\bot}$, then let $\M^{'} = \clo{\pi(\A)e^{'}}$ and $\M^{'} \bot \M$. If $\M^{'} \neq \M^{\bot}$, we can continue above process again. By induction and the Zorn's Lemma, the theorem can be obtained.
\end{proof}

\begin{thm} \label{thm8}
	If $\A$ is a \Cs and $\I$ is an ideal of $\A$, then every representation $\rho \colon \I \sto \oper$ can be extended to a representation $\tilde{\rho} \colon \A \sto \oper$. Moreover, if $\rho$ is non-degenerate, then $\tilde{\rho}$ is unique.
\end{thm}
\begin{proof}
	Firstly, assume $\rho$ is non-degenerate and by above theorem we can also assume $\rho$ is cyclic with the cyclic vector $e$. Then for any $a \in \A$, we define
	\begin{center}
		\begin{tabular}{l c c l}
			$\tilde{\rho(a)} \colon$ & $\rho(\I)e$ & $\longrightarrow$ & $\Hs$ \\
			~ & $\rho(x)e$ & $\longmapsto$ & $\rho(ax)e$
		\end{tabular}
	\end{center}
	And since for an approximate identity $\{e_i\}$ for $\I$, we have
	\begin{eqnarray*}
		\norm{\rho(ax)e} &=& \lim_{i} \norm{\rho(ae_ix)e} \\
		&\leqslant& \lim_{i} \norm{\rho(ae_i)\rho(x)e} \\
		&\leqslant& \sup_{i} \norm{ae_i} \norm{\rho(x)e} \\
		&\leqslant& \norm{a} \norm{\rho(x)e}
	\end{eqnarray*}
	i.e. $\norm{\tilde{\rho}(a)} \leqslant \norm{a}$, $\tilde{\rho}(a)$ can extend to $\Hs$. Then, we can check that $\tilde{\rho} \colon \A \sto \oper$ is indeed a representation. \\
	If $T \in \oper$ s.t. for any $x \in \I$, $T\rho(x) = \rho(ax)$, then
	\begin{equation*}
		T\rho(x)h = \tilde{\rho}\rho(x)h,~~ \forall~ x \in \I ~\&~ \forall~ h \in \Hs
	\end{equation*}
	Since $\rho$ is non-degenerate, $T=\tilde{\rho}(a)$.\\
	Now, if $\rho$ is degenerate, let $\Hs_0 = \clo{\rho{\I}}$, and $\rho_0(x) = \rho(x)|_{\Hs_0}$, then $\rho_0 \colon \I \sto \fml{B}(\Hs_0)$ is a non-degenerate representation. Therefore, it can extend to $\tilde{\rho_0} \colon \A \sto \fml{B}(\Hs_0)$. Let $\kappa \colon \A \sto \fml{B}(\Hs_0^{\bot})$ be any representation. Then $\tilde{\rho} = \tilde{\rho_0} \oplus \kappa$ is the extension.
\end{proof}

\begin{cor}
	If $\A$ is a \Cs and $\I$ is an ideal of $\A$, and $(\rho,\Hs)$ and $(\kappa,\fml{K})$ are two representations of $\A$ s.t.
	\begin{enumerate}[label=\arabic*)]
		\item $\rho|_{\I}$ is a non-degenerate representation of $\I$.
		\item $\rho|_{\I} \cong \kappa|_{\I}$.
	\end{enumerate}
	then $\rho \cong \kappa$
\end{cor}


We know the irreducible representation play a important role in the representation theory. Therefore, we want to find the necessary and sufficient conditions make a representation be irreducible.

\begin{defn}
	If $\A \subset \oper$ is a $\st{C}$-subalgebra, the commutant of $\A$
	\begin{equation*}
		\Ac = \{~T \in \oper \colon TS=ST,~\forall~S \in \A~\}
	\end{equation*}
\end{defn}
\begin{rem}
	Clearly, $\Ac$ is a $\st{C}$-algebra.
\end{rem}

\begin{thm}
	Let $\A$ be a \Cs and $(\rho,\Hs)$ be a representation, the following statements are equivalent. Then $\rho$ is irreducible if and only if ${\rho(\A)}^{'} = \C$
\end{thm}
\begin{proof}
	If $\rho$ is irreducible and $\C \neq {\rho(\A)}^{'}$, then there is a $T \in {\rho(\A)}^{'} \backslash \C$. Therefore, there is a proper projection $P \in {\rho(\A)}^{'} \backslash \C$. Then $P\Hs$ is a proper invariant space for $\rho$, which is a contradiction. The converse is clearly true by using similar prove as above.
\end{proof}
\begin{rem}
	The existence of $P$ is because that ${\rho(\A)}^{'}$ is a von Neumann algebra.
\end{rem}

Firstly, by the GNS construction, the representations generated by positive linear functionals are important. So we need more properties of these representations.

\begin{prop}
	Let $\A$ be a \Cs and $\psi$ and $\phi$ be two positive linear functionals on $\A$ and $(\pi,\Hs,e)$ be the representation generated by $\phi$. Then $\psi \leqslant \phi$ if and only if there is unique a $T \in {\pi(\A)}^{'}$ with $0 \leqslant T \leqslant 1$, s.t. $\psi(a) = \langle \pi(a)Te, e \rangle$ for all $a \in \A$. 
\end{prop}
\begin{proof}
	Assume there is a $T \in {\pi(\A)}^{'}$ with $0 \leqslant T \leqslant 1$. If $a \in \A_{+}$, then
	\begin{equation*}
		0 \leqslant T^{\frac{1}{2}}\pi(a)T^{\frac{1}{2}} = \pi(a)T = \pi(a)^{\frac{1}{2}}T\pi(a)^{\frac{1}{2}} \leqslant \pi(a)
	\end{equation*}
	Therefore, $\psi(a) \leqslant \phi(a)$. 
	\item Conversely, we define a sesquilinear form on $\pi(\A)e$,
	\begin{equation*}
		f(\pi(a)e, \pi(b)e) = \psi(\st{b}a)
	\end{equation*}
	And since $\psi \leqslant \phi$, by CBS Inequality, we have
	\begin{eqnarray*}
		\abs{f(\pi(a)e, \pi(b)e)} &=& \abs{\psi(\st{b}a)}^2 \\
		&\leqslant& \psi(\st{a}a)\psi(\st{b}b) \leqslant \phi(\st{a}a)\phi(\st{b}b) \\
		&=& \norm{\pi(a)e}^2\norm{\pi(b)e}^2
	\end{eqnarray*}
	Therefore, $f$ is bounded and can be extended to $\Hs = \clo{\pi(\A)e}$. By \textbf{Theorem} \ref{thm7} in the subsection \textbf{1.5.1}, we have $T \in \oper$, s.t.
	\begin{equation*}
		\psi(\st{b}a) = \langle T\pi(a)e, \pi(b)e \rangle
	\end{equation*}
	In particular, $\psi(a) = \langle T\pi(a)e, e \rangle$.
	For $a,b,c \in \A$,
	\begin{eqnarray*}
		\langle T\pi(a)\pi(b)e, \pi(c)e \rangle &=& \langle T\pi(ab)e, \pi(c)e \rangle = \psi(\st{c}(ab)) \\
		&=& \psi(\st{(\st{a}c)}b) = \langle T\pi(b)e, \pi(\st{a}c)e \rangle \\
		&=& \langle \pi(a)T\pi(b)e, \pi(c)e \rangle
	\end{eqnarray*}
	And by the fact that $\Hs = \clo{\pi(\A)e}$, $T\pi(a)=\pi(a)T$ i.e. $T \in {\pi(\A)}^{'}$. And the uniqueness of $T$ can be easily checked.
\end{proof}

\begin{prop}
	Let $(\pi_j,\Hs_j,e_j)$ for $j=1,2$ be two cyclic representations of a \Cs $\A$. Then $\pi_1 \cong \pi_2$ by a unitary $U \colon \Hs_1 \sto \Hs_2$ satisfying $Ue_1=e_2$ if and only if for any $a \in \A$
	\begin{equation*}
		\langle \pi_1(a)e_1, e_1 \rangle = \langle \pi_2(a)e_2, e_2 \rangle
	\end{equation*}
\end{prop}
\begin{proof}
	We define for any $a \in \A$
	\begin{center}
		\begin{tabular}{l c c l}
			$U \colon$ & $\pi_1(\A)e_1$ & $\longrightarrow$ & $\pi_2(\A)e_2$ \\
			~ & $\pi_1(a)e_1$ & $\longmapsto$ & $\pi_2(a)e_2$
		\end{tabular}
	\end{center}
	If $\langle \pi_1(a)e_1, e_1 \rangle = \langle \pi_2(a)e_2, e_2 \rangle$, 
	\begin{equation*}
		\norm{\pi_2(a)e_2}^2 = \langle \pi_2(a)e_2, \pi_2(a)e_2 \rangle = \langle \pi_1(a)e_1, \pi_1(a)e_1 \rangle = \norm{\pi_1(a)e_1}^2
	\end{equation*}
	$U$ is a unitary on $\pi_1(\A)e_1$ and $U$ can extend to $\Hs_1$. And since
	\begin{equation*}
		U(\pi_1(a)\pi_1(b)e_1) = \pi_2(ab)e_2 = \pi_2(a)\pi_2(b)e_2=\pi_2(a)U(\pi_1(b)e_1)
	\end{equation*}
	$U$ is the unitary make $\pi_1 \cong \pi_2$. The converse is trivial.
\end{proof}
\begin{rem}
	Using this proposition, we can strengthen the result of the \textbf{Corollary} \ref{cor6} in the subsection \textbf{2.2.6}. In fact, every state on an closed ideal $\I$ of a \Cs $\A$ can extend to a unique state on $\A$. 
	If there are two extensions, they can generate two cyclic representations. And these two representations are equivalent on the ideal, thus these two representations are equivalent. And by above proposition, these two extensions are equal.
\end{rem}

By above two propositions, we can get the relationship between positive functionals and the generated representations.

\begin{cor}
	Let $\A$ be a \Cs and $\psi$ and $\phi$ be two positive linear functionals on $\A$ with $\psi \leqslant \phi$. Then $\pi_{\psi}$ is equivalent to a subrepresentation of $\pi_{\phi}$.
\end{cor}
\begin{proof}
	By above proposition, there is a $T \in {\pi(\A)}^{'}$ s.t. 
	\begin{equation*}
		\psi(a) = \langle \pi_{\phi}(a)Te_{\phi}, e_{\phi} \rangle
	\end{equation*}
	Put $\M = \clo{\pi_{\phi}(\A)T^{\frac{1}{2}}e_{\phi}}$. Clearly, $\M$ reduces $\pi_{\phi}(\A)$. Since
	\begin{equation*}
		\langle \pi_{\psi}(a)e_{\psi}, e_{\psi} \rangle = \psi(a) = \langle \pi_{\phi}(a)T^{\frac{1}{2}}e_{\phi}, T^{\frac{1}{2}}e_{\phi} \rangle
	\end{equation*}
	by above proposition $\pi_{\psi} \cong \pi_{\phi}|_{\M}$
\end{proof}

\begin{defn}
	A state $\phi \in \St_{\A}$ on some \Cs $\A$ is called pure if for any positive linear functional $\psi$ on $\A$ with $\psi \leqslant \phi$, there is a sclalar $\lambda \in [0,1]$ s.t. $\psi = \lambda \phi$.
\end{defn}

\begin{thm}
	Let $\A$ be a \Cs and $(\rho,\Hs)$ be a representation, the following statements are equivalent. Then $\rho$ is irreducible if and only if $\rho$ is equivalent to some cyclic representation generated by a pure state.
\end{thm}
\begin{proof}
	Suppose $\rho$ is irreducible. Then for any nonzero unit vector $h \in \Hs$, $\clo{\rho(\A)h} = \Hs$. Define 
	\begin{center}
		\begin{tabular}{l c c l}
			$\phi \colon$ & $\A$ & $\longrightarrow$ & $\C$ \\
			~ & $a$ & $\longmapsto$ & $\langle \rho(a)h,h \rangle$
		\end{tabular}
	\end{center}
	Then $\phi$ is clearly a state. If $(\pi,\fml{K},e)$ is the representation generated by $\phi$, then
	\begin{equation*}
		\langle \pi(a)e,e \rangle = \langle \rho(a)h,h \rangle
	\end{equation*}
	Thus, by above proposition, $\pi \cong \rho$. \\
	If there is a positive linear functional $\psi \leqslant \phi$, then
	\begin{equation*}
		\exists~ T \in {\pi(\A)}^{'},\text{ s.t. } \psi(a) = \langle \pi(a)Te,e \rangle
	\end{equation*}
	But by above theorem, ${\pi(\A)}^{'} = \C$, therefore $T = \lambda \in \C$, $\psi = \lambda \phi$.
	\item Conversely, if $(\rho,\Hs,e) \cong (\pi_{\phi}, \Hs_{\phi},e_{\phi})$ for some pure state $\phi$, 
	\begin{equation*}
		\phi(a) = \langle \rho(a)e, e \rangle =\langle \pi_{\phi}e_{\phi}, e_{\phi} \rangle
	\end{equation*}
	Then if a projection $P \in {\rho(\A)}^{'}$, then
	\begin{equation*}
		\psi(a) = \langle \rho(a)Pe, Pe \rangle = \langle \rho(a)Pe, e \rangle
	\end{equation*}
	is a positive linear functional and $\psi \leqslant \phi$, therefore $\psi = \lambda \phi$ for some $\lambda \in [0,1]$, i.e.
	\begin{equation*}
		\langle \rho(a)Pe, e \rangle = \langle Pe, \rho(\st{a})e \rangle =\lambda\langle \rho(a)e, e \rangle = \langle \lambda e, \rho(\st{a})e \rangle
	\end{equation*}
	Since $\Hs = \clo{\rho(\A)e}$, $P = \lambda$, thus $P=1$ or $0$. Therefore $\rho$ has no proper invariant space.
\end{proof} 

\begin{prop}
	Let $\A$ be a \Cs and $\phi \in \St_{\A}$. Then $\phi$ is pure if and only if $\phi$ is the extreme point of  $\St_{\A}$.
\end{prop}
\begin{rem}
	This proposition can be obtained by definition. In fact, we usually use the extreme point to define the pure state.
\end{rem}

We have similar results for pure states as general states. Firstly, we need a lemma.

\begin{prop}
	Let $\A$ be a \Cs and $\I$ be a closed ideal of $\A$. 
	\begin{enumerate}[label=\arabic*)]
		\item If $(\rho, \Hs)$ is an irreducible representation of $\I$, then it can extend to a unique irreducible representation $(\tilde{\rho},\Hs)$ on $\A$.
		\item If $(\rho, \Hs)$ is an irreducible representation of $\A$, then $(\rho|_{\I},\Hs)$ is an irreducible representation of $\I$.
	\end{enumerate}
\end{prop}
\begin{proof}
	For $1)$, since $\rho$ is irreducible, $\rho$ is non-degenerate. By the \textbf{Theorem} \ref{thm8} in this subsection, there is a unique extension $\tilde{\rho}$ of $\A$. And clearly, 
	\begin{equation*}
		\C \subset {\tilde{\rho}(\A)}^{'} \subset {\rho(\I)}^{'} = \C
	\end{equation*} 
	${\tilde{\rho}(\A)}^{'} = \C$, i.e. $\tilde{\rho}$ is irreducible.
	\item For $2)$, $\rho$ can be generated by a pure state $\phi$ on $\A$. Then $\phi|_{\I}$ is a pure state on $\I$ since if $\phi|_{\I} = s\psi_1 + (1-s)\psi_2$ for some $\psi_1 ,\psi_2 \in \St_{\I}$ and $s \in (0,1)$, then $\psi_1, \psi_2$ have extension $\psi_1^{'}, \psi_2^{'}$ on $\A$, $\phi = s\psi_1^{'} + (1-s)\psi_2^{'}$ by the uniqueness of extensions, which is a contradiction. Therefore, $\phi|_{\I}$ can generate an irreducible representation $\kappa$ on $\I$ and $\kappa \cong \rho|_{\I}$.
\end{proof}

\begin{prop}
	Let $\B$ be a $\st{C}$-subalgebra of $\A$. Then every pure state on $\B$ can extend to a pure state on $\A$.
\end{prop}
\begin{proof}
	Define $T \colon \St_{\A} \sto \St_{\B}$ by $\phi \sto \phi|_{\B}$. Then $T$ is surjective, affine and $\st{wk}$-continuous. Therefore, $T$ maps extreme points to extreme points. Thus the proposition holds.
\end{proof}

\begin{cor}
	Let $\A$ be a \Cs and $a \in \A$. Then
	\begin{equation*}
		\norm{a}^2 = \sup{\{~\phi(\st{a}a) \colon \phi \text{ is a pure state on } \A~\}}
	\end{equation*}
\end{cor}
\begin{proof}
	If $\A=C(X)$ for some compact space $X$, this corollary is true. And then by above proposition, any pure state on $\Cg{\st{a}a}$ can extend to a pure state on $\A$. Therefore, this corollary holds.
\end{proof}

\begin{thm}
	If $\A$ is a $\st{C}$-algebra, then there is a family of irreducible representations $\{\pi_i\}_{i\in I}$ s.t. $\oplus_{i \in I} \pi_{i}$ is a faithful representation of $\A$.
\end{thm}
\begin{proof}
	Let $\{\pi_i\}_{i\in I}$ be generated by all pure state. Then by above corollary, $\oplus_{i \in I} \pi_{i}$ is faithful.
\end{proof}

\section{Operators on Hilbert Spaces}

We have known that each $\st{C}$-algebra can be isometrically imbedded a $\oper$ for some Hilbert space $\Hs$. Therefore, we just research operator algebras other than $\st{C}$-algebras. Firstly, we need some general theories of operators on a Hilbert space.

\subsection{Spectrums for Operators}

By using the fact that $T \sto \st{T}$ in $\oper$ is bijective and isometrically antilinear, we have the following proposition.
\begin{prop}
	Let $\Hs$ be a Hilbert space and $T \in \oper$. Then
	\begin{equation*}
		\sigma(\st{T}) = \st{\sigma(T)} 
	\end{equation*}
\end{prop}

For self-adjoint operators in $\oper$, we have some extral properties related to the inner product.

\begin{prop}
	Let $\Hs$ be a Hilbert space and $T \in \oper$. Then $T$ is self-adjoint if and only if $\langle Th,h \rangle \in \R$ for any $h \in \Hs$.
\end{prop}
\begin{proof}
	If $\langle Th,h \rangle \in \R$ for any $h \in \Hs$, then for any $h,g \in \Hs$ and $\alpha \in \C$,
	\begin{eqnarray*}
		\langle T(h+\alpha g),h+\alpha g \rangle = \langle Th,h \rangle + \clo{\alpha}\langle Th,g \rangle + \alpha \langle Tg,h \rangle + \abs{\alpha}^2 \langle Tg,g \rangle \in \R
	\end{eqnarray*}
	Then by taking its conjugate,
	\begin{eqnarray*}
		\clo{\alpha}\langle Th,g \rangle + \alpha \langle Tg,h \rangle &=& \clo{\alpha} \langle h,Tg \rangle + \alpha \langle g,Th \rangle \\
		&=& \clo{\alpha} \langle \st{T}h,g \rangle + \alpha \langle \st{T}g,h \rangle
	\end{eqnarray*}
	Taking $\alpha = 1$ and $i$, we have
	\begin{eqnarray*}
		\langle Th,g \rangle + \langle Tg,h \rangle &=& \langle \st{T}h,g \rangle + \langle \st{T}g,h \rangle \\
		i\langle Th,g \rangle - i\langle Tg,h \rangle &=& -i\langle \st{T}h,g \rangle + i\langle \st{T}g,h \rangle
	\end{eqnarray*}
	Therfore, $\langle Th,g \rangle=\langle \st{T}g,h \rangle$ for any $h,g \in \Hs$.\\
	The converse is clearly true by definition.
\end{proof}

\begin{prop} \label{prop9}
	Let $\Hs$ be a Hilbert space and $T \in \oper$ be self-adjoint. Then
	\begin{equation*}
		\sigma(T) \subset \{~\langle Th,h \rangle \colon \norm{h} = 1~\}
	\end{equation*}
\end{prop}
\begin{proof}
	If $\lambda \notin \{~\langle Th,h \rangle \colon \norm{h} = 1~\}$, then there is a $\varepsilon > 0$, s.t.
	\begin{equation*}
		\varepsilon < \abs{\lambda - \langle Th,h \rangle} \leqslant \norm{Th},~~\forall~ h \in \Hs,~\norm{h}=1
	\end{equation*}
	Therefore, by the \textbf{Lemma} \ref{lem1} in the subsection \textbf{1.4.4},
	\begin{equation*}
		\ker{(T - \lambda)}=\{0\} ~\&~ \ran{(T - \lambda)} \text{ is closed}
	\end{equation*}
	Since $T$ is self-adjoint, $\clo{\lambda} \notin \sigma(T)$, so $\ker{(\clo{\lambda} - T)} = \{0\}$. Thus 
	\begin{equation*}
		\ran{(T - \lambda)}^{\bot} = \ker{(T - \clo{\lambda})} = \{0\}
	\end{equation*}
	i.e. $\ran{(T - \lambda)} = \Hs$. Then by the Inverse Mapping Theorem, $(T - \lambda)^{-1} \in \oper$, i.e. $\lambda \notin \sigma(a)$.
\end{proof}

\begin{cor} \label{cor8}
	Let $\Hs$ be a Hilbert space and $T \in \oper$ be self-adjoint.
	\begin{enumerate}[label=\arabic*)]
		\item $\norm{T} = \sup{\{\langle Th,h \rangle \colon \norm{h} = 1\}}$.
		\item If $\langle Th,h \rangle = 0$ for any $h \in \Hs$, then $T=0$.
	\end{enumerate}
\end{cor}
\begin{proof}
	For $1)$, 
	\begin{equation*}
		\norm{T} = r(T) \leqslant \sup{\{\langle Th,h \rangle \colon \norm{h} = 1\}} \leqslant \norm{T}
	\end{equation*}
	$2)$ is the direct result of $1)$.
\end{proof}
\begin{rem}
	In fact, for any $T \in \oper$ with $\langle Th,h \rangle = 0$ for any $h \in \Hs$, then $T = 0$ since any $T$ can be the linear combination of two self-adjoint operators.
\end{rem}

By above proposition, we see the spectrum of operators can be described by using another method. Firstly, we can generize the consequence in \textbf{Proposition} \ref{prop9}.
\begin{prop}
	Let $\Hs$ be a Hilbert space and $T \in \oper$. Then let 
	\begin{equation*}
		\chi(\lambda) = \inf{\{~\norm{(T-\lambda)h} \colon \norm{h} = 1~\}}
	\end{equation*}
	\begin{equation*}
		\sigma_l(T) =\{~\lambda \in \C \colon \chi(\lambda) = 0~\}
	\end{equation*}
\end{prop}
\begin{proof}
	By above proposition, we have seen if $\chi(\lambda) > 0$, 
	\begin{equation*}
		\ker{(T - \lambda)}=\{0\} ~\&~ \ran{(T - \lambda)} \text{ is closed}
	\end{equation*}
	Then let $\fml{K} = \ran{(T - \lambda)}$, then $T - \lambda \colon \Hs \sto \fml{K}$ is a bijection and\\ $(T-\lambda)^{-1} \colon \fml{K} \sto \Hs$ is bounded. Let
	\begin{equation*}
		B = (T-\lambda)^{-1}P \colon \Hs \sto \Hs
	\end{equation*}
	where $P$ is a projection to $\fml{K}$. Then $B \in \oper$ and $B(T - \lambda) = 1$. Therefore, $\sigma_l(T) \subset \{~\lambda \in \C \colon \chi(\lambda) = 0~\}$.
	\item Conversely, if there is a $B \in \oper$ s.t. $B(T-\lambda)=1$, then
	for $h \in \Hs$ with $\norm{h} = 1$
	\begin{equation*}
		1 = \norm{h} = \norm{B(T-\lambda)h} \leqslant \norm{B}\norm{(T-\lambda)h}
	\end{equation*}
	Therefore, $\norm{(T-\lambda)h} \geqslant \norm{B}^{-1}$, i.e. $\chi(\lambda) > 0$.
\end{proof}

\begin{defn}
	Let $\Hs$ be a Hilbert space and $T \in \oper$.
	\begin{enumerate}[label=\arabic*)]
		\item The point spectrum of $T$ is defined as
		\begin{equation*}
			\sigma_p(T) = \{~ \lambda \in \C \colon \ker{(T-\lambda)} \neq \{0\}~\}
		\end{equation*}
		\item The approximate point spectrum of $T$ is defined as 
		\begin{equation*}
			\sigma_{ap}(T) =\{~\lambda \in \C \colon \exists~\text{units }{x_n} \subset \oper ~\text{s.t.}~ \norm{(T-\lambda)x_n} \sto 0~\}
		\end{equation*}
	\end{enumerate}
\end{defn}

\begin{prop}
	Let $\Hs$ be a Hilbert space and $T \in \oper$.
	\begin{enumerate}
		\item $\lambda \notin \sigma_{ap}(T)$ if and only if $\ker{(T - \lambda)}=\{0\}$ and $\ran{(T - \lambda)}$ is closed.
		\item $\partial \sigma(T) \subset \sigma_{ap}(T)$.
		\item $\partial \sigma(T) \subset \sigma_l(T) \cap \sigma_r(T) = \sigma_{ap}(T) \cap \sigma_{ap}(\st{T})$.
	\end{enumerate}
\end{prop}
\begin{proof}
	For $1)$, if $\lambda \notin \sigma_{ap}(T)$, then there is a constant $C > 0$ s.t.
	\begin{equation*}
		\norm{(T-\lambda)x} \geqslant C\norm{x},~~\forall~x \in \Hs
	\end{equation*}
	Thus by above proposition, $\ker{(T - \lambda)}=\{0\}$ and $\ran{(T - \lambda)}$ is closed. Conversely, the condition that means $(T-\lambda)$ is left invertible, thus 
	\begin{equation*}
		\norm{(T-\lambda)x} \geqslant \norm{B}^{-1}\norm{x},~~\forall~ x \in \Hs,~\norm{x}=1
	\end{equation*}
	where $B$ is the left inverse of $T$. Therefore, $\lambda \notin \sigma_{ap}(T)$.
	\item For $2)$, let $\lambda \in \partial \sigma(T)$ and $\{\lambda_n\} \subset \rho(T)$ s.t. $\lambda_n \sto \lambda$. Then we know $\norm{(T-\lambda_n)^{-1}} \sto \infty$. Otherwise, if $\norm{(T-\lambda_n)^{-1}} < M$ and $\abs{\lambda_n-\lambda} < M^{-1}$ for sufficiently large $n$,
	\begin{equation*}
		\norm{(T-\lambda_n)-(T-\lambda)} < \norm{(T-\lambda_n)^{-1}}^{-1}
	\end{equation*}
	by the perturbation of inverse, $T-\lambda$ is also invertible. Let $\{x_n\}$ by any sequence of unit vectors in $\Hs$. Define
	\begin{equation}
		y_n = \frac{(T-\lambda_n)^{-1}x_n}{\norm{(T-\lambda_n)^{-1}x_n}}
	\end{equation}
	Then $\norm{(T-\lambda)y_n} \sto 0$, i.e. $\lambda \in \sigma_{ap}(T)$.
	\item For $3)$, by $1)$ we have known $\sigma_l(T)=\sigma_{ap}(T)$ and by taking the complex conjugate, $\sigma_r(\st{T})=\sigma_{ap}(\st{T})$. Therefore, $3)$ holds.
\end{proof}

Now, we can prove the inverse of the \textbf{Corollary} \ref{cor7} in the subsection \textbf{2.2.3}.

\begin{thm}
	Let $\Hs$ be a Hilbert space and $T \in \oper$. If for any $h \in \Hs$, 
	\begin{equation*}
		\langle Th, h \rangle \geqslant 0
	\end{equation*}
	then $T$ is positive.
\end{thm}
\begin{proof}
	Firstly, $T$ is self-adjoint by above proposition. And for $\lambda < 0$ and $\norm{h} =1$
	\begin{eqnarray*}
		\norm{(T-\lambda)h} &=& \norm{Th}^2 - 2\lambda \langle Th,h \rangle + \lambda^2\norm{h}^2 \\
		&\geqslant& - 2\lambda \langle Th,h \rangle + \lambda^2\norm{h}^2 \\
		&\geqslant& \lambda^2\norm{h}^2
	\end{eqnarray*}
	Therefore, $\lambda \notin \sigma_l(T)$. Since $T$ is self-adjoint, $\lambda \notin \sigma_r(T)$. Thus  $\lambda \notin \sigma(T)$, i.e. $\sigma(T) \subset [0,\infty)$.
\end{proof}

\subsection{Polar Decomposition}

For a $z \in \C$, it can be decomposed as $z = \abs{z}e^{i\theta}$. Similarly, a operator in $\oper$ can also be decomposed as a complex number does. Firstly, we need to find the element playing same role as $e^{i\theta}$.

\begin{defn}
	Let $\Hs$ be a Hilbert space and $U \in \oper$ and $\M$ be a closed subspace of $\Hs$. If $U|_{\M}$ is an isometry and $U|_{\M^{\bot}} = 0$, then $U$ is called a partial isometry. $\M$ is called the initial space and $\fml{N}=U(\M)$ is called the final space.
\end{defn}

\begin{prop}
	If $U$ is a partial isometry, then $\st{U}U$ is the projection on the initial space $\M$ and $U\st{U}$ is the projection on the final space $\fml{N}$. Moreover, $\st{U}$ is the partial isometry with the initial space $\fml{N}$ and final space $\M$.
\end{prop}
\begin{proof}
	For any $g,h \in \Hs$, then $g = g_{\shortparallel} + g_{\bot}$ and $h = h_{\shortparallel} + h_{\bot}$, where $g_{\shortparallel},h_{\shortparallel} \in \M$ and $g_{\bot},h_{\bot} \in \M^{\bot}$
	\begin{eqnarray*}
		\langle \st{U}U(g_{\shortparallel} + g_{\bot}), h_{\shortparallel} + h_{\bot} \rangle &=& \langle Ug_{\shortparallel},Uh_{\shortparallel} \rangle = \langle g_{\shortparallel},h_{\shortparallel} \rangle \\
		&=& \langle P_{\M}(g_{\shortparallel} + g_{\bot}),h_{\shortparallel} + h_{\bot} \rangle
	\end{eqnarray*}
	Then $\st{U}U = P_{\M}$. $P_{\M^{\bot}} = 1-\st{U}U$ implies $U(1-\st{U}U) = 0$, i.e. $U = U\st{U}U$ and $\ker{\st{U}} = (\ran{U})^{\bot} = \fml{N}^{\bot}$, therefore, $U\st{U}$ is the projection on $\fml{N}$. For $x \in \fml{N}$.
	\begin{equation}
		\langle \st{U}x, \st{U}x \rangle = \langle U\st{U}x, x \rangle = \langle x, x \rangle
	\end{equation}
	And $\ran{\st{U}} = (\ker{U})^{\bot} = \M$, thus $\st{U}$ is a partial isometry with the initial space $\fml{N}$ and the final space $\M$.
\end{proof}

\begin{cor}
	If $U \in \oper$, then the following statements are equivalent.
	\begin{enumerate}[label=\arabic*)]
		\item $U$ is a partial isometry.
		\item $\st{U}U$ is a projection.
		\item $U = U\st{U}U$.
		\item $\st{U}$ is a partial isometry.
		\item $U\st{U}$ is a projection.
		\item $\st{U} = \st{U}U\st{U}$.
	\end{enumerate}
\end{cor}

\begin{thm}[Polar Decomposition]
	Let $\Hs$ be a Hile]bert space and $T \in \oper$. Then there is a partial isometry $U \in \oper$ with the initial space $(\ker{T})^{\bot}$ and the final space $\clo{\ran{T}}$, s.t. $T = U\abs{T}$. Moreover, if $T=WS$, where $S \geqslant 0$ and $W$ is a partial isometry with $\ker{W}=\ker{S}$, then $S = \abs{T}$ and $W=U$.
\end{thm}
\begin{proof}
	Define 
	\begin{center}
		\begin{tabular}{l c c l}
			$U \colon$ & $\ran{\abs{T}}$ & $\longrightarrow$ & $\ran{T}$ \\
			~ & $\abs{T}h$ & $\longmapsto$ & $Th$
		\end{tabular}
	\end{center}
	Since 
	\begin{equation*}
		\norm{Th}^2=\langle Th,Th \rangle = \langle \st{T}Th,h \rangle = \langle \abs{T}h,\abs{T}h \rangle = \norm{\abs{T}h}^2
	\end{equation*}
	$U$ is a isometry. And thus $U$ can extend to $\clo{\ran{\abs{T}}}$ and onto $\clo{\ran{T}}$. Then extending $U$ to $\Hs$ by setting $U|_{\clo{\ran{\abs{T}}}^{\bot}} = 0$. $U$ is a partial isometry with the initial space $\ker{\abs{T}}^{\bot}$ and the final space $\clo{\ran{T}}$. Clearly, $\ker{\abs{T}} = \ker{T}$ and $T=U\abs{T}$.\\
	If $T=WS$, then $\st{T}T=S\st{W}WS = SP_{\ker{S}^{\bot}}S = S^2$. Therefore, $S = \abs{T}$. And $W\abs{T}=U\abs{T}$, thus $W$ ans $U$ has same initial space, then $W=U$.
\end{proof}

\subsection{Topologies on \texorpdfstring{$\oper$}{Operator}}

In $\oper$, the norm topology may be too stronger, which means norm closed subalgebras in $\oper$ do not contain "enough" elements. Therefore, we need some weaker topologies

\begin{defn}
	Let $\Hs$ be a Hilbert space. 
	\begin{enumerate}[label=\arabic*)]
		\item The strong operator topology on $\oper$, denoted by $SOT$, is generated by a family of seminorms $\{p_x\}_{x \in \Hs}$, where $p_x(T) = \norm{Tx}$.
		\item The weak operator topology on $\oper$, denoted by $WOT$, is denerated by a fanmily of seminorms $\{p_{gh}\}_{g,h \in \Hs}$, where $p_{gh}(T) = \abs{\langle Tg,h \rangle}$.
	\end{enumerate}
\end{defn}
\begin{rem}
	Firstly, by definition and the \textbf{Corollary} \ref{cor8} in the subsection \textbf{2.3.1}, we can see 
	\begin{equation*}
		\bigcap_{x \in \Hs} \ker{p_x}=\{0\} ~\&~ \bigcap_{g,h \in \Hs} \ker{p_{gh}}=\{0\}
	\end{equation*}
	Therefore, both $SOT$ and $WOT$ can let $\oper$ be a Hausdorff locally convex t.v.s.. And by the CBS Inequatlity, $SOT$ is stronger than $WOT$. Moreover,
	\begin{enumerate}[label=\arabic*)]
		\item the subbasis of the $WOT$ at $T_0$ is like 
			\begin{equation*}
				U_{gh\varepsilon}(T_0) = \{T \in \oper \colon \abs{\langle (T-T_0)g,h \rangle} < \varepsilon\}
			\end{equation*}
		Therefore, a net $\{T_\alpha\}$ in $\oper$ converges in $WOT$ to $T_0$ if and only if $\langle T_{\alpha}g,h \rangle \sto \langle T_0g,h \rangle $ for all $g,h \in \Hs$.
		\item the subbasis of the $SOT$ at $T_0$ is like 
			\begin{equation*}
				V_{x\varepsilon}(T_0) = \{T \in \oper \colon \norm{(T-T_0)x} < \varepsilon\}
			\end{equation*}
		Therefore, a net $\{T_\alpha\}$ in $\oper$ converges in $TOT$ to $T_0$ if and only if $\norm{T_{\alpha}x} \sto \norm{T_0x}$ for all $x \in \Hs$.
	\end{enumerate}
	Comparing with the norm topology, the $SOT$ is related to the pointwise convergence, where the norm topology is related to the uniform convergence.
\end{rem}

\begin{prop} \label{prop13}
	Let $\Hs$ be a Hilbert space and $L$ is a linear functional on $\oper$. Then the following statements are equivalent.
	\begin{enumerate}[label=\arabic*)]
		\item $L$ is $SOT$-continuous.
		\item $L$ is $WOT$-continuous.
		\item There are vectors $f_1,\cdots,f_n,g_1,\cdots,g_n \in \Hs$ s.t.
		\begin{equation*}
			L(T) = \sum_{i=1}^{n} \langle Tg_i,f_i \rangle,~~\forall~T \in \oper
		\end{equation*}
	\end{enumerate}
\end{prop}
\begin{proof}
	Clearly, $3) \Rightarrow 2) \Rightarrow 1)$, thus assume $1)$, then by the \textbf{Theorem} \ref{thm2} in the subsection \textbf{1.2.4}, there are $g_1,\cdots,g_n \in \Hs$ and $\alpha_1,\cdots,\alpha_n > 0$ s.t.
	\begin{equation*}
		\norm{L(T)} \leqslant \sum_{i=1}^{n} \alpha_i \norm{Tg_i}
	\end{equation*}
	Replacing $g_i$ by $\alpha_i g_i$ and $g_i$ by $\sqrt{n} g_i$ then
	\begin{equation*}
		\norm{L(T)} \leqslant \sum_{i=1}^{n} \norm{Tg_i} \leqslant \sqrt{n}(\sum_{i=1}^{n} \norm{Tg_i}^2)^{\frac{1}{2}} = (\sum_{i=1}^{n} \norm{Tg_i}^2)^{\frac{1}{2}}
	\end{equation*}
	Let $\fml{K} = \clo{\{\oplus_{i=1}^{n}Tg_i \colon T \in \oper\}}$ be a closed subspace of $\Hs^{(n)}$. 
	Define a linear functional $F$ on $\fml{K}$ by $F(\oplus_{i=1}^{n}Tg_i) = L(T)$, since
	\begin{equation*}
		\abs{F(\oplus_{i=1}^{n}Tg_i)} = \norm{L(T)} \leqslant (\sum_{i=1}^{n} \norm{Tg_i}^2)^{\frac{1}{2}}
	\end{equation*}
	Therefore, $F$ can extends to $\Hs^{(n)}$ with the conincided inner product and norm. By the Riesz Theorem, there is a $f_1, \cdots, f_n \in \Hs$ s.t.
	\begin{equation*}
		F(\oplus_{i=1}^{n}h_i) = \langle \oplus_{i=1}^{n}h_i, \oplus_{i=1}^{n}f_i \rangle,~~\forall~\oplus_{i=1}^{n}h_i \in \Hs^{(n)}
	\end{equation*}
	In particular, restricting $F$ to $\fml{K}$
	\begin{equation*}
		L(T) = F(\oplus_{i=1}^{n}Tg_i) = \langle \oplus_{i=1}^{n}Tg_i, \oplus_{i=1}^{n}f_i \rangle = \sum_{i=1}^{n} \langle Tg_i,f_i \rangle \qedhere
	\end{equation*}

\end{proof}

\begin{prop}
	If $S$ is a convex subset of $\oper$, then
	\begin{equation*}
		\clo{S}^{WOT} = \clo{S}^{SOT} 
	\end{equation*}
\end{prop}
\begin{proof}
	Clearly, 
	\begin{equation*}
		\fml{T}_{WOT} \subset \fml{T}_{SOT}
	\end{equation*}
	Therefore,
	\begin{equation*}
		\clo{S}^{SOT} \subset \clo{S}^{WOT}
	\end{equation*}
	Since for linear functionals, $SOT$-continuity implies $WOT$-continuity, 
	\begin{equation*}
		\clo{(S,WOT)}^{wk} \subset \clo{(S,SOT)}^{wk}
	\end{equation*}
	And we have known $\clo{(S,WOT)}^{wk} = \clo{(S,WOT)} = \clo{S}^{WOT}$ and $\clo{(S,SOT)}^{wk} = \clo{(S,SOT)} = \clo{S}^{SOT}$, therefore
	\begin{equation*}
		\clo{S}^{SOT} \subset \clo{S}^{WOT} \subset \clo{S}^{SOT} \qedhere
	\end{equation*}
\end{proof}

$WOT$ also have a similar property to $wk$-topology, the Alaoglu's Theorem.
\begin{thm}
	The closed unit ball in $\oper$ is $WOT$-compact.
\end{thm}
\begin{proof}
	Let $X_h$ be closed unit ball in $\Hs$ with the $wk$-topology and indexed with $h \in \Hs$ and $B$ be the closed unit ball in $\Hs$ with the $WOT$. Define
	\begin{center}
		\begin{tabular}{l c c l}
			$\tau \colon$ & $B$ & $\longrightarrow$ & $\prod\{X_h \colon \norm{h} \leqslant 1\}$ \\
			~ & $T$ & $\longmapsto$ & $\tau(T)$
		\end{tabular}
	\end{center}
	where $\tau(T)_h = Th$. Then using similarl method, we can prove $B$ is compact.
\end{proof}





