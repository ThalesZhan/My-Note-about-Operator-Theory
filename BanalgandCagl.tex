\chapter{Banach Algebras and \texorpdfstring{$C^{*}$-Algebras}{C*-Algebras}}
\section{Banach Algebras}

Let $\fml{H}$ be a Hilbert space. Then $\oper$ is indeed a Banach space. But we have more sructure on it. Any two element $S,T \in \oper$ can do multiplication, defined as $ST = S \circ T$, then $ST \in \oper$ by the definition of norm, and moreover $\norm{ST} \leqslant \norm{S}\norm{T}$. Therefore, $\oper$ is an algebra with the extra property of the multiplication, called a Banach algebra.

\subsection{Elementary Properties}

\begin{defn}
	A Banach algebra $\alg{A}$ is an algebra over $\C$ with a norm $\norm{\cdot}$ relative to which $\alg{A}$ is a Banach space and s.t. for all $a,b \in \alg{A}$,
	\begin{equation*}
		\norm{ab} \leqslant \norm{a}\norm{b}.
	\end{equation*}
\end{defn}
\begin{rem}
	The extra condition guarantees the multiplication is norm continuous. In fact, $\alg{A}$ with the multiplication and the norm is a topological semigroup.
\end{rem}

If $\alg{A}$ has an identity $1$, it assumes $\norm{1}=1$. But if $\alg{A}$ does not have an identity, we can add an identity to it.

\begin{prop}
	If $\alg{A}$ is a Banach algebra without the identity. The let $\tilde{\alg{A}}= \alg{A} \oplus \C$ be an induced vector space. Then we define a norm on $\tilde{\alg{A}}$ as
	\begin{equation*}
		\norm{(a,\lambda)} = \norm{a} + \abs{\lambda}
	\end{equation*}
	and define a multiplication on $\tilde{\alg{A}}$ as
	\begin{equation*}
		(a,\alpha)(b,\beta) = (ab+\alpha b + \beta a, \alpha \beta)
	\end{equation*}
	Then $\alg{A}$ is a Banach algebra with the identity $(0,1)$.
\end{prop}

Then $\alg{A}$ can be isometrically imbedded into $\tilde{\alg{A}}$. Therefore, we can always assume $\alg{A}$ has an identity. For a unit algebra, the invertibility of a element is important. 

\begin{thm}
	Let $\alg{A}$ be a Banach algebra and $a \in \alg{A}$. If $\norm{a-1} < 1$, then $a$ is invertible.
\end{thm}
\begin{proof}
	For a nonzero real number $\lambda \in \R$ with $\abs{\lambda} < 1$, we know 
	\begin{equation*}
		(1-\lambda)^{-1} = \sum_{n=0}^{\infty} \lambda^{n}
	\end{equation*}
	Similar, for $a \in \alg{A}$ with $\norm{a-1} < 1$, set
	\begin{equation*}
		b= \sum_{n=0}^{\infty} (1-a)^{n}
	\end{equation*}
	Firstly, since $\norm{(1-a)^{n}} \leqslant \norm{(1-a)}^{n}$, $b$ is well-defined. Then we can prove $b = (1-(1-a))^{-1} = a^{-1}$.
\end{proof}
\begin{rem}
	This result is important. It says a small perturbation of an invertible element is also invertible. It is because that the multiplication is norm continuous. And by the continuity of multiplication, this result can be true at any point other than 1.
\end{rem}

\begin{cor}
	Let $\alg{A}$ be a Banach algebra and 
	\begin{eqnarray*}
		G_l &=& \{~a \in \alg{A} \colon a \text{ is left invertible}~\}
		G_r &=& \{~a \in \alg{A} \colon a \text{ is right invertible}~\}
		G &=& G_l \bigcap G_r = \{~a \in \alg{A} \colon a \text{ is invertible}~\}
	\end{eqnarray*}
	Then $G_l$ and $G_r$ and $G$ are open. Moreover, the map $a \sto a^{-1}$ from $G$ to $G$ is continuous.
\end{cor}
\begin{proof}
	$G_l$ and $G_r$ and $G$ are open by above theorem. \\
	We just need to check this map is continuous at $1$ because of the continuity of multiplication. For $\{a_n\} \subset G$ with $a_n \sto 1$, thus $\norm{1-a_n}<\delta<1$. Since
	\begin{equation*}
		a_n^{-1} = (1-(1-a_n))^{-1} = \sum_{k=0}^{\infty} (1-a_n)^{k} = 1+ \sum_{k=1}^{\infty} (1-a_n)^{k}
	\end{equation*}
	Therefore,we have
	\begin{eqnarray*}
		\norm{1-a_n^{-1}} &=& \norm{\sum_{k=1}^{\infty} (1-a_n)^{k}}\\
		&\leqslant&  \sum_{k=1}^{\infty} \norm{1-a_n}^{k}
		&<& \frac{\delta}{1+\delta} < \delta = \norm{1-a_n}
	\end{eqnarray*}
	i.e. $\lim a_n = 1$.
\end{proof}

\begin{cor}
	Let $\alg{A}$ be a Banach algebra.
	\begin{enumerate}[label=\arabic*)]
		\item The closure of a proper ideal is a proper ideal.
		\item A maximal ideal is closed.
		\item every ideal contained in a maximal ideal.
	\end{enumerate}
\end{cor}

If $\alg{B}$ is a closed ideal of a Banach algebra $\alg{A}$, then then quotient algbra $\alg{A}/\alg{B}$ with the induced norm is also a Banach algebra since 
\begin{equation*}
	\norm{(a+\alg{B})(b+\alg{B})} = \norm{ab+\alg{B}} \leqslant \norm{(a+b_1)(b+b_2)} \leqslant \norm{(a+b_1)}\norm{(b+b_2)}
\end{equation*}
for any $b_1, b_2 \in \alg{B}$.

\subsection{Spectrum}

\begin{defn}
	Let $\alg{A}$ be a Banach algebra and $a \in \alg{A}$. The spectrum of $a$, denoted by $\sigma(a)$ defined as
	\begin{equation*}
		\sigma(a) = \{~ \lambda \in \C \colon a-\lambda \text{ is invertible}~\}
	\end{equation*}	
	And the resolvents of $a$, $\rho(a) = \C \backslash \sigma(a)$.\\
	Moreover, we can define the spectral radius of $a$ as
	\begin{equation*}
		r(a) = \sup{\{~\abs{\lambda} \colon \lambda \in \sigma(a)~\}}
	\end{equation*}
\end{defn}

Firstly, there are some elementary properties of the spectrum.

\begin{thm}
	Let $\alg{A}$ be a Banach algebra and $a \in \alg{A}$.
	\begin{enumerate}[label=\arabic*)]
		\item If $\abs{\lambda} > \norm{a}$, then $\lambda \notin \sigma(a)$.
		\item $\sigma(a)$ is a compact subset of $\C$.
		\item the map $\lambda \mapsto (a-\lambda)^{-1}$ from $\rho(a)$ to $\alg{A}$ is analytic and $\sigma(a)$ is nonempty.
		\item $r(a)=\lim_{n \sto \infty} \norm{a^{n}}^{\frac{1}{n}}$.
	\end{enumerate}
\end{thm}
\begin{proof}
	$1)$ holds by above theorem. \\
	For $2)$, since $\lambda \sto a-\lambda$ is continuous from $\C$ to $\alg{A}$ and $G$ is open, $\rho(a)$ is open i.e. $\sigma(a)=\C \backslash \rho(a)$ is closed. Then by $1)$, $\sigma(a)$ is compact.\\
	For $3)$, by the identity $a^{-1} - b^{-1} = a^{-1}(b-a)b^{-1}$ and the continuity of $a \sto a^{-1}$, we can compute the derivative of $F(\lambda) = (a-\lambda)^{-1}$,
	\begin{equation*}
		F^{'}(\lambda) = (a-\lambda)^{-2}
	\end{equation*}
	And clearly, $F^{'}(\lambda)$ is continuous. Thus it is analytic and it vanishes at $\infty$. By the Liouville's Theorem, if $\rho(a) = \C$, $F$ is constant. Therefore, $\rho(a) \notin \C$ i.e. $\sigma(a) \neq \varnothing$.\\
	For $4)$, let $U=\{\lambda \in \C \colon \lambda = 0 \text{ or } \lambda^{-1} \in \rho(a)\}$ and 
	\begin{equation*}
		f(\lambda) = 
		\begin{cases}
			(\lambda^{-1}-a)^{-1} & x \neq 0,\\
			0,& x = 0
		\end{cases}
	\end{equation*}
	Then $f$ is analytic on $U$, i.e $f(\lambda)=\lambda\sum_{n=0}^{\infty} \lambda^{n} a^{n}$ is well-defined. Therefore, the convergent radius $R = r(a)^{-1}$
	\begin{equation*}
		R^{-1} = \limsup_{n \sto \infty} \norm{a^{n}}^{\frac{1}{n}} = r(a)
	\end{equation*}
	Conversely, by the identity $(a^{n}-\lambda^{n}) = (a-\lambda)(a^{n-1}+\lambda a^{n-2}+\lambda^{2} a^{n-3} + \cdots + \lambda^{n-1})$. Then, if $(a^{n}-\lambda^{n})$ is invertible, then $(a-\lambda)$ is invertible, i.e. $\sigma(a) \subset \sigma(a^{n})$. Thus $\abs{\lambda}^{n} \leqslant \norm{a^{n}}$ for any $\lambda \in \sigma{a}$. $r(a)=\liminf_{n \sto \infty}\norm{a^{n}}^{\frac{1}{n}}$. Therefore, $r(a)=\lim_{n \sto \infty} \norm{a^{n}}^{\frac{1}{n}}$.
\end{proof}

If $\alg{B} \subset \alg{A}$ is a subalgebra with the same identity of a Banach algebra $\alg{A}$, then we know for any element $b \in \alg{B}$, $\sigma_{\alg{A}}(b) \subset \sigma_{\alg{B}}(b)$. Then we can have more results other than it. Since the spectrum is a subset of $\C$, we need some topological properties results of $\C$.

\begin{lem}
	If $K$ is any compact subset of $\C$, then $\C \backslash K$ has a countable components, only one of which is unbounded. And the boundary of each component is in $K$.
\end{lem}
\begin{proof}
	Let $\tilde{K} = \C \backslash K$, then $\tilde{K}$ is open.
	\item Firstly, the connected component of open set in $\C$ is open. \\
		  Let $U$ be an connected component in $\tilde{K}$ and $x \in U$. For any point $x \in U$, Since any open neighbourhood of $x$ is connected, and $K$ is open, there is a open neighbourhood $V$ of $x$ s.t. $V \subset U$.
	\item Secondly, $\C$ has just at most countable many open sets, which are pairwise disjoint.\\
		  This result is because any open set in $\C$ contains a rational point.
	\item For any two disjoint open sets $A$ and $B$ in $\C$, $\partial A \bigcap B = \varnothing$. Thus the boundary of some compoment of $\tilde{K}$ can not intersect in any component of $\tilde{K}$, i.e. it is contained $K$.
	\item Finally, since $K$ is bounded, there is a closed ball $B$ containing $K$. But the complement of $B$ is connected, thus there is only one component of $\tilde{K}$ containing $B$. Thus the other components of $\tilde{K}$ are bounded.
\end{proof}
\begin{rem}
	The bounded component of $\C \backslash K$ is called a hole of $K$.
\end{rem}

\begin{defn}
	If $f \colon A \sto \C$, where $A$ is a set, then the norm of $f$ on $A$ is defined as
	\begin{equation*}
		\norm{f}_A = \sup{\{\abs{f(x)} \colon x \in A\}}
	\end{equation*}
	For a compact set $K \in \C$, the polynomially convex hull of $K$ is defined as
	\begin{equation*}
		\hat{K} = \{~z \in \C \colon \abs{p(z)} \leqslant \norm{p}_K \text{ for any polynomial } p ~\}
	\end{equation*}
	If $K = \hat{K}$, $K$ is called polynomially convex.
\end{defn}

\begin{prop}
	Let $K$ be a compact subset of $\C$. Then $\C \backslash \hat{K}$ is the unbounded component of $\C \backslash K$. Therefore, $K$ is polynomially convex if and only if $\C \backslash K$ is connected.
\end{prop}
\begin{proof}
	Let $L$ be the set containing $K$ and all bounded component of $\C \backslash K$. Then by the Maximal Principle, $L \subset \hat{K}$. Conversely, if $\alpha \notin L$, then $(z-\alpha)^{-1}$ is analytic in a neighbourhood of $L$. Therefore, there is a sequence of polynomials $\{p_n\}$ s.t. $p_n \sto (z-\alpha)^{-1}$. Let $q_n=(z-\alpha)p_n$. Then $q_n \sto 1$, i.e. $\norm{q_n-1} < \frac{1}{2}$ for some $n$. But $\abs{q_n(\alpha)-1}=1$, this implies $\alpha \notin \hat{K}$, i.e. $\hat{K} \subset L$.
\end{proof}

By above results, now we can provide the relationships betweem $\sigma_{\alg{A}}(b)$ and $\sigma_{\alg{B}}(b)$.

\begin{thm}
	If $\alg{A}$ and $\alg{B}$ are Banach algebras with same identity s.t. $\alg{B} \subset \alg{A}$ and $b \in \alg{B}$, then
	\begin{enumerate}[label=\arabic*)]
		\item $\sigma_{\alg{A}}(b) \subset \sigma_{\alg{B}}(b)$ and $\partial\sigma_{\alg{B}}(b) \subset \partial\sigma_{\alg{A}}(b)$
		\item $\hat{\sigma_{\alg{A}}(b)} = \hat{\sigma_{\alg{B}}(b)}$
		\item if $G$ is a hole of $\sigma_{\alg{A}}(b)$, then $G \subset \sigma_{\alg{B}}(b)$ or $G \bigcap \sigma_{\alg{B}}(b) = \varnothing$
	\end{enumerate}
\end{thm}
\begin{proof}
	\item For $1)$, let $\lambda \in \partial\sigma_{\alg{B}}(b)$. Since $\inte{\sigma_{\alg{A}}(b)} \subset \inte{\sigma_{\alg{B}}(b)}$, it is sufficient to show $\lambda \in \sigma_{\alg{A}}(b)$. Suppose $\lambda \notin \sigma_{\alg{A}}(b)$, i.e. $(b-\lambda)$ is invertible in $\alg{A}$. But since $\lambda \in \partial\sigma_{\alg{B}}(b)$, there are $\lambda_n \in \C \backslash \alg{B}$ with $\lambda_n \sto \lambda$. Thus $(b-\lambda_n)^{-1} \in \alg{B}$. But $(b-\lambda_n)^{-1} \sto (b-\lambda)^{-1} \in \sigma_{\alg{B}}(b)$, contradicting to $\lambda \in \sigma_{\alg{A}}(b)$.
	\item $2)$ holds because of the result of $1)$ and the Maxiamal Principle.
	\item For $3)$, let $G_1 = G \bigcap \sigma_{\alg{B}}(b)$ and $G_2 = G \backslash \sigma_{\alg{B}}(b)$. Since $\partial\sigma_{\alg{B}}(b) \subset \sigma_{\alg{A}}(b)$ and $G \bigcap \sigma_{\alg{A}}(b) = \varnothing$, $G_1 = G \bigcap \inte{\sigma_{\alg{B}}(b)}$ is open. By the facts that $G_2$ is clearly open and $G = G_1 \bigcup G_2$ and $G_1 \bigcap G_2 = \varnothing$, either $G_1$ or $G_2$ is empty.
\end{proof}

Then we can have some useful corollaries.

\begin{cor}
	Let $\alg{A}$ and $\alg{B}$ be Banach algebras with same identity s.t. $\alg{B} \subset \alg{A}$ and $b \in \alg{B}$.
	\begin{enumerate}[label=\arabic*)]
		\item If $\sigma_{\alg{A}}(b)$ has no holes, then $\sigma_{\alg{A}}(b)=\sigma_{\alg{B}}(b)$.
		\item If $\sigma_{\alg{B}}(b) \subset \R$, then $\sigma_{\alg{A}}(b)=\sigma_{\alg{B}}(b)$.
		\item $\sigma_{\alg{A}}(b)=\sigma_{\alg{B}}(b)$ if and only if $\rho_{\alg{A}}(b)$ is connected.
	\end{enumerate}
\end{cor}
\begin{proof}
	$1)$ is clearly true since ubbouded component does not intersect $\sigma_{\alg{B}}(b)$. $2)$ is because $\C \backslash \sigma_{\alg{A}}(b)$ has no holes. $3)$ is similar as $2)$.
\end{proof}

\subsection{Riesz Functional Calculus}
 
For any polynomial $p$ with complex coefficients,  
\begin{equation*}
	p(z) = \sum_{k=0}^{n} \alpha_k z^{k}
\end{equation*}
we can define $p(a)$ for some $a \in \alg{A}$, where $\alg{A}$ is a Banach algebra
\begin{equation*}
	p(a) = \sum_{k=0}^{n} \alpha_k a^{k}
\end{equation*}
Clearly, $p(a)$ is well-defined. But we can do more. If $f$ is an analytic funcion on $A \subset \C$, then $f$ can be approximated by a sequence of polynomials
\begin{equation*}
	f(z) = \sum_{n=0}^{\infty} \alpha_n z^{n}
\end{equation*}
Similarly, we can define $f(a)$ for $a \in \alg{A}$ as
\begin{equation*}
	f(a) = \sum_{n=0}^{\infty} \alpha_n a^{n}
\end{equation*}
If the radius of convergence of this sequence is $R$, then it can be well-defined for $\norm{a} \leqslant R$. By the fact that $r(a) \leqslant \norm{a}$, for the analytic function $f$, if $\sigma(a) \subset A$, $f(a)$ can be well-defined.\\
Let $\hol{(a)}$ denote all functions that are analytic in a neighbourhood of $\sigma(a)$. Then there is a map from $\hol{(a)}$ to $\alg{A}$ defined as $f \mapsto f(a)$. Now, we can find more properties of this map. Firstly, we can give another formula of $f(a)$.\\
If $f \in \hol{(a)}$, then for any closed curve $\gamma$ which encloses $\hol{(a)}$ and any point $z_0 \in \hol{(a)}$, 
\begin{equation*}
	f(z_0) = \frac{1}{2 \pi i}\int_{\gamma} f(z) (z-z_0)^{-1} dz
\end{equation*}
Therefore, replacing $z_0$ by $a \in \alg{A}$, then we have 
\begin{equation*}
	f(a) = \frac{1}{2 \pi i}\int_{\gamma} f(z) (z-a)^{-1} dz
\end{equation*}
Clearly, by the Cauchy's Integral Formula, this definition is well-defined and is coincided with above definition. But this definition can provide us a conivient method to research the map $f \mapsto f(a)$.

\begin{thm}[Riesz Functional Calculus]
	Let $\alg{A}$ be a Banach algebra and $a \in \alg{A}$ and the map
	\begin{center}
		\begin{tabular}{l c c l}
			$\rho \colon$ & $\hol{(a)}$ & $\longrightarrow$ & $\alg{A}$ \\
			~ & $f$ & $\longmapsto$ & $f(a)=\frac{1}{2 \pi i}\int_{\gamma} f(z) (z-a)^{-1} dz$
		\end{tabular}
	\end{center}
	has the following properties.
	\begin{enumerate}[label=\arabic*)]
		\item $\rho$ is an algebra homomorphism.
		\item $\rho(1) = 1$ and $\rho(z)=a$.
		\item If $\{f_n\} \subset \hol{(a)}$ and $f \in \alg{A}$ with $f_n \sto f$ uniformly on a compact set of $\hol{(a)}$, then $\rho(f_n) \sto \rho(a)$ in norm.
	\end{enumerate}
	Moreover, if any map $\tau \colon \hol{(a)} \sto \alg{A}$ satifies above conditions, then $\tau = \rho$. 
\end{thm}
\begin{proof}
	\item For $1)$, $\rho$ is clearly linear. And
		\begin{eqnarray*}
			f(a)g(a)&=& -\frac{1}{4 \pi^{2}}\int_{\gamma_1} f(z) (z-a)^{-1} dz\int_{\gamma_2} g(\zeta) (\zeta-a)^{-1} d\zeta \\
			&=& -\frac{1}{4 \pi^{2}}\int_{\gamma_1}\int_{\gamma_2} f(z)g(\zeta) \frac{(z-a)^{-1}-(\zeta-a)^{-1}}{\zeta-z} d\zeta dz
			&=& -\frac{1}{4 \pi^{2}}\int_{\gamma_1}f(z) \int_{\gamma_2} \frac{g(\zeta)}{\zeta-z} d\zeta (z-a)^{-1} dz + \frac{1}{4 \pi^{2}}\int_{\gamma_2}g(\zeta) \int_{\gamma_1} \frac{f(z)}{\zeta-z} dz (\zeta-a)^{-1} d\zeta
		\end{eqnarray*}
		We can choose $\gamma_2$ to enclose $\gamma_1$, thus 
		\begin{equation*}
			\int_{\gamma_1} \frac{f(z)}{\zeta-z} dz =0,~ \int_{\gamma_2} \frac{g(\zeta)}{\zeta-z} d\zeta] = 2\pi i g(z)
		\end{equation*}
		Therefore, 
		\begin{equation*}
			f(a)g(a) = {2 \pi i}\int_{\gamma_1} f(z)g(z)(z-a)^{-1} dz = (fg)(a)
		\end{equation*}
	\item For $2)$, let $f(z) = z^k$ and $\gamma = R e^{2 \pi i t}$, where $R>\norm{a}$ and $t \in [0,1]$, then
	\begin{eqnarray*}
		f(a) &=& \frac{1}{2 \pi i}\int_{\gamma} z^k (z-a)^{-1} dz \\
		&=& \frac{1}{2 \pi i}\int_{\gamma} z^{k-1} (1-\frac{a}{z})^{-1} dz \\
		&=& \frac{1}{2 \pi i}\int_{\gamma} z^{k-1} \sum_{n=0}^{\infty} \frac{a^{n}}{z^{n}}dz \\
		&=& \sum_{n=0}^{\infty} (\frac{1}{2 \pi i}\int_{\gamma} \frac{1}{z^{n-k+1}}) a^{n} \\
		&=& a^{k}
	\end{eqnarray*}
	\item For $3)$, 
	\begin{eqnarray*}
		\norm{\int_{\gamma} f_n(z) (z-a)^{-1} dz - \int_{\gamma} f(z) (z-a)^{-1} dz} \\
		&=& \norm{\int_{0}^{1}(f_n(\gamma(t))-f(\gamma(t)))(\gamma(t)-a)^{-1}d\gamma(t)}\\
		&\leqslant& \int_{0}^{1} \abs{f_n(\gamma(t))-f(\gamma(t))}\norm{(\gamma(t)-a)^{-1}}d\abs{\gamma}(t) \\
		&\leqslant& M \norm{\gamma} \sup{\{\abs{f_n(z)-f(z)} \colon z \in \gamma(t)\}}
	\end{eqnarray*}
	where $M$ is the bound of $\norm{(\gamma(t)-a)^{-1}}$ since $t \mapsto \norm{(\gamma(t)-a)^{-1}}$ is continuous on $\gamma(t)$.\\
	Therefore, by the fact that $f_n \sto f$ uniformly, $\norm{f_n(a)-f(a)} \sto 0$.
	\item Finally, the uniquness is because any $f \in \hol{(a)}$ can be approximated uniformly by a sequence of polynomials. Thus, $1)$ and $2)$ means $\tau(p) = \rho(p)$ for any polynomial $p$, and $3)$ provides the fact that $\tau(f) = \rho(f)$ for any $f \in \hol{(a)}$.
\end{proof}
\begin{rem}
	we have mentioned that the integral definition is coincided with the convergent difinition. In fact, by $2)$, this statement can be proved rigorously.
\end{rem}

\begin{thm}[Spectral Mapping Theorem]
	If $a \in \alg{A}$ and $f \in \hol{(a)}$, then
	\begin{equation*}
		\sigma(f(a)) = f(\sigma(a))
	\end{equation*}
\end{thm}
\begin{proof}
	Firstly, there is a $g \in \hol{(a)}$ s.t. for $\alpha \in \sigma(a)$, $f(z)-f(\alpha) = (z-\alpha)g(z)$, that means $f(\sigma(a)) \subset \sigma(f(a))$.
	\item Conversely, if $\alpha \notin f(\sigma(a))$, $g(z)=(f(z)-\alpha)^{-1} \in \hol{(a)}$. Thus, $g(a)(f(a)-\alpha) = 1$. Therefore, $\alpha \notin \sigma(f(a))$.
\end{proof}

\begin{prop}
	Let $\alg{A}$ be a Banach algebra and $a \in \alg{A}$. $\sigma(a) = F_1 \bigcup F_2$, where $F_1$ and $F_2$ are disjoint nonempty closed sets. Then there is a nontrial idempotent $e$, i.e. $e^{2}=e$, s.t.
	\begin{enumerata}[label=\arabic*)]
		\item if $ab=ba$, then $eb=be$.
		\item if $a_1=ae$ and $a_2=a(1-e)$, then $a_1a_2=a_2a_1=0$.
		\item $\sigma(a_1)=F_1 \bigcup \{0\}$ and $\sigma(a_2)=F_2 \bigcup \{0\}$.
	\end{enumerata}
\end{prop}
\begin{proof}
	Since $F_1$ and $F_2$ are disjoint closed set, there are two disjoint open sets $G_1$ and $G_2$ separating $F_1$ and $F_2$. Let $f$ be the characteristic function of $G_1$ and $e=f(a)$. Thus $e^{2}=e$ by $f^{2}=f$.
	\item For $1)$, there is a more genera result, $f(a)b=bf(a)$ for any $f \in \hol{(a)}$. It is because by extending the fact $p(a)b=bp(a)$ for any polynomial $p$.
	\item $2)$ is clearly true.
	\item Let $f_1(z)=zf(z)$ and $f_2(z)=z(1-f(z))$. Then $a_j = f_j(a)$ for $j=1,2$. Then by the Spectral Mapping Theorem $\sigma(a_j) = f_j(\sigma(a_j)) = F_j \bigcup \{0\}$.
\end{proof}

\subsection{Abelian Banach Algebras}

\begin{thm}[Gelfand-Mazur Theorem]
	If $\alg{A}$ is a Banach algebra and a division ring, then $\alg{A} = \C$.
\end{thm}
\begin{proof}
	It is because that for any $a \in \alg{A}$, $\sigma(a) \neq \varnothing$.
\end{proof}

Next, we reach the structure of an abelian Banach algebra. The structure of abelian Banach algebras can be explicit by constructing a map from an abelian Banach algebra to a continuous function space on a compact space. Firstly, we can find this compact space. Let 
\begin{eqnarray*}
	\Sigma(\alg{A}) &=& \{ \text{all algebra homomorphism } h \colon \alg{A} \sto \C\}\\
	\fml{M} &=& \{\text{all maximal ideals of } \alg{A}\}
\end{eqnarray*}
for an abelian Banach algebra $\alg{A}$. Then we can find the relationship between $\Sigma(\alg{A})$ and $\fml{M}$.

\begin{thm}
	Let $\alg{A}$ be an abelian Banach algebra. Define a map
	\begin{center}
		\begin{tabular}{l c c l}
			$\gamma \colon$ & $\Sigma(\alg{A})$ & $\longrightarrow$ & $\fml{M}$ \\
			~ & $h$ & $\longmapsto$ & $\ker{h}$
		\end{tabular}
	\end{center}
	Then $\gamma$ is a bijection.
\end{thm}
\begin{proof}
	Since $\alg{A} / \ker{h} \cong \C$, $\ker{h} \in \fml{M}$, i.e. $\gamma$ is well-defined.
	\item Check: $\alg{A} / M \cong \C$ for any $M \in \fml{M}$\\
		Let $\pi \colon \alg{A} \sto \alg{A} / M$. If $\pi(a)$ is not invertible, then $\pi(a\alg{A})$ is a proper ideal in $\alg{A} / M$. Thus $I = \pi^{-1}(\pi(a\alg{A}))$ is a proper ideal in $\alg{A}$ and $M \subset I$. Then by the maximality of $M$, $I=M$, i.e. $\pi(a)=0$. In fact, for any commutative ring, this result is true. Therefore, by Gelfand-Mazur Theorem, $\alg{A} / M \cong \C$.
	\item Check: $\gamma$ is surjective.
		Let $M \in \fml{M}$. Define $\tilde{h} \colon \alg{A} / M \sto \C$ as the algebraic isomorphism. Then $h=\pi \circ \tilde{h} \in \Sigma(\alg{A})$ with $\ker{h} = M$.
	\item Check: $\gamma$ is injective.
		If $\ker{h} = \ker{h^{'}}$ for $h,h^{'} \in \Sigma(\alg{A})$, then by the \textbf{Propostion} \ref{prop2} in the subsection \textbf{1.4.2}, $h = \alpha h^{'}$. And since $h(1) = h^{'}(1)=1$, $h=h^{'}$.
\end{proof}

Then we have some porperties of $h \in \Sigma(\alg{A})$.

\begin{prop}
	Let $\alg{A}$ be an abelian Banach algebra and $h \in \Sigma(\alg{A})$.
	\begin{enumerate}[label=\arabic*)]
		\item $h$ is continuous.
		\item $\norm{h}=1$ for $h \neq 0$.
	\end{enumerate}
\end{prop}
\begin{proof}
	$1)$ holds since $\ker{h}$ is maximal, thus it is closed.
	\item Let $\lambda = h(a)$. Suppose $\abs{\lambda} > \norm{a}$. Then $1-\frac{1}{\lambda}$ is invertible. Set $b = (1-\frac{1}{\lambda})^{-1}$, then 
	\begin{equation*}
		1=h(b(1-\frac{1}{\lambda})) = h(b) - \frac{h(b)h(a)}{\lambda} =0
	\end{equation*}
	Therefore, $\abs{h(a)} \leqslant \norm{a}$ i.e $\norm{n} \leqslant 1$. Since $h(1) = 1$, $\norm{h}=1$.
\end{proof}

\begin{defn}
	Let $\alg{A}$ be an abelian Banach algebra. Then $\Sigma(\alg{A}) \subset \alg{A}^{*}$ endowed with the induced $wk^{*}$-topology, is called the maximal ideal space of $\alg{A}$.
\end{defn}

\begin{prop}
	If $\alg{A}$ is an abelian Banach algebra, then $\Sigma(\alg{A})$ is a compact Hausdorff space. Moreover, if $a \in \alg{A}$, then
	\begin{equation*}
		\sigma(a) = \Sigma(a) = \{~ h(a) \colon h \in \Sigma(\alg{A}) ~\}
	\end{equation*}
\end{prop}
\begin{proof}
	Since $\Sigma(\alg{A}) \subset \alg{A}^{*}$, we just need to show $\Sigma(\alg{A})$ is $wk^{*}$-closed. Let $\{h_i\}$ be a net in $\Sigma(\alg{A})$ s.t. $h_i \sto h$ $wk^{*}$ for some $h$ in the unit closed ball of $\alg{A}^{*}$. Then for $a,b \in \alg{A}$,
	\begin{equation*}
		h(ab) = \lim_{i} h_i(ab) = \lim_{i} h_i(a)h_i(b) = h(a)h(b)
	\end{equation*}
	and $h(1)=\lim_{i}h_i(1)=1$, thus $h \in \Sigma(\alg{A})$. $\Sigma(\alg{A})$ is compact.
	\item If $h\in \Sigma(\alg{A})$ and $h-h(a) \in \ker{h} \in \fml{M}$, then $h-h(a)$ is not invertible, i.e. $\Sigma(a) \subset \sigma(a)$. Conversely, if $a-\lambda$ is not invertible, $(a-\lambda)\alg{A}$ is a proper ideal, which can be contained in a maximal ideal. Then $(a-\lambda) \in \ker{h}$ with some $h \in \Sigma(\alg{A})$, $\lambda = h(a) \in \Sigma(a)$.
\end{proof}

Therefore, $\Sigma(\alg{A})$ is the compact space we need. Then we define the map from $\alg{A}$ to $C(\Sigma(\alg{A}))$.

\begin{thm}
	If $\alg{A}$ is an abelian Banach algebra, the Gelfand transform is defined as
	\begin{center}
		\begin{tabular}{l c c l}
			$\Gamma \colon$ & $\alg{A}$ & $\longrightarrow$ & $C(\Sigma(\alg{A}))$ \\
			~ & $a$ & $\longmapsto$ & $\hat{a} = \Gamma(a)$
		\end{tabular}
	\end{center}
	where $\hat{a}(h)=h(a)$.
	\begin{enumerate}[label=\arabic*)]
		\item $\Gamma$ is a continuous homomorphsim.
		\item $\norm{\Gamma}=1$.
		\item $\ker{\Gamma}=\bigcap\{M \colon M \in \fml{M}\}$.
		\item $\norm{\hat{a}}_{\infty}=r(a)$.
	\end{enumerate}
\end{thm}
\begin{proof}
	Firstly, for $1)$ if $h_i \sto h$ in $wk^{*}$, $\hat{a}(h_i) = h_i(a) \sto h(a) = \hat{a}(h)$. Thus $\Gamma$ is well-defined. And
	\begin{equation*}
		\Gamma(ab)(h) = h(ab) = h(a)h(b) = \Gamma(a)(h)\Gamma(b)(h)
	\end{equation*}
	Thus $\Gamma$ is a homormophsim.
	\item For $2)$, since $\abs{\hat{a}(h)} = \abs{h(a)} \leqslant \norm{a}$, $\norm{\hat{a}}_{\infty} \leqslant \norm{a}$. Then $\norm{\Gamma} \leqslant 1$. By $\Gamma(1) = 1$, $\norm{\Gamma} = 1$.
	\item For $3)$, $a \in \ker{\Gamma}$ if and only if $h(a) = 0$ for any $h \in \Sigma(\alg{A})$, i.e. $a \in \bigcap\{M \colon M \in \fml{M}\}$. 
	\item For $4)$, it holds since $\sigma(a) = \{~ h(a) \colon h \in \Sigma(\alg{A}) ~\}$.
\end{proof}

If $a \in \alg{A}$ s.t.  
\begin{proof}
	In fact, $\tau$ can be defined as $\tau(h)=h(a)$. By above mention, $\tau$ is continuous and surjective. If $\tau{h_1}=\tau{h_2}$, then $ h_1(a) = h_2(a)$. By the fact that $h_1,h_2 \in \alg{A}$ and $a$ is a generator of $\alg{A}$, $h_1=h_2$. Thus $\tau$ is a bijection. And because $\Sigma(\alg{A})$ is compact, $f$ is a closed map, i.e. $f$ is a homeomorphism.
	\begin{equation*}
		\Gamma(p(a)(h) = p(\Gamma(a))(h) = p(\Gamma(a)(h)) = p(\tau(h))
	\end{equation*}
\end{proof}
\begin{rem}
	If $\alg{A}$ is generate by $a$, then $\Gamma \colon \alg{A} \sto C(\sigma(a))$ can be defined as $\Gamma(p(a)) = p$. 
\end{rem}

In fact, above proposition can extends to $n$ generators. If $\{a_i\}_{i=1}^{n}$ are generators of $\alg{A}$, i.e $\clo{\{p(a_1,\cdots,a_n) \colon p \text{ is any } n \text{ variables polynomial}\}} = \alg{A}$, then we have similar results as above proposition.




\section{\texorpdfstring{$C^{*}$-Algebras}{C*-Algebras}}

Now, we have known $\oper$ is a Banach algebra. But there is another algebraic operation on $\oper$, which let $\oper$ be more interesting than the general Banach algebra. This operation is a map $T \sto T^{*}$ on $\oper$, called an involution, and moreover, it satisfies the condition $\norm{T} = \norm{T^{*}}$. This identity provides a strong relation between the topological structure and the algebraic structure on $\oper$. In fact, the topology is completely determined by the algebraic structure on $\oper$. In order to research this structure, we firstly define an general algebra satisfying above condition, called a $C^{*}$-algebra. By digging its topological structures and algebraic structures, we can embed it into $\oper$ for some Hilbert space $\fml{H}$. Therefore, any $C^{*}$-algebra can be regarded as a subalgebra of $\oper$.

\subsection{Elementary Properties}
\begin{defn}
	If $\alg{A}$ is a Banach algebra, an involution is a map $a \sto a^{*}$ from $\alg{A}$ to $\alg{A}$ satisfying for any $a,b \in \alg{A}$ and any $\alpha \in \C$, 
	\begin{enumerate}[label=\arabic*)]
		\item $(a^{*})^{*}=a$,
		\item $(ab)^{*}=b^{*}a^{*}$,
		\item $(\alpha a+b)^{*} = \clo{\alpha} a^{*}+b^{*}$.
	\end{enumerate}
\end{defn}

\begin{defn}
	A $C^{*}$-algebra is a Banach algebra $\alg{A}$ with an involution s.t. for every $a \in \alg{A}$,
	\begin{equation*}
		\norm{\st{a}a} = \norm{a}^{2}
	\end{equation*}
\end{defn}

Then we can get some easy properties for the norm and the involution.

\begin{prop}
	Let $\A$ be a \Cs and $a \in \A$.
	\begin{enumerate}[label=\arabic*)]
		\item $\norm{\st{a}} = \norm{a}$.
		\item $\norm{a\st{a}}=\norm{a}^{2}$.
		\item $\norm{a}=\sup{\{\norm{ax} \colon \norm{x} \leqslant 1\}} = \sup{\{\norm{xa} \colon \norm{x} \leqslant 1\}}$.
	\end{enumerate}
\end{prop}
\begin{proof}
	For $1)$, $\norm{a}^{2} = \norm{\st{a}a} \leqslant \norm{\st{a}}\norm{a}$, thus $\norm{a} \leqslant \norm{\st{a}}$. Taking the involution, $\norm{\st{a}} \leqslant \norm{a}$.
	\item $2)$, $\norm{a\st{a}} = \norm{\st{(\st{a})}} = \norm{\st{a}}^{2} = \norm{a}^{2}$.
	\item For $3)$, let $\alpha$ be the supremum, then $\alpha \leqslant \norm{a}$. $a = 0$ is clearly true. For $a \neq 0$, let $x = \st{a} / \norm{a}$. Thus $\alpha \geqslant \norm{a}$.
\end{proof}

If a \Cs $\A$ is without the identity, we can use same method of the Banach algebra to extend it to a unit \Cs $\tilde{\A}$. The only thing we need to prove is the identity. And this can be obtained the result in above proposition. Therefore, we always assume a \Cs is with the identity.

\begin{defn}
	Let $\A$ be a \Cs and $a \in \A$.
	\begin{enumerate}[label=\arabic*)]
		\item $a$ is self-adjoint if $a = \st{a}$.
		\item $a$ is normal if $a\st{a} = \st{a}a$.
		\item $a$ is unitary if $a\st{a}=\st{a}a=1$.
		\item $a$ is a projection if $a$ is self-adjoint and $a=a^{2}$.
	\end{enumerate}
\end{defn}

Then we can see the algebraic strucuture on a \Cs completely determine its norm topology.

\begin{thm}
	Let $\A$ be a \Cs and $a \in \A$. If $a$ is self-adjoint, then
	\begin{equation*}
		r(a) = \norm{a}
	\end{equation*}
\end{thm}
\begin{proof}
	Since $a$ is sef-sdjoint,
	\begin{equation*}
		\norm{a}^{2} = \norm{\st{a}a} = \norm{a^{2}}
	\end{equation*}
	Thus by induction, we have $\norm{a}^{2n} = \norm{a^{2n}}$. Then
	\begin{equation*}
		r(a) = \lim_{n \sto \infty} \norm{a^{n}}^{\frac{1}{n}} = \lim_{n \sto \infty} \norm{a^{2n}}^{\frac{1}{2n}} = \norm{a}
	\end{equation*}
\end{proof}
\begin{rem}
	For any $b \in \A$, we know $\st{b}b$ is self-adjoint,
	\begin{equation*}
		r(\st{b}b) = \norm{\st{b}b} = \norm{b}^{2}
	\end{equation*}
	Thus, the norm in a \Cs is completely determined by the spectral radius, which is totally an algebraic trait.
\end{rem}

Now, we can see how the algebraic property influences the topological structure.

\begin{prop}
	Let $\rho \colon \A \sto \B$ be a *-homomorphsim between two $C^{*}$-algebras.
	\begin{enumerate}[label=\arabic*)]
		\item $\rho$ is continuous, and moreover $\norm{\rho(a)} \leqslant \norm{a}$.
		\item If $\rho$ is a *-isomorphism, then $\rho$ is an isometry.
	\end{enumerate}
\end{prop}
\begin{proof}
	$2)$ is the direct corollary from $1)$. For $1)$, clearly $\sigma(\rho(a)) \subset \sigma(a)$, thus
	\begin{equation*}
		\norm{\rho(a)}^{2} = r(\rho(\st{a}a)) \leqslant r(\st{a}a) = \norm{a}^{2}
	\end{equation*}
\end{proof}













