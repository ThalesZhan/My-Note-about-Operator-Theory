\chapter{Banach Algebras and \texorpdfstring{$C^{*}$-Algebras}{C*-Algebras}}
\section{Banach Algebras}

Let $\fml{H}$ be a Hilbert space. Then $\oper$ is indeed a Banach space. But we have more sructure on it. Any two element $S,T \in \oper$ can do multiplication, defined as $ST = S \circ T$, then $ST \in \oper$ by the definition of norm, and moreover $\norm{ST} \leqslant \norm{S}\norm{T}$. Therefore, $\oper$ is an algebra with the extra property of the multiplication, called a Banach algebra.

\subsection{Elementary Properties}

\begin{defn}
	A Banach algebra $\alg{A}$ is an algebra over $\C$ with a norm $\norm{\cdot}$ relative to which $\alg{A}$ is a Banach space and s.t. for all $a,b \in \alg{A}$,
	\begin{equation*}
		\norm{ab} \leqslant \norm{a}\norm{b}.
	\end{equation*}
\end{defn}
\begin{rem}
	The extra condition guarantees the multiplication is norm continuous. In fact, $\alg{A}$ with the multiplication and the norm is a topological semigroup.
\end{rem}

If $\alg{A}$ has an identity $1$, it assumes $\norm{1}=1$. But if $\alg{A}$ does not have an identity, we can add an identity to it.

\begin{prop}
	If $\alg{A}$ is a Banach algebra without the identity. The let $\tilde{\alg{A}}= \alg{A} \oplus \C$ be an induced vector space. Then we define a norm on $\tilde{\alg{A}}$ as
	\begin{equation*}
		\norm{(a,\lambda)} = \norm{a} + \abs{\lambda}
	\end{equation*}
	and define a multiplication on $\tilde{\alg{A}}$ as
	\begin{equation*}
		(a,\alpha)(b,\beta) = (ab+\alpha b + \beta a, \alpha \beta)
	\end{equation*}
	Then $\alg{A}$ is a Banach algebra with the identity $(0,1)$.
\end{prop}

Then $\alg{A}$ can be isometrically imbedded into $\tilde{\alg{A}}$. Therefore, we can always assume $\alg{A}$ has an identity. For a unit algebra, the invertibility of a element is important. 

\begin{thm}
	Let $\alg{A}$ be a Banach algebra and $a \in \alg{A}$. If $\norm{a-1} < 1$, then $a$ is invertible.
\end{thm}
\begin{proof}
	For a nonzero real number $\lambda \in \R$ with $\abs{\lambda} < 1$, we know 
	\begin{equation*}
		(1-\lambda)^{-1} = \sum_{n=0}^{\infty} \lambda^{n}
	\end{equation*}
	Similar, for $a \in \alg{A}$ with $\norm{a-1} < 1$, set
	\begin{equation*}
		b= \sum_{n=0}^{\infty} (1-a)^{n}
	\end{equation*}
	Firstly, since $\norm{(1-a)^{n}} \leqslant \norm{(1-a)}^{n}$, $b$ is well-defined. Then we can prove $b = (1-(1-a))^{-1} = a^{-1}$.
\end{proof}
\begin{rem}
	This result is important. It says a small perturbation of an invertible element is also invertible. It is because that the multiplication is norm continuous. And by the continuity of multiplication, this result can be true at any point other than 1.
\end{rem}

\begin{cor}
	Let $\alg{A}$ be a Banach algebra and 
	\begin{eqnarray*}
		G_l &=& \{~a \in \alg{A} \colon a \text{ is left invertible}~\}
		G_r &=& \{~a \in \alg{A} \colon a \text{ is right invertible}~\}
		G &=& G_l \bigcap G_r = \{~a \in \alg{A} \colon a \text{ is invertible}~\}
	\end{eqnarray*}
	Then $G_l$ and $G_r$ and $G$ are open. Moreover, the map $a \sto a^{-1}$ from $G$ to $G$ is continuous.
\end{cor}
\begin{proof}
	$G_l$ and $G_r$ and $G$ are open by above theorem. \\
	We just need to check this map is continuous at $1$ because of the continuity of multiplication. For $\{a_n\} \subset G$ with $a_n \sto 1$, thus $\norm{1-a_n}<\delta<1$. Since
	\begin{equation*}
		a_n^{-1} = (1-(1-a_n))^{-1} = \sum_{k=0}^{\infty} (1-a_n)^{k} = 1+ \sum_{k=1}^{\infty} (1-a_n)^{k}
	\end{equation*}
	Therefore,we have
	\begin{eqnarray*}
		\norm{1-a_n^{-1}} &=& \norm{\sum_{k=1}^{\infty} (1-a_n)^{k}}\\
		&\leqslant&  \sum_{k=1}^{\infty} \norm{1-a_n}^{k}
		&<& \frac{\delta}{1+\delta} < \delta = \norm{1-a_n}
	\end{eqnarray*}
	i.e. $\lim a_n = 1$.
\end{proof}

\begin{cor}
	Let $\alg{A}$ be a Banach algebra.
	\begin{enumerate}[label=\arabic*)]
		\item The closure of a proper ideal is a proper ideal.
		\item A maximal ideal is closed.
		\item every ideal contained in a maximal ideal.
	\end{enumerate}
\end{cor}

If $\alg{B}$ is a closed ideal of a Banach algebra $\alg{A}$, then then quotient algbra $\alg{A}/\alg{B}$ with the induced norm is also a Banach algebra since 
\begin{equation*}
	\norm{(a+\alg{B})(b+\alg{B})} = \norm{ab+\alg{B}} \leqslant \norm{(a+b_1)(b+b_2)} \leqslant \norm{(a+b_1)}\norm{(b+b_2)}
\end{equation*}
for any $b_1, b_2 \in \alg{B}$.

\subsection{Spectrum}

\begin{defn}
	Let $\alg{A}$ be a Banach algebra and $a \in \alg{A}$. The spectrum of $a$, denoted by $\sigma(a)$ defined as
	\begin{equation*}
		\sigma(a) = \{~ \lambda \in \C \colon a-\lambda \text{ is invertible}~\}
	\end{equation*}	
	And the resolvents of $a$, $\rho(a) = \C \backslash \sigma(a)$.\\
	Moreover, we can define the spectral radius of $a$ as
	\begin{equation*}
		r(a) = \sup{\{~\abs{\lambda} \colon \lambda \in \sigma(a)~\}}
	\end{equation*}
\end{defn}

Firstly, there are some elementary properties of the spectrum.

\begin{thm}
	Let $\alg{A}$ be a Banach algebra and $a \in \alg{A}$.
	\begin{enumerate}[label=\arabic*)]
		\item If $\abs{\lambda} > \norm{a}$, then $\lambda \notin \sigma(a)$.
		\item $\sigma(a)$ is a compact subset of $\C$.
		\item the map $\lambda \mapsto (a-\lambda)^{-1}$ from $\rho(a)$ to $\alg{A}$ is analytic and $\sigma(a)$ is nonempty.
		\item $r(a)=\lim_{n \sto \infty} \norm{a^{n}}^{\frac{1}{n}}$.
	\end{enumerate}
\end{thm}
\begin{proof}
	$1)$ holds by above theorem. \\
	For $2)$, since $\lambda \sto a-\lambda$ is continuous from $\C$ to $\alg{A}$ and $G$ is open, $\rho(a)$ is open i.e. $\sigma(a)=\C \backslash \rho(a)$ is closed. Then by $1)$, $\sigma(a)$ is compact.\\
	For $3)$, by the identity $a^{-1} - b^{-1} = a^{-1}(b-a)b^{-1}$ and the continuity of $a \sto a^{-1}$, we can compute the derivative of $F(\lambda) = (a-\lambda)^{-1}$,
	\begin{equation*}
		F^{'}(\lambda) = (a-\lambda)^{-2}
	\end{equation*}
	And clearly, $F^{'}(\lambda)$ is continuous. Thus it is analytic and it vanishes at $\infty$. By the Liouville's Theorem, if $\rho(a) = \C$, $F$ is constant. Therefore, $\rho(a) \notin \C$ i.e. $\sigma(a) \neq \varnothing$.\\
	For $4)$, let $U=\{\lambda \in \C \colon \lambda = 0 \text{ or } \lambda^{-1} \in \rho(a)\}$ and 
	\begin{equation*}
		f(\lambda) = 
		\begin{cases}
			(\lambda^{-1}-a)^{-1} & x \neq 0,\\
			0,& x = 0
		\end{cases}
	\end{equation*}
	Then $f$ is analytic on $U$, i.e $f(\lambda)=\lambda\sum_{n=0}^{\infty} \lambda^{n} a^{n}$ is well-defined. Therefore, the convergent radius $R = r(a)^{-1}$
	\begin{equation*}
		R^{-1} = \limsup_{n \sto \infty} \norm{a^{n}}^{\frac{1}{n}} = r(a)
	\end{equation*}
	Conversely, by the identity $(a^{n}-\lambda^{n}) = (a-\lambda)(a^{n-1}+\lambda a^{n-2}+\lambda^{2} a^{n-3} + \cdots + \lambda^{n-1})$. Then, if $(a^{n}-\lambda^{n})$ is invertible, then $(a-\lambda)$ is invertible, i.e. $\sigma(a) \subset \sigma(a^{n})$. Thus $\abs{\lambda}^{n} \leqslant \norm{a^{n}}$ for any $\lambda \in \sigma{a}$. $r(a)=\liminf_{n \sto \infty}\norm{a^{n}}^{\frac{1}{n}}$. Therefore, $r(a)=\lim_{n \sto \infty} \norm{a^{n}}^{\frac{1}{n}}$.
\end{proof}

If $\alg{B} \subset \alg{A}$ is a subalgebra with the same identity of a Banach algebra $\alg{A}$, then we know for any element $b \in \alg{B}$, $\sigma_{\alg{A}}(b) \subset \sigma_{\alg{B}}(b)$. Then we can have more results other than it. Since the spectrum is a subset of $\C$, we need some topological properties results of $\C$.

\begin{lem}
	If $K$ is any compact subset of $\C$, then $\C \backslash K$ has a countable components, only one of which is unbounded. And the boundary of each component is in $K$.
\end{lem}
\begin{proof}
	Let $\tilde{K} = \C \backslash K$, then $\tilde{K}$ is open.
	\item Firstly, the connected component of open set in $\C$ is open. \\
		  Let $U$ be an connected component in $\tilde{K}$ and $x \in U$. For any point $x \in U$, Since any open neighbourhood of $x$ is connected, and $K$ is open, there is a open neighbourhood $V$ of $x$ s.t. $V \subset U$.
	\item Secondly, $\C$ has just at most countable many open sets, which are pairwise disjoint.\\
		  This result is because any open set in $\C$ contains a rational point.
	\item For any two disjoint open sets $A$ and $B$ in $\C$, $\partial A \bigcap B = \varnothing$. Thus the boundary of some compoment of $\tilde{K}$ can not intersect in any component of $\tilde{K}$, i.e. it is contained $K$.
	\item Finally, since $K$ is bounded, there is a closed ball $B$ containing $K$. But the complement of $B$ is connected, thus there is only one component of $\tilde{K}$ containing $B$. Thus the other components of $\tilde{K}$ are bounded.
\end{proof}
\begin{rem}
	The bounded component of $\C \backslash K$ is called a hole of $K$.
\end{rem}

\begin{defn}
	If $f \colon A \sto \C$, where $A$ is a set, then the norm of $f$ on $A$ is defined as
	\begin{equation*}
		\norm{f}_A = \sup{\{\abs{f(x)} \colon x \in A\}}
	\end{equation*}
	For a compact set $K \in \C$, the polynomially convex hull of $K$ is defined as
	\begin{equation*}
		\hat{K} = \{~z \in \C \colon \abs{p(z)} \leqslant \norm{p}_K \text{ for any polynomial } p ~\}
	\end{equation*}
	If $K = \hat{K}$, $K$ is called polynomially convex.
\end{defn}

\begin{prop}
	Let $K$ be a compact subset of $\C$. Then $\C \backslash \hat{K}$ is the unbounded component of $\C \backslash K$. Therefore, $K$ is polynomially convex if and only if $\C \backslash K$ is connected.
\end{prop}
\begin{proof}
	Let $L$ be the set containing $K$ and all bounded component of $\C \backslash K$. Then by the Maximal Principle, $L \subset \hat{K}$. Conversely, if $\alpha \notin L$, then $(z-\alpha)^{-1}$ is analytic in a neighbourhood of $L$. Therefore, there is a sequence of polynomials $\{p_n\}$ s.t. $p_n \sto (z-\alpha)^{-1}$. Let $q_n=(z-\alpha)p_n$. Then $q_n \sto 1$, i.e. $\norm{q_n-1} < \frac{1}{2}$ for some $n$. But $\abs{q_n(\alpha)-1}=1$, this implies $\alpha \notin \hat{K}$, i.e. $\hat{K} \subset L$.
\end{proof}

By above results, now we can provide the relationships betweem $\sigma_{\alg{A}}(b)$ and $\sigma_{\alg{B}}(b)$.

\begin{thm}
	If $\alg{A}$ and $\alg{B}$ are Banach algebras with same identity s.t. $\alg{B} \subset \alg{A}$ and $b \in \alg{B}$, then
	\begin{enumerate}[label=\arabic*)]
		\item $\sigma_{\alg{A}}(b) \subset \sigma_{\alg{B}}(b)$ and $\partial\sigma_{\alg{B}}(b) \subset \partial\sigma_{\alg{A}}(b)$
		\item $\hat{\sigma_{\alg{A}}(b)} = \hat{\sigma_{\alg{B}}(b)}$
		\item if $G$ is a hole of $\sigma_{\alg{A}}(b)$, then $G \subset \sigma_{\alg{B}}(b)$ or $G \bigcap \sigma_{\alg{B}}(b) = \varnothing$
	\end{enumerate}
\end{thm}
\begin{proof}
	\item For $1)$, let $\lambda \in \partial\sigma_{\alg{B}}(b)$. Since $\inte{\sigma_{\alg{A}}(b)} \subset \inte{\sigma_{\alg{B}}(b)}$, it is sufficient to show $\lambda \in \sigma_{\alg{A}}(b)$. Suppose $\lambda \notin \sigma_{\alg{A}}(b)$, i.e. $(b-\lambda)$ is invertible in $\alg{A}$. But since $\lambda \in \partial\sigma_{\alg{B}}(b)$, there are $\lambda_n \in \C \backslash \alg{B}$ with $\lambda_n \sto \lambda$. Thus $(b-\lambda_n)^{-1} \in \alg{B}$. But $(b-\lambda_n)^{-1} \sto (b-\lambda)^{-1} \in \sigma_{\alg{B}}(b)$, contradicting to $\lambda \in \sigma_{\alg{A}}(b)$.
	\item $2)$ holds because of the result of $1)$ and the Maxiamal Principle.
	\item For $3)$, let $G_1 = G \bigcap \sigma_{\alg{B}}(b)$ and $G_2 = G \backslash \sigma_{\alg{B}}(b)$. Since $\partial\sigma_{\alg{B}}(b) \subset \sigma_{\alg{A}}(b)$ and $G \bigcap \sigma_{\alg{A}}(b) = \varnothing$, $G_1 = G \bigcap \inte{\sigma_{\alg{B}}(b)}$ is open. By the facts that $G_2$ is clearly open and $G = G_1 \bigcup G_2$ and $G_1 \bigcap G_2 = \varnothing$, either $G_1$ or $G_2$ is empty.
\end{proof}

Then we can have some useful corollaries.

\begin{cor}
	Let $\alg{A}$ and $\alg{B}$ be Banach algebras with same identity s.t. $\alg{B} \subset \alg{A}$ and $b \in \alg{B}$.
	\begin{enumerate}[label=\arabic*)]
		\item If $\sigma_{\alg{A}}(b)$ has no holes, then $\sigma_{\alg{A}}(b)=\sigma_{\alg{B}}(b)$.
		\item If $\sigma_{\alg{B}}(b) \subset \R$, then $\sigma_{\alg{A}}(b)=\sigma_{\alg{B}}(b)$.
		\item $\sigma_{\alg{A}}(b)=\sigma_{\alg{B}}(b)$ if and only if $\rho_{\alg{A}}(b)$ is connected.
	\end{enumerate}
\end{cor}
\begin{proof}
	$1)$ is clearly true since ubbouded component does not intersect $\sigma_{\alg{B}}(b)$. $2)$ is because $\C \backslash \sigma_{\alg{A}}(b)$ has no holes. $3)$ is similar as $2)$.
\end{proof}

\subsection{Riesz Functional Calculus}








