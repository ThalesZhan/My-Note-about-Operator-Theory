\chapter{Von Neumann Algebras}

Abelian von Neuman algebras can be explicitly described as $L^{\infty}(\mu)$. For general von Neumann algebras, they have more complex structure but more interesting properties. Firstly, we have seen the relation between the $WOT$ and the ultraweak topology, therefore, the ultraweak topology can provide some different things of the von Neumann algebra. Then, since the weak closedness of von Neumann algebras, all projections in a von Neumann algebra can completely generate the whole algebra. Thus projections are fundamental elements in the von Neumann algebra. Researching the properties of them and classifying them can help us to classify general von Neumann algebras. By the properties of the projections contained in a von Neumann algebras, general von Neumann algebras can be classified as $5$ different categories and we can see that the abelian one is belong to what category.

\section{General von Neumann Algebras}

\subsection{Elementary Properties}

\begin{prop}
	If $\{A_i\}$ is an incresing net of self-adjoint operators on $\Hs$ s.t. $\sup_i{\norm{A_i}} < \infty$, then there is an operator $A$ s.t. the following hold.
	\begin{enumerate}[label = \arabic*)]
		\item $A_i \leqslant A$ for all $i$ and if $B$ is any self-adjoint operator with $A_i \leqslant B$ for all $i$, then $A \leqslant B$. Thus $A = \sup_i A_i$.
		\item $A_i \sto A$ in $WOT$.
		\item $A_i \sto A$ in $SOT$.
		\item $A_i \sto A$ in $weak^{*}$.
	\end{enumerate}
\end{prop}
\begin{proof}
	Without loss of generality, assume all $A_i$ are positive.\\
	For any $h \in \Hs$, $\{\langle A_i h, h \rangle\}$ is an increasing positive sequence and bounded by $\alpha \norm{h}^2$ for $\alpha = \sup_i{\norm{A_i}}$. Then
	\begin{equation*}
		F(h,h) = \lim_i \langle A_i h, h \rangle,~ \abs{F(h,h)} \leqslant \alpha \norm{h}^2
	\end{equation*}
	Clearly, $F$ is a sesquilinear and by the Polar Identity, 
	\begin{equation*}
		\abs{F(g,h)} \leqslant \alpha \norm{g} \norm{h}
	\end{equation*}
	Then there is a $A \in \oper$ s.t. 
	\begin{equation*}
		\langle Ag,h \rangle = F(g,h) = \lim_i \langle A_i g, h \rangle
	\end{equation*}
	Therefore, $A$ is positive, $A \sto A_i$ in $WOT$, $A_i \leqslant A$ and $\norm{A} \leqslant \alpha$. Then
	\begin{equation*}
		0 \leqslant A - A_i \leqslant A \Rightarrow (A-A_i)^{\frac{1}{2}} \leqslant A^{\frac{1}{2}},~ \forall~ i
	\end{equation*}
	If $h \in \Hs$, then
	\begin{eqnarray*}
		\norm{(A-A_i)h}^2 &\leqslant& \norm{(A-A_i)^{\frac{1}{2}}}^2 \norm{(A-A_i)^{\frac{1}{2}}h}^2 \\
		&\leqslant& \norm{A} \norm{(A-A_i)^{\frac{1}{2}}h}^2 \\
		&=& \norm{A} \langle (A-A_i)h,h  \rangle \sto 0
	\end{eqnarray*}
	Thus $A_i \sto A$ in $SOT$. \\
	Finally, to see that $A_i \sto A$ in $weak^{*}$, consider $\{A_i^{(\infty)}\}$, and by $2)$, there is a $A^{(\infty)}$ s.t. $A_i^{(\infty)} \sto A^{(\infty)}$ in $WOT$, therefore, $A_i \sto A$ in $weak^{*}$.
\end{proof}

\begin{cor}
	If $\A$ is a $\st{C}$-algebra contained in $\oper$ but without the identity and $\A$ is $weak^{*}$-closed, then there is a projection $P$ in $\A$ s.t.
	\begin{equation*}
		A = PA = AP,~ \forall~ A \in \A
	\end{equation*}
	Therefore, $P$ is the identity for $\A$ and $\A|_{P}$ is a von Neumann algebra.
\end{cor}
\begin{proof}
	There is an approximate identity $\{A_i\}$. By above proposition, 
	\begin{equation*}
		P = \sup_i A_i \in \A, \text{ and } A_i \sto P \text{ in } weak^{*}
	\end{equation*}
	Then for any $A \in \A$, $\norm{AA_i - A} \sto 0$ and $AA_i \sto AP$ in $weak^{*}$. Thus $A = AP =PA$.
\end{proof}

\begin{prop}
	\begin{enumerate}[label = \arabic*)]
		\item If $\A$ is a von Neumann algebra and $\fml{E}$ is a family of projections in $\A$, then $\bigvee \fml{E}$ and $\bigwedge \fml{E}$ belong to $\A$.
		\item If $\A$ is a von Neumann algebra and $1 \leqslant d \leqslant \infty$, then $\A^{(d)}$ is a von Neumann algebra and 
		\begin{equation*}
			(\A^{(d)})^{'} = \{~T=(T_{ij}) \in \fml{B}(\Hs^{(d)}) \colon T_{ij} \in \A^{'} ~\}
		\end{equation*}
		\item If $\A_i$ is a von Neumann algebra in $\fml{B}(\Hs_i)$ for all $i$ and $\Hs = \bigoplus_{i} \Hs_i$, then
		\begin{equation*}
			\A = \bigoplus_{i} \A_i = \{~ \bigoplus_i A_i \colon A_i \in \A_i,~ \sup_i{\norm{A_i}} < \infty~\}
		\end{equation*}
	\end{enumerate}
\end{prop}

There are some results of the $WOT$



















