\chapter{Von Neumann Algebras}

Abelian von Neuman algebras can be explicitly described as $L^{\infty}(\mu)$. For general von Neumann algebras, they have more complex structure but more interesting properties. Firstly, we have seen the relation between the $WOT$ and the ultraweak topology, therefore, the ultraweak topology can provide some different things of the von Neumann algebra. Then, since the weak closedness of von Neumann algebras, all projections in a von Neumann algebra can completely generate the whole algebra. Thus projections are fundamental elements in the von Neumann algebra. Researching the properties of them and classifying them can help us to classify general von Neumann algebras. By the properties of the projections contained in a von Neumann algebras, general von Neumann algebras can be classified as $5$ different categories and we can see that the abelian one is belong to what category.

\section{General von Neumann Algebras}

\subsection{Elementary Properties}

\begin{prop}
	If $\{A_i\}$ is an incresing net of self-adjoint operators on $\Hs$ s.t. $\sup_i{\norm{A_i}} < \infty$, then there is an operator $A$ s.t. the following hold.
	\begin{enumerate}[label = \arabic*)]
		\item $A_i \leqslant A$ for all $i$ and if $B$ is any self-adjoint operator with $A_i \leqslant B$ for all $i$, then $A \leqslant B$. Thus $A = \sup_i A_i$.
		\item $A_i \sto A$ in $WOT$.
		\item $A_i \sto A$ in $SOT$.
		\item $A_i \sto A$ in $weak^{*}$.
	\end{enumerate}
\end{prop}
\begin{proof}
	Without loss of generality, assume all $A_i$ are positive.\\
	For any $h \in \Hs$, $\{\langle A_i h, h \rangle\}$ is an increasing positive sequence and bounded by $\alpha \norm{h}^2$ for $\alpha = \sup_i{\norm{A_i}}$. Then
	\begin{equation*}
		F(h,h) = \lim_i \langle A_i h, h \rangle,~ \abs{F(h,h)} \leqslant \alpha \norm{h}^2
	\end{equation*}
	Clearly, $F$ is a sesquilinear and by the Polar Identity, 
	\begin{equation*}
		\abs{F(g,h)} \leqslant \alpha \norm{g} \norm{h}
	\end{equation*}
	Then there is a $A \in \oper$ s.t. 
	\begin{equation*}
		\langle Ag,h \rangle = F(g,h) = \lim_i \langle A_i g, h \rangle
	\end{equation*}
	Therefore, $A$ is positive, $A \sto A_i$ in $WOT$, $A_i \leqslant A$ and $\norm{A} \leqslant \alpha$. Then
	\begin{equation*}
		0 \leqslant A - A_i \leqslant A \Rightarrow (A-A_i)^{\frac{1}{2}} \leqslant A^{\frac{1}{2}},~ \forall~ i
	\end{equation*}
	If $h \in \Hs$, then
	\begin{eqnarray*}
		\norm{(A-A_i)h}^2 &\leqslant& \norm{(A-A_i)^{\frac{1}{2}}}^2 \norm{(A-A_i)^{\frac{1}{2}}h}^2 \\
		&\leqslant& \norm{A} \norm{(A-A_i)^{\frac{1}{2}}h}^2 \\
		&=& \norm{A} \langle (A-A_i)h,h  \rangle \sto 0
	\end{eqnarray*}
	Thus $A_i \sto A$ in $SOT$. \\
	Finally, to see that $A_i \sto A$ in $weak^{*}$, consider $\{A_i^{(\infty)}\}$, and by $2)$, there is a $A^{(\infty)}$ s.t. $A_i^{(\infty)} \sto A^{(\infty)}$ in $WOT$, therefore, $A_i \sto A$ in $weak^{*}$.
\end{proof}

\begin{cor} \label{cor12}
	If $\A$ is a $\st{C}$-algebra contained in $\oper$ but without the identity and $\A$ is $weak^{*}$-closed, then there is a projection $P$ in $\A$ s.t.
	\begin{equation*}
		A = PA = AP,~ \forall~ A \in \A
	\end{equation*}
	Therefore, $P$ is the identity for $\A$ and $\A|_{P}$ is a von Neumann algebra.
\end{cor}
\begin{proof}
	There is an approximate identity $\{A_i\}$. By above proposition, 
	\begin{equation*}
		P = \sup_i A_i \in \A, \text{ and } A_i \sto P \text{ in } weak^{*}
	\end{equation*}
	Then for any $A \in \A$, $\norm{AA_i - A} \sto 0$ and $AA_i \sto AP$ in $weak^{*}$. Thus $A = AP =PA$.
\end{proof}

\begin{prop}
	\begin{enumerate}[label = \arabic*)]
		\item If $\A$ is a von Neumann algebra and $\fml{E}$ is a family of projections in $\A$, then $\bigvee \fml{E}$ and $\bigwedge \fml{E}$ belong to $\A$.
		\item If $\A$ is a von Neumann algebra and $1 \leqslant d \leqslant \infty$, then $\A^{(d)}$ is a von Neumann algebra and 
		\begin{equation*}
			(\A^{(d)})^{'} = \{~T=(T_{ij}) \in \fml{B}(\Hs^{(d)}) \colon T_{ij} \in \A^{'} ~\}
		\end{equation*}
		\item If $\A_i$ is a von Neumann algebra in $\fml{B}(\Hs_i)$ for all $i$ and $\Hs = \bigoplus_{i} \Hs_i$, then
		\begin{equation*}
			\A = \bigoplus_{i} \A_i = \{~ \bigoplus_i A_i \colon A_i \in \A_i,~ \sup_i{\norm{A_i}} < \infty~\}
		\end{equation*}
		is a von Neumann algebra.
	\end{enumerate}
\end{prop}

\begin{prop}
	Let $\{T_i\}$ and $\{S_i\}$ be nets in $\oper$.
	\begin{enumerate}[label=\arabic*)]
		\item If $T_i \sto 0$ in $WOT$ and $S_i \sto$ in $SOT$, then $S_iT_i,T_iS_i \sto 0$ in $SOT$.
		\item If $T_i \sto 0$ in $SOT$ and $K \in \coper$, then $T_iK$ and $KT_i$ convergence to $0$ in norm. If $T_i \sto 0$ in $WOT$ then $T_iK \sto 0$ in norm and $KT_i \sto 0$ in $SOT$.
	\end{enumerate}
\end{prop}
\begin{proof}
	For $1)$, since $T_i \sto 0$ in $WOT$, by the Principle of Uniform Boundedness, $\{T_i\}$ is norm boundedness. Since for any $h \in \Hs$, $\norm{S_i h} \sto 0$,
	\begin{equation*}
		\norm{T_iS_i h} \leqslant \norm{T_i} \norm{S_i h} \sto 0
	\end{equation*}
	And $\norm{S_iT_i h} \sto 0$ clearly.
	To prove $2)$, firstly for finite rank $K_n$ and let $h_1,\cdots,h_n$ and $g_1,\cdots,g_n$ be in $\Hs$,
	\begin{equation*}
		K_n = \sum_{i=1}^n h_i \otimes g_i
	\end{equation*}
	That means $\ran{K_n} = \spn{\{h_1, \cdots, h_n\}}$. Let $P_n$ be the projection onto $\ran{K_n}$, then $T_iK_n = T_iP_n$, 
	\begin{equation*}
		T_iK_n = T_iP_n \colon \ran{K_n} \longrightarrow \spn{\{T_ih_1, \cdots, T_ih_n\}}
	\end{equation*}
	And since the $WOT$ and the $SOT$ and the norm topology are same on a finite dimensional space, $T_iK_n \sto 0$ in norm for $T_i \sto 0$ in $WOT$ or in $SOT$. For any $K \in \coper$, there is a sequence of finite rank operators $\{K_n\}$ s.t. $K_n \sto K$ in norm. Since
	\begin{equation*}
		\norm{T_iK} = \norm{T_iK - T_iK_n} + \norm{T_iK_n}
	\end{equation*}
	$T_iK$ convergences to $0$ in norm for $T_i \sto 0$ in $WOT$ or in $SOT$.
	\begin{equation*}
		K_nT_i = \sum_{l=1}^{n} h_n \otimes (\st{T_i}g_n)
	\end{equation*}
	Thus,
	\begin{eqnarray*}
		\norm{K_nT_i} &\leqslant& \sum_{l=1}^{n} \norm{h_n \otimes (\st{T_i}g_n)} \\
		&=& \sum_{l=1}^n \norm{h_n}\norm{\st{T_i} g_n}
	\end{eqnarray*}
	If $T_i \sto 0$ in $SOT$, then $\norm{\st{T_i} g_n} \sto 0$ for all $n$, i.e. $\norm{K_nT_i} \sto 0$. Thus for $K \in \coper$, $KT_i \sto 0$ in norm. If $T_i \sto 0$ in $WOT$, for any $x \in \Hs$,
	\begin{eqnarray*}
		\langle K_n T_i x, K_n T_i x \rangle &=& \langle \sum_{l=1}^n \langle T_i x, h_l \rangle g_l, \sum_{m=1}^n \langle T_i x, h_m \rangle g_m \rangle \\
		&=& \sum_{l=1}^n \sum_{m=1}^n \langle T_i x, h_l \rangle \langle T_i x, h_m \rangle \langle g_l, g_m \rangle
	\end{eqnarray*}
	Therefore, if $T_i \sto 0$ in $WOT$, $\norm{KT_i x} \sto 0$ for $x \in \Hs$, i.e. $KT_i \sto 0$ in $SOT$.
\end{proof}

\subsection{Central Cover}

\begin{defn}
	For a von Neumann algebra $\A$, if $\fml{Z} = \A \cap \A^{'}$ is the center of $\A$, the central cover or central cover of $A \in \A$ is the projection
	\begin{equation*}
		C_A = \inf{\{~C \colon C \in \fml{Z} \text{ is a projection},~AC = A~\}}
	\end{equation*}
\end{defn}
\begin{rem}
	By above proposition, $C_A$ is definitely a projection satisfying
	\begin{equation*}
		AC_A = C_A = A
	\end{equation*}
\end{rem}

\begin{prop}
	Let $\A$ be a von Neumann algebra.
	\begin{enumerate}[label=\arabic*)]
		\item If $A \in \A$, $E = (\ker{A})^{\bot}$ and $F = \clo{\ran{A}}$, then
		\begin{equation*}
			C_A = C_{\st{A}} = C_{\st{A}A} = C_{A\st{A}} = C_E = C_F
		\end{equation*}
		\item If $E$ is a projection in $\A$, then
		\begin{equation*}
			C_E = \clo{\spn{\{Ah \colon h \in E,~ A \in \A\}}}
		\end{equation*}
	\end{enumerate}
\end{prop}
\begin{proof}
	For $1)$, $C_A = C_{\st{A}}$ is trivial by the definition. If $C$ is a central projection s.t. $AC = A$, then $\st{A}AC = \st{A}A$. Conversely, if $\st{A}AC = \st{A}A$, then $\abs{A}C = \abs{A}$ by taking the root. By the Polar Decomposition, $AC = A$. Thus $C_A = C_{\st{A}A}$. \\
	Firstly, $E \in \A$. If $C$ is a central projection s.t. $AC = A$, then 
	\begin{equation*}
		C(\ker{A}) \subset \ker{A} \Rightarrow EC = CE = E \Rightarrow E \leqslant C
	\end{equation*}
	Therefore, $C_E \leqslant C_A$. Conversely, if $C$ is a central projection s.t. $E \leqslant C$, then 
	\begin{equation*}
		AC = AEC = AE = A
	\end{equation*}
	Thus $C_A \leqslant C_E$. Moreover, $C_F = C_E$.
	\item For $2)$, let
	\begin{equation*}
		P = \clo{\spn{\{Ah \colon h \in E,~ A \in \A\}}}
	\end{equation*}
	and clearly $P \in \A^{'}$. Moreover, for any $B \in \A^{'}$ and $h \in E$,
	\begin{equation*}
		BAh = ABh \in P \Rightarrow BP \subset P \Rightarrow P \in \A^{''} = \A
	\end{equation*}
	Therefore, $P$ is a central projection with $E \leqslant P$, thus $C_E \leqslant P$. Conversely, if $C$ is a central projection s.t. $EC = E$, then for $A \in \A$ and $h \in E$,
	\begin{equation*}
		C(Ah) = AC(Eh) = AEh = Ah \Rightarrow CP = P = PC
	\end{equation*}
	Thus $P \leqslant C$, then $P \leqslant C_E$.
\end{proof}

\begin{defn}
	If $\A$ is a con Neumann algebra and $E$ is a projection, defining
	\begin{equation*}
		\A_{E} = \{~A_E \colon A \in \A~\}
	\end{equation*}
	where $A_E = EA|_E$.
\end{defn}
\begin{rem}
	In general, $\A_E$ is not an algebra. However, if $E \in \A$, there is a $*$-isomorphism between $\A_E$ and $E\A E$. If $E \in \A^{'}$, there is a $*$-homomorphism between $\A_E$ and $E\A E$. $E \A E$ is clear a $\st{C}$-subalgebra of $\A$.
\end{rem}

\begin{prop}
	Let $\A$ be a von Neumann algebra.
	\begin{enumerate}[label=\arabic*)]
		\item If $E \in \A^{'}$ is a projection, then $\A_E$ is a von Neumann algebra in $\fml{B}(E)$ and $(\A_E)^{'} = (\A)^{'}_E$.
		\item If $E \in \A$ is a projection, then $\A_E$ is a von Neumann algebra in $\fml{B}(E)$ and $(\A_E)^{'} = (\A)^{'}_E$.
		\item If $\fml{Z}$ is the center of $\A$ and $E$ is a projection in $\A$ or $\A^{'}$, then the center of $\A_E$ is $\fml{Z}_E$.
	\end{enumerate}
\end{prop}
\begin{proof}
	For $1)$, it is clear since $E$ reduces $\A$. 
	\item Assume that $E \in \A$. $\A_E$ and $(\A^{'})_E$ are $*$-algebra clearly. If $B \in \A^{'}$ and $A \in \A$, then for $h \in E$, 
	\begin{equation*}
		B_E A_E h = EBEAh = EAEBh = A_E B_E h
	\end{equation*}
	Therefore, $(\A)^{'}_E \subset (\A_E)^{'}$.
	\item Check: $(\A_E)^{'} \subset (\A)^{'}_E$ \\
	Let $U \in (\A_E)^{'} \subset \fml(B)(E)$ be a unitry, $A_1,\cdots,A_n \in \A$ and $h_1,\cdots,h_n \in E$, then
	\begin{eqnarray*}
		\norm{\sum_{k=1}^n A_k U h_k}^2 &=& \sum_{i,j}^n \langle A_iUh_i,A_jUh_j \rangle = \sum_{i,j}^n \langle (E\st{A_j}A_iE)Uh_i, Uh_j \rangle \\
		&=& \sum_{i,j}^n \langle U(E\st{A_j}A_iE)h_i, Uh_j \rangle = \sum_{i,j}^n \langle E\st{A_j}A_iEh_i, h_j \rangle \\
		&=& \norm{\sum_{k=1}^n A_k h_k}^2
	\end{eqnarray*}
	Therefore, for $C = \clo{\spn{\{Ah \colon h \in E,~ A \in \A\}}}$, there is a partical isometry with the initial space $C$,
	\begin{equation*}
		W(\sum_{k=1}^n A_k h_k) = \sum_{k=1}^n A_k U h_k
	\end{equation*}
	and $WC^{\bot} = 0$. By above proposition $C = C_E$. Therefore, for any $A \in \A$,
	\begin{equation*}
		WAC^{\bot} = WC^{\bot}A = 0 = AWC^{\bot}
	\end{equation*}
	and on $C$,
	\begin{equation*}
		WA \sum_{k=1}^n A_k h_k = W \sum_{k=1}^n A A_k h_k = A \sum_{k=1}^n A_kUh_k = AW\sum_{k=1}^n A_k h_k
	\end{equation*}
	Therefore, $W \in \A^{'}$ with $Wh = Uh$ for $h \in E$. So $W_E = U$. Since any element in $(\A_E)^{'}$ can be the linear combination of four unitaries, $(\A_E)^{'} \subset (\A)^{'}_E$.
	\item Check: $\A_E \subset \fml{B}(E)$ is von Neumann. \\
	Let $T \in (\A_E)^{''}$ and extending $T$ on $\Hs$ by setting $T = 0$ on $E^{\bot}$. If $B \in \A^{'}$, then $B_E \in (\A_E)^{'}$. For $h \in E$,
	\begin{equation*}
		TBh = (TE)Bh = T B_E h = B_E Th = B_E Th =B Th
	\end{equation*}
	And on the other hand,
	\begin{equation*}
		TBE^{\bot} = TE^{\bot}B = 0 = BTE^{\bot}
	\end{equation*}
	Thus $T \in \A^{''} = \A$, and $(\A_E)^{''} = \A_E$.
	\item For $3)$, by above mention, 
	\begin{equation*}
		\A_E \cap (\A_E)^{'} = \A_E \cap (\A)^{'}_E = \fml{Z}_E \qedhere
	\end{equation*}
\end{proof}

\subsection{Kaplansky Density Theorem}

For $\st{C}$-algebra $\A$, let $\mathcal{B}(\A)$ be the norm-closed unit ball in $\A$.

\begin{thm}[Kaplansky Density Theorem]
	If $\B$ is a $\st{C}$-subalgebra of $\oper$ and $\A = \clo{\B}^{SOT}$, then 
	\begin{enumerate}[label = \arabic*)]
		\item $\mathcal{B}(\A)$ is the $SOT$ closure of $\mathcal{B}(\B)$.
		\item $\mathcal{B}(\Rea{\A})$ is the $SOT$ closure of $\mathcal{B}(\Rea{\B})$.
		\item $\mathcal{B}(\A)_{+}$ is the $SOT$ closure of $\mathcal{B}(\B)_{+}$.
		\item the $SOT$ closure of the unitaries in $\B$ contains the unitaries of $\A$.
	\end{enumerate}
\end{thm}

\begin{lem}
	If $f \in C(\R)$ with $f(0)=0$ and there are positive constants $a,b$ s.t. 
	\begin{equation*}
		\abs{f(t)} \leqslant a\abs{t} + b,~ \forall~ t \in \R
	\end{equation*}
	then for any net $\{T_i\} \subset \Rea{\A}$ with $T_i \sto T$ in $SOT$, $f(T_i) \sto f(T)$ in $SOT$.
\end{lem}
\begin{proof}
	For any $g \in C(\R)$, then the map $g \colon \Rea{\oper} \sto \Rea{\oper}$ makes sence. Denote the set of all such $g$ s.t. $g$ is continuous w.r.t. the $SOT$ by $\mathcal{S}$ and let $\mathcal{S}_b$ be set of all norm bounded elements in $\mathcal{S}$. Clearly, $\mathcal{S}_b \mathcal{S} \subset \mathcal{S}$.\\
	Let $e(t) = t / (1+t^2)$ and $A,B \in \Rea{\oper}$,
	\begin{eqnarray*}
		e(A)-e(B) &=& (1+A^2)^{-1}(A(1+B^2)-(1+A^2)B)(1+B^2)^{-1} \\
		&=& (1+A^2)^{-1}(A-B)(1+B^2)^{-1} \\
		&& \negmedspace{} + (1+A^2)^{-1}A(B-A)B(1+B^2)^{-1}
	\end{eqnarray*}
	Then for any $h \in \Hs$,
	\begin{equation*}
		\norm{e(A)-e(B)} \leqslant \norm{(A-B)(1+B^2)^{-1}h} + \norm{(A-B)(1+B^2)^{-1}h}
	\end{equation*}
	Therefore, $e \in \mathcal{S}_b$. Put $e_{\alpha}(t) = e(\alpha)$, then $e_{\alpha}(t) \in \mathcal{S}_b$ for each $\alpha > 0$. Since $\{e_{\alpha} \colon \alpha > 0\}$ can separate the points in $\R \backslash \{0\}$ and $\mathcal{S}_b$ is closed,
	\begin{equation*}
		C_0(\R \backslash \{0\}) \subset \mathcal{S}_b
	\end{equation*}
	If $f$ satisfy above assumption, $t \sto f(t) / (1+t^2)$ belongs to $C_0(\R \backslash \{0\})$, and hence to $\mathcal{S}_b$. Since the identity funtion also belongs to $\mathcal{S}$ and the condition $f$ satisfies,
	\begin{equation*}
		t \longrightarrow \frac{tf(t)}{1+t^2} \in \mathcal{S}_b
	\end{equation*}
	And thus 
	\begin{equation*}
		t \longrightarrow \frac{t^2f(t)}{1+t^2} \in \mathcal{S}
	\end{equation*}
	Therefore,
	\begin{equation*}
		f(t) = \frac{t^2f(t)}{1+t^2} + \frac{f(t)}{1+t^2} \in \mathcal{S} \qedhere
	\end{equation*}
\end{proof}

Then by using above lemma and constructing some special functions, we can easily prove the Kaplansky Theorem.

\begin{proof}[proof of Kaplansky Theorem]
	Fix a $A \in \mathcal{B}(\Rea{\A})$ and a net $\{B_i\} \subset \Rea{\B}$ s.t. $B_i \sto A$ in $SOT$.
	\item For $2)$, if $f = (t \wedge 1) \vee (-1)$, then $f$ satisfies above lemma. Therefore,
	\begin{equation*}
		f(B_i) \longrightarrow f(A) =  A \text{ in } SOT
	\end{equation*}
	but $f(B_i) \in \mathcal{B}(\Rea{\B})$.
	\item Similarly, $3)$ holds by constructing the function $f(t) = (t \wedge 1) \vee 0$.
	\item Let $U \in \A$ be a unitary. Then by the functional calculus, there is a $A \in \Rea{\A}$ s.t.
	\begin{equation*}
		U = \exp{i A} = \cos{A} + i \sin{A}
	\end{equation*}
	Then let $\{B_i\} \subset \Rea{\B}$ be a net s.t. $B_i \sto A$ in $SOT$. Since both $\sin{t}$ and $\cos{t} - 1$  satisfy above lemma, 
	\begin{equation*}
		U_i =  \cos{B_i} + i \sin{B_i} \longrightarrow U \text{ in } SOT
	\end{equation*}
	and $U_i \in \B$ is a unitary.
	\item For $1)$, fix a $A \in \mathcal{B}(\A)$ and consider the $\st{C}-$algebras $M_2(\B)$ and $M_2(\A)$, then $M_2(\A)$ is the $SOT$-closure of $M_2(\B)$.
	\begin{equation*}
		\left(
			\begin{array}{cc}
				0 & A \\
				\st{A} & 0
			\end{array}
		\right)
		\in \mathcal{B}(\Rea{M_2(\A)})
	\end{equation*}
	therefore, by $1)$ there is a net $\{T_i\} \subset \mathcal{B}(\Rea{M_2(\B)})$ converges to above element in $SOT$. Put
	\begin{equation*}
		T_i = \left(
			\begin{array}{cc}
				X_i & B_i \\
				\st{B}_i & Y_i
			\end{array}
		\right)
	\end{equation*}
	Since $\norm{T_i} \leqslant 1$, $\norm{B_i} \leqslant 1$. Clearly, $B_i \sto A$  in $SOT$.
\end{proof}

\section{SOT on von Neumann Algebras}

\subsection{Pedersen Up-Down Theorem}

The order of all self-adjoint operators in von Neumann algebras can provide another important property of von Neumann algebras.

\begin{defn}
	For any subset $\mathcal{S} \subset \Rea{\oper}$, define
	\begin{eqnarray*}
		\mathcal{S}^{\sigma} &=& \{~T \colon \exists \{S_n\} \subset \mathcal{S}  \text{  increasing sequence, } S_n \sto T \text{ in } SOT~\} \\
		\mathcal{S}_{\sigma} &=& \{~T \colon \exists \{S_n\} \subset \mathcal{S}  \text{  decreasing sequence, } S_n \sto T \text{ in } SOT~\}
	\end{eqnarray*}
\end{defn}
\begin{rem}
	Note that $\mathcal{S}  \subset \mathcal{S}^{\sigma}$ and $\mathcal{S}_{\sigma} = -(-\mathcal{S})^{\sigma}$.
\end{rem}

\begin{lem}
	If $A$ and $B$ are two positive operators with $A  \leqslant B$, then
	\begin{equation*}
		A(1+A)^{-1} \leqslant B(1+B)^{-1}
	\end{equation*}
\end{lem}
\begin{proof}
	Because of $A(1+A)^{-1} = 1  - (1+A)^{-1}$ and the fact if $0 \leqslant A \leqslant B$, $B^{-1} \leqslant A^{-1}$, the lemma holds.
\end{proof}

\begin{lem}
	Let $\Hs$ be separable. If $\B$ is a $\st{C}$-algebra and $\A = \clo{\B}^{SOT}$ and $P$ is a projection in $\A$, then for each sequence of unit $\{h_n\}$ in $\Hs$ there is a $A \in ((\mathcal{B}(\B_+))^{\sigma})_{\sigma}$ s.t.
	\begin{equation*}
		A(1-P)h_n = 0  = (1-A)Ph_n,~ \forall  n \geqslant 1
	\end{equation*}
\end{lem}
\begin{proof}
	By the Kaplansky Density Theorem and the sparability of $\Hs$, there is a sequence $\{C_n\}$ in $\mathcal{B}(\B_+)$ s.t. $C_n \sto P$ in $SOT$. Then it can assume that for $1 \leqslant j \leqslant  n$,
	\begin{equation*}
		\norm{C_n(1-P)h_j} < \frac{1}{n2^n},~ \norm{(1-C_n)Ph_j} < \frac{1}{n}
	\end{equation*}
	For $n < m$ define
	\begin{equation*}
		A_{nm} = \left(1+ \sum_{k=n}^m kC_k \right)^{-1}\sum_{k=n}^m kC_k
	\end{equation*}
	Then each $A_{nm} \in \mathcal{B}(\B_+)$ and $A_{nm} \leqslant \sum_{k=n}^m kC_k$. Thus for $j \leqslant n$,
	\begin{equation*}
		\langle A_{nm}(1-P)h_j, (1-P)h_j \rangle \leqslant \sum_{k=n}^m \frac{1}{2^k} < \frac{1}{2^{n-1}}
	\end{equation*}
	Since $\sum_{k=n}^m kC_k \geqslant mC_m$, by above lemma
	\begin{equation*}
		A_{nm} \leqslant (1+mC_m)^{-1} (mC_m) \Rightarrow 1-A_{nm} \leqslant (1+mC_m)^{-1}
	\end{equation*}
	By the fact for $0 \leqslant t \leqslant 1$, $(1+mt)^{-1} \leqslant (1+m)^{-1}(1+m(1-t))$ and $C_m \in \mathcal{B}(\B_+)$, 
	\begin{equation*}
		1- A_{nm} \leqslant (1+m)^{-1}(1+m(1-C_m))
	\end{equation*}
	Then it impilies that
	\begin{eqnarray*}
		\langle (1-A_{nm})Ph_j,Ph_j \rangle &\leqslant& \frac{1}{m} \langle (1+m(1-C_m))Ph_j,Ph_j \rangle \\
		&\leqslant& \frac{1}{m} (\norm{Ph_j}+m\norm{(1-C_m)Ph_j}) \\
		&\leqslant& \frac{2}{m}
	\end{eqnarray*}
	Fix a $n$, $\{A_{nm} \colon m > n\}$ is increasing by above lemma. Since it is bounded, there is a $A_n$  s.t. $A_{nm} \sto A_n$ fin $SOT$ and $A_n \in (\mathcal{B}(\B_+))^{\sigma}$. On the other hand, when $n+1 < m$, also by above lemma, it implies that $A_{n+1,m} \leqslant A_{nm}$, thus $A_{n+1} \leqslant A_n$. Thus there is a operator $A$ s.t. $A_n \sto A$ in $SOT$, and thus $A \in ((\mathcal{B}(\B_+))^{\sigma})_{\sigma}$. Moreover, by above
	\begin{eqnarray*}
		\langle A_n(1-P)h_j, (1-P)h_j \rangle &\leqslant& \frac{1}{2^{n-1}} \\
		\langle (1-A_n)Ph_j,Ph_j \rangle &\leqslant& 0
	\end{eqnarray*}
	As $n \sto \infty$, $A(1-P)h_n = 0  = (1-A)Ph_n$ for all $n$.
\end{proof}

\begin{thm}[Pedersen Up-Down Theorem]
	If $\Hs$ is a separable,  $\B$ is a \Cs contained in $\oper$, and $\A = \clo{\B}^{SOT}$, then
	\begin{equation*}
		\mathcal{B}(\A_+) = ((\mathcal{B}(\B_+))^{\sigma})_{\sigma} \text{ and } \Rea{\A} = (\Rea{\B})^{\sigma})_{\sigma}
	\end{equation*}
\end{thm}
\begin{proof}
	Firstly, by above lemma, for any projection $P \in \A$ and a dense subset in $\{h \in \Hs \colon \norm{h} =1\}$, there is a $A \in ((\mathcal{B}(\B_+))^{\sigma})_{\sigma}$ s.t. $A(1-P) = 0  = (1-A)P$.\\
	For $A \in (\mathcal{B}(\A_+)$ with the spectral measure $E$ and $k \leqslant 1$,
	\begin{equation*}
		P_k = E \left(\bigcup_{j=1}^{2^k-1}(\frac{j}{2^k}, \frac{j+1}{2^k}] \right)
	\end{equation*}
	then by the integral, $A  =  \sum_{k=1}^{\infty}  2^{-k}P_k$ converging by the norm. Then for each $k$, there is a decreasing sequence $\{Z_{kn}\}$ in $(\mathcal{B}(\B_+))^{\sigma}$ s.t. $Z_{kn} \sto  P_k$ in $SOT$, define
	\begin{equation*}
		T_n = \sum_{k=1}^n \frac{1}{2^k} Z_{kn} + \frac{1}{2^n} \geqslant A
	\end{equation*}
	Since $(\mathcal{B}(\B_+))^{\sigma}$ is convex, $T_n \in (\mathcal{B}(\B_+))^{\sigma}$. Moreover, because of $Z_{+1,n+1} < 1$,
	\begin{equation*}
		T_{n} - T_{n+1}  = \sum_{k=1}^n \frac{1}{2^k} (Z_{kn} -  Z_{k,n+1}) + \frac{1}{2^n} (\frac{1}{2^{n+1}}Z_{n+1,n+1} + \frac{1}{2^{n+1}}) \leqslant 0
	\end{equation*}
	Thus $\{T_n\}$ is decreasing and there is some $T \in ((\mathcal{B}(\B_+))^{\sigma})_{\sigma}$ with $T \geqslant A$ s.t. $T_n \sto T$ in $SOT$.
	\item Check: $A = T \in ((\mathcal{B}(\B_+))^{\sigma})_{\sigma}$ \\
	Since $0 \leqslant Z_{kn} - P_k \leqslant 1$, the series $\sum_k 2^{-k}(Z_{kn}-P_k)$ is convergent. Then for any unit $h \in \Hs$ and $\varepsilon > 0$, chose $N$ s.t. $2^{-N} < \varepsilon$, then there is an $n_0$ s.t. 
	\begin{equation*}
		0 \leqslant \langle (Z_{kn}-P_k) h,h \rangle < \frac{\varepsilon}{2}, \text{ for } 1 \leqslant k \leqslant N \text{ and } n \geqslant n_0
	\end{equation*}
	Therefore,
	\begin{equation*}
		\lim_{n \sto \infty} \sum_{k=1}^{\infty} \frac{1}{2^k} \langle (Z_{kn}-P_k) h,h \rangle = 0,~ \forall h \in \Hs
	\end{equation*}
	However,
	\begin{equation*}
		0 \leqslant T_n - A = T_n - \sum_{k=1}^{\infty} \frac{1}{2^k} P_k \leqslant \frac{1}{2^n} + \sum_{k=1}^{\infty} \frac{1}{2^k}(Z_{kn}-P_k)
	\end{equation*}
	Therefore, $T_n \sto A$ in $WOT$, but $T_n \sto T$ in $SOT$, thus $T = A$. 
	\item Check: $\Rea{\A} = (\Rea{\A})^{\sigma})_{\sigma}$ \\
	Fix $A \in \Rea{\A}$ and let $\alpha = -\norm{A}$.  Therefore, $A+\alpha \leqslant 0$. Let $\beta = \norm{A+\alpha}$, so $T=\beta^{-1}(A+\alpha) \in \mathcal{B}(\A_+)$ and $A = \beta T -\alpha$. But $\beta T \in ((\mathcal{B}(\B_+))^{\sigma})_{\sigma}$  and $-\alpha \in \mathcal{B}(\B)$. Thus $A  \in ((\mathcal{B}(\Rea{\B}))^{\sigma})_{\sigma}$.
\end{proof}

The Up-Down Theorem can reveal the basic property of general von Neumann algebras.

\begin{thm}
	If $\A$ is a \Cs contained in $\oper$, then $\A$ is $weak^*$-closed if and only if it contains the supremum of every bounded increasing net of self-adjoint operators in the algebra.
\end{thm}
\begin{proof}
	$\A$ is $weak^*$-closed, thus $\A$ is  a von Neumann algebra. Therefore, by above proposition $\A$ has this property. Assume that $\A$ is a \Cs and if $\{A_i\}$ is an increasing net in $\Rea{\A}$ with $A_i \sto A$ in $SOT$, then $A \in \A$. Then by taking the nagatives of the increasing parts, it is also true for that if $\{A_i\}$ is an decreasing net in $\Rea{\A}$ with $A_i \sto A$ in $SOT$, then $A \in \A$. Therefore, 
	\begin{equation*}
		\Rea{\A} = ((\mathcal{B}(\Rea{\A}))^{\sigma})_{\sigma}
	\end{equation*}
	Then if $\Hs$ is separable, by the Up-Down Theorem $\A$ is $weak^*$-closed. \\
	For the general case, let $\B = \clo{\A}^{SOT}$. It is sufficient to show that any projection contained in $\B$ is contained in $\A$. Let $P$ be the projection in $\B$ and $g \in P$ and $h \in P^{\bot}$. Then by above lemma, there is an $T \in \A_+$ s.t. $Tg = g$ and $Th = 0$. Let $R$ be projection onto $\clo{\ran{T}}$ and $Rg = g$ and $Rh = 0$, moreover $R \in \A$ because of the hypothesis and the Spectral Theorem. For any $g \in P$ and $h \in P^{\bot}$, denote the $R$ by $R_{gh}$. Fix $g \in P$ and 
	\begin{equation*}
		\fml{F} = \{~h_1, h_2, \cdots, h_m\} \subset P^{\bot}
	\end{equation*}
	then $R_{\fml{F}} = R_{gh_1} \wedge R_{gh_2} \wedge \cdots \wedge R_{gh_m} \in \A$. And $\{R_{\fml{F}}\}$ is a decreasing net and thus $R_g = \lim-SOT R_{\fml{F}} \in \A$. And since $R_g P^{\bot} = 0$, $R_g \leqslant P$ for all $g \in P$. Then let
	\begin{equation*}
		\fml{G} = \{~g_1, g_2, \cdots, g_n\} \subset P
	\end{equation*}
	$R_{\fml{G}} = R_{g_1} \vee R_{g_2} \vee \cdots \vee R_{g_n} \in \A$. And $\{R_{\fml{G}}\}$ is an increasing net and thus $R = \lim-SOT R_{\fml{F}} \in \A$. Therefore, $R = P \in \A$.
\end{proof}

\subsection{Normal Homomorphisms}

\begin{defn}
	If $\A$ and $\B$ are von Neumann algebras and $\rho \colon \A \sto \B$ is a linear positive map, then $\rho$ is normal if for any increasing net $\{A_i\}$ in $\A$ with $A_i \sto A$ in $SOT$, $\rho(A_i) \sto \rho(A)$ in $SOT$.
\end{defn}
\begin{rem}
	In particular, if $E \in \A$, $A \mapsto EAE$ is a normal homomorphism. The normal homomorphsims should be postive linear maps that are continuous with respect to the $SOT$, but in above definition, there is an extra condition. In fact, this extra condition has an equivalent expression.
\end{rem}

If $\A$ is a von Neumann algebra, the $\A$ is $weak^*$-closed, therefore $\A$ is the dual space of some Banach space. Let $\A_*$ be the space of all $weak^*$-continuous fuctionals on $\A$, then $\A  = (\A_*)^*$. And also 
\begin{equation*}
	\A_* \cong \toper / \A_{\bot},
\end{equation*}
where $\A_{\bot}$ is the space of all trace class annihilating the $\A$. Therefore, for any $L \in \A_*$, there is $T \in \toper$ s.t. for all $A \in \A$, 
\begin{equation*}
	L(A) = \tr{(AT)}
\end{equation*}
In fact, if $L$ is positive, then $T$ can be chosen as a positive element.

\begin{lem}
	If $\phi$ and $\psi$ are positive functionals on a von Neumann algebra $\A$ s.t. there is an operator $A \in \A_+$ with $\phi(A)  < \psi(A)$, then there is an operator $B \in \A_+$ s.t. $B \leqslant A$ and $\phi(T) < \psi(T)$ for all $T \in \A$ with $0 <  T \leqslant  B$. If $A$ is projection, then $B$ can be chosen as a projection.
\end{lem}
\begin{proof}
	Let $\mathcal{C}$ be the set of all $C \in \A_+$ s.t. $C \leqslant A$ and $\phi(C) \leqslant \psi(C)$. And give $\mathcal{C}$ with the usual order. Then if $\{C_i\}$ is the chain in $\mathcal{C}$ and let $C = \sup_i C_i \in \A_+$  and $C \leqslant A$. Since $\phi$ and $\psi$ is positive, $C \in \mathcal{C}$. Then by the Zorn's Lemma, there is a maximal element $C$ in $\mathcal{C}$. Put $B = A-C$. If $T \in \A_+$ with $0 < T \leqslant B$, by the maximality of $C$, $T \notin \mathcal{C}$, thus $\phi(T) < \psi(T)$. Moreover, if $A$ is a projection, $B \leqslant A$ implies that $\clo{\ran{B}} \leqslant A$. If $B = \int_0^1 t dP(t)$ be the spectral decomposition, then for any $\varepsilon >  0$, $P = P([\varepsilon,1])$. Then $P$ satisfies above condition.
\end{proof}

\begin{thm}
	If $\psi$ is a positive linear functional on the von Neumann algebra $\A$, then the following statements are equvilaten.
	\begin{enumerate}[label = \arabic*)]
		\item $\psi$ is normal.
		\item If $\{E_i\}$ is a pairwise orthogonal family of projections in $\A$, then 
		\begin{equation*}
			\psi(\sum E_i) = \sum \psi(E_i)
		\end{equation*}
		\item $\psi$ is $weak^*$-continous.
		\item There is a positive trace class operator $C$ s.t. $\psi(A) = \tr{(AT)}$ for all $A \in \A$.
	\end{enumerate}
\end{thm}
\begin{proof}
	Without loss the generality, assuming that $\psi(1) = 1$.
	\item $1) \Rightarrow 2)$: It is trivial.
	\item $2) \Rightarrow 3)$: Since $0 \leqslant \psi(E) \leqslant 1$ for any non-zero projection $E$, for any $h \in \Hs$,
	\begin{equation*}
		\psi(E) \leqslant \langle Eh,h \rangle
	\end{equation*}
	Therefore, by preceding lemma, there is a projection $F$ with $0 \leqslant F \leqslant E$ s.t.
	\begin{equation*}
		\psi(T) \leqslant \langle hT,h \rangle,~ \forall 0 \leqslant  T \leqslant F
	\end{equation*}
	Since $FAF \leqslant \norm{A}F$, 
	\begin{equation*}
		\psi(FAF) \leqslant \norm{A} \langle FAFh,h \rangle
	\end{equation*}
	Then by the CBS inequality, for any $A \in \mathcal{B}(\A)$
	\begin{equation*}
		\abs{\psi(AF)}^2 \leqslant \psi(F\st{A}AF) \leqslant \norm{A}^2 \langle F\st{A}AFh,h \rangle = \norm{A}^2 \norm{AFh}^2
	\end{equation*}
	Therefore, $\psi(\cdot F) \colon \A \sto \C$ is $SOT$-continuous. \\
	Let $\{E_i\}$ be the maximal family of pairwise orthogonal projections in $\A$ s.t. $\psi(\cdot E_i)$ is $SOT$-continous on $\A$. Let $E = \sum_i E_i$. Clearly, $E = 1$. If $E \neq 1$, then there is a projection $F \leqslant E^{\bot}$ s.t. $\psi(\cdot F)$ is $SOT$-continous on $\A$, which contradicts the maximality of $\{E_i\}$. Then by $2)$,
	\begin{equation*}
		\psi(\sum_i E_i) = \psi(1) = 1
	\end{equation*}  
	Therefore, for any $\varepsilon > 0$, there is a finite set $I_0$ of index s.t. for any $J$ with $I_0 \subset J$, and let $P_J = \sum_J E_j$,
	\begin{equation*}
		\psi{P_J^{\bot}} = 1- \psi(P_J) < \varepsilon
	\end{equation*}
	Then for any $A \in \mathcal{B}(\A)$, 
	\begin{equation*}
		\abs{\psi(AP_J^{\bot})}^2  \leqslant \psi(A\st{A})\psi(P_J) < \varepsilon
	\end{equation*}
	And thus for any $J$ with $I_0 \subset J$,
	\begin{equation*}
		\norm{\psi - \psi(\cdot P_J)} = \sup{\{~\abs{\psi(AP_J)} \colon A \in \mathcal{B}(\A)~\}} < \sqrt{\varepsilon}
	\end{equation*}
	Since $\psi(\cdot P_J) \in \A_*$, $\psi \in \A_*$.
	\item $3) \Rightarrow 4)$: Firstly, assume that $\psi(A) = \langle Ag,h \rangle$ for some $g,h \in \Hs$. \\
	Check: $\psi(A) = \langle Af, f \rangle$ for some $f \in \Hs$. \\
	By the fact that $A \in \A_+$, $\langle Ag,h \rangle \leqslant 0$, 
	\begin{eqnarray*}
		4\langle Ag,h \rangle &=& \langle A(g+h),g+h \rangle - \langle A(g-h),g-h \rangle \\
		&\leqslant& \langle A(g+h),g+h \rangle
	\end{eqnarray*}
	Let $\phi(A) = \frac{1}{4}\langle A(g+h),g+h \rangle$, then $\psi(A) \leqslant \phi(A)$. By the \textbf{Proposition} \ref{prop17} in the subsection \textbf{3.2.7}, there is a $T \in \A^{'}$ with $0 \leqslant T \leqslant 1$ s.t. 
	\begin{equation*}
		\psi(A) =  \langle AT(g+h),g+h \rangle
	\end{equation*}
	Let $f = T^{\frac{1}{2}}(g+h)$, then $\psi(A) = \langle Af, f \rangle =  \tr{(Af \otimes f)}$, where $f \otimes f$ is a positive trace class. \\
	For general case, since $\psi \in \A_*$, by above mention, there is a $D \in \coper$ s.t. $\psi(A) = \tr{(AD)}$. By the \textbf{Theorem} \ref{thm15} in the subsection \textbf{5.3.3}, there are $g, h \in \Hs^{(\infty)}$  s.t.
	\begin{equation*}
		\psi(A)  = \tr{(AD)} = \langle A^{(\infty)}g,h \rangle = \langle A^{(\infty)}f,f \rangle
	\end{equation*} 
	Then there is a $C \in \coper$, s.t.
	\begin{equation*}
		\langle A^{(\infty)}f,f \rangle = \tr{(AC)}
	\end{equation*}
	And in fact, if $f = (f_i) \in \Hs^{(\infty)}$, $C = \sum_i f_i \otimes f_i \geqslant 0$.
	\item $4) \Rightarrow 1)$: It is trivial by definition.
\end{proof}

Then by using the above theorem, the equivalent statement as the following corollary shows can make the definition of normal homomorphism more nature.

\begin{cor}
	If $\A$ and $\B$ are von Neumann algebras and $\rho \colon \A \sto \B$ is a positive linear map, then $\rho$ is normal if and only if it is $weak^*$-continuous. 
\end{cor}
\begin{proof}
	If $\rho$ is $weak^*$-continuous, then it is clearly normal. Conversely, if $\rho$ is normal and $\phi \in \B_*$ such that $\phi$ is positive, then $\phi \circ  \rho$ is a normal functional on $\A$. By preceding theorem and the fact that all element in $\B_*$ can be linear combination of four positive elments, $\phi \circ  \rho$ is $weak^*$-continuous for all $\phi \in \B_*$. Therefore, by the Hahn-Banach Theorem, $\rho$ is $weak^*$-continuous.
\end{proof}

\begin{prop}
	Every $*$-isomorphism between von Neumann algebras is normal.
\end{prop}
\begin{proor}
	Let $\rho \colon \A \sto \B$ be $*$-isomorphism between von Neumann algebras $\A$ and $\B$. If $\{\A_i\}$ is an increasing net of self-adjoint operators in $\A$ with $A = \sup_i A_i$, then $\{\rho(\A_i)\}$ is an increasing net in $\B$ with $\rho(\A_i) \leqslant \rho(A)$ for all $A_i$. Thus 
	\begin{equation*}
		B = \sup_i  \rho(A_i) \leqslant \rho(A)
	\end{equation*}
	Conversely, $A_i = \rho^{-1}\rho(A_i)  \leqslant \rho^{-1}(B)$ for all $A_i$. That means $A \leqslant \rho^{-1}(B)$. Therefore, $A =  \rho(B)$.
\end{proor}

\subsection{Ideals}

Rather than considering norm closed ideals in $\st{C}$-algebras, in von Neumann algebras, $WOT$-closed ideal should be paid more attention on, or equivalently, $weak^*$-closed ideal.

\begin{thm}
	Let $\A$ be a von Neumann algebra.
	\begin{enumerate}[label=\arabic*)]
		\item $\B$ is a $weak^*$-closed hereditary subalgebra of $\A$ if and only if there is a unique projection $P$ in $\A$ s.t. $\B = P \A P$.
		\item $\I$ is a $weak^*$-closed left ideal of $\A$ if and only if there is a unique projection $P$ in $\A$ s.t. $\I = \A P$.
		\item $\I$ is a $weak^*$-closed ideal of $\A$ if and only if there is a unique central projection $P$ in $\A$ s.t. $\I = \A P = P \A$.
	\end{enumerate}
\end{thm}
\begin{proof}
	For $1)$, if $\B = P \A P = \A P \cap P \A = \A P \cap \st{(\A P)}$, $\B$ is hereditary by the fact that $\A P$ is a norm closed left ideal and \textbf{Theorem} \ref{thm6} in the subsection \textbf{3.2.5}. And clearly, $\B$ is $weak^*$-closed subalgebra. Conversely, if $\B$ is a $weak^*$-closed hereditary subalgebra, then by the \textbf{Corollary} \ref{cor12} in the subsection \textbf{6.1.1}, there is a central projection $P \in \B$ s.t. $\B = P \B = \B P$, thus $\B \subset P \A P$. If $A \in \A_+$, then $PAP  \leqslant \norm{A} P \in \B$. Since $\B$ is hereditary, $PAP \in \B$. Therefore, $\B =  P \A P$. The uniqueness of $P$ is because any such $P$ is the identity for $\B$.
	\item For $2)$, if $\I$ is a $weak^*$-closed left ideal of $\A$, then $\B = \I \cap \st{\I}$ is a $weak^*$-closed hereditary subalgebra. Then there is a projection $P \in \B \subset \I$ s.t. $\B =  P \A P$. Thus $\A P \subset \I$. Conversely, if $T \in \I$, then $\st{T}T \in \B_+ = \I_+$. Therefore, $\st{T}T P = \st{T}T$ and so $\abs{T}P = \abs{T}$. By the Polar Decomposition, $T = TP \in \A P$. Hence, $\I = \A P$. And the unqueness of $P$ is because the uniqueness of $P$ in $\B$.
	\item For $3)$, That $\I$ is a $weak^*$-closed ideal of $\A$, also norm closed, implies that $\I = \st{I}$. Then by $1)$ and $2)$, there is a projection $P \in \I$ s.t. $\I = \A P = P \A$. Moreover, for any $A \in \A$,
	\begin{equation*}
		PA = P(PA) = P (PAP) = (PA) P = AP
	\end{equation*}
	Therefore, $P$ is in the center of $\A$.
\end{proof}

For the ideal in a von Neumann algebra, it is always assumed as the $weak^*$-closed ideal. Thus, there is a direct corollary of above theorem about the simple von Neumann algbras.

\begin{cor}
		A von Neumann algebra $\A$ is simple if and only if 
		\begin{equation*}
			\fml{Z} = \A \cap \A^{'} = \C I
		\end{equation*}
\end{cor}

\begin{defn}
	A von Neumann algebra $\A$ is called factor, if 
	\begin{equation*}
		\fml{Z} = \A \cap \A^{'} = \C I
	\end{equation*}
\end{defn}
\begin{rem}
	By above corollary, factors are simple von Neumann algebras.
\end{rem}




















