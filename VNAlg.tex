\chapter{Von Neumann Algebras}

Abelian von Neuman algebras can be explicitly described as $L^{\infty}(\mu)$. For general von Neumann algebras, they have more complex structure but more interesting properties. Firstly, we have seen the relation between the $WOT$ and the ultraweak topology, therefore, the ultraweak topology can provide some different things of the von Neumann algebra. Then, since the weak closedness of von Neumann algebras, all projections in a von Neumann algebra can completely generate the whole algebra. Thus projections are fundamental elements in the von Neumann algebra. Researching the properties of them and classifying them can help us to classify general von Neumann algebras. By the properties of the projections contained in a von Neumann algebras, general von Neumann algebras can be classified as $5$ different categories and we can see that the abelian one is belong to what category.

\section{General von Neumann Algebras}

\subsection{Elementary Properties}

\begin{prop}
	If $\{A_i\}$ is an incresing net of self-adjoint operators on $\Hs$ s.t. $\sup_i{\norm{A_i}} < \infty$, then there is an operator $A$ s.t. the following hold.
	\begin{enumerate}[label = \arabic*)]
		\item $A_i \leqslant A$ for all $i$ and if $B$ is any self-adjoint operator with $A_i \leqslant B$ for all $i$, then $A \leqslant B$. Thus $A = \sup_i A_i$.
		\item $A_i \sto A$ in $WOT$.
		\item $A_i \sto A$ in $SOT$.
		\item $A_i \sto A$ in $weak^{*}$.
	\end{enumerate}
\end{prop}
\begin{proof}
	Without loss of generality, assume all $A_i$ are positive.\\
	For any $h \in \Hs$, $\{\langle A_i h, h \rangle\}$ is an increasing positive sequence and bounded by $\alpha \norm{h}^2$ for $\alpha = \sup_i{\norm{A_i}}$. Then
	\begin{equation*}
		F(h,h) = \lim_i \langle A_i h, h \rangle,~ \abs{F(h,h)} \leqslant \alpha \norm{h}^2
	\end{equation*}
	Clearly, $F$ is a sesquilinear and by the Polar Identity, 
	\begin{equation*}
		\abs{F(g,h)} \leqslant \alpha \norm{g} \norm{h}
	\end{equation*}
	Then there is a $A \in \oper$ s.t. 
	\begin{equation*}
		\langle Ag,h \rangle = F(g,h) = \lim_i \langle A_i g, h \rangle
	\end{equation*}
	Therefore, $A$ is positive, $A \sto A_i$ in $WOT$, $A_i \leqslant A$ and $\norm{A} \leqslant \alpha$. Then
	\begin{equation*}
		0 \leqslant A - A_i \leqslant A \Rightarrow (A-A_i)^{\frac{1}{2}} \leqslant A^{\frac{1}{2}},~ \forall~ i
	\end{equation*}
	If $h \in \Hs$, then
	\begin{eqnarray*}
		\norm{(A-A_i)h}^2 &\leqslant& \norm{(A-A_i)^{\frac{1}{2}}}^2 \norm{(A-A_i)^{\frac{1}{2}}h}^2 \\
		&\leqslant& \norm{A} \norm{(A-A_i)^{\frac{1}{2}}h}^2 \\
		&=& \norm{A} \langle (A-A_i)h,h  \rangle \sto 0
	\end{eqnarray*}
	Thus $A_i \sto A$ in $SOT$. \\
	Finally, to see that $A_i \sto A$ in $weak^{*}$, consider $\{A_i^{(\infty)}\}$, and by $2)$, there is a $A^{(\infty)}$ s.t. $A_i^{(\infty)} \sto A^{(\infty)}$ in $WOT$, therefore, $A_i \sto A$ in $weak^{*}$.
\end{proof}

\begin{cor}
	If $\A$ is a $\st{C}$-algebra contained in $\oper$ but without the identity and $\A$ is $weak^{*}$-closed, then there is a projection $P$ in $\A$ s.t.
	\begin{equation*}
		A = PA = AP,~ \forall~ A \in \A
	\end{equation*}
	Therefore, $P$ is the identity for $\A$ and $\A|_{P}$ is a von Neumann algebra.
\end{cor}
\begin{proof}
	There is an approximate identity $\{A_i\}$. By above proposition, 
	\begin{equation*}
		P = \sup_i A_i \in \A, \text{ and } A_i \sto P \text{ in } weak^{*}
	\end{equation*}
	Then for any $A \in \A$, $\norm{AA_i - A} \sto 0$ and $AA_i \sto AP$ in $weak^{*}$. Thus $A = AP =PA$.
\end{proof}

\begin{prop}
	\begin{enumerate}[label = \arabic*)]
		\item If $\A$ is a von Neumann algebra and $\fml{E}$ is a family of projections in $\A$, then $\bigvee \fml{E}$ and $\bigwedge \fml{E}$ belong to $\A$.
		\item If $\A$ is a von Neumann algebra and $1 \leqslant d \leqslant \infty$, then $\A^{(d)}$ is a von Neumann algebra and 
		\begin{equation*}
			(\A^{(d)})^{'} = \{~T=(T_{ij}) \in \fml{B}(\Hs^{(d)}) \colon T_{ij} \in \A^{'} ~\}
		\end{equation*}
		\item If $\A_i$ is a von Neumann algebra in $\fml{B}(\Hs_i)$ for all $i$ and $\Hs = \bigoplus_{i} \Hs_i$, then
		\begin{equation*}
			\A = \bigoplus_{i} \A_i = \{~ \bigoplus_i A_i \colon A_i \in \A_i,~ \sup_i{\norm{A_i}} < \infty~\}
		\end{equation*}
		is a von Neumann algebra.
	\end{enumerate}
\end{prop}

\begin{prop}
	Let $\{T_i\}$ and $\{S_i\}$ be nets in $\oper$.
	\begin{enumerate}[label=\arabic*)]
		\item If $T_i \sto 0$ in $WOT$ and $S_i \sto$ in $SOT$, then $S_iT_i,T_iS_i \sto 0$ in $SOT$.
		\item If $T_i \sto 0$ in $SOT$ and $K \in \coper$, then $T_iK$ and $KT_i$ convergence to $0$ in norm. If $T_i \sto 0$ in $WOT$ then $T_iK \sto 0$ in norm and $KT_i \sto 0$ in $SOT$.
	\end{enumerate}
\end{prop}
\begin{proof}
	For $1)$, since $T_i \sto 0$ in $WOT$, by the Principle of Uniform Boundedness, $\{T_i\}$ is norm boundedness. Since for any $h \in \Hs$, $\norm{S_i h} \sto 0$,
	\begin{equation*}
		\norm{T_iS_i h} \leqslant \norm{T_i} \norm{S_i h} \sto 0
	\end{equation*}
	And $\norm{S_iT_i h} \sto 0$ clearly.
	To prove $2)$, firstly for finite rank $K_n$ and let $h_1,\cdots,h_n$ and $g_1,\cdots,g_n$ be in $\Hs$,
	\begin{equation*}
		K_n = \sum_{i=1}^n h_i \otimes g_i
	\end{equation*}
	That means $\ran{K_n} = \spn{\{h_1, \cdots, h_n\}}$. Let $P_n$ be the projection onto $\ran{K_n}$, then $T_iK_n = T_iP_n$, 
	\begin{equation*}
		T_iK_n = T_iP_n \colon \ran{K_n} \longrightarrow \spn{\{T_ih_1, \cdots, T_ih_n\}}
	\end{equation*}
	And since the $WOT$ and the $SOT$ and the norm topology are same on a finite dimensional space, $T_iK_n \sto 0$ in norm for $T_i \sto 0$ in $WOT$ or in $SOT$. For any $K \in \coper$, there is a sequence of finite rank operators $\{K_n\}$ s.t. $K_n \sto K$ in norm. Since
	\begin{equation*}
		\norm{T_iK} = \norm{T_iK - T_iK_n} + \norm{T_iK_n}
	\end{equation*}
	$T_iK$ convergences to $0$ in norm for $T_i \sto 0$ in $WOT$ or in $SOT$.
	\begin{equation*}
		K_nT_i = \sum_{l=1}^{n} h_n \otimes (\st{T_i}g_n)
	\end{equation*}
	Thus,
	\begin{eqnarray*}
		\norm{K_nT_i} &\leqslant& \sum_{l=1}^{n} \norm{h_n \otimes (\st{T_i}g_n)} \\
		&=& \sum_{l=1}^n \norm{h_n}\norm{\st{T_i} g_n}
	\end{eqnarray*}
	If $T_i \sto 0$ in $SOT$, then $\norm{\st{T_i} g_n} \sto 0$ for all $n$, i.e. $\norm{K_nT_i} \sto 0$. Thus for $K \in \coper$, $KT_i \sto 0$ in norm. If $T_i \sto 0$ in $WOT$, for any $x \in \Hs$,
	\begin{eqnarray*}
		\langle K_n T_i x, K_n T_i x \rangle &=& \langle \sum_{l=1}^n \langle T_i x, h_l \rangle g_l, \sum_{m=1}^n \langle T_i x, h_m \rangle g_m \rangle \\
		&=& \sum_{l=1}^n \sum_{m=1}^n \langle T_i x, h_l \rangle \langle T_i x, h_m \rangle \langle g_l, g_m \rangle
	\end{eqnarray*}
	Therefore, if $T_i \sto 0$ in $WOT$, $\norm{KT_i x} \sto 0$ for $x \in \Hs$, i.e. $KT_i \sto 0$ in $SOT$.
\end{proof}

\subsection{Central Cover}

\begin{defn}
	For a von Neumann algebra $\A$, if $\fml{Z} = \A \cap \A^{'}$ is the center of $\A$, the central cover or central cover of $A \in \A$ is the projection
	\begin{equation*}
		C_A = \inf{\{~C \colon C \in \fml{Z} \text{ is a projection},~AC = A~\}}
	\end{equation*}
\end{defn}
\begin{rem}
	By above proposition, $C_A$ is definitely a projection satisfying
	\begin{equation*}
		AC_A = C_A = A
	\end{equation*}
\end{rem}

\begin{prop}
	Let $\A$ be a von Neumann algebra.
	\begin{enumerate}[label=\arabic*)]
		\item If $A \in \A$, $E = (\ker{A})^{\bot}$ and $F = \clo{\ran{A}}$, then
		\begin{equation*}
			C_A = C_{\st{A}} = C_{\st{A}A} = C_{A\st{A}} = C_E = C_F
		\end{equation*}
		\item If $E$ is a projection in $\A$, then
		\begin{equation*}
			C_E = \clo{\spn{\{Ah \colon h \in E,~ A \in \A\}}}
		\end{equation*}
	\end{enumerate}
\end{prop}
\begin{proof}
	For $1)$, $C_A = C_{\st{A}}$ is trivial by the definition. If $C$ is a central projection s.t. $AC = A$, then $\st{A}AC = \st{A}A$. Conversely, if $\st{A}AC = \st{A}A$, then $\abs{A}C = \abs{A}$ by taking the root. By the Polar Decomposition, $AC = A$. Thus $C_A = C_{\st{A}A}$. \\
	Firstly, $E \in \A$. If $C$ is a central projection s.t. $AC = A$, then 
	\begin{equation*}
		C(\ker{A}) \subset \ker{A} \Rightarrow EC = CE = E \Rightarrow E \leqslant C
	\end{equation*}
	Therefore, $C_E \leqslant C_A$. Conversely, if $C$ is a central projection s.t. $E \leqslant C$, then 
	\begin{equation*}
		AC = AEC = AE = A
	\end{equation*}
	Thus $C_A \leqslant C_E$. Moreover, $C_F = C_E$.
	\item For $2)$, let
	\begin{equation*}
		P = \clo{\spn{\{Ah \colon h \in E,~ A \in \A\}}}
	\end{equation*}
	and clearly $P \in \A^{'}$. Moreover, for any $B \in \A^{'}$ and $h \in E$,
	\begin{equation*}
		BAh = ABh \in P \Rightarrow BP \subset P \Rightarrow P \in \A^{''} = \A
	\end{equation*}
	Therefore, $P$ is a central projection with $E \leqslant P$, thus $C_E \leqslant P$. Conversely, if $C$ is a central projection s.t. $EC = E$, then for $A \in \A$ and $h \in E$,
	\begin{equation*}
		C(Ah) = AC(Eh) = AEh = Ah \Rightarrow CP = P = PC
	\end{equation*}
	Thus $P \leqslant C$, then $P \leqslant C_E$.
\end{proof}

\begin{defn}
	If $\A$ is a con Neumann algebra and $E$ is a projection, defining
	\begin{equation*}
		\A_{E} = \{~A_E \colon A \in \A~\}
	\end{equation*}
	where $A_E = EA|_E$.
\end{defn}
\begin{rem}
	In general, $\A_E$ is not an algebra. However, if $E \in \A$, there is a $*$-isomorphism between $\A_E$ and $E\A E$. If $E \in \A^{'}$, there is a $*$-homomorphism between $\A_E$ and $E\A E$. $E \A E$ is clear a $\st{C}$-subalgebra of $\A$.
\end{rem}

\begin{prop}
	Let $\A$ be a von Neumann algebra.
	\begin{enumerate}[label=\arabic*)]
		\item If $E \in \A^{'}$ is a projection, then $\A_E$ is a von Neumann algebra in $\fml{B}(E)$ and $(\A_E)^{'} = (\A)^{'}_E$.
		\item If $E \in \A$ is a projection, then $\A_E$ is a von Neumann algebra in $\fml{B}(E)$ and $(\A_E)^{'} = (\A)^{'}_E$.
		\item If $\fml{Z}$ is the center of $\A$ and $E$ is a projection in $\A$ or $\A^{'}$, then the center of $\A_E$ is $\fml{Z}_E$.
	\end{enumerate}
\end{prop}
\begin{proof}
	For $1)$, it is clear since $E$ reduces $\A$. 
	\item Assume that $E \in \A$. $\A_E$ and $(\A^{'})_E$ are $*$-algebra clearly. If $B \in \A^{'}$ and $A \in \A$, then for $h \in E$, 
	\begin{equation*}
		B_E A_E h = EBEAh = EAEBh = A_E B_E h
	\end{equation*}
	Therefore, $(\A)^{'}_E \subset (\A_E)^{'}$.
	\item Check: $(\A_E)^{'} \subset (\A)^{'}_E$ \\
	Let $U \in (\A_E)^{'} \subset \fml(B)(E)$ be a unitry, $A_1,\cdots,A_n \in \A$ and $h_1,\cdots,h_n \in E$, then
	\begin{eqnarray*}
		\norm{\sum_{k=1}^n A_k U h_k}^2 &=& \sum_{i,j}^n \langle A_iUh_i,A_jUh_j \rangle = \sum_{i,j}^n \langle (E\st{A_j}A_iE)Uh_i, Uh_j \rangle \\
		&=& \sum_{i,j}^n \langle U(E\st{A_j}A_iE)h_i, Uh_j \rangle = \sum_{i,j}^n \langle E\st{A_j}A_iEh_i, h_j \rangle \\
		&=& \norm{\sum_{k=1}^n A_k h_k}^2
	\end{eqnarray*}
	Therefore, for $C = \clo{\spn{\{Ah \colon h \in E,~ A \in \A\}}}$, there is a partical isometry with the initial space $C$,
	\begin{equation*}
		W(\sum_{k=1}^n A_k h_k) = \sum_{k=1}^n A_k U h_k
	\end{equation*}
	and $WC^{\bot} = 0$. By above proposition $C = C_E$. Therefore, for any $A \in \A$,
	\begin{equation*}
		WAC^{\bot} = WC^{\bot}A = 0 = AWC^{\bot}
	\end{equation*}
	and on $C$,
	\begin{equation*}
		WA \sum_{k=1}^n A_k h_k = W \sum_{k=1}^n A A_k h_k = A \sum_{k=1}^n A_kUh_k = AW\sum_{k=1}^n A_k h_k
	\end{equation*}
	Therefore, $W \in \A^{'}$ with $Wh = Uh$ for $h \in E$. So $W_E = U$. Since any element in $(\A_E)^{'}$ can be the linear combination of four unitaries, $(\A_E)^{'} \subset (\A)^{'}_E$.
	\item Check: $\A_E \subset \fml{B}(E)$ is von Neumann. \\
	Let $T \in (\A_E)^{''}$ and extending $T$ on $\Hs$ by setting $T = 0$ on $E^{\bot}$. If $B \in \A^{'}$, then $B_E \in (\A_E)^{'}$. For $h \in E$,
	\begin{equation*}
		TBh = (TE)Bh = T B_E h = B_E Th = B_E Th =B Th
	\end{equation*}
	And on the other hand,
	\begin{equation*}
		TBE^{\bot} = TE^{\bot}B = 0 = BTE^{\bot}
	\end{equation*}
	Thus $T \in \A^{''} = \A$, and $(\A_E)^{''} = \A_E$.
	\item For $3)$, by above mention, 
	\begin{equation*}
		\A_E \cap (\A_E)^{'} = \A_E \cap (\A)^{'}_E = \fml{Z}_E \qedhere
	\end{equation*}
\end{proof}

\subsection{Kaplansky Density Theorem}

For $\st{C}$-algebra $\A$, let $\mathcal{B}(\A)$ be the norm-closed unit ball in $\A$.

\begin{thm}[Kaplansky Density Theorem]
	If $\B$ is a $\st{C}$-subalgebra of $\oper$ and $\A = \clo{\B}^{SOT}$, then 
	\begin{enumerate}[label = \arabic*)]
		\item $\mathcal{B}(\A)$ is the $SOT$ closure of $\mathcal{B}(\B)$.
		\item $\mathcal{B}(\Rea{\A})$ is the $SOT$ closure of $\mathcal{B}(\Rea{\B})$.
		\item $\mathcal{B}(\A)_{+}$ is the $SOT$ closure of $\mathcal{B}(\B)_{+}$.
		\item the $SOT$ closure of the unitaries in $\B$ contains the unitaries of $\A$.
	\end{enumerate}
\end{thm}

\begin{lem}
	If $f \in C(\R)$ with $f(0)=0$ and there are positive constants $a,b$ s.t. 
	\begin{equation*}
		\abs{f(t)} \leqslant a\abs{t} + b,~ \forall~ t \in \R
	\end{equation*}
	then for any net $\{T_i\} \subset \Rea{\A}$ with $T_i \sto T$ in $SOT$, $f(T_i) \sto f(T)$ in $SOT$.
\end{lem}
\begin{proof}
	For any $g \in C(\R)$, then the map $g \colon \Rea{\oper} \sto \Rea{\oper}$ makes sence. Denote the set of all such $g$ s.t. $g$ is continuous w.r.t. the $SOT$ by $\mathcal{S}$ and let $\mathcal{S}_b$ be set of all norm bounded elements in $\mathcal{S}$. Clearly, $\mathcal{S}_b \mathcal{S} \subset \mathcal{S}$.\\
	Let $e(t) = t / (1+t^2)$ and $A,B \in \Rea{\oper}$,
	\begin{eqnarray*}
		e(A)-e(B) &=& (1+A^2)^{-1}(A(1+B^2)-(1+A^2)B)(1+B^2)^{-1} \\
		&=& (1+A^2)^{-1}(A-B)(1+B^2)^{-1} + (1+A^2)^{-1}A(B-A)B(1+B^2)^{-1}
	\end{eqnarray*}
\end{proof}














