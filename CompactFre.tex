\chapter{Compact Operators, Fredholm Theory and Perturbations}

The compact operator is as similar as the finite dimensional operator with respect to the spectrums, the representations and so on. In fact, all compact operators can form the unique closed ideal in $\oper$. Moverover, it contains some important classes of operators, like the trace class, which provides an extral topology on $\oper$ and this topology plays an important role in the von Neumann algebra. In fact, the topology is conincided with the $WOT$ in some cases, but it can give a new method to obtain some properties of general von Neumann algebras. For non-compact operators, there is a "weak version" of the compact operator, the Fredholm operator. By researching the Fredholm operators, it provides us an approach to find some "invariant variable" under the compact perturbations. Finally, we have known that finite dimensional normal operator can always be diagnolizable, but for infinite dimensional case, this statement is not true. But by some small compact perturbation, the normal operator can be diagnalizable, and this result has some applications of representation theory. 

\section{Spectrums}

\subsection{Elementary Properties}

\begin{defn}
	Let $\Hs$ be a Hilbert space and $B_{\Hs}$ be the closed unit ball in $\Hs$ and $T \in \oper$. $T$ is compact if and only if the closure of $T(B_{\Hs})$ is compact. Then let $\coper$ denote the set of all compact operators.
\end{defn}
\begin{rem}
	Clearly, if $T$ is finite rank, $T \in \coper$. Denote $\foper$ as the set of all finite rank operators. Therefore, $\foper \subset \coper$.
	\begin{equation*}
		\foper = \{~T \in \oper \colon \ran{T} \text{ is finite dimension}~\}
	\end{equation*}
\end{rem}

Since $\Hs$ is a complete metric space, the compactness of $\clo{T(B_{\Hs})}$ is equivalent to that any sequence in $\clo{T(B_{\Hs})}$ has a convergent subsequence with the convergent point in $\clo{T(B_{\Hs})}$. By this fact, we can get the following theorem.

\begin{thm}
	Let $\Hs$ be a Hilbert space.
	\begin{enumerate}[label=\arabic*)]
		\item $\coper$ is a linear space.
		\item $\coper$ is closed in norm.
		\item $\coper$ is an ideal in $\oper$.
		\item If $T \in \coper$, then $\st{T} \in \coper$.
	\end{enumerate}
\end{thm}
\begin{rem}
	Therefore, $\coper$ is a $\st{C}$-subalgebra, and moreover, $\coper$ is a closed ideal of $\oper$.
\end{rem}

\begin{defn}
	Let $\Hs$ be a Hilbert space and $T \in \oper$. $T$ is called completely continuous if for any sequence $\{x_n\} \in \Hs$ with $x_n \sto x$ weakly, then $Tx_n \sto Tx$ in norm.
\end{defn}

\begin{prop}
	$T \in \oper$ is compact if and only if $T$ is completely continuous. 
\end{prop}
\begin{proof}
	Let $T$ be a compact operator and $\{x_n\} \in \Hs$ with $x_n \sto 0$ weakly. By the Principle of Uniform Boundedness, $M = \sup_n{\norm{x_n}} < \infty$. Assuming $M \leqslant 1$,
	\begin{equation*}
		\{Tx_n\} \subset \clo{T(B_{\Hs})}
	\end{equation*}
	By the fact that $\clo{T(B_{\Hs})}$ is compact, there is a subsequence $\{x_{n_k}\}$ s.t. $Tx_{n_k}  \sto y$ in norm. Since $T$ is also weakly continuous, $Tx_n \sto 0$ weakly. Thus $y = 0$. Then $T$ is completely continuous.\\
	Conversely, assume $T$ is completely continuous. \\
	Firstly, if $\Hs$ is separable and since $\Hs$ is reflexive, $B_{\Hs}$ is a weak-compact metric space. Therefore, for any $\{x_n\} \subset B_{\Hs}$, there is a subsequence $\{x_{n_k}\}$ and $x$ s.t. $x_{n_k} \sto x$ weakly. Therefore, $Tx_{n_k} \sto x$ in norm, i.e. $T$ is compact.\\
	For general case, let $\{x_n\} \subset B_{\Hs}$, then $\Hs_1 = \clo{\spn{\{x_n\}}}$ is separable. Therefore, since
	\begin{equation*}
		T|_{\Hs_1}(\{x_n\}) = T(\{x_n\}) \subset T|_{\Hs_1}(B_{\Hs_1})
	\end{equation*}
	$T$ is compact.
\end{proof}

By above theorem, we can see the power of compact operators. Intuitively, compact operators can "strengthen" the topology. They map the weakly convergent sequences be norm convergent sequence. How can they do that? We have known the weak topology is agree with the norm topology on the finite dimensional space. So, compact operators may be the "extension" of finite rank operators. In fact, we can descibe this property more rigorously.

\begin{thm}
	$T \in \oper$ is compact if and only if there is a sequence $\{T_n\} \subset \foper$ s.t. $T_n \sto T$ in norm.
\end{thm}
\begin{proof}
	Assume $T$ is compact. Therefore, $\clo{T(B_{\Hs})}$ is separable. Let $\{e_n\}_{n=1}^{\infty}$ be the basis of the dense subspace of $\clo{T(B_{\Hs})}$. Then define projections
	\begin{equation*}
		P_n \colon \Hs \longrightarrow \spn{\{e_1, e_2, \cdots, e_n\}}
	\end{equation*}
	and let $T_n = P_nT$. Clearly, $T_n \in \foper$. Now, clearly, $\norm{Th - T_nh} \sto 0$ for any $h \in \Hs$. Since $\clo{T(B_{\Hs})}$ is compact, for any given $\varepsilon > 0$, there are $h_1, \cdots, h_m \in \Hs$ s.t.
	\begin{equation*}
		T(B_{\Hs}) \subset \bigcup_{i=1}^{m}B_{\varepsilon}(Th_i)
	\end{equation*}
	where $B_{\varepsilon}(h_i)$ is the open ball centred at $Th_i$ with radius $\varepsilon$. Therefore, for any $h \in B_{\Hs}$, choose $h_j$ s.t. $\norm{Th-Th_j} < \varepsilon$.
	\begin{eqnarray*}
		\norm{Th-T_nh} &\leqslant& \norm{Th - Th_j} + \norm{Th_j - T_nh_j} + \norm{P_n(Th_j-Th)} \\
		&\leqslant& 2\norm{Th - Th_j} +  \norm{Th_j - T_nh_j} \\
		&<& 2\varepsilon +  \norm{Th_j - T_nh_j}
	\end{eqnarray*}
	Since $\norm{Th_j - T_{n}h_j} < \varepsilon$ for any $h_j$ and $n > n_0$ for some $n_)$,  
	\begin{equation*}
		\norm{Th-T_nh} < 3\varepsilon \text{ for } n > n_0
	\end{equation*}
	Then $T_n \sto T$ uniformly, i.e. $T_n \sto T$ in norm. \\
	The converse is clearly since $\foper \subset \coper$ and $\coper$ is norm closed.
\end{proof}
\begin{cor}
	All projections $\{P_i\}_{i \in I}$ in $\oper$ form an approximate identity for the ideal $\coper$.
\end{cor}

\subsection{Spectrums of Compact Operators}

The spectrum of a compact operator also has similar properties as a finite rank operator.

\begin{thm}
	If $T \in \coper$ and $\lambda \neq 0$ satisfying
	\begin{equation*}
		\inff{\{~\norm{(T-\lambda)h} \colon \norm{h}=1~\}} = 0
	\end{equation*}
	then $\lambda \in \sigma_p(T)$.
\end{thm}
\begin{proof}
	There is a sequence $\{h_n\}$ with $\norm{h_n}=1$ s.t. $\norm{(T-\lambda)h_n} \sto 0$. Since $T$ is compact, there is a subsequence $\{h_{n_k}\}$ and a $h_0$ s.t. $\norm{Th_{n_k}-h_0} \sto 0$, therefore
	\begin{equation*}
		h_{n_k} = \frac{1}{\lambda}((\lambda-T)h_{n_k}+Th_{n_k}) \sto \frac{1}{\lambda}h_0
	\end{equation*}
	Then $\norm{\frac{1}{\lambda}h_0}=1=\abs{\frac{1}{\lambda}}\norm{h_0}$, thus $\norm{h_0} \neq 0$. Since $Th_{n_k} \sto \frac{1}{\lambda}Th_0$ and $Th_{n_k} \sto h_0$,
	\begin{equation*}
		h_0 = \frac{1}{\lambda}Th_0, \text{ i.e. } Th_0 = \lambda h_0
	\end{equation*}
\end{proof}

In fact, we have known that
\begin{equation*}
	\sigma_{ap}(T) = \sigma_l(T) = \{~\lambda \in \C \colon \inff{\{\norm{(T-\lambda)h} \colon \norm{h}=1\}} = 0~\}
\end{equation*}
and $\lambda \notin \sigma_{ap}(T)$ is equivalent to that $\ker{(T-\lambda)} = \{0\}$ and $\ran{(T-\lambda)}$ is closed. Combining with the Riesz Theorem, any closed and bounded subset in a normed space is compact if and only if the normed space is finite dimensional, we have the following corollary.

\begin{cor}
	If $T \in \coper$ and $\lambda \neq 0$ and $\ker{(T-\lambda)} = \{0\}$, then $\ran{(T-\lambda)}$ is closed.
\end{cor}

Here is another important corollary, which says the point spectrum play an important role.
\begin{cor}
	If $T \in \coper$, $\lambda \notin \sigma_p(T)$ with $\lambda \neq 0$ and $\clo{\lambda} \notin \sigma_p(\st{T})$, then $\lambda \notin \sigma(T)$.
\end{cor}
\begin{proof}
	By above theorem, $\lambda \notin \sigma_p(T)$ with $\lambda \neq 0$ means $\lambda \notin \sigma_l(T)$, thus $\ker{(T-\lambda)} = \{0\}$ and $\ran{(T-\lambda)}$ is closed.\\ Similarly, for $\st{T}$, $\ker{(\st{T}-\clo{\lambda})} = \{0\}$ and $\ran{(\st{T}-\lambda)}$ is closed. Therefore,
	\begin{equation*}
		\ran{(T-\lambda)} = \clo{\ran{(T-\lambda)}} = (\ker{(\st{T}-\clo{\lambda})})^{\bot} = \Hs
	\end{equation*}
	Thus $T-\lambda$ is a bijection, and by the Inverse Mapping Theorem, $(T-\lambda)^{-1} \in \oper$, i.e. $\lambda \notin \sigma(T)$.
\end{proof}
\begin{rem}
	In other words, for a compact operator $T$, if $\lambda \in \sigma(T)$ with $\lambda \neq 0$, then $\lambda \in \sigma_p(T)$ or $\lambda \in \sigma_p(\st{T})$.
\end{rem}

In fact, any nonzero point in $\sigma(T)$ for a compact operator $T$ is isolated and moreover, it is an eigenvalue. To prove this, we need some lemmas.

\begin{lem}
	If $\M$ and $\fml{N}$ are two closed linear subspaces of $\Hs$ with $\M \subset \fml{N}$, then for any $\varepsilon > 0$, there exists a $y \in \fml{N}$ with $\norm{y} = 1$ s.t. $\dist{(y,\M)} \leqslant 1-\varepsilon$.
\end{lem}
\begin{proof}
	For $y \in \fml{N}$, define $\delta(y) = \dist{(y,\M)}$. Choosing $y_1 \in \fml{N} \backslash \M$, there exists a $x_0 \in \M$ s.t.
	\begin{equation*}
		\delta(y_1) \leqslant \norm{x_0-y_1} \leqslant (1+\varepsilon)\delta(y_1)
	\end{equation*}
	Let $y_2 = y_1-x_0$, then
	\begin{equation*}
		(1+\varepsilon)\delta(y_2) = (1+\varepsilon)\inff{\{\norm{y_1-x_0-x} \colon x \in \M\}} = (1+\varepsilon)\delta(y_1)
	\end{equation*}
	Thus $(1+\varepsilon)\delta(y_2) > \norm{x_0 - y_1} = \norm{y_2}$. Let $y = \norm{y_2}^{-1}y_2$, then $y_2 \in \fml{N}$ with $\norm{y_2}=1$.
	\begin{eqnarray*}
		\norm{y-x} &=& \norm{\norm{y_2}^{-1}y_2-x} = \norm{y_2}^{-1}\norm{y_2-\norm{y_2}x} \\
		&>& ((1+\varepsilon)\delta(y_2))^{-1} \norm{y_2-\norm{y_2}x} \\
		&\geqslant& (1+\varepsilon)^{-1} > 1 - \varepsilon
	\end{eqnarray*}
\end{proof}

\begin{prop}
	If $T \in \coper$ and $\{\lambda_n\}$ is a sequence of distinct elements in $\sigma_p(T)$, then $\lim_{n \sto \infty} \lambda_n = 0$.
\end{prop}
\begin{proof}
	For each $n$, choosing a nonzero $x_n \in \ker{(T-\lambda_n)}$,
	\begin{equation*}
		\M_n = \spn{\{~x_1,x_2,\cdots,x_n~\}}
	\end{equation*}
	By preceding lemma, there is a $y_n \in \M_n$ with $\norm{y_n}=1$ s.t.
	\begin{equation*}
		\dist{(y_n,\M_{n-1})} > \frac{1}{2}
	\end{equation*}
	Let $y_n = \sum_{i=1}^{n} \alpha_i x_n$, thus
	\begin{equation*}
		(T-\lambda_n)y_n = \sum_{i=1}^{n-1} \alpha_i (\lambda_i-\lambda_n)x_i \in \M_{n-1}
	\end{equation*}
	Therefore, for $n > m$,
	\begin{eqnarray*}
		T(\lambda_n^{-1}y_n)-T(\lambda_m^{-1}y_m) &=& \lambda_n^{-1}(T-\lambda_n)y_n - \lambda_m^{-1}(T-\lambda_m)y_m + (y_n - y_m) \\
		&=& y_n - (y_m + \lambda_n^{-1}(T-\lambda_n)y_n + \lambda_m^{-1}(T-\lambda_m)y_m)
	\end{eqnarray*}
	Since the part in the bracketed is in $\M_{n-1}$,
	\begin{equation*}
		\norm{T(\lambda_n^{-1}y_n)-T(\lambda_m^{-1}y_m)} \leqslant \dist{(y_n,\M_{n-1})} > \frac{1}{2}
	\end{equation*}
	Thus $\{\lambda_n^{-1}y_n\}$ has no bounded subset by the fact that $T$ is compact.
	\begin{equation*}
		\norm{\lambda_n^{-1}y_n} = \abs{\lambda_n}^{-1} \sto \infty
	\end{equation*}
	That means $\lim_{n \sto \infty} \lambda_n = 0$.
\end{proof}

\begin{prop}
	If $T \in \coper$ and a nonzero $\lambda \in \sigma(T)$, then $\lambda$ is a isolated point in $\sigma(T)$.
\end{prop}
\begin{proof}
	If there is a sequence $\{\lambda_n\} \subset \sigma(T)$ s.t. $\lambda_n \sto \lambda$, then each $\lambda_n$ is in either $\sigma_p(T)$ or $\sigma_p(\st{T})$. Then if there exists a subsequence $\{\lambda_{n_k}\} \subset \sigma_p(T)$. But by above proposition, $\lim_{k \sto \infty} \lambda_{n_k} =0$, which is a contradiction. Similarly, if $\{\lambda_{n_k}\} \subset \sigma_p(\st{T})$, $\lim_{k \sto \infty} \lambda_{n_k} =0$, which is also a contradiction.
\end{proof}

Now, we can see any nonzero point in the spectrum of a compact operator is a eigenvalue.

\begin{lem}
	If $T \in \coper$ and a nonzero $\lambda \in \sigma(T)$, then $\ker{(T-\lambda)}$ is finite dimentional.
\end{lem}
\begin{proof}
	If there is an infinite orthonormal sequence $\{e_n\}$ in $\ker{(T-\lambda)}$, then $\{Te_n\}$ has a convergent subsequence $\{Te_{n_k}\}$, thus it is Cauchy.
	\begin{equation*}
		\norm{Te_{n_k}-Te_{n_j}}^2 = \norm{\lambda e_{n_k}-\lambda e_{n_j}}^2 = 2 \abs{\lambda} > 0
	\end{equation*}
	which is a contradiction.
\end{proof}

\begin{thm}
	If $T \in \coper$ and a nonzero $\lambda \in \sigma(T)$, then $\lambda \in \sigma_p(T)$ and $\dim{(T-\lambda)} < \infty$.
\end{thm}
\begin{proof}
	By the \textbf{Proposition} \ref{prop10} in the subsection \textbf{2.1.3}, since each $\lambda \in \sigma(T)$ is isolated, we can define
	\begin{equation*}
		E(\lambda) = \int_{\Gamma(\{\lambda\})} (z-T)^{-1} dz 
	\end{equation*}
	where $\Gamma(\{\lambda\})$ is the closed curve enclosed $\{\lambda\}$ and disjoint with other $\lambda$ in $\sigma(T)$. Then, each $E(\lambda)$ is a projection.
	\begin{equation*}
		\Hs_{\lambda} = E(\lambda)\Hs,~ T_{\lambda} = T|_{\Hs_{\lambda}} \colon \Hs_{\lambda} \sto \Hs_{\lambda}
	\end{equation*}
	Since $\sigma(T_{\lambda}) = \{\lambda\}$ and $\lambda \neq 0$, $T_{\lambda}$ is invertible. But clearly, $T_{\lambda}$ is compact. That means any bounded and closed subset in $\Hs_{\lambda}$ is compact, thus $\dim{\Hs_{\lambda}} < \infty$ by the Riesz Theorem. Therefore, by the result of finite linear algebra, $\lambda$ is the eigenvalue of $T_{\lambda}$, i.e. $\lambda \in \sigma_p(T)$. By above lemma, $\dim{(T-\lambda)} < \infty$.
\end{proof}

Combining all of above results, we can get an explicit structure of the spectrum of compact operator.

\begin{thm}[Riesz]
	If $\Hs$ is a infinite dimensional Hilbert space and $T \in \coper$, then one and only one of the following possibilities occurs.
	\begin{enumerate}[label=\arabic*)]
		\item $\sigma(T) = \{0\}$.
		\item $\sigma(T) = \{0,\lambda_1,\cdots,\lambda_n\}$, where each $\lambda_k \neq 0$, and each $\lambda_k$ is the eigenvalue of $T$ with $\dim{\ker{(T-\lambda_k)}}<\infty$.
		\item $\sigma(T) = \{0,\lambda_1,\lambda_2,\cdots\}$, where each $\lambda_k \neq 0$, and each $\lambda_k$ is the eigenvalue of $T$ with $\dim{\ker{(T-\lambda_k)}}<\infty$, and moreover, $\lim_{n \sto \infty} \lambda_n =0$.
	\end{enumerate}
\end{thm}

Because of its discrete spectrum, we can using the spectral measure to decompose the compact normal operator.

\begin{thm}[Spectral Theorem]
	If $T$ is a compact normal operator on a Hilbert space $\Hs$, then $T$ has at most countable eigenvalues $\{\lambda_n\}$, and there are corresponding projections $P_n \colon \Hs \sto \ker{(T-\lambda_n)}$ with $P_nP_m=P_mP_n=0$ s.t.
	\begin{equation*}
		T = \sum_{n=1}^{\infty} \lambda_n P_n
	\end{equation*}
	where $\{\lambda_n\}$ are all distinctive eigenvalues and the convergence is with respect to the norm topology.
\end{thm}
\begin{proof}
	Let $E$ be the spectral measure of $T$, then we see
	\begin{equation*}
		N = \int_{\sigma(N)} z dE
	\end{equation*}
	If $\dim{\Hs} = \infty$, since $\sigma(N)$ is consisted of at most countable eigenvalues $\{\lambda_n\}_{n=1}^{\infty}$ and $0$, then we can set $P_n = E(\{\lambda_n\})$ for $n = 1,2,\cdots$, which is well-defined by the \textbf{Proposition} \ref{prop11} in the subsection \textbf{3.1.2}, above integral can be
	\begin{eqnarray*}
		N &=& \lim_{n \sto \infty} (\int_{B_{\frac{1}{n}}(0)} z dz + \sum_{k=1}^{n} \lambda_k P_k) \\
		&=& \sum_{n=1}^{\infty} \lambda_n P_n
	\end{eqnarray*} 
	And moreover, $P_n$ is the projection from $\Hs$ onto $\ker{(T-\lambda_n)}$. \\
	If $\dim{\Hs} < \infty$, above theorem is clearly true.
\end{proof}

Also, we can see the Functional Calculus of the compact normal operators. Firstly, Taking same notation as above theorem and let $P_0 = 1-\sum_{n=1}^{\infty}$, in fact, $P_0$ is the projection from $\Hs$ to $\ker{T}$. By the Functional Calculus of normal operators, we know
\begin{equation*}
	\st{W}(T) = \{~\phi(T) \colon \phi \in \lfs{\infty}(\sigma(T))~\}
\end{equation*}
Since $\sigma(T)$ is discrete, $\lfs{\infty}(\sigma(T)) = l^{\infty}(\C)$, and for any $(a_n)_{n=0}^{\infty} \in l^{\infty}(\C)$,
\begin{equation*}
	(a_n)(T) = a_0 P_0 + \sum_{n=1}^{\infty} a_n P_n
\end{equation*}
Thus we can see the separating vector for $\st{W}(T)$, if $e$ is a separating vector and decomposing $e$ to $\{P_n\}_{n=0}^{\infty}$
\begin{equation*}
	e = \sum_{n=0}^{\infty} e_n,~ e_n \in P_n \text{ for each n}
\end{equation*} 
Then for $(a_n)_{n=0}^{\infty} \in l^{\infty}(\C)$
\begin{equation*}
	(a_n)(T)e = (a_0 P_0 + \sum_{n=1}^{\infty} a_n P_n)(\sum_{n=0}^{\infty} e_n) = \sum_{n=0}^{\infty} a_n e_n
\end{equation*}
Since $e$ is separating, $\sum_{n=0}^{\infty} a_n e_n = 0$ for any $(a_n) \in l^{\infty}(\C)$ implies $a_n = 0$ for all $n$. That means that $e_n \neq 0$ for any $n$. Then the corresponding measure $\mu$ is defined as
\begin{equation*}
	\mu(\{\lambda_n,\lambda_m\}) = \norm{E(\{\lambda_n,\lambda_m\})e}^2 = \norm{e_n+e_m}^2
\end{equation*}

Now, we can see the multiplicity function of compact normal operator. By similar construction in the \textbf{Theorem} \ref{thm10} and \textbf{Theorem} \ref{thm9} in the section \textbf{3.4}, the multiplicity function is like
\begin{equation*}
	m_T(\lambda) = \dim{\ker{(T-\lambda)}}
\end{equation*}
That means two compact normal operators are equivalent if and only if they have same dimension of all eigenspaces.

Finally,, we can prove $\coper$ is the unique closed ideal in $\oper$ if $\Hs$ is separable. 
\begin{prop}
	If $N$ is a normal operator in $\oper$ with the specture measure $E$, then $N$ is compact if and only if for any $\varepsilon > 0$,
	\begin{equation*}
		\dim{E(\{z \in \C \colon \abs{z} > \varepsilon\})} < \infty
	\end{equation*}
\end{prop}
\begin{proof}
	Let $\Delta_{\varepsilon} = \{z \in \C \colon \abs{z} > \varepsilon\}$ and $E_{\varepsilon} = E(\Delta_{\varepsilon})$.\\
	Assume that for any given $\varepsilon > 0$, $\dim{E_{\varepsilon}} < \infty$, then
	\begin{eqnarray*}
		N - NE_{\varepsilon} &=& \int z dE - \int z\chi_{\Delta_{\varepsilon}} dE \\
		&=& \int z\chi_{\C \backslash \Delta_{\varepsilon}} dE = \phi(N)
	\end{eqnarray*}
	where $\phi(z) = z\chi_{\C \backslash \Delta_{\varepsilon}}(z)$. Therefore,
	\begin{equation*}
		\norm{N - NE_{\varepsilon}} = \sup{\{\abs{z} \colon \C \backslash \Delta_{\varepsilon}\}} < \varepsilon
	\end{equation*}
	Thus $N \in \coper$. \\
	Conversely, if $N$ is compact, then for any $\varepsilon > 0$, put $\phi(z) = z^{-1}\chi_{\Delta_{\varepsilon}}$, then
	\begin{equation*}
		N\phi(N) &=& \int \chi_{\Delta_{\varepsilon}} dE = E_{\varepsilon} 
	\end{equation*}
	Since $E_{\varepsilon}$ is a compact projection, i.e. $\ran{E_{\varepsilon}}$ is closed, by the Riesz Theorem, $\dim{\ran{E_{\varepsilon}}} < \infty$.
\end{proof}

\begin{thm}
	If $\Hs$ is a separable Hilbert space and $\I$ is an ideal of $\oper$ that contains a non-compact operator, then $\I = \oper$.
\end{thm}
\begin{proof}
	Let $T \in \I \backslash \coper$ then 
	\begin{equation*}
		\st{T}T = \int_{\sigma(\st{T}T)} t dE(t)
	\end{equation*}
	By above theorem, there is an $\varepsilon > 0$ s.t. $P = E(\varepsilon,\infty)$ has infinite rank. 
	\begin{equation*}
		P = (\int t^{-1}\chi_{(\varepsilon,\infty)}(t)dE(t))\st{T}T \in \I
	\end{equation*}
	Since $\Hs$ is separable, $\dim{P\Hs} = \dim{\Hs} = \aleph_0$, there is a unitary $U$ from $\Hs$ to $P\Hs$. Therefore, $1 = \st{U}PU \in \I$. $\I = \oper$.
\end{proof}

\begin{prop}
	If $\I$ is a closed ideal of $\oper$, the $\coper \subset \I$ or $\I = \{0\}$.
\end{prop}
\begin{proof}
	Since $\I$ is closed, $\I$ is self-adjoint by the  \textbf{Proposition} \ref{prop12} in the subsection \textbf{2.2.5}. Then by the \textbf{Theorem} \ref{thm6} in the subsection \textbf{2.2.5}, we know $\I = \I \cap \st{\I}$ is a hereditary subalgebra. \\
	Therefore, if $\I$ is nonzero and $T \in \I$ is nonzero, then there is a finite rank projection $P$ s.t.
	\begin{equation*}
		0 \leqslant P \leqslant \st{T}T
	\end{equation*}
	Thus, $P \in \I$. Morover, any finite rank projection is in $\I$. Since $\I$ is norm closed, $\coper \subset \I$.
\end{proof}

Combining above propositions and theorems, we can get the final result.
\begin{cor}
	If $\Hs$ is a separable Hilbert space, then the only nontrivial closed ideal of $\oper$ is $\coper$.
\end{cor}

\section{Representations of Compact Operators}










