\chapter{Compact Operators, Fredholm Theory and Perturbations}

The compact operator is as similar as the finite dimensional operator with respect to the spectrums, the representations and so on. In fact, all compact operators can form the unique closed ideal in $\oper$. Moverover, it contains some important classes of operators, like the trace class, which provides an extral topology on $\oper$ and this topology plays an important role in the von Neumann algebra. In fact, the topology is conincided with the $WOT$ in some cases, but it can give a new method to obtain some properties of general von Neumann algebras. For non-compact operators, there is a "weak version" of the compact operator, the Fredholm operator. By researching the Fredholm operators, it provides us an approach to find some "invariant variable" under the compact perturbations. Finally, we have known that finite dimensional normal operator can always be diagnolizable, but for infinite dimensional case, this statement is not true. But by some small compact perturbation, the normal operator can be diagnalizable, and this result has some applications of representation theory. 

\section{Spectrums}

\subsection{Elementary Properties}

\begin{defn}
	Let $\Hs$ be a Hilbert space and $B_{\Hs}$ be the closed unit ball in $\Hs$ and $T \in \oper$. $T$ is compact if and only if the closure of $T(B_{\Hs})$ is compact. Then let $\coper$ denote the set of all compact operators.
\end{defn}
\begin{rem}
	Clearly, if $T$ is finite rank, $T \in \coper$. Denote $\foper$ as the set of all finite rank operators. Therefore, $\foper \subset \coper$.
	\begin{equation*}
		\foper = \{~T \in \oper \colon \ran{T} \text{ is finite dimension}~\}
	\end{equation*}
\end{rem}

Since $\Hs$ is a complete metric space, the compactness of $\clo{T(B_{\Hs})}$ is equivalent to that any sequence in $\clo{T(B_{\Hs})}$ has a convergent subsequence with the convergent point in $\clo{T(B_{\Hs})}$. By this fact, we can get the following theorem.

\begin{thm}
	Let $\Hs$ be a Hilbert space.
	\begin{enumerate}[label=\arabic*)]
		\item $\coper$ is a linear space.
		\item $\coper$ is closed in norm.
		\item $\coper$ is an ideal in $\oper$.
		\item If $T \in \coper$, then $\st{T} \in \coper$.
	\end{enumerate}
\end{thm}
\begin{rem}
	Therefore, $\coper$ is a $\st{C}$-subalgebra, and moreover, $\coper$ is a closed ideal of $\oper$.
\end{rem}

\begin{defn}
	Let $\Hs$ be a Hilbert space and $T \in \oper$. $T$ is called completely continuous if for any sequence $\{x_n\} \in \Hs$ with $x_n \sto x$ weakly, then $Tx_n \sto Tx$ in norm.
\end{defn}

\begin{prop}
	$T \in \oper$ is compact if and only if $T$ is completely continuous. 
\end{prop}
\begin{proof}
	Let $T$ be a compact operator and $\{x_n\} \in \Hs$ with $x_n \sto 0$ weakly. By the Principle of Uniform Boundedness, $M = \sup_n{\norm{x_n}} < \infty$. Assuming $M \leqslant 1$,
	\begin{equation*}
		\{Tx_n\} \subset \clo{T(B_{\Hs})}
	\end{equation*}
	By the fact that $\clo{T(B_{\Hs})}$ is compact, there is a subsequence $\{x_{n_k}\}$ s.t. $Tx_{n_k}  \sto y$ in norm. Since $T$ is also weakly continuous, $Tx_n \sto 0$ weakly. Thus $y = 0$. Then $T$ is completely continuous.\\
	Conversely, assume $T$ is completely continuous. \\
	Firstly, if $\Hs$ is separable and since $\Hs$ is reflexive, $B_{\Hs}$ is a weak-compact metric space. Therefore, for any $\{x_n\} \subset B_{\Hs}$, there is a subsequence $\{x_{n_k}\}$ and $x$ s.t. $x_{n_k} \sto x$ weakly. Therefore, $Tx_{n_k} \sto x$ in norm, i.e. $T$ is compact.\\
	For general case, let $\{x_n\} \subset B_{\Hs}$, then $\Hs_1 = \clo{\spn{\{x_n\}}}$ is separable. Therefore, since
	\begin{equation*}
		T|_{\Hs_1}(\{x_n\}) = T(\{x_n\}) \subset T|_{\Hs_1}(B_{\Hs_1})
	\end{equation*}
	$T$ is compact.
\end{proof}

By above theorem, we can see the power of compact operators. Intuitively, compact operators can "strengthen" the topology. They map the weakly convergent sequences be norm convergent sequence. How can they do that? We have known the weak topology is agree with the norm topology on the finite dimensional space. So, compact operators may be the "extension" of finite rank operators. In fact, we can descibe this property more rigorously.

\begin{thm}
	$T \in \oper$ is compact if and only if there is a sequence $\{T_n\} \subset \foper$ s.t. $T_n \sto T$ in norm.
\end{thm}
\begin{proof}
	Assume $T$ is compact. Therefore, $\clo{T(B_{\Hs})}$ is separable. Let $\{e_n\}_{n=1}^{\infty}$ be the basis of the dense subspace of $\clo{T(B_{\Hs})}$. Then define projections
	\begin{equation*}
		P_n \colon \Hs \longrightarrow \spn{\{e_1, e_2, \cdots, e_n\}}
	\end{equation*}
	and let $T_n = P_nT$. Clearly, $T_n \in \foper$. Now, clearly, $\norm{Th - T_nh} \sto 0$ for any $h \in \Hs$. Since $\clo{T(B_{\Hs})}$ is compact, for any given $\varepsilon > 0$, there are $h_1, \cdots, h_m \in \Hs$ s.t.
	\begin{equation*}
		T(B_{\Hs}) \subset \bigcup_{i=1}^{m}B_{\varepsilon}(Th_i)
	\end{equation*}
	where $B_{\varepsilon}(h_i)$ is the open ball centred at $Th_i$ with radius $\varepsilon$. Therefore, for any $h \in B_{\Hs}$, choose $h_j$ s.t. $\norm{Th-Th_j} < \varepsilon$.
	\begin{eqnarray*}
		\norm{Th-T_nh} &\leqslant& \norm{Th - Th_j} + \norm{Th_j - T_nh_j} + \norm{P_n(Th_j-Th)} \\
		&\leqslant& 2\norm{Th - Th_j} +  \norm{Th_j - T_nh_j} \\
		&<& 2\varepsilon +  \norm{Th_j - T_nh_j}
	\end{eqnarray*}
	Since $\norm{Th_j - T_{n}h_j} < \varepsilon$ for any $h_j$ and $n > n_0$ for some $n_)$,  
	\begin{equation*}
		\norm{Th-T_nh} < 3\varepsilon \text{ for } n > n_0
	\end{equation*}
	Then $T_n \sto T$ uniformly, i.e. $T_n \sto T$ in norm. \\
	The converse is clearly since $\foper \subset \coper$ and $\coper$ is norm closed.
\end{proof}
\begin{cor}
	All projections $\{P_i\}_{i \in I}$ in $\oper$ form an approximate identity for the ideal $\coper$.
\end{cor}

\subsection{Spectrums of Compact Operators}

The spectrum of a compact operator also has similar properties as a finite rank operator.

\begin{thm}
	If $T \in \coper$ and $\lambda \neq 0$ satisfying
	\begin{equation*}
		\inff{\{~\norm{(T-\lambda)h} \colon \norm{h}=1~\}} = 0
	\end{equation*}
	then $\lambda \in \sigma_p(T)$.
\end{thm}
\begin{proof}
	There is a sequence $\{h_n\}$ with $\norm{h_n}=1$ s.t. $\norm{(T-\lambda)h_n} \sto 0$. Since $T$ is compact, there is a subsequence $\{h_{n_k}\}$ and a $h_0$ s.t. $\norm{Th_{n_k}-h_0} \sto 0$, therefore
	\begin{equation*}
		h_{n_k} = \frac{1}{\lambda}((\lambda-T)h_{n_k}+Th_{n_k}) \sto \frac{1}{\lambda}h_0
	\end{equation*}
	Then $\norm{\frac{1}{\lambda}h_0}=1=\abs{\frac{1}{\lambda}}\norm{h_0}$, thus $\norm{h_0} \neq 0$. Since $Th_{n_k} \sto \frac{1}{\lambda}Th_0$ and $Th_{n_k} \sto h_0$,
	\begin{equation*}
		h_0 = \frac{1}{\lambda}Th_0, \text{ i.e. } Th_0 = \lambda h_0
	\end{equation*}
\end{proof}

In fact, we have known that
\begin{equation*}
	\sigma_{ap}(T) = \sigma_l(T) = \{~\lambda \in \C \colon \inff{\{\norm{(T-\lambda)h} \colon \norm{h}=1\}} = 0~\}
\end{equation*}
and $\lambda \notin \sigma_{ap}(T)$ is equivalent to that $\ker{(T-\lambda)} = \{0\}$ and $\ran{(T-\lambda)}$ is closed. Combining with the Riesz Theorem, any closed and bounded subset in a normed space is compact if and only if the normed space is finite dimensional, we have the following corollary.

\begin{cor}
	If $T \in \coper$ and $\lambda \neq 0$ and $\ker{(T-\lambda)} = \{0\}$, then $\ran{(T-\lambda)}$ is closed.
\end{cor}

Here is another important corollary, which says the point spectrum play an important role.
\begin{cor}
	If $T \in \coper$, $\lambda \notin \sigma_p(T)$ with $\lambda \neq 0$ and $\clo{\lambda} \notin \sigma_p(\st{T})$, then $\lambda \notin \sigma(T)$.
\end{cor}
\begin{proof}
	By above theorem, $\lambda \notin \sigma_p(T)$ with $\lambda \neq 0$ means $\lambda \notin \sigma_l(T)$, thus $\ker{(T-\lambda)} = \{0\}$ and $\ran{(T-\lambda)}$ is closed.\\ Similarly, for $\st{T}$, $\ker{(\st{T}-\clo{\lambda})} = \{0\}$ and $\ran{(\st{T}-\lambda)}$ is closed. Therefore,
	\begin{equation*}
		\ran{(T-\lambda)} = \clo{\ran{(T-\lambda)}} = (\ker{(\st{T}-\clo{\lambda})})^{\bot} = \Hs
	\end{equation*}
	Thus $T-\lambda$ is a bijection, and by the Inverse Mapping Theorem, $(T-\lambda)^{-1} \in \oper$, i.e. $\lambda \notin \sigma(T)$.
\end{proof}
\begin{rem}
	In other words, for a compact operator $T$, if $\lambda \in \sigma(T)$ with $\lambda \neq 0$, then $\lambda \in \sigma_p(T)$ or $\lambda \in \sigma_p(\st{T})$.
\end{rem}

In fact, any nonzero point in $\sigma(T)$ for a compact operator $T$ is isolated and moreover, it is an eigenvalue. To prove this, we need some lemmas.

\begin{lem}
	If $\M$ and $\fml{N}$ are two closed linear subspaces of $\Hs$ with $\M \subset \fml{N}$, then for any $\varepsilon > 0$, there exists a $y \in \fml{N}$ with $\norm{y} = 1$ s.t. $\dist{(y,\M)} \leqslant 1-\varepsilon$.
\end{lem}
\begin{proof}
	For $y \in \fml{N}$, define $\delta(y) = \dist{(y,\M)}$. Choosing $y_1 \in \fml{N} \backslash \M$, there exists a $x_0 \in \M$ s.t.
	\begin{equation*}
		\delta(y_1) \leqslant \norm{x_0-y_1} \leqslant (1+\varepsilon)\delta(y_1)
	\end{equation*}
	Let $y_2 = y_1-x_0$, then
	\begin{equation*}
		(1+\varepsilon)\delta(y_2) = (1+\varepsilon)\inff{\{\norm{y_1-x_0-x} \colon x \in \M\}} = (1+\varepsilon)\delta(y_1)
	\end{equation*}
	Thus $(1+\varepsilon)\delta(y_2) > \norm{x_0 - y_1} = \norm{y_2}$. Let $y = \norm{y_2}^{-1}y_2$, then $y_2 \in \fml{N}$ with $\norm{y_2}=1$.
	\begin{eqnarray*}
		\norm{y-x} &=& \norm{\norm{y_2}^{-1}y_2-x} = \norm{y_2}^{-1}\norm{y_2-\norm{y_2}x} \\
		&>& ((1+\varepsilon)\delta(y_2))^{-1} \norm{y_2-\norm{y_2}x} \\
		&\geqslant& (1+\varepsilon)^{-1} > 1 - \varepsilon
	\end{eqnarray*}
\end{proof}

\begin{prop}
	If $T \in \coper$ and $\{\lambda_n\}$ is a sequence of distinct elements in $\sigma_p(T)$, then $\lim_{n \sto \infty} \lambda_n = 0$.
\end{prop}
\begin{proof}
	For each $n$, choosing a nonzero $x_n \in \ker{(T-\lambda_n)}$,
	\begin{equation*}
		\M_n = \spn{\{~x_1,x_2,\cdots,x_n~\}}
	\end{equation*}
	By preceding lemma, there is a $y_n \in \M_n$ with $\norm{y_n}=1$ s.t.
	\begin{equation*}
		\dist{(y_n,\M_{n-1})} > \frac{1}{2}
	\end{equation*}
	Let $y_n = \sum_{i=1}^{n} \alpha_i x_n$, thus
	\begin{equation*}
		(T-\lambda_n)y_n = \sum_{i=1}^{n-1} \alpha_i (\lambda_i-\lambda_n)x_i \in \M_{n-1}
	\end{equation*}
	Therefore, for $n > m$,
	\begin{eqnarray*}
		T(\lambda_n^{-1}y_n)-T(\lambda_m^{-1}y_m) &=& \lambda_n^{-1}(T-\lambda_n)y_n - \lambda_m^{-1}(T-\lambda_m)y_m + (y_n - y_m) \\
		&=& y_n - (y_m + \lambda_n^{-1}(T-\lambda_n)y_n + \lambda_m^{-1}(T-\lambda_m)y_m)
	\end{eqnarray*}
	Since the part in the bracketed is in $\M_{n-1}$,
	\begin{equation*}
		\norm{T(\lambda_n^{-1}y_n)-T(\lambda_m^{-1}y_m)} \leqslant \dist{(y_n,\M_{n-1})} > \frac{1}{2}
	\end{equation*}
	Thus $\{\lambda_n^{-1}y_n\}$ has no bounded subset by the fact that $T$ is compact.
	\begin{equation*}
		\norm{\lambda_n^{-1}y_n} = \abs{\lambda_n}^{-1} \sto \infty
	\end{equation*}
	That means $\lim_{n \sto \infty} \lambda_n = 0$.
\end{proof}

\begin{prop}
	If $T \in \coper$ and a nonzero $\lambda \in \sigma(T)$, then $\lambda$ is a isolated point in $\sigma(T)$.
\end{prop}
\begin{proof}
	If there is a sequence $\{\lambda_n\} \subset \sigma(T)$ s.t. $\lambda_n \sto \lambda$, then each $\lambda_n$ is in either $\sigma_p(T)$ or $\sigma_p(\st{T})$. Then if there exists a subsequence $\{\lambda_{n_k}\} \subset \sigma_p(T)$. But by above proposition, $\lim_{k \sto \infty} \lambda_{n_k} =0$, which is a contradiction. Similarly, if $\{\lambda_{n_k}\} \subset \sigma_p(\st{T})$, $\lim_{k \sto \infty} \lambda_{n_k} =0$, which is also a contradiction.
\end{proof}

Now, we can see any nonzero point in the spectrum of a compact operator is a eigenvalue.

\begin{lem}
	If $T \in \coper$ and a nonzero $\lambda \in \sigma(T)$, then $\ker{(T-\lambda)}$ is finite dimentional.
\end{lem}
\begin{proof}
	If there is an infinite orthonormal sequence $\{e_n\}$ in $\ker{(T-\lambda)}$, then $\{Te_n\}$ has a convergent subsequence $\{Te_{n_k}\}$, thus it is Cauchy.
	\begin{equation*}
		\norm{Te_{n_k}-Te_{n_j}}^2 = \norm{\lambda e_{n_k}-\lambda e_{n_j}}^2 = 2 \abs{\lambda} > 0
	\end{equation*}
	which is a contradiction.
\end{proof}

\begin{thm}
	If $T \in \coper$ and a nonzero $\lambda \in \sigma(T)$, then $\lambda \in \sigma_p(T)$ and $\dim{(T-\lambda)} < \infty$.
\end{thm}
\begin{proof}
	By the \textbf{Proposition} \ref{prop10} in the subsection \textbf{2.1.3}, since each $\lambda \in \sigma(T)$ is isolated, we can define
	\begin{equation*}
		E(\lambda) = \int_{\Gamma(\{\lambda\})} (z-T)^{-1} dz 
	\end{equation*}
	where $\Gamma(\{\lambda\})$ is the closed curve enclosed $\{\lambda\}$ and disjoint with other $\lambda$ in $\sigma(T)$. Then, each $E(\lambda)$ is a projection.
	\begin{equation*}
		\Hs_{\lambda} = E(\lambda)\Hs,~ T_{\lambda} = T|_{\Hs_{\lambda}} \colon \Hs_{\lambda} \sto \Hs_{\lambda}
	\end{equation*}
	Since $\sigma(T_{\lambda}) = \{\lambda\}$ and $\lambda \neq 0$, $T_{\lambda}$ is invertible. But clearly, $T_{\lambda}$ is compact. That means any bounded and closed subset in $\Hs_{\lambda}$ is compact, thus $\dim{\Hs_{\lambda}} < \infty$ by the Riesz Theorem. Therefore, by the result of finite linear algebra, $\lambda$ is the eigenvalue of $T_{\lambda}$, i.e. $\lambda \in \sigma_p(T)$. By above lemma, $\dim{(T-\lambda)} < \infty$.
\end{proof}

Combining all of above results, we can get an explicit structure of the spectrum of compact operator.

\begin{thm}[Riesz]
	If $\Hs$ is a infinite dimensional Hilbert space and $T \in \coper$, then one and only one of the following possibilities occurs.
	\begin{enumerate}[label=\arabic*)]
		\item $\sigma(T) = \{0\}$.
		\item $\sigma(T) = \{0,\lambda_1,\cdots,\lambda_n\}$, where each $\lambda_k \neq 0$, and each $\lambda_k$ is the eigenvalue of $T$ with $\dim{\ker{(T-\lambda_k)}}<\infty$.
		\item $\sigma(T) = \{0,\lambda_1,\lambda_2,\cdots\}$, where each $\lambda_k \neq 0$, and each $\lambda_k$ is the eigenvalue of $T$ with $\dim{\ker{(T-\lambda_k)}}<\infty$, and moreover, $\lim_{n \sto \infty} \lambda_n =0$.
	\end{enumerate}
\end{thm}

Because of its discrete spectrum, we can using the spectral measure to decompose the compact normal operator.

\begin{thm}[Spectral Theorem]
	If $T$ is a compact normal operator on a Hilbert space $\Hs$, then $T$ has at most countable eigenvalues $\{\lambda_n\}$, and there are corresponding projections $P_n \colon \Hs \sto \ker{(T-\lambda_n)}$ with $P_nP_m=P_mP_n=0$ s.t.
	\begin{equation*}
		T = \sum_{n=1}^{\infty} \lambda_n P_n
	\end{equation*}
	where $\{\lambda_n\}$ are all distinctive eigenvalues and the convergence is with respect to the norm topology.
\end{thm}
\begin{proof}
	Let $E$ be the spectral measure of $T$, then we see
	\begin{equation*}
		N = \int_{\sigma(N)} z dE
	\end{equation*}
	If $\dim{\Hs} = \infty$, since $\sigma(N)$ is consisted of at most countable eigenvalues $\{\lambda_n\}_{n=1}^{\infty}$ and $0$, then we can set $P_n = E(\{\lambda_n\})$ for $n = 1,2,\cdots$, which is well-defined by the \textbf{Proposition} \ref{prop11} in the subsection \textbf{3.1.2}, above integral can be
	\begin{eqnarray*}
		N &=& \lim_{n \sto \infty} (\int_{B_{\frac{1}{n}}(0)} z dz + \sum_{k=1}^{n} \lambda_k P_k) \\
		&=& \sum_{n=1}^{\infty} \lambda_n P_n
	\end{eqnarray*} 
	And moreover, $P_n$ is the projection from $\Hs$ onto $\ker{(T-\lambda_n)}$. \\
	If $\dim{\Hs} < \infty$, above theorem is clearly true.
\end{proof}

Also, we can see the Functional Calculus of the compact normal operators. Firstly, Taking same notation as above theorem and let $P_0 = 1-\sum_{n=1}^{\infty}$, in fact, $P_0$ is the projection from $\Hs$ to $\ker{T}$. By the Functional Calculus of normal operators, we know
\begin{equation*}
	\st{W}(T) = \{~\phi(T) \colon \phi \in \lfs{\infty}(\sigma(T))~\}
\end{equation*}
Since $\sigma(T)$ is discrete, $\lfs{\infty}(\sigma(T)) = l^{\infty}(\C)$, and for any $(a_n)_{n=0}^{\infty} \in l^{\infty}(\C)$,
\begin{equation*}
	(a_n)(T) = a_0 P_0 + \sum_{n=1}^{\infty} a_n P_n
\end{equation*}
Thus we can see the separating vector for $\st{W}(T)$, if $e$ is a separating vector and decomposing $e$ to $\{P_n\}_{n=0}^{\infty}$
\begin{equation*}
	e = \sum_{n=0}^{\infty} e_n,~ e_n \in P_n \text{ for each n}
\end{equation*} 
Then for $(a_n)_{n=0}^{\infty} \in l^{\infty}(\C)$
\begin{equation*}
	(a_n)(T)e = (a_0 P_0 + \sum_{n=1}^{\infty} a_n P_n)(\sum_{n=0}^{\infty} e_n) = \sum_{n=0}^{\infty} a_n e_n
\end{equation*}
Since $e$ is separating, $\sum_{n=0}^{\infty} a_n e_n = 0$ for any $(a_n) \in l^{\infty}(\C)$ implies $a_n = 0$ for all $n$. That means that $e_n \neq 0$ for any $n$. Then the corresponding measure $\mu$ is defined as
\begin{equation*}
	\mu(\{\lambda_n,\lambda_m\}) = \norm{E(\{\lambda_n,\lambda_m\})e}^2 = \norm{e_n+e_m}^2
\end{equation*}

Now, we can see the multiplicity function of compact normal operator. By similar construction in the \textbf{Theorem} \ref{thm10} and \textbf{Theorem} \ref{thm9} in the section \textbf{3.4}, the multiplicity function is like
\begin{equation*}
	m_T(\lambda) = \dim{\ker{(T-\lambda)}}
\end{equation*}
That means two compact normal operators are equivalent if and only if they have same dimension of all eigenspaces.

Finally,, we can prove $\coper$ is the unique closed ideal in $\oper$ if $\Hs$ is separable. 
\begin{prop}
	If $N$ is a normal operator in $\oper$ with the specture measure $E$, then $N$ is compact if and only if for any $\varepsilon > 0$,
	\begin{equation*}
		\dim{E(\{z \in \C \colon \abs{z} > \varepsilon\})} < \infty
	\end{equation*}
\end{prop}
\begin{proof}
	Let $\Delta_{\varepsilon} = \{z \in \C \colon \abs{z} > \varepsilon\}$ and $E_{\varepsilon} = E(\Delta_{\varepsilon})$.\\
	Assume that for any given $\varepsilon > 0$, $\dim{E_{\varepsilon}} < \infty$, then
	\begin{eqnarray*}
		N - NE_{\varepsilon} &=& \int z dE - \int z\chi_{\Delta_{\varepsilon}} dE \\
		&=& \int z\chi_{\C \backslash \Delta_{\varepsilon}} dE = \phi(N)
	\end{eqnarray*}
	where $\phi(z) = z\chi_{\C \backslash \Delta_{\varepsilon}}(z)$. Therefore,
	\begin{equation*}
		\norm{N - NE_{\varepsilon}} = \sup{\{\abs{z} \colon \C \backslash \Delta_{\varepsilon}\}} < \varepsilon
	\end{equation*}
	Thus $N \in \coper$. \\
	Conversely, if $N$ is compact, then for any $\varepsilon > 0$, put $\phi(z) = z^{-1}\chi_{\Delta_{\varepsilon}}$, then
	\begin{equation*}
		N\phi(N) &=& \int \chi_{\Delta_{\varepsilon}} dE = E_{\varepsilon} 
	\end{equation*}
	Since $E_{\varepsilon}$ is a compact projection, i.e. $\ran{E_{\varepsilon}}$ is closed, by the Riesz Theorem, $\dim{\ran{E_{\varepsilon}}} < \infty$.
\end{proof}

\begin{thm}
	If $\Hs$ is a separable Hilbert space and $\I$ is an ideal of $\oper$ that contains a non-compact operator, then $\I = \oper$.
\end{thm}
\begin{proof}
	Let $T \in \I \backslash \coper$ then 
	\begin{equation*}
		\st{T}T = \int_{\sigma(\st{T}T)} t dE(t)
	\end{equation*}
	By above theorem, there is an $\varepsilon > 0$ s.t. $P = E(\varepsilon,\infty)$ has infinite rank. 
	\begin{equation*}
		P = (\int t^{-1}\chi_{(\varepsilon,\infty)}(t)dE(t))\st{T}T \in \I
	\end{equation*}
	Since $\Hs$ is separable, $\dim{P\Hs} = \dim{\Hs} = \aleph_0$, there is a unitary $U$ from $\Hs$ to $P\Hs$. Therefore, $1 = \st{U}PU \in \I$. $\I = \oper$.
\end{proof}

\begin{prop}
	If $\I$ is a closed ideal of $\oper$, the $\coper \subset \I$ or $\I = \{0\}$.
\end{prop}
\begin{proof}
	Since $\I$ is closed, $\I$ is self-adjoint by the  \textbf{Proposition} \ref{prop12} in the subsection \textbf{2.2.5}. Then by the \textbf{Theorem} \ref{thm6} in the subsection \textbf{2.2.5}, we know $\I = \I \cap \st{\I}$ is a hereditary subalgebra. \\
	Therefore, if $\I$ is nonzero and $T \in \I$ is nonzero, then there is a finite rank projection $P$ s.t.
	\begin{equation*}
		0 \leqslant P \leqslant \st{T}T
	\end{equation*}
	Thus, $P \in \I$. Morover, any finite rank projection is in $\I$. Since $\I$ is norm closed, $\coper \subset \I$.
\end{proof}

Combining above propositions and theorems, we can get the final result.
\begin{cor}
	If $\Hs$ is a separable Hilbert space, then the only nontrivial closed ideal of $\oper$ is $\coper$.
\end{cor}

\section{Compact Operator Algebras}

For a normal operator, it can be decomposed as the direct sum of $*$-cyclic normal operators. Therefore, we want to reseach the similar property of compact operators. Like researching the normal operator, we firstly begin with the representation of compact operators. Let $\A$ denote a $\st{C}$-subalgebra in $\coper$.

\subsection{Minimal Projections}

\begin{defn}
	Let $\A$ is a $\st{C}$-subalgebra in $\coper$
	\begin{enumerate}[label=\arabic*)]
		\item A projection $E$ in $\A$ is minimal if $E \neq 0$ and there are no nonzero projection $P$ in $\A$ s.t. $P < E$.
		\item $\A$ is called irreducible if $\A$ has no proper reducing subspaces.
	\end{enumerate}
\end{defn}

\begin{prop}
	Let $E$ be a projection in $\A$.
	\begin{enumerate}[label=\arabic*)]
		\item $E$ is minimal if and only if
		\begin{equation*}
			E\A E = \{~\lambda E \colon \lambda \in \C~\}
		\end{equation*}
		\item Every projection in $\A$ is the direct sum of a finite number of pairwise orthogonal minimal projections in $\A$.
	\end{enumerate}
\end{prop}
\begin{proof}
	For $1)$, clearly, $E \A E$ is a $\st{C}$-subalgebra of $\coper$. Thus for any self-adjoint operator $A \in E \A E$, each nonzero spectrum point $\lambda$ of $A$ is isolated. That means $\chi_{\{\lambda\}}$ is continuous, thus $E(\{\lambda\})$ is a projection in $E \A E$. However, the minimality of $E$ implies that $E(\{\lambda\}) = E$. Therefore, by the fact that $E \A E$ can be generated by all self-adjoint operators in $E \A E$, $E\A E = \{\lambda E \colon \lambda \in \C\}$. The converse is trivial.
	\item For $2)$, since each projection in $P$ $\A$ is compact, $P$ has finite rank. Thus $P$ can he direct sum of a finite number of pairwise orthogonal minimal projections.
\end{proof}

Using the minimal projection, we can get an important trait the irreducible compact subalgebra has.
\begin{thm}
	If $\A$ is an irreducible $\st{C}$-subalgebra of $\coper$, then $\A = \coper$.
\end{thm}
\begin{proof}
	Let $E$ be a minimal projection in $\A$.\\
	Claim: The rank of $E$ is one. \\
	If $\rank{E} > 1$, then there are nonzero $g,h \in E$ with $g \bot h$. Let $A \in \A$ and $\lambda \in \C$ s.t. $EAE = \lambda A$. Then
	\begin{equation*}
		\langle Ag,h \rangle = \langle EAEg,h \rangle = \lambda \langle g,h \rangle = 0
	\end{equation*}
	Thus, $h \bot \clo{\A g}$. But since $\A$ is irreducible, $\clo{\A g} = \Hs$. So $h = 0$ contradicted to the assumption.
	\item Claim: If $P$ is any rank one projection, then $P \in \A$. \\
	Let $e \in E$ and $p \in P$. By the irreducibility of $\A$, $\Hs = \clo{\A e}$. Then for any given $\varepsilon > 0$, 
	\begin{equation*}
		\norm{Ae - p} < \varepsilon,~ \text{ for some } A \in \A
	\end{equation*}
	Let $P^{'} = AE$, then $P^{'} \in \A$. And moreover, for any $h \in \Hs$
	\begin{eqnarray*}
		\norm{(P-P^{'})h} &=& \norm{\langle h,p \rangle p - \langle h, Ae \rangle Ae}\\ 
		&=& \norm{\langle h,p-Ae \rangle Ae + \langle h,p \rangle (p-Ae)} \\
		&\leqslant& \norm{h} \norm{Ae} \varepsilon + \norm{h}\norm{p} \varepsilon
	\end{eqnarray*}
	Thus by the fact that $\A$ is norm closed, $P \in \A$. \\
	Therefore, any finite rank projection in $\A$, i.e. $\A = \coper$.
\end{proof}

\begin{cor}
	For any Hilbert space $\Hs$, $\coper$ is simple.
\end{cor}
\begin{proof}
	Let $\I$ be an closed ideal in $\coper$. Then $\clo{\I e}$ for any nonzero $e$ is a reducing space for $\coper$. Thus $\clo{\I e} = \Hs$ since $\coper$ is irreducible. That means $\I$ is irreducible. Thus by above theorem, $\I = \coper$.
\end{proof}
\begin{rem}
	In fact, the converse of the result in above proof is also true. $\B$ is a irreducible subalgebra contained in $\oper$ if and only if $\clo{\B e} = \Hs$ for any $e \in \Hs$. And because of this, we can get the genera version of the corollary that if $\B$ is a irreducible subalgebra contained in $\oper$, then any ideal of it is irreducible.
\end{rem}

\begin{cor}
	Let $\Hs$ be a Hilbert space. If $\B$ is a irreducible $\st{C}$-subalgebra of $\oper$ and $\B \cap \coper \neq 0$, then $\coper \subset \B$.
\end{cor}
\begin{proof}
	Let $\I = \B \cap \coper$. Then $\I$ is a closed ideal in $\B$. By above corollary, $\I = \coper$, i.e. $\coper \subset \B$.
\end{proof}


Now, the unit element of a compact operator subalgebra can be provided.
\begin{prop}
	If $E$ is a minimal projection in $\A$, $e$ is any nonzero vector in $\Hs$ and $\Hs_e = \clo{\A e}$, then $\A|_{\Hs_e} = \fml{B}_0(\Hs_e)$.
\end{prop}
\begin{proof}
	Let $\A_e = \A|_{\Hs_e}$. We just need to show $\A_e$ is irreducible in $\fml{B}(\Hs_e)$. Let $P$ be a projection in $\fml{B}(\Hs_e)$ commuting with $\A_e$. Put 
	\begin{equation*}
		T = P - \langle Pe,e \rangle 1
	\end{equation*}
	Then $T \in \A_e^{'}$, for any $A,B \in \A$,
	\begin{eqnarray*}
		\langle TAe, Be \rangle &=& \langle T \st{B}A E e, E e \rangle \\
		&=& \langle T E \st{B}A E e,e \rangle
	\end{eqnarray*}
	Since $E$ is minimal, $E \st{B}A E = \lambda E$ for some $\lambda \in \C$. Therefore,
	\begin{equation*}
		\langle TAe, Be \rangle = \lambda \langle Te, e \rangle = \lambda (\langle Pe, e \rangle - \langle Pe, e \rangle) = 0
	\end{equation*}
	Since $A,B$ are arbitrary and $\Hs_e = \clo{\A e}$, $T = 0$, that means $P = \langle Pe,e \rangle 1$. Thus, $P = 0$ or $1$.
\end{proof}

\subsection{Representations of Compact Operator Algebras}

For the representation of a compact operator subalgebra, it can be decomposed as the direct sum of irreducible parts.
\begin{thm}
	If $(\rho, \fml{K})$ is a non-degenerate representation of $\A$, then there are irreducible representations $\{\rho_i\}_{i \in I}$ s.t. each $\rho_i$ is equivalent to a subrepresentation of the identity representation $i \colon \A \sto \oper$.
\end{thm}
\begin{proof}
	Claim: There is a minimal projection $E \in \A$ s.t. $\rho(E) \neq 0$. \\
	In fact, there is a self-adjoint operator $A \in \A$ s.t. $\rho(A) \neq 0$. Otherwise, $\rho$ is the zero representation. Then there is a spectral projection $F$ for $A$ with $\rho(F) \neq 0$. Then since $F$ is the direct sum of finite numbers minimal projections, that means $\rho$ must not vanish on at least one of them.
	\item For this $E$, choosing a unit vector $e_0 \in \rho(E)$ and define $\fml{K}_0 = \clo{\rho(\A)e_0}$ and $\rho_0(A) = \rho(A)|_{\fml{K}_0}$. Then \\
	Claim: $(\rho_0,\fml{K}_0)$ is equivalent to a subrepresentation of the identity representation of $\A$.\\
	Define the unitary $U$ from $\fml{K}_0$ to $\Hs_0 = \clo{\A e}$, where $e$ is the unit vector in $E$. 
	\begin{center}
		\begin{tabular}{l c c l}
			$U \colon$ & $\fml{K}_0$ & $\longrightarrow$ & $\Hs_0$ \\
			~ & $\rho(A)e_0$ & $\longmapsto$ & $Ae$
		\end{tabular}
	\end{center}
	Since $E$ is minimal, there exists a unique $\lambda$ s.t. $E \st{A}A E = \lambda E$ for $A \in \A$.
	\begin{eqnarray*}
		\norm{\rho(A)e_0}^2 &=& \norm{\rho(AE)e_0}^2 = \langle \rho(E \st{A}AE)e_0,e_0 \rangle\\
		&=& \lambda \langle e_0,e_0 \rangle = \lambda \langle e,e \rangle \\
		&=& \langle E \st{A}AE e_0,e_0 \rangle \\
		&=& \norm{Ae}^2
	\end{eqnarray*}
	Therefore, $U$ is surjective isometry, and can extend to $\fml{K}_0$. And moreover,
	\begin{equation*}
		U\rho(A)\rho(B)e_0 = A Be ~\Rightarrow~ U\rho(A) \st{U} Be = A|_{\Hs_0} Be
	\end{equation*}
	Thus, $U\rho_0(A)\st{U} = A|_{\Hs_0}$. By above proposition, $\rho_0$ is irreducible.
	\item Then by Zorn's Lemma, we can get a maximal family of $(\rho_i, \fml{K}_i)_{i \in I}$ as above construction. And by the maximality and the non-degenerality, 
	\begin{equation*}
		\rho = \oplus_{i \in I} \rho_i,~ \fml{K} = \oplus_{i \in I} \fml{K}_i \qedhere
	\end{equation*}
\end{proof}
\begin{rem}
	If $\rho$ is degenerate, then let $\Hs_0 = \clo{\rho(\A) \Hs}$. The representation $(\rho_0(A) = \rho(A)|_{\Hs_0}, \Hs_0)$ is a non-degenerate representation, thus above theorem can be applied to it. Moreover, $\rho = \rho_0 \oplus 0$. 
\end{rem}

Now, we can use above theorem to get the structure of any finite dimensional $\st{C}$-algebra.
\begin{cor}
	For any finite dimensional $\st{C}$-algebra $\A$ and let $M(n)$ be the set of $n \times n$ matrices with complex entries acting on the inner product space $\C^{n}$., there are $n_1,\cdots,n_p \in \N$, s.t.
	\begin{equation*}
		\A \cong M(n_1) \oplus M(n_2) \oplus \cdots \oplus M(n_p)
	\end{equation*}
\end{cor}

Also, there are some other corollaries of above theorem.
\begin{cor} \label{cor10}
	\begin{enumerate}[label=\arabic*)]
		\item If $(\rho, \fml{K})$ is an irreducible representation of $\A$, then $\rho(\A) = \fml{B}_0(\fml{K})$.
		\item If $\Hs$ and $\fml{K}$ are Hilbert spaces and $\rho \colon \coper \sto \fml{B}_0(K)$ is an $*$-isomorphism, then there is a unitary $U \colon \Hs \sto \fml{K}$, s.t.
		\begin{equation*}
			\rho(T) = U T \st{U}, \text{ for any } T \in \coper
		\end{equation*}
		\item If $\Hs$ and $\fml{K}$ are Hilbert spaces and $\rho \colon \oper \sto \fml{B}(K)$ is an $*$-isomorphism, then there is a unitary $U \colon \Hs \sto \fml{K}$, s.t.
		\begin{equation*}
			\rho(A) = U A \st{U}, \text{ for any } A \in \oper
		\end{equation*}
	\end{enumerate}
\end{cor}
\begin{proof}
	$1)$ is the direct result from above theorem. And $2)$ can be obtained by $1)$. \\
	For $3)$, since $\rho(\coper)$ is an ideal of $\fml{K}$ and $\fml{K}$ is irreducible, $\rho(\coper)$ is irreducible by above corollary. Thus $\rho(\coper) = \fml{B}_0(\fml{K})$. Then by $2)$, there is a unitary $U \colon \Hs \sto \fml{K}$, s.t.
	\begin{equation*}
		\rho(A) = U A \st{U}, \text{ for any } A \in \coper
	\end{equation*}
	Let $\{E_i\}$ be the approximatel identity for $\coper$ consisting with all finite projections. For any $A \leqslant 0$ in $\Hs$, then it can see
	\begin{equation*}
		A_i = A^{\frac{1}{2}} E_i A^{\frac{1}{2}} \sto A \text{ in } SOT
	\end{equation*}
	Therefore, $\rho(A_i) = U A_i \st{U} \sto U A \st{U} = T$ in $SOT$. Since $A_i \leqslant A$, $\rho(A_i) \leqslant \rho(A)$ and thus $T \leqslant \rho(A)$ by the fact that $\{A_i\}$ is increasing. Conversely, $\rho(A_i) \leqslant T$ implies that $A_i \leqslant \rho^{-1}(T)$. Thus $A \leqslant \rho^{-1}(T)$, thus $\rho(A) \leqslant T$.
\end{proof}

Here is an important example, which can be used to constract the $AF$ algebra.

\begin{exam}
	Let $M(n)$ be the set of $n \times n$ matrices with complex entries acting on the inner product space $\C^{n}$. If
	\begin{equation*}
		\rho \colon M(m) \longrightarrow M(n)
	\end{equation*}
	is a $*$-homomorphism, then there is an integer $k$ s.t. $km \leqslant n$ and a unitary $U$ in $M(n)$ and 
	\begin{equation*}
		\rho(x) = U(\underbrace{x \oplus x \oplus \cdots \oplus x}_k \oplus 0)\st{U}
	\end{equation*}
	In general, define $\mathbf{m} = (m_1, m_2, \cdots, m_p)$, where $m_i \in \N$, and
	\begin{equation*}
		M(\mathbf{m}) = M(m_1) \oplus M(m_2) \oplus \cdots M(m_p)
	\end{equation*}
	Then for $\mathbf{m} = (m_1, m_2, \cdots, m_p)$ and $\mathbf{n}=(n_1, n_2, \cdots, n_q)$, if $\rho \colon M(\mathbf{m}) \rightarrow M(\mathbf{n})$ is a $*$-homomorphism, then there this a $q \times p$ matirx $[k_{ij}]$ with integer entries s.t.
	\begin{equation*}
		\rho(x_1,\cdots,x_p) = U_1(\underbrace{x_1 \oplus \cdots \oplus x_1}_{k_{11}} \oplus 0)\st{U}_1 \oplus \cdots \oplus U_q(\underbrace{x_p \oplus \cdots \oplus x_p}_{k_{qp}} \oplus 0)\st{U}_q
	\end{equation*}
	By above corollary, we see for any finite dimensional $\st{C}$-algebras $\A$, there exists a $\mathbf{n}=(n_1, n_2, \cdots, n_q)$ s.t. $\A \cong M(\mathbf{n})$. Then we can get the form of the $*$-homomorphism between any two finite dimensional $\st{C}$-algebras. \\
	If $\rho$ is a $*$-homomorphism between two finite dimensional $\st{C}$-algebras\\ $\A \cong M(\mathbf{m})$ and $\B \cong M(\mathbf{n})$, where $\mathbf{m} = (m_1, m_2, \cdots, m_p)$ and $\mathbf{n}=(n_1, n_2, \cdots, n_q)$, then $\rho$ is determined up to a $q \times p$ matirx $[k_{ij}]$ with integer entries. In fact, we can use it to construct the $AF$ algebras.
\end{exam}

\subsection{Decompositions of Compact Operator Algebras}

By above theorem and the Zorn's Lemma, for the non-degenerate $\A$, there is a maximal family $\{E_i\}_{i \in I}$ of pairwise orthogonal minimal projections in $\A$ and let $\Hs_i = \clo{\A E_i \Hs}$ s.t. 
\begin{equation*}
	\Hs = \oplus_{i \in I} \Hs_i~,~ \A|_{\Hs_i} = \fml{B}_0(\Hs_i)
\end{equation*}
But we cannot just using this as the decomposition of the compact operator algebra $\A$, since there are some "equivalent" relationships of these $\Hs_i$. Firstly, we say that $\Hs_i$ does not dependent on $\Hs_j$, if there is a $A \in \A$ s.t. $\A|_{\Hs_j} = 0$ and $\A|_{\Hs_i} \neq 0$.

 \begin{prop}
 	$\Hs_i$ does not dependent on $\Hs_j$ if and only if
 	\begin{equation*}
 		A|_{\Hs_i \oplus \Hs_j} = \fml{B}_0(\Hs_i) \oplus \fml{B}_0(\Hs_j)
 	\end{equation*}
 \end{prop}
 \begin{proof}
 	Assume that $\Hs_i$ does not dependent on $\Hs_j$. Let
 	\begin{equation*}
 		\I = \{~A \in \A \colon A|_{\Hs_j} = 0~\}
 	\end{equation*}
 	Since $\Hs_j$ is reducing, $\I$ is a closed ideal. Moreover, $\I|_{\Hs_i}$ is a closed ideal of $\A|_{\Hs_i} = \fml{B}_0(\Hs_i)$. By the assumption, $\I|_{\Hs_i}$ is not zero, thus $\I|_{\Hs_i} = \fml{B}_0(\Hs_i)$. Let $K_i$ and $K_j$ be arbitrary compact operators on $\Hs_i$ and $\Hs_j$. And $A \in \A$ s.t. $A|_{\Hs_j} = K_j$. Since $\I|_{\Hs_i} = \fml{B}_0(\Hs_i)$, there exists $B \in \I$ s.t. $B|_{\Hs_i} = K_i - A|_{\Hs_i}$. Thus,
 	\begin{equation*}
 		(A+B)|_{\Hs_i} = K_i,~~\&~~ (A+B)|_{\Hs_j} = K_j
 	\end{equation*}
 	The converse is trivial.
\end{proof}
\begin{rem}
	By this proposition, if $\Hs_i$ does not dependent with $\Hs_j$, $\Hs_j$ does not $\Hs_i$. $\Hs_i$ and $\Hs_j$ are independent if they are not dependent.
\end{rem}

Now we want to research the same properties of the dependent subspaces have, and then we can devide all ${\Hs_i}_{i \in I}$ into dependency class.

\begin{prop}
	If $\Hs_i$ and $\Hs_j$ are dependent, then
	\begin{equation*}
		\norm{A|_{\Hs_i}} = \norm{A|_{\Hs_j}}~, \text{ for any } A \in \A
	\end{equation*}
	Moreover, two dependent spaces are isomorphic, and each dependency class is finite.
\end{prop}
\begin{proof}
	By the hypothesis, if $A \in \A$ s.t. $A|_{\Hs_i} = 0$, then $A|_{\Hs_j} = 0$. Then we can define the map
	\begin{equation*}
		\rho \colon  \fml{B}_0(\Hs_j)  \longrightarrow  \fml{B}_0(\Hs_i)
	\end{equation*}
	For any $K \in \fml{B}_0(\Hs_j)$, there is a $A \in \A$ s.t. $A|_{\Hs_j} = K$, then we define that $\rho(K) = A|_{\Hs_i}$.
	\item Claim: $\rho$ is well-defined. \\
	If $B \in \A$ s.t. $B|_{\Hs_j} = K$, then $(A-B)|_{\Hs_j} = 0$. By the dependency of $\Hs_j$ and $\Hs_i$, $(A-B)|_{\Hs_i} = 0$, i.e. $A|_{\Hs_i} = B|_{\Hs_i}$.
	\item $\rho$ is a $*$-isomorphism. \\
	If $K \in \fml{B}_0(\Hs_j)$ s.t. $\rho(K) = 0$, then there is a $A \in \A$ s.t. $A|_{\Hs_j} = K$ and $A|_{\Hs_i} = 0$. Also, by the dependency, $K = A|_{\Hs_j} = 0$. Clearly, $\rho$ is a $*$-homomorphism, then $\rho$ is a $*$-monomorphism, i.e. $\rho$ is an isometry. And the surjectivity of $\rho$ is similar by the dependency of $\Hs_j$ and $\Hs_i$.
	\item Then by the \textbf{Corollary} \ref{cor10} in above subsection, we can find a unitary from $\Hs_j$ to $\Hs_i$. Thus $\dim{\Hs_j} = \dim{\Hs_i}$. And if there are infinite element in a dependency class, by the compactness of $A$, $\norm{A|_{\Hs_i}} \sto 0$, which is a contradiction.
\end{proof}

\begin{thm}
	If $\A$ is a $\st{C}$-subalgebra of compact operators, then
	\begin{equation*}
		\Hs \cong \Hs_0 \oplus \oplus_{d \in D} \Hs_d^{(k_d)}
	\end{equation*}
	and
	\begin{equation*}
		\A \cong \{~0 \oplus \oplus_{d \in D} K_d^{(k_d)} \colon K_d \in \fml{B}_0(\Hs_d)~\}
	\end{equation*}
\end{thm}
\begin{proof}
	Firstly, by above mention, if $\A$ is non-degenerate, there are $\{\Hs_i\}_{i \in I}$ s.t. 
	\begin{equation*}
		\Hs = \oplus_{i \in I} \Hs_i~,~ \A|_{\Hs_i} = \fml{B}_0(\Hs_i)
	\end{equation*}
	But by above theorem, we can divide $\{\Hs_i\}$ into the dependency classes, and re-labeled as 
	\begin{equation*}
		\{~\underbrace{\Hs_1,\cdots,\Hs_1}_{k_1},\underbrace{\Hs_2,\cdots,\Hs_2}_{k_2}, \cdots~\}
	\end{equation*}
	Thus, for the non-degenerate $\A$,
	\begin{equation*}
		\Hs \cong \oplus_{d \in D} \Hs_d^{(k_d)}, \A \cong \{~\oplus_{d \in D} K_d^{(k_d)} \colon K_d \in \fml{B}_0(\Hs_d)~\}
	\end{equation*}
	If $\A$ is degenerate, let $\Hs_0 = (\clo{\A \Hs})^{\bot}$, then this theorem holds.
\end{proof}

For a non-degenerate $\A$, let $\hat{\A}$ denote the set of all equivalent classes of irreducible representations of $\A$. For any $\zeta \in \hat{\A}$, let $\rho_{\zeta} \in \zeta$. By the theorem, for any representation $\rho$ of $\A$, there are irreducible representations $\{\rho_i\}$ s.t. $\rho = \oplus_i \rho_i$, then we define the mutiplicity function of $\rho$,
\begin{equation*}
	m_{\rho}(\zeta) = \#\{i \colon \rho_i \in \zeta\}
\end{equation*}
Now, we can easily apply above theorem to the representation. 

\begin{thm}
	If $\rho$ is a representation of $\A$ and $m_{\rho}$ is the multiplicity function, then
	\begin{equation*}
		\rho \cong \oplus \{~\rho_{\zeta}^{(m_{\rho}(\zeta))} \colon \zeta \in \hat{\A}~\}
	\end{equation*}
	Moreover, any two representations of $\A$ are equivalent if and only if they have same multiplicity functions.
\end{thm}

\section{Trace Class and Ultraweak Topology}

There some interesting subalgebras in the compact operator algebras, and some of them can provide extra topologies on operator algebras, which will be helpful in the research of general von Neumann algebras.

\subsection{Trace Class and Hilbert-Schmit Operators}

\begin{defn}
	Let $T \in \oper$. If there is a a orthonormal basis $\fml{E}$ of $\Hs$ s.t.
	\begin{equation*}
		\sum_{e \in \fml{E}} \langle \abs{T}e,e \rangle < \infty
	\end{equation*}
	then $T$ is called trace class. Let $\toper$ denote the set of all trace classes operators.
\end{defn}

In this definition, the orthonormal basis just need to exist. But how can that condition garantees for any orthonormal basis of $\Hs$ the condition is always satisfied?

\begin{prop}
	If $\fml{E}$ and $\fml{F}$ are two orthonormal bases for $\Hs$, then for any $T \in \oper$,
	\begin{equation*}
		\sum_{e \in \fml{E}} \norm{Te}^2 = \sum_{f \in \fml{F}} \norm{\st{T}f}^2 = \sum_{e \in \fml{E}}\sum_{f \in \fml{F}} \abs{\langle Te,f \rangle}^2
	\end{equation*}
\end{prop}
\begin{proof}
	By the Parseval's Identity,
	\begin{equation*}
		\norm{Te}^2 = \sum_{f \in \fml{F}} \abs{\langle Te,f \rangle}^2
	\end{equation*}
	Then we get above identity.
\end{proof}

\begin{cor}
	The sum 
	\begin{equation*}
		\sum_{e \in \fml{E}} \langle \abs{T}e,e \rangle
	\end{equation*}
	is independent with the choise of the $\fml{E}$.
\end{cor}

Therefore, by this corollary, above definition is valid. Moreover, we can define one more norm on $\toper$, for $T \in \toper$ and some orthonormal basis $\fml{E}$,
\begin{equation*}
	\norm{T}_1 = \sum_{e \in \fml{E}} \langle \abs{T}e,e \rangle
\end{equation*}

Then by using this norm, called the trace norm, on $\toper$, the $\toper$ can be a Banach space.

\begin{defn}
	$T \in \oper$ is called a Hilbert-Schimdt operator if $\abs{T}^2$ is trace class. Let $\hoper$ denote the set of all Hilbert-Schimdt operators. 
\end{defn}

Also, we can define the norm on $\hoper$, for any orthonormal basis $\fml{E}$ in $\Hs$,
\begin{equation*}
	\norm{T}_2 = (\sum_{e \in \fml{E}} \langle \abs{T}^2e,e \rangle)^{\frac{1}{2}} = (\sum_{e \in \fml{E}} \norm{\abs{T}e}^2)^{\frac{1}{2}} = (\sum_{e \in \fml{E}} \norm{Te}^2)^{\frac{1}{2}} = \norm{\abs{T}^2}_1^{\frac{1}{2}}
\end{equation*}
And $\hoper$ can be also a Banach space.

In order to research the trace class, it is convinient to get some properties of the Hilbert-Schmidt operators.

\begin{prop}
	Let $T \in \hoper$.
	\begin{enumerate}[label=\arabic*)]
		\item $\norm{T}_2 = (\sum_{e \in \fml{E}} \norm{Te}^2)^{\frac{1}{2}}$.
		\item $\norm{\st{T}}_2 = \norm{T}_2$.
		\item $\norm{T} \leqslant \norm{T}_2$.
		\item If $A \in \oper$, then $AT,TA \in \hoper$ and 
		\begin{equation*}
			\norm{AT}_2, \norm{TA}_2 \leqslant \norm{A} \norm{T}_2
		\end{equation*}
		\item $\hoper$ is an ideal of $\oper$ and $\norm{\cdot}_2$ is a norm on $\hoper$.
	\end{enumerate}
\end{prop}
\begin{proof}
	The first three results are trivial. For $4)$, fix an orthonormal basis $\fml{E}$ and an $A \in \oper$, for a $e \in \fml{E}$,
	\begin{equation*}
		\norm{ATe}^2 \leqslant \norm{A}^2 \norm{Te}^2 = \norm{A}^2 \norm{\abs{T}e}^2
	\end{equation*}
	Therefore, $\norm{AT}_2 \leqslant \norm{A} \norm{T}_2$.  \\
	For $5)$, we just need to prove the addition is continuous and closed. For a fixed normal basis $\fml{E}$ and $T,S \in \hoper$, then 
	\begin{equation*}
		\{\norm{Te} \colon e \in \fml{E}\}, \{\norm{Se} \colon e \in \fml{E}\} \in l^{2}(\fml{E})
	\end{equation*}
	By the triangle inequality for $l^{2}(\fml{E})$, 
	\begin{equation*}
		\norm{S+T}_2 = \left(\sum_{E}(\norm{Te}+\norm{Se})^2 \right)^{\frac{1}{2}} \leqslant \norm{T}_2 + \norm{S}_2 \qedhere
	\end{equation*}
\end{proof}
\begin{rem}
	Therefore, $\hoper$ with the fixed involution and the norm $\norm{\cdot}_2$ is a $\st{C}$-algebra.
\end{rem}

By using these results, we can see the Hilbert-Schmidt operator is compact.

\begin{cor}
	If $T \in \hoper$ and $\varepsilon > 0$, there is a $A \in \foper$ s.t.
	\begin{equation*}
		\norm{T-A}_2 \leqslant \varepsilon
	\end{equation*}
	Consequently, every Hilbert-Schmidt operator is compact.
\end{cor}
\begin{proof}
	For this fixed $\varepsilon > 0$, since $T \in \hoper$, there is a finite set $I \in \fml{E}$ s.t.
	\begin{equation*}
		\sum_{\fml{E} \backslash I} \norm{Te}^2 < \varepsilon^2
	\end{equation*}
	Let $\hat{E} = \spn{E}$ and $B = T|_{\hat{E}} \in \foper$, then
	\begin{equation*}
		\norm{T-B}_2 = (\sum_{\fml{E} \backslash I} \norm{Te}^2)^{\frac{1}{2}} < \varepsilon \qedhere
	\end{equation*}
\end{proof}

Then, there are similar consequences of the $\toper$.

\begin{prop}
	If $T \in \oper$, then the following statements are equivalent.
	\begin{enumerate}[label=\arabic*)]
		\item $T \in \toper$.
		\item $\abs{T}^{\frac{1}{2}} \in \hoper$.
		\item $T$ is the product of two Hilbert-Schmidt operators.
		\item $\abs{T}$ is the product of two Hilbert-Schmidt operators.
	\end{enumerate}
\end{prop}
\begin{proof}
	Since $\norm{\abs{T}^{\frac{1}{2}}}{e}^2$, $1)$ implies $2)$. And by the Polar Decomposition,
	\begin{equation*}
		T = W \abs{T} = (W \abs{T}^{\frac{1}{2}})\abs{T}^{\frac{1}{2}}
	\end{equation*}
	thus $2)$ implies $3)$. And $3)$ implies $3)$, because of the Polar Decomposition. \\
	Suppose $\abs{T} = BC$ for $B,C \in \hoper$, then for any orthonormal $\fml{E}$ and $e \in \fml{E}$
	\begin{equation*}
		\langle \abs{T}e, e \rangle = \langle Ce \st{B}e \rangle \leqslant \norm{Ce}\norm{\st{B}e}
	\end{equation*}
	Therefore, we have 
	\begin{eqnarray*}
		\sum_{\fml{E}} \langle \abs{T}e, e \rangle &\leqslant& \sum_{\fml{E}} \norm{Ce}\norm{\st{B}e} \\
		&\leqslant& (\sum_{\fml{E}} \norm{Ce}^2)^{\frac{1}{2}} (\sum_{\fml{E}} \norm{Be}^2)^{\frac{1}{2}} \\ 
		&=& \norm{C}_2 \norm{B}_2
	\end{eqnarray*}
	Thus, $T \in \toper$.
\end{proof}

\begin{defn}
	If $T \in \toper$ and $\fml{E}$ is any orthonormal basis, the trace of $T$ is defined as
	\begin{equation*}
		\tr{T} = \sum_{\fml{E}} \langle Te,e \rangle
	\end{equation*}
\end{defn}

It need to check that this definition is well-defined.

\begin{prop}
	If $T \in \toper$ and $\fml{E}$ is any orthonormal basis, then $\sum_{\fml{E}} \abs{\langle Te,e \rangle} < \infty$, and the sum $\sum_{\fml{E}} \langle Te,e \rangle$ is independent with the choice of $\fml{E}$.
\end{prop}
\begin{proof}
	By above proposition, $T = \st{C}B$ for $C,B \in \hoper$. Since $\norm{(C-\lambda B)e}^2 \leqslant 0$ for any $\lambda \in \C$,
	\begin{equation*}
		2 \Rea{\clo{\lambda} \langle Be,Ce \rangle} \leqslant \norm{Be}^2+\abs{\lambda} \norm{Ce}^2
	\end{equation*}
	Choosing a $\lambda$ s.t. $\abs{\lambda} = 1$ and
	\begin{equation*}
		\clo{\lambda} \langle Be,Ce \rangle = \abs{\langle Be,Ce \rangle}
	\end{equation*}
	Then we have,
	\begin{equation*}
		\abs{\langle Te,e \rangle} = \abs{\langle Be,Ce \rangle} \leqslant \frac{1}{2}(\norm{Be}^2+\norm{Ce}^2)
	\end{equation*}
	Thus $\sum_{\fml{E}} \abs{\langle Te,e \rangle} \leqslant \frac{1}{2}(\norm{B}_2^2+\norm{C}_2^2)$. \\ 
	And since 
	\begin{eqnarray*}
		\Rea{\langle Te,e \rangle} &=& \frac{1}{4} (\norm{(B+C)e}^2 - \norm{(B+C)e}^2) \\
		\Img{\langle Te,e \rangle} &=& \frac{1}{4} (\norm{(iB+C)e}^2 - \norm{(iB+C)e}^2)
	\end{eqnarray*}
	then
	\begin{equation*}
		\sum_{\fml{E}} \langle Te,e \rangle = \Rea{\langle Te,e \rangle} + i\Img{\langle Te,e \rangle}
	\end{equation*}
	Therefore, the sum is independent with the choice of the orthonormal basis $\fml{E}$.
\end{proof}

For convinience, there is another notation to denote the rank one operators.
\begin{defn}
	For two vectors $g,h \in \Hs$, the rank one operator $g \otimes h$ is defined as
	\begin{equation*}
		g \otimes h (f) = \langle f,h \rangle g
	\end{equation*}
\end{defn}

There are some elementary properties for this rank one operator.
\begin{prop}
	Let $g,h \in \Hs$.
	\begin{enumerate}[label=\arabic*)]
		\item $e \otimes e$ is the projection onto $\C e$.
		\item $\ran{g \otimes h} = \C g$ and $\ker{g \otimes h} = (\C h)^{\bot}$ for $g \neq 0$.
		\item $(g \otimes h)^{*} = h \otimes g$.
		\item The map $(g,h) \sto g \otimes h$ is a sesquilinear from $\Hs \times \Hs$ to $\oper$.
		\item $\norm{g \otimes h} = \norm{g} \norm{h}$.
		\item If $T \in \oper$, then $T(g \otimes h) = (Tg) \otimes h$ and $(g \otimes h) T = g \otimes (\st{T}h)$.
		\item Every any finite rank operator can be expressed as
		\begin{equation*}
			\sum_{k=1}^{n} g_k \otimes h_k
		\end{equation*}
		for some $g_1, \cdots, g_n \in \Hs$ and $h_1, \cdots, h_n \in \Hs$.
	\end{enumerate}
\end{prop}

There are more interesting properties of the trace class.

\begin{thm}
	\begin{enumerate}[label=\arabic*)]
		\item $\toper \subset \coper$ and conversely, if $A \in \coper$ and $\{\alpha_n\}$ are the eigenvalues of $\abs{A}$, then $A \in \toper$ if and only if $(\alpha_n) \in l^{1}$. In this case, $\norm{A}_1 = \sum \alpha_n$.
		\item $\toper$ is an ideal of $\oper$.
		\item $\tr{} \colon \toper \sto \C$ is a positive non-degenerate linear functional.
		\item $\foper$ is a dense subset in $\toper$ with respect to $\norm{\cdot}_1$.
		\item If $T \in \toper$, then for any $A \in \oper$,
		\begin{equation*}
			\tr{TA} = \tr{AT},~\text{and } \abs{\tr{AT}} \leqslant \norm{A}\norm{T}_1
		\end{equation*}
		\item $\norm{T}_1 = \norm{\st{T}}_1$ for any $T \in \toper$.
		\item If $T \in \toper$ and $A \in \oper$, then
		\begin{equation*}
			\norm{TA}_1,~ \norm{AT}_1 \leqslant \norm{A} \norm{T}_1
		\end{equation*}
	\end{enumerate}
\end{thm}
\begin{proof}
	If $A \in \coper$, also $\abs{A} \in \coper$, then there is an orthonormal basis $\{e_n\}$ for $\Hs$ and the correponding eigenvalues $\{\alpha_n\} \in l^{\infty}(\C)$ s.t.
	\begin{equation*}
		\abs{A} = \sum_{n} \alpha_n P_{e_n} = \sum_{n} \alpha_n e_n \otimes e_n
	\end{equation*}
	Then we have that
	\begin{equation*}
		\sum_n \langle \abs{A}e_n, e_n \rangle = \sum_n \alpha_n
	\end{equation*}
	By the Polar Decomposion, $A \in \toper$ is equivalent the $\abs{A} \in \toper$. Thus, $\{\alpha_n\} \in l^{1}(\C)$ is equivalent to $A \in \toper$, and $\norm{A}_1 = \sum \alpha_n$.
	\item For $2)$, similarly as the $\hoper$, we just need to prove that the addition is continuous and closed. Let $A \in \toper$ and $B \in \toper$, and by the Polar Decomposition,
	\begin{equation*}
		A = W\abs{A},~ B = V \abs{B},~ A+B = U\abs{A+B}
	\end{equation*}
	And $\abs{A+B} = \sum_{n} \gamma_n e_n \otimes e_n$ for an orthonormal basis $\{e_n\}$. By $1)$, we just need to check that $\{\gamma_n\} \in l^{1}(\C)$.
	\begin{eqnarray*}
		\sum_n \gamma_n &=& \sum_n \langle \abs{A+B}e_n,e_n \rangle = \sum_n \abs{\langle Ae_n, Ue_n \rangle + \langle Be_n, Ue_n \rangle} \\
		&=& \sum_n \abs{\langle \abs{A}e_n, \st{W}Ue_n \rangle + \langle \abs{B}e_n, \st{V}Ue_n \rangle} \\
		&=& \sum_n \abs{\langle \abs{A}^{\frac{1}{2}}e_n, \abs{A}^{\frac{1}{2}}\st{W}Ue_n \rangle + \langle \abs{B}^{\frac{1}{2}}e_n, \abs{B}^{\frac{1}{2}}\st{V}Ue_n \rangle} \\
		&\leqslant& \sum_n (\norm{\abs{A}^{\frac{1}{2}}e_n}\norm{\abs{A}^{\frac{1}{2}}\st{W}Ue_n}+\norm{\abs{B}^{\frac{1}{2}}e_n}\norm{\abs{B}^{\frac{1}{2}}\st{V}Ue_n}) \\
		&\leqslant& \left(\sum_n \norm{\abs{A}^{\frac{1}{2}}e_n}^2 \right)^{\frac{1}{2}}\left(\sum_n \norm{\abs{A}^{\frac{1}{2}}\st{W}Ue_n}^2 \right)^{\frac{1}{2}} \\
		&& \negmedspace{} + \left(\sum_n \norm{\abs{B}^{\frac{1}{2}}e_n}^2 \right)^{\frac{1}{2}}\left(\sum_n \norm{\abs{B}^{\frac{1}{2}}\st{V}Ue_n}^2 \right)^{\frac{1}{2}} \\
		&\leqslant& \norm{\abs{A}^{\frac{1}{2}}}_2^2 + \norm{\abs{B}^{\frac{1}{2}}}_2^2 \\
		&=& \norm{A}_1 + \norm{B}_1
	\end{eqnarray*}
	\item $3)$ is trivial and $4)$ can be obtained by similar argument as the $\hoper$.
	\item For $5)$, let $T \in \toper$ and $T = \st{C}B$ for some $B,C \in \hoper$, then by above mention,
	\begin{eqnarray*}
		\Rea{\tr{\st{C}B}} &=& \frac{1}{4} (\norm{(B+C)e}^2 - \norm{(B+C)e}^2) \\
		&=& \frac{1}{4} (\norm{(\st{B}+\st{C})e}^2 - \norm{(\st{B}+\st{C})e}^2) \\
		&=& \Rea{\tr{C\st{B}}}
	\end{eqnarray*}
	And similarly, $\Img{\tr{\st{C}B}} = - \Img{\tr{C\st{B}}}$, thus
	\begin{equation*}
		\tr{\st{C}B} = \clo{\tr{C\st{B}}}
	\end{equation*}
	Therefore, we have that for any $A \in \oper$ 
	\begin{equation*}
		\tr{AT} = \tr{(A\st{C})B} = \clo{\tr{C\st{A} \st{B}}} = \tr{(\st{C})BA} = \tr{TA}
	\end{equation*}
	Let $T = W \abs{T}$ be the Polar Decomposition. Then by the CBS Inequality,
	\begin{eqnarray*}
		\abs{\tr{AT}} &\leqslant& \sum_{\fml{E}}  \norm{\abs{T}^{\frac{1}{2}}e}\norm{\abs{T}^{\frac{1}{2}}\st{W}\st{A}e} \\
		&\leqslant& \left(\sum \norm{\abs{T}^{\frac{1}{2}}e}^2 \right)^{\frac{1}{2}}\left(\sum \norm{\abs{T}^{\frac{1}{2}}\st{W}\st{A}e}^2 \right)^{\frac{1}{2}} \\
		&\leqslant& \norm{\abs{T}^{\frac{1}{2}}}_2\norm{\abs{T}^{\frac{1}{2}}\st{W}\st{A}}_2 \\
		&\leqslant& \norm{\abs{T}^{\frac{1}{2}}}_2^2 \norm{\st{W}\st{A}} \\
		&\leqslant& \norm{T}_1 \norm{A}
	\end{eqnarray*}


\end{proof}

















