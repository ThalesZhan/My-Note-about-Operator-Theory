\chapter{Normal Operators and Abelian Von Neumann Algebras}

For a normal operator $N$, it can generate an abelian norm closed $\st{C}$-subalgebra of $\oper$, which is isometrically $*$-isomorphic to $C(\sigma(N))$. But the norm closed $\st{C}$-subalgebra may be too "small" to get more interesting results. Therefore, we want to see the $WOT$-closed $\st{C}$-subalgebra $\st{W}(N)$ generated by a normal operator $N$. This extension can be naturally done by following processes. Firstly, we will extend $C(X)$ to $B(X,\Omega)$, where $B(X,\Omega)$ denotes all $\Omega$-measurable functions on $X$, and find the map from $B(X,\Omega)$ to $\oper$. Then using these method, we can extend the Continuous Functional Calculus to the Borel Functional Calculus. In order to illuminate the structure of $\st{W}(N)$, we need more properties of $WOT$-closed $\st{C}$-subalgebras, i.e. von Neumann algebras, and abelian von Neumann algebras. After that, we can build the isomorphism from $\st{W}(N)$ to $\lfs{\infty}(\sigma(N))$. Finally, by above methods, we can get the multiplicity theory of normal operators.

\section{Spectral Theorem}

\subsection{Spectral Measures}

Firstly, we want to construct the map from $B(X,\Omega)$ to $\oper$. Thus we need to define the operator-valued measure.

\begin{defn}
	If $X$ is a set and $\Omega$ is a $\sigma$-algebra of subsets of $X$, and $\Hs$ is a Hilbert space, then a spectral measure for $(X,\Omega,\Hs)$ is a function $E \colon \Omega \sto \oper$ satisfying the following properties.
	\begin{enumerate}[label=\arabic*)]
		\item $E(\Delta)$ is a projection for any $\Delta \in \Omega$.
		\item $E(\varnothing) = 0$ and $E(X) = 1$.
		\item $E(\Delta_1\cap\Delta_1)=E(\Delta_1)E(\Delta_2)$ for any $\Delta_1, \Delta_2 \in \Omega$.
		\item If $\{\Delta_n\}_{n=1}^{\infty}$ is a sequence of pairwise disjoint sets in $\Omega$,
		\begin{equation*}
			E(\bigcup_{n=1}^{\infty}\Delta_n) = \sum_{n=1}^{\infty} E(\Delta_n)
		\end{equation*}
	\end{enumerate}
\end{defn}
\begin{rem}
	In $4)$, this convergence is about $SOT$, and since $E(\Delta)$ is a projection, it is also about $WOT$.
\end{rem}

Then we can see the relation between the spectral measure and the complex measure.
\begin{prop}
	If $E$ is a spectral measure for $(X,\Omega,\Hs)$ and $g,h \in \Hs$, then
	\begin{equation*}
		E_{gh}(\Delta) = \langle E(\Delta)g, h \rangle
	\end{equation*}
	is a complex measure of $(X,\Omega)$. Moreover, $\norm{E_{gh}} \leqslant \norm{g}\norm{h}$.
\end{prop}
\begin{proof}
	$\mu=E_{gh}(\Delta)$ is a complex measure by definition.
	\begin{equation*}
		\abs{\mu}(\Delta) = \abs{\langle E(\Delta)g, h \rangle} \leqslant \norm{E(\Delta)g}\norm{h} \leqslant \norm{g}\norm{h} \qedhere
	\end{equation*}
\end{proof}

Using the spectral measure, we can define the operator-valued integral.

\begin{thm}
	If $E$ is a spectral measure for $(X,\Omega,\Hs)$ and $\phi \colon X \sto \C$ is a bounded $\Omega$-measurable function, then there is a unique operator $T \in \oper$ s.t. for any $\varepsilon > 0$, there is a $\Omega$-partition $\{\Delta_i\}_{i=1}^{n}$ of $X$ with
	\begin{equation*}
		 \sup{\{\abs{\phi(x)-\phi(y)} \colon x, y \in \Delta_k\}}
	\end{equation*}
	for $1 \leqslant k \leqslant n$ s.t. for any $x_k \in \Delta_k$
	\begin{equation*}
		\norm{T-\sum_{k=1}^{n}\phi(k)E(\Delta_k)} < \varepsilon
	\end{equation*}
\end{thm}
\begin{proof}
	Define
	\begin{center}
		\begin{tabular}{l c c l}
			$B \colon$ & $\Hs \times \Hs$ & $\longrightarrow$ & $\C$ \\
			~ & $(g,h)$ & $\longmapsto$ & $\int \phi dE_{gh}$
		\end{tabular}
	\end{center}
	Since $\norm{B(g,h)} \leqslant \norm{\phi}_{\infty} \norm{g} \norm{h}$, $B$ is a bounded sesquilinear form. Then by the Riesz Theorem, there is a $B \in \oper$ s.t. $B(g,h) = \langle Tg,h \rangle$ for all $g,h \in \Hs$ and $\norm{T} \leqslant \norm{\phi}_{\infty}$. Then for and $g,h \in \Hs$ and $x_k \in \Delta_k$ with the giving partition
	\begin{eqnarray*}
		\abs{\langle Tg,h \rangle - \sum_{k=1}^{n}\phi(x_k)\langle E(\Delta_k)g, h \rangle} &=& \abs{\sum_{k=1}^{n} \int_{\Delta_k} (\phi(x)-\phi(x_k))dE_{gh}(x)} \\
		&\leqslant& \sum_{k=1}^{n} \int_{\Delta_k} \abs{\phi(x)-\phi(x_k)} d\abs{E_{gh}}(x) \\
		&\leqslant& \varepsilon \int d\abs{E_{gh}}(x) \leqslant \varepsilon \norm{g}\norm{h}
	\end{eqnarray*}
	Therefore, by the Riezs Theorem, 
	\begin{equation*}
		\norm{T-\sum_{k=1}^{n}\phi(k)E(\Delta_k)} < \varepsilon \qedhere
	\end{equation*}
\end{proof}
\begin{rem}
	We define $T = \int \phi dE$, and $\langle Tg,h \rangle = \int \phi dE_{gh}$.
\end{rem}

We have one more property on $B(X)$ for a compact space $X$. Let $B(X)$ denote all bounded Borel measurable functions on $X$ and $M(X)$ be all Borel measures.

\begin{lem}
	If $X$ is a compact space and $\phi \in B(X)$, then there is a net $\{u_i\} \subset C(X)$ with $\norm{u_i}_{\infty} \leqslant \norm{\phi}$ s.t. $\int u_i d\mu \sto \int \phi d\mu$ for any $\mu$ in $M(X)$.
\end{lem}
\begin{proof}
	$C(X)^{*} = M(X)$ and $C(X) \subset C(X)^{**} = \st{M(X)}$.  Therefore, the unit ball in $C(X)$ is $wk$-dense in the unit ball in $\st{M(X)}$ (by using Hahn-Banach Theorem). Identifying $B(X)$ be the subspace of $\st{M(X)}$, we have this lemma.
\end{proof}
\begin{rem}
	In fact, $C(X)$ is norm closed in $C(X)^{**} = \st{M(X)}$ like that the $\st{C}$-algebra is norm closed in $\oper$. But by above lemma, we see in $\st{M(X)}$
	\begin{equation*}
		\clo{C(X)}^{wk} = B(X)
	\end{equation*}
	Thus we may apply this to extending $\st{C}$-algebras.
\end{rem}


$B(X,\Omega)$ with the supremum norm and complex conjugate can become a $\st{C}$-algebra. Then by above theorem, we can provide the map from $B(X,\Omega)$ to operator algebras.

\begin{prop}
	If $E$ is a spectral measure for $(X,\Omega,\Hs)$, then
	\begin{center}
		\begin{tabular}{l c c l}
			$\rho \colon$ & $B(X,\Omega)$ & $\longrightarrow$ & $\oper$ \\
			~ & $\phi$ & $\longmapsto$ & $\int \phi dE$
		\end{tabular}
	\end{center}
	is a representation of $B(X,\Omega)$.
\end{prop}
\begin{proof}
	Firstly, we need check that $\rho$ preserves the involution. For $g,h \in \Hs$,
	\begin{equation*}
		E_{gh}(\Delta) = \langle E(\Delta)g, h \rangle = \langle g, E(\Delta)h \rangle = \clo{\langle E(\Delta)h, g \rangle} = \clo{E_{hg}(\Delta)}
	\end{equation*}
	Therefore, for any $g,h \in \Hs$
	\begin{eqnarray*}
		\langle \st{(\int \phi dE)}g,h \rangle &=& \langle g, (\int \phi dE)h \rangle \\
		&=& \clo{\langle (\int \phi dE)h, g \rangle} = \clo{\int \phi dE_{hg}} \\
		&=& \int \st{\phi} d \clo{E_{hg}} = \int \st{\phi} dE_{gh} \\ 
		&=& \langle (\int \st{\phi} dE)g,h \rangle
	\end{eqnarray*}
	\item Also, $\rho$ must be multiplicative. Fix $\phi,\psi \in B(X)$ and $\varepsilon >0$, let $\{\Delta_i\}_{i=1}^{n}$ be the $\Omega$-partition of $X$ s.t. the oscillations of $\phi,\psi,\phi\psi$ on each $\Delta_i$ are less than $\varepsilon$.
	By the fact $E(\Delta_i)E(\Delta_j) = 0$ for $i \neq j$, 
	\begin{eqnarray*}
		\lefteqn{\norm{\int \phi\psi dE - (\int \phi dE)(\int \psi dE)}} \\
		&\leqslant& \varepsilon+ \norm{\sum_{k=1}^{n}\phi(x_k)\psi(x_k)E(\Delta_k) - (\sum_{i=1}^{n}\phi(x_i)E(\Delta_i))(\sum_{j=1}^{n}\psi(x_j)E(\Delta_j))} \\
		&& \negmedspace{} + \norm{(\sum_{i=1}^{n}\phi(x_i)E(\Delta_i))(\sum_{j=1}^{n}\psi(x_j)E(\Delta_j))-(\int \phi dE)(\int \psi dE)} \\
		&\leqslant& \varepsilon + \norm{(\sum_{i=1}^{n}\phi(x_i)E(\Delta_i))(\sum_{j=1}^{n}\psi(x_j)E(\Delta_j)-\int \psi dE)} \\
		&& \negmedspace{} + \norm{(\sum_{i=1}^{n}\phi(x_i)E(\Delta_i)-\int \phi dE)(\sum_{j=1}^{n}\psi(x_j)E(\Delta_j))} \\
		&\leqslant& \varepsilon(1+\norm{\phi}_{\infty}+\norm{\psi}_{\infty})
	\end{eqnarray*}
	Then $\rho$ is indeed multiplicative.
\end{proof}

We have already there is a map from $C(X)$ to $\oper$, then by combining this proposition and above lemma, if the map is $wk$-continuous on $C(X)$, we may extend it from $M(X)$ to $\oper$.

\begin{thm}
	If $X$ is a compact space and $\rho \colon C(X) \sto \oper$ is a representation, then there is a unique spectral measure $E$ defined on the Borel sets of $X$ s.t. $\rho$ can be expressed as 
	\begin{equation*}
		\rho(\phi) = \int \phi dE ~~\forall~\phi \in C(X)
	\end{equation*}
	Moreover, $T \in \oper$ commutes with $\rho(\phi)$ for any $\phi \in C(X)$ if and only if $T$ commutes with $E(\Delta)$ for any Borel set $\Delta$.
\end{thm}
\begin{proof}
	Constructing $\tilde{\rho}$ from $\rho$: \\
	For any $g,h \in \Hs$, the map $u \sto \langle \rho(u)g,h \rangle$ from $C(X)$ to $\oper$ is bounded with $\norm{g}\norm{h}$. Therefore, by Riesz Theorem, there exists a unique $\mu_{gh} \in M(X)$ with $\norm{\mu_{gh}} \leqslant \norm{g}\norm{h}$
	\begin{equation*}
		\langle \rho(u)g,h \rangle = \int u d\mu_{gh},~~\forall~u \in C(X)
	\end{equation*}
	For any $\phi \in B(X)$, $(g,h) \sto \int \phi d\mu_{gh}$ is a bounded sesquilinear on $\Hs \times \Hs$. Therefore, there exsits a unique $T \in \oper$, s.t.
	\begin{equation*}
		\int \phi d\mu_{gh} = \langle Tg,h \rangle
	\end{equation*}
	Then we define $\tilde{\rho}(\phi) = T$ and it is well-defined. Clearly, $\tilde{\rho}(u) = \rho(u)$ for any $u$ in $C(X)$.
	\item Check: $\tilde{\rho} \colon B(X) \sto \oper$ is a representation \\
	By the definiation, $\tilde{\rho}$ is clearly linear and preserves the involution, thus we just need to prove the multiplication. Let $\phi$ and $\psi$ in $B(X)$. There is a sequence $\{u_i\} \subset C(X)$ s.t. $u_i \sto \phi$ in $wk$-topology. Then since $\int u_i\psi d\mu \sto \int \phi\psi$ for any $\mu \in M(X)$,  $\tilde{\rho(u_i\psi)} \sto \rho(\phi\psi)$ in $WOT$. For any $\psi \in C(X)$,
	\begin{equation*}
		\tilde{\rho}(\phi\psi) = \lim \tilde{\rho}(u_i\psi) = \lim \rho(u_i)\rho(\psi) = \tilde{\rho}(\phi)\rho(\psi)
	\end{equation*}
	Therefore, if $\phi,\psi \in B(X)$, $\tilde{\rho}(\phi\psi) = \tilde{\rho}(\phi)\tilde{\rho}(\psi)$.
	\item Contructing the spectral measure: \\
	Let $E(\Delta) = \rho(\chi_{\Delta}(x))$ for all Borel set $\Delta$. Clearly, $E(\Delta)$ is a projection for any $\Delta$. For disjoint Borel sets $\Delta_1, \Delta_2$,
	\begin{equation*}
		E(\Delta_1 \cap \Delta_2) = \tilde{\rho}(\chi_{\Delta_1 \cap \Delta_2}) = \tilde{\rho}(\chi(\Delta_1)\chi(\Delta_2)) =\tilde{\rho}(\chi(\Delta_1)) = E(\Delta_1)E(\Delta_2)
	\end{equation*}
	Let $\{\Delta_n\}_{n=1}^{\infty}$ be disjoint Borel sets and $\Delta = \cup_{n=1}^{\infty}\Delta_n$ and $\Lambda_n = \cup_{k=n+1}^{\infty}\Delta_k$. By the fact that $E$ is finitely additive, for any $h \in \Hs$
	\begin{eqnarray*}
	\norm{E(\Delta)h-\sum_{k=1}^{n}E(\Delta_k)h} &=& \langle E(\Lambda_n)h, E(\Lambda_n)h \rangle \\
	&=& \langle E(\Lambda_n)h, h \rangle = \langle \tilde{\rho}(\Lambda_n)h, h \rangle \\
	&=& \int \chi_{\Lambda_n} d\mu_{hh} \\
	&\leqslant& \mu_{hh}(\Lambda_n) \sto 0
	\end{eqnarray*}
	Then by above theorem, with respect to this spectral measure $E$, for any $\phi \in B(X)$,
	\begin{equation*}
		\tilde{\rho}(\phi) = \int \phi dE
	\end{equation*}
	In particular, for $u \in C(X)$, it is also true.
	\item Check: The uniqueness of $E$.
	If there is another $E^{'}$ satisfying above conditions, then for any $g,h \in \Hs$
	\begin{equation*}
		\int \phi dE^{'}_{gh} = \int \phi dE_{gh}
	\end{equation*}
	Therefore, $E^{'} = E$.
	\item If $T$ commutes with $\tilde{\rho}(\phi)$ for any $\phi \in C(X)$, we can use the $WOT$-convergence to get that $T$ commutes with $\tilde{\rho}(\phi)$ for any $\phi \in B(X)$. Therefore, clearly $T$ commutes with $E(\Delta)$ for any Borel set $\Delta$. Conversely, if $T$ commutes with $E(\Delta)$ for any Borel set $\Delta$, 
	\begin{equation*}
		\langle E(\Delta)Tg,h \rangle = \langle E(\Delta)g,\st{T}h \rangle
	\end{equation*}
	i.e. $\mu_{Tg,h} = \mu_{g, \st{T}h}$. Thus for any $\phi \in B(X)$,
	\begin{equation*}
		\int \phi d\mu_{Tg,h} = \int \phi d\mu_{g,\st{T}h},~\text{i.e.}~\langle \phi Tg,h \rangle = \langle \phi g,\st{T}h \rangle
	\end{equation*}
	Then $\langle \tilde{\rho}(\phi) Tg,h \rangle = \langle T \tilde{\rho}(\phi) g,h \rangle$ for any $g,h \in \Hs$.
\end{proof}
\begin{rem}
	Thus, we extend $\rho \colon C(X) \sto \oper$ to $\tilde{\rho}$ defined on $B(X)$ Moreover, by above proof we know the $*$-homomorphism
	\begin{center}
		\begin{tabular}{l c c l}
			$\tilde{\rho} \colon$ & $(B(X),wk)$ & $\longrightarrow$ & $(\oper,WOT)$ \\
			~ & $\phi$ & $\longmapsto$ & $\int \phi dE$
		\end{tabular}
	\end{center}
	is continuous and therefore $\tilde{\rho}(B(X))$ is $WOT$-closed in $\oper$.
\end{rem}

\subsection{Spectral Theorem for Normal Operators}

Now we can apply above theorem to a normal operator and extend the Continuous Functional Calculus to the Borel Functional Calculus. Moreover, it will promote us to consider the structure of the $WOT$-closed operator algebra generated by a normal operator.

\begin{thm}
	If $N$ is a normal operator, then there is a unique spectral measure $E$ s.t. 
	\begin{center}
		\begin{tabular}{l c c l}
			$\rho \colon$ & $B(\sigma(N))$ & $\longrightarrow$ & $\oper$ \\
			~ & $\phi$ & $\longmapsto$ & $\phi(N) = \int \phi dE$
		\end{tabular}
	\end{center}
	and $\rho|_{C(\sigma(N))}$ is the Continuous Functional Calculus. Moreover,
	\begin{enumerate}[label=\arabic*)]
		\item $\tilde{\rho} \colon (B(X),wk) \sto (\oper,WOT)$ is continuous.
		\item if $G$ is nonempty and relatively open subset in $\sigma(N)$, $E(G) \neq 0$.
		\item if $T \in \oper$, then $TN=NT$ and $T\st{N}=\st{N}T$ if and only if T commutes with $E(\Delta)$ for any Borel set $\Delta$.
	\end{enumerate}
\end{thm}
\begin{proof}
	By above theorem, this $\rho$ is in fact the extension of the Continuous Functional Calculus. And $\rho$ is $wk-WOT$ continuous. 
	\item For $2)$, it is because that $\rho|_{C(\sigma(N))}$ is an isometry. If $\chi_{G} > 0$, then there is a continuous function s.t. $0<f<\chi_{G}$. Therefore, $0 < \rho(f) \leqslant \rho(\chi_{G})$.
	\item Since the condition that $T$ commutes with $N$ and $\st{N}$ is equivalent to that $T$ commutes with $\phi(N)$ for any $\phi \in C(\sigma(N))$, by above theorem, $3)$ holds.
\end{proof}
\begin{rem}
	Now, for a normal operator $N$, we have
	\begin{equation*}
		N = \int_{\sigma(N)} z dE = z(N) = \int_{\Gamma} (z-N)^{-1} dz
	\end{equation*}
\end{rem}

There are some properties of the spectral measure of normal operator $N$.

\begin{prop}
	Let $N$ be a normal operator with the spectral measure $E$.
	\begin{enumerate}[label=\arabic*)]
		\item $\lambda \in \sigma_p(N) \Leftrightarrow E(\{\lambda\}) \neq 0$.
		\item If $\lambda \in \sigma_p(N)$, then $\ran{E(\{\lambda\})}=\ker{(N-\lambda)}$.
		\item If $\lambda \in \sigma_p(N)$ and $\phi \in B(\sigma(N))$, then for any $h \in \Hs$,
		\begin{equation*}
			\phi(N)h = \phi(\lambda)h
		\end{equation*}
	\end{enumerate}
\end{prop}
\begin{proof}
	For $1)$, if $\lambda \in \sigma(N)$, then there is a nonzero vector $h \in \Hs$ s.t. $(N-\lambda)h=0$.
	\begin{equation*}
		0 = \norm{(N-\lambda)h}^2=\int_{\sigma(N)} \abs{z-\lambda}^2 d E_{hh} = \int_{\sigma(N) \backslash \{\lambda\}} \abs{z-\lambda}^2 d E_{hh}
	\end{equation*}
	Then since $E_{hh}$ is a positive measure,
	\begin{equation*}
		\int_{\sigma(N) \backslash \{\lambda\}} d E_{hh} = \norm{E(\sigma(N) \backslash \{\lambda\})h}^2 = 0
	\end{equation*} 
	And since $E(\sigma(N))=1$, 
	\begin{equation*}
		E(\{\lambda\})h = h,~\text{i.e } E(\{\lambda\}) \neq 0
	\end{equation*}
	Conversely, if $E(\{\lambda\}) \neq 0$, then for nonzero $h \in \ran{E(\{\lambda\})}$, $E(\{\lambda\})h = h$,
	\begin{eqnarray*}
		\int_{\sigma(N)} \abs{z-\lambda}^2 d \langle E(z)h,h \rangle &=&  \int_{\sigma(N)} \abs{z-\lambda}^2 d \langle E(z)E(\{\lambda\})h,h \rangle \\
		&=& \int_{\{\lambda\}} \abs{z-\lambda}^2 d \langle E(z)h,h \rangle =0
	\end{eqnarray*}
	That means
	\begin{equation*}
		\norm{(N-\lambda)h}^2 = \int_{\sigma(N)} \abs{z-\lambda}^2 d \langle E(z)h,h \rangle = 0
	\end{equation*}
	Then $h \in \ker{(T-\lambda)}$, i.e. $\lambda \in \sigma_p(N)$. \\
	$2)$ and $3)$ can be obtained directly by above proof.
\end{proof}

Now, we can extend the \textbf{Example} \ref{exam2} in the subsection \textbf{2.2.2}.

\begin{exam}
	Let $\mu$ be a compactly supported, regular Borel measure on $\C$ and ($X,\Omega,\mu$) be the measure space. For each $\pi \in \lfs{\infty}(\mu)$, we define the map
	\begin{center}
		\begin{tabular}{l c c l}
			$M_{\phi} \colon$ & $\lfs{2}(\mu)$ & $\longrightarrow$ & $\lfs{2}(\mu)$ \\
			~ & $f(z)$ & $\longmapsto$ & $\phi(z)f(z)$
		\end{tabular} 
	\end{center}
	\begin{enumerate}[label=\arabic*)]
		\item the spectral measure for $M_{\phi}$ is $M_{\chi_{\phi^{-1}(\Delta)}}$.
		\item If $\psi \in B(\sigma(M_{\phi}))$, then $\psi(M_{\phi}) = M_{\psi \circ \phi}$.
	\end{enumerate}
\end{exam}

\section{Abelian Von Neumann Algebras}

In above section, we built the representation $\rho$ from $B(X)$ to $\oper$, and more over, $\rho$ is $wk-WOT$ continuous. Thus we want to prove $\rho(B(X))$ is indeed a $WOT$-closed $\st{C}$-subalgebra of $\oper$. If $X = \sigma(N)$, $\rho(B(\sigma(N)))$ is the $WOT$-closed $\st{C}$-subalgebra generated by $N$, denoted by $\st{W}(N)$. We will provide more rigorous proof of this statement. After that, we will make the structure of $\st{W}(N)$ more explicit. But before doing these, we some properties of $WOT$-closed subalgebra.

\subsection{Double Commutant Theorem}















