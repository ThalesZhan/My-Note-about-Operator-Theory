\chapter{Introduction}

Most of this report is based on the two books, A Course in Functional Analysis and A Course in Operator Theory by John B. Conway. However, I reorganized the contents under my understanding and added some other materials to help me understand the structure of the knowledge of the fundamental operator theory.

My understanding of this structure is based on the "General-to-special" order. First, I studied the properties of general spaces. Then by adding more other structures on it, it can get a more special space with more interesting properties. Moreover, these properties also revealed the essential traits of the more special space. Then it can endow the special space more traits, thus a more special space is born. Using this order, the appearance of $\st{C}$-algebra and von Neumann algebra is very natural for me.\\
Firstly, I have seen the structure of finite-dimensional linear space, but how to research the infinite-dimensional linear space is a question. Clearly, the algebraic structure cannot provide enough information about the infinite-dimensional case. However, by learning topological spaces, I know that the topology can provide a method to research the infinite sets, thus equipping the infinite-dimensional linear space with an appropriate topology is a good idea. Besides, this topology should be compatible with the original linear structure. Therefore, topological vector space plays an important role, and there are many interesting properties generated by the merge of the topology and algebra. If the topology of the T.V.S. can be more special, like that the pre-"length" of elements is continuous with respect to the topology, then it can be a locally convex topological vector space. When learning the finite-dimensional linear space, I saw the corresponding dual space is very "similar" as the original space. But how do these properties change in the infinite-dimensional case? The Hahn-Banach Theorem provides an answer.

Moreover, locally convex T.V.S. can become more special. If there is only one seminorm generating the locally convex T.V.S., this space is different. In order to make space be Hausdorff, the seminorm need to a norm. Besides, since complete topological space is more interesting, this special space should be complete. Therefore, we get a Banach space. And then all of the properties of locally convex T.V.S. can be applied to the Banach space and have a more powerful form. Moreover, the space consisting of all continuous maps between the Banach spaces can be equipped with a more interesting structure, which can also become a Banach space with the induced norm.

The norm generating a Banach space can also be more special. If the norm can be generated by an inner product, then the Banach space with this inner product becomes a Hilbert space. Because of this extra inner product structure, the Hilbert space has the Projection Theorem, thus the dual space is as same as the original space. Then the space consisting of all continuous maps between the same Hilbert space can contain the adjoint version of each of them. This space is the operator space.

For researching the operator space, similarly, we can firstly research a more general version. The main difference between the operator space and the Banach space is that the operator space can do multiplication, defined as composition. Moreover, this multiplication is continuous with respect to the norm topology. Therefore, the general version of the operator space is a Banach space with an extra continuous multiplication, then it becomes a Banach algebra. Comparing with the Banach space, the Banach algebra has one more algebraic structure, and the invertibility is an important property with respect to this structure. Thus, the appearance of the concept of the spectrum is natural. Besides, the commutative Banach algebra can have an explicit expression.

Besides, the operator algebra has one more operation, the involution. And the involution is not only continuous with respect to the norm topology. In fact, it provides an identity, $\st{C}$-identity. That means the topology is determined by the algebraic structure. Then, the general version is $\st{C}$-algebra, whose topology is completely determined by the algebraic structure it has. Because of the strong relationship between the topology and the algebra, $\st{C}$-algebra has some similar properties to the complex number system. By the GNS construction, any $\st{C}$-algebra can be embedded in an operator algebra. It is amazing since the general version of operator algebra is itself. Thus, researching operator algebras is enough.

However, a $\st{C}$-subalgebras of the operator algebra may be too small to contain enough "interesting" elements. Thus, it can find a weaker topology and then extending the $\st{C}$-algebra with respect to the new topology to a larger subalgebra is a good idea. But how can we do is? Continuous Functional Calculus can provide a clue. We have known that the continuous function space on a compact space is weakly dense in the bounded Borel function space and the Gelfand transform maps the continuous function space to the operator algebra. So it is natural to find a method of extending the map from the continuous function space to the bounded Borel function space. Therefore, by defining the operator-valued measure, the integral of the operator makes sense. Comparing this process to the process embedding the continuous function space in the bounded Borel function space, the needed new topologies appear naturally. That is the weak operator topology and the strong operator topology. Then the new algebra is born, von Neumann algebra. In fact, these "interesting" elements are projections. Since the norm topology is determined by the algebraic structure,  von Neumann algebras can be constructed by algebraic properties. Similarly, the structure of an abelian von Neumann algebra is easier. There is a particular abelian von Neumann algebra, which is generated by the normal operator. By extending the Gelfand transform, the structure of this von Neumann algebra can become concrete. Moreover, any abelian von Neumann algebra has the same structure.

From finite rank operators to infinite rank operators, there is a special class of operators, which can be approximated by a sequence of finite rank operators, the compact operators. Because of it, compact operators have a lot of properties as similar to finite rank operators, such as the spectrum, the representations and so on. There are also some interesting classes contained in the compact operators, like the trace class. The importance of trace class is that it is the dual space of the operator algebra, and thus it can provide an extra topology on the operator algebra. This topology has some relations with the weak operator topology, thus it can provide a new method to research the von Neumann algebra. By the quotient map, there is another interesting class of operators, the Fredholm operators. Finally, we have known that any normal operator can be diagonalizable in the finite-dimensional case, but in the infinite-dimensional case, we need the compact perturbation in order to get a similar result. 

For the general von Neumann algebra, combining the $WOT$ and the ultraweak topology, it can get more properties than a general $\st{C}$-algebra has, like the structure of the ideals. Moreover, it provides the equivalent condition the simple von Neumann algebra has. The von Neumann algebra is generated by all of its projections, thus these projections play an important role. By researching the relations of projections and the properties they have, it can classify von Neumann algebras into corresponding classes. In fact, these projections can be classified as abelian ones, non-abelian ones and pure infinite ones, and the corresponding von Neumann algebras are Type \RNum{1}, Type \RNum{2} and \RNum{3}. Type \RNum{1} is easier, but Type \RNum{2} and \RNum{2} do not have trivial example.
