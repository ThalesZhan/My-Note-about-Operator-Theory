\chapter{Introduction}

Most of this report is based on the two books, A Course in Functional Analysis and A Course in Operator Theory by John B. Conway. However, I reorganized the contents under my understanding and added some other materials to help me understand the structure of the knowledge of the fundamental operator theory.

My understanding of this structure is based on the "General-to-special" order. Firstly, I learned the properties of one general space. Then by endowing more other structures on it, it can become a more special space with more interesting properties. Moreover, these properties also revealed the essence of this space. So they can endow this special space more traits. Then a more special space is born. Using this order, the appearances of $\st{C}$-algebras and von Neumann algebras are very natural for me.\\
Firstly, the structure of finite-dimensional linear space is clear, but how to research the infinite-dimensional linear space is a question. The algebraic structure cannot provide enough information about the infinite-dimensional case. However, by learning topological spaces, I know that the topology can provide a method to research the infinite sets. Thus equipping the infinite-dimensional linear space with an appropriate topology is a good idea. Besides, this topology should be compatible with the original linear structure. Therefore, a topological vector space appears, and there are many interesting properties generated by the merge of the topology and algebra. If the topology of the T.V.S. can be more special, like that the pre-"length" of elements is continuous concerning the topology, that is, semi-norms, then it can be a locally convex topological vector space. For the finite-dimensional linear space, the corresponding dual space is very ``similar" as the original space, but how do these properties change in the infinite-dimensional case? The Hahn-Banach Theorem provides an answer.

Moreover, locally convex T.V.S. can become more special. If there is only one seminorm generating the locally convex T.V.S., this space is different. The seminorm needs to a norm to make space be Hausdorff. Then, by taking completion, a Banach space appears. So all of the properties of locally convex T.V.S. can be applied to the Banach space and have a more powerful form. Moreover, the space consisting of all continuous maps between the Banach space can be equipped with the linear structure, and it can also become a Banach space with the induced norm.

The norm generating a Banach space can also be more special. If the norm can be generated by an inner product, then the Banach space with this inner product becomes a Hilbert space. Because of this extra inner product structure, the Hilbert space has the Projection Theorem that makes the dual space as same as the original space. Furthermore, the set consisting of all continuous linear transformations between the Hilbert space contains the adjoint version of each element. Then this space is called operator space.

For researching the operator space, similarly, it is necessary to research a more general version. The main difference between operator space and the general Banach space is that elements in operator space can do multiplication, defined as composition. Moreover, this multiplication is continuous under the norm topology. Therefore, the general version of the operator space is a Banach space with a continuous multiplication, then it is called a Banach algebra. Compared with Banach space, Banach algebra has one more algebraic structure, and the invertibility is an important property for this structure. Thus, the appearance of the concept of the spectrum is natural. Besides, commutative Banach algebras have an explicit expression.

Besides, operator algebra has one more operation, the involution. And the involution is not only continuous concerning the norm topology. In fact, it provides an identity, $\st{C}$-identity. That means the topology is determined by the algebraic structure. Then, the general version is $\st{C}$-algebra, whose topology is completely determined by the algebraic structure it has. Because of the strong relationship between the topology and the algebra, $\st{C}$-algebra has some similar properties to the complex number system. By the GNS construction, any $\st{C}$-algebra can be embedded in an operator algebra. It is amazing since the general version of operator algebra is itself. Thus, researching operator algebras is significant.

However, a $\st{C}$-subalgebra of the operator algebra may be too small to contain enough ``useful" elements. Thus, it should find a weaker topology, and the $\st{C}$-subalgebra can extend to a larger subalgebra under this new topology. But how can we do is? Continuous Functional Calculus provides a clue. The continuous function space on a compact space is weakly dense in the bounded Borel function space, and the Gelfand Transform maps the continuous function space to operator algebra. So it is natural to find a method of extending the map from the continuous function space to the bounded Borel function space. Therefore, by defining the operator-valued measure, the integral of operators makes sense. Comparing this process to the process of embedding the continuous function space in the bounded Borel function space, the needed new topology appears naturally, which is the weak operator topology. So the new algebra is born, that is, von Neumann algebra. These ``useful" elements are projections. Since the norm topology is determined by the algebraic structure,  von Neumann algebras can be constructed by algebraic properties. Similarly, the structure of an abelian von Neumann algebra is easier, and there is a particular abelian von Neumann algebra, which is generated by the normal operator. By extending the Gelfand transform, the structure of this von Neumann algebra can become concrete. Moreover, any abelian von Neumann algebra has the same structure.

In infinite rank operators, there is a particular class of operators that can be approximated by a sequence of finite rank operators, that is, compact operators. Because of this, compact operators have some properties similar to finite rank operators, like the spectrum and the representations. There are also some interesting classes contained in the compact operators, like the trace class. The significance of trace class is that it is the dual space of the operator algebra, so it can provide an extra topology on the operator algebra. Moreover, this new topology has some relationships with the weak operator topology, so it can provide a new way of researching the von Neumann algebra. Besides, there is another interesting class of operators, the Fredholm operators. Finally, any normal operator can be diagonalizable in the finite-dimensional case, but in the infinite-dimensional case, it needs the compact perturbation to get a similar result. 

For the general von Neumann algebra, by combining the $WOT$ and the ultraweak topology, it can get more properties than a general $\st{C}$-algebra has, like the structure of the ideals. Moreover, it provides the equivalent condition the simple von Neumann algebra has. The von Neumann algebra is generated by all of its projections, so these projections play an important role. By researching the relations of projections and the properties they have, it can classify von Neumann algebras into different classes. These projections can be classified as abelian ones, non-abelian ones and pure infinite ones, and the corresponding von Neumann algebras are Type \RNum{1}, Type \RNum{2} and \RNum{3}. Type \RNum{1} is easier, but Type \RNum{2} and \RNum{2} do not have trivial example.

