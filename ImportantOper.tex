\chapter{Normal Operators and Abelian Von Neumann Algebras}

For a normal operator $N$, it can generate an abelian norm closed $\st{C}$-subalgebra of $\oper$, which is isometrically $*$-isomorphic to $C(\sigma(N))$. But the norm closed $\st{C}$-subalgebra may be too "small" to get more interesting results. Therefore, we want to see the $WOT$-closed $\st{C}$-subalgebra $\st{W}(N)$ generated by a normal operator $N$. This extension can be naturally done by following processes. Firstly, we will extend $C(X)$ to $B(X,\Omega)$, where $B(X,\Omega)$ denotes all $\Omega$-measurable functions on $X$, and find the map from $B(X,\Omega)$ to $\oper$. Then using these method, we can extend the Continuous Functional Calculus to the Borel Functional Calculus. In order to illuminate the structure of $\st{W}(N)$, we need more properties of $WOT$-closed $\st{C}$-subalgebras, i.e. von Neumann algebras, and abelian von Neumann algebras. After that, we can build the isomorphism from $\st{W}(N)$ to $\lfs{\infty}(\sigma(N))$. Finally, by above methods, we can get the multiplicity theory of normal operators.

\section{Spectral Theorem}

\subsection{Spectral Measures}

Firstly, we want to construct the map from $B(X,\Omega)$ to $\oper$. Thus we need to define the operator-valued measure.

\begin{defn}
	If $X$ is a set and $\Omega$ is a $\sigma$-algebra of subsets of $X$, and $\Hs$ is a Hilbert space, then a spectral measure for $(X,\Omega,\Hs)$ is a function $E \colon \Omega \sto \oper$ satisfying the following properties.
	\begin{enumerate}[label=\arabic*)]
		\item $E(\Delta)$ is a projection for any $\Delta \in \Omega$.
		\item $E(\varnothing) = 0$ and $E(X) = 1$.
		\item $E(\Delta_1\cap\Delta_1)=E(\Delta_1)E(\Delta_2)$ for any $\Delta_1, \Delta_2 \in \Omega$.
		\item If $\{\Delta_n\}_{n=1}^{\infty}$ is a sequence of pairwise disjoint sets in $\Omega$,
		\begin{equation*}
			E(\bigcup_{n=1}^{\infty}\Delta_n) = \sum_{n=1}^{\infty} E(\Delta_n)
		\end{equation*}
	\end{enumerate}
\end{defn}
\begin{rem}
	In $4)$, this convergence is about $SOT$, and since $E(\Delta)$ is a projection, it is also about $WOT$.
\end{rem}

Then we can see the relation between the spectral measure and the complex measure.
\begin{prop}
	If $E$ is a spectral measure for $(X,\Omega,\Hs)$ and $g,h \in \Hs$, then
	\begin{equation*}
		E_{gh}(\Delta) = \langle E(\Delta)g, h \rangle
	\end{equation*}
	is a complex measure of $(X,\Omega)$. Moreover, $\norm{E_{gh}} \leqslant \norm{g}\norm{h}$.
\end{prop}
\begin{proof}
	$\mu=E_{gh}(\Delta)$ is a complex measure by definition.
	\begin{equation*}
		\abs{\mu}(\Delta) = \abs{\langle E(\Delta)g, h \rangle} \leqslant \norm{E(\Delta)g}\norm{h} \leqslant \norm{g}\norm{h} \qedhere
	\end{equation*}
\end{proof}

Using the spectral measure, we can define the operator-valued integral.

\begin{thm}
	If $E$ is a spectral measure for $(X,\Omega,\Hs)$ and $\phi \colon X \sto \C$ is a bounded $\Omega$-measurable function, then there is a unique operator $T \in \oper$ s.t. for any $\varepsilon > 0$, there is a $\Omega$-partition $\{\Delta_i\}_{i=1}^{n}$ of $X$ with
	\begin{equation*}
		 \sup{\{\abs{\phi(x)-\phi(y)} \colon x, y \in \Delta_k\}}
	\end{equation*}
	for $1 \leqslant k \leqslant n$ s.t. for any $x_k \in \Delta_k$
	\begin{equation*}
		\norm{T-\sum_{k=1}^{n}\phi(k)E(\Delta_k)} < \varepsilon
	\end{equation*}
\end{thm}
\begin{proof}
	Define
	\begin{center}
		\begin{tabular}{l c c l}
			$B \colon$ & $\Hs \times \Hs$ & $\longrightarrow$ & $\C$ \\
			~ & $(g,h)$ & $\longmapsto$ & $\int \phi dE_{gh}$
		\end{tabular}
	\end{center}
	Since $\norm{B(g,h)} \leqslant \norm{\phi}_{\infty} \norm{g} \norm{h}$, $B$ is a bounded sesquilinear form. Then by the Riesz Theorem, there is a $B \in \oper$ s.t. $B(g,h) = \langle Tg,h \rangle$ for all $g,h \in \Hs$ and $\norm{T} \leqslant \norm{\phi}_{\infty}$. Then for and $g,h \in \Hs$ and $x_k \in \Delta_k$ with the giving partition
	\begin{eqnarray*}
		\abs{\langle Tg,h \rangle - \sum_{k=1}^{n}\phi(x_k)\langle E(\Delta_k)g, h \rangle} &=& \abs{\sum_{k=1}^{n} \int_{\Delta_k} (\phi(x)-\phi(x_k))dE_{gh}(x)} \\
		&\leqslant& \sum_{k=1}^{n} \int_{\Delta_k} \abs{\phi(x)-\phi(x_k)} d\abs{E_{gh}}(x) \\
		&\leqslant& \varepsilon \int d\abs{E_{gh}}(x) \leqslant \varepsilon \norm{g}\norm{h}
	\end{eqnarray*}
	Therefore, by the Riezs Theorem, 
	\begin{equation*}
		\norm{T-\sum_{k=1}^{n}\phi(k)E(\Delta_k)} < \varepsilon \qedhere
	\end{equation*}
\end{proof}
\begin{rem}
	We define $T = \int \phi dE$, and $\langle Tg,h \rangle = \int \phi dE_{gh}$.
\end{rem}

We have one more property on $B(X)$ for a compact space $X$. Let $B(X)$ denote all bounded Borel measurable functions on $X$ and $M(X)$ be all Borel measures.

\begin{lem}
	If $X$ is a compact space and $\phi \in B(X)$, then there is a net $\{u_i\} \subset C(X)$ with $\norm{u_i}_{\infty} \leqslant \norm{\phi}$ s.t. $\int u_i d\mu \sto \int \phi d\mu$ for any $\mu$ in $M(X)$.
\end{lem}
\begin{proof}
	$C(X)^{*} = M(X)$ and $C(X) \subset C(X)^{**} = \st{M(X)}$.  Therefore, the unit ball in $C(X)$ is $wk$-dense in the unit ball in $\st{M(X)}$ (by using Hahn-Banach Theorem). Identifying $B(X)$ be the subspace of $\st{M(X)}$, we have this lemma.
\end{proof}
\begin{rem}
	In fact, $C(X)$ is norm closed in $C(X)^{**} = \st{M(X)}$ like that the $\st{C}$-algebra is norm closed in $\oper$. But by above lemma, we see in $\st{M(X)}$
	\begin{equation*}
		\clo{C(X)}^{wk} = B(X)
	\end{equation*}
	Thus we may apply this to extending $\st{C}$-algebras.
\end{rem}


$B(X,\Omega)$ with the supremum norm and complex conjugate can become a $\st{C}$-algebra. Then by above theorem, we can provide the map from $B(X,\Omega)$ to operator algebras.

\begin{prop}
	If $E$ is a spectral measure for $(X,\Omega,\Hs)$, then
	\begin{center}
		\begin{tabular}{l c c l}
			$\rho \colon$ & $B(X,\Omega)$ & $\longrightarrow$ & $\oper$ \\
			~ & $\phi$ & $\longmapsto$ & $\int \phi dE$
		\end{tabular}
	\end{center}
	is a representation of $B(X,\Omega)$.
\end{prop}
\begin{proof}
	Firstly, we need check that $\rho$ preserves the involution. For $g,h \in \Hs$,
	\begin{equation*}
		E_{gh}(\Delta) = \langle E(\Delta)g, h \rangle = \langle g, E(\Delta)h \rangle = \clo{\langle E(\Delta)h, g \rangle} = \clo{E_{hg}(\Delta)}
	\end{equation*}
	Therefore, for any $g,h \in \Hs$
	\begin{eqnarray*}
		\langle \st{(\int \phi dE)}g,h \rangle &=& \langle g, (\int \phi dE)h \rangle \\
		&=& \clo{\langle (\int \phi dE)h, g \rangle} = \clo{\int \phi dE_{hg}} \\
		&=& \int \st{\phi} d \clo{E_{hg}} = \int \st{\phi} dE_{gh} \\ 
		&=& \langle (\int \st{\phi} dE)g,h \rangle
	\end{eqnarray*}
	\item Also, $\rho$ must be multiplicative. Fix $\phi,\psi \in B(X)$ and $\varepsilon >0$, let $\{\Delta_i\}_{i=1}^{n}$ be the $\Omega$-partition of $X$ s.t. the oscillations of $\phi,\psi,\phi\psi$ on each $\Delta_i$ are less than $\varepsilon$.
	By the fact $E(\Delta_i)E(\Delta_j) = 0$ for $i \neq j$, 
	\begin{eqnarray*}
		\lefteqn{\norm{\int \phi\psi dE - (\int \phi dE)(\int \psi dE)}} \\
		&\leqslant& \varepsilon+ \norm{\sum_{k=1}^{n}\phi(x_k)\psi(x_k)E(\Delta_k) - (\sum_{i=1}^{n}\phi(x_i)E(\Delta_i))(\sum_{j=1}^{n}\psi(x_j)E(\Delta_j))} \\
		&& \negmedspace{} + \norm{(\sum_{i=1}^{n}\phi(x_i)E(\Delta_i))(\sum_{j=1}^{n}\psi(x_j)E(\Delta_j))-(\int \phi dE)(\int \psi dE)} \\
		&\leqslant& \varepsilon + \norm{(\sum_{i=1}^{n}\phi(x_i)E(\Delta_i))(\sum_{j=1}^{n}\psi(x_j)E(\Delta_j)-\int \psi dE)} \\
		&& \negmedspace{} + \norm{(\sum_{i=1}^{n}\phi(x_i)E(\Delta_i)-\int \phi dE)(\sum_{j=1}^{n}\psi(x_j)E(\Delta_j))} \\
		&\leqslant& \varepsilon(1+\norm{\phi}_{\infty}+\norm{\psi}_{\infty})
	\end{eqnarray*}
	Then $\rho$ is indeed multiplicative.
\end{proof}

We have already there is a map from $C(X)$ to $\oper$, then by combining this proposition and above lemma, if the map is $wk$-continuous on $C(X)$, we can extend it from $M(X)$ to $\oper$. 

\begin{thm}
	If $X$ is a compact space and $\rho \colon C(X) \sto \oper$ is a representation, then there is a unique spectral measure $E$ defined on the Borel sets of $X$ s.t. $\rho$ can extend to $\tilde{\rho}$ on $B(X)$, where 
	\begin{equation*}
		\tilde{\rho}(\phi) = \int \phi dE ~~\forall~\phi \in B(X)
	\end{equation*}
\end{thm}























