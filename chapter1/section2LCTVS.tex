\section{Locally Convex Topological Vector Spaces}

The locally convex topological vector space is a topological vector spaces whose topology is generated by a family of seminorms, thus it can provide more properties.

\subsection{Definition by Convex Sets}

Firstly, the conception of locally convex space can be obtained by convex sets. So, we need to research some elementary traits of convex sets.

\begin{defn}
	~
	\begin{enumerate}[label=\arabic*)]
		\item Let $S$ be any subset of a vector space $X$ over $\K$. The convex hull of $S$, $conv(S)$, is the smallest convex subset containing $S$. In fact, 
		\begin{equation*}
			conv(S) = \{~ \sum_{i=1}^{n}\lambda_i x_i \colon x_i \in S,~ \lambda_i \in [0,1],~ \sum_{i=1}^{n}\lambda_i = 1,~ n \in \N ~\}
		\end{equation*}
		\item A subset $S$ of a vector space $X$ over $\K$ is absolutely convex, if $S$ is convex and balanced
		\item A subset $S$ of a vector space $X$ over $\K$ is called a barrel if $S$ is closed, absorbing and absolutely convex.
	\end{enumerate}
\end{defn}

\begin{prop}
	~
	\begin{enumerate}[label=\arabic*)]
		\item Arbitrary intersections of convex sets are convex sets, and the sum of two convex sets is convex, and linear maps preserve convex.
		\item The convex hull of a balanced set is balanced.
		\item The closure and the interior of a convex set in a t.v.s. is convex.
		\item Every neighbourhood of the origin of a t.v.s. is contained in a neighbourhood of the origin which is a barrel.
	\end{enumerate}
\end{prop}
\begin{proof}
	$1)$ and $2)$ are easily obtained by the definition. \\
	To prove $3)$, for any $\lambda \in [0,1]$, we define the map
	\begin{center}
		\begin{tabular}{l c c l}
			$\phi ~ \colon$ & $X \times X$ & $\longrightarrow$ & $X$ \\
			~ & $(x,y)$ & $\mapsto$ & $\lambda x + (1-\lambda) y$
		\end{tabular}
	\end{center}
	Using the fact that $\phi(S \times S) \subset S$, $S \times S \subset \phi^{-1}(S) \phi^{-1}(\clo{S})$. And since $X$ is a t.v.s., $\phi$ is continuous. Thus $\phi^{-1}(\clo{S})$ is closed, i.e. $\clo{S} \times \clo{S} \subset \phi^{-1}(\clo{S}) \Rightarrow \phi(\clo{S} \times \clo{S}) \subset \clo{S}$. $\clo{S}$ is convex.\\
	Since $z+u = \lambda x + (1-\lambda) y + \lambda u + (1-\lambda) u$ if $z = \lambda x + (1-\lambda) y$, $x + U ~\&~ y + U \subset S \Rightarrow z + U \subset S$. That means $\inte{S}$ is convex.\\
	For $4)$, $\forall ~ U \in \mathscr{F}(e)$, the set
	\begin{equation*}
		\clo{conv\left(\bigcup_{\abs{\lambda} \leqslant 1}\lambda U\right)}
	\end{equation*}
	is a barrel.
\end{proof}
\begin{cor}
	Every neighbourhood of the origin in a t.v.s. $X$ is contained in a neighbourhood of the origin which is absolutely convex.
\end{cor}

Now, we can get the definition of a locally convex t.v.s..
\begin{defn}
	A t.v.s. $X$ is said to be locally convex if there is a basis of neighbourhoods of the origin in $X$ consisting of convex sets.
\end{defn}

By this definition, the structure of the neighbourhoods of the origin in a locally convex t.v.s. can be more explicit by using above proposition.

\begin{prop}
	Let $X$ be a locally convex t.v.s..
	\begin{enumerate}[label=\arabic*)]
		\item $X$ has a basis of neighbourhoods of origin consisting with oepn absorbing absolutely convex sets.
		\item  $X$ has a basis of neighbourhoods of origin consisting with barrels.
	\end{enumerate}
\end{prop}

\begin{thm}
	If $X$ is a locally convex t.v.s., then there exists a basis $\mathscr{B}$ of neighbourhoods of origin consisting of absorbing absolutely convex set s.t.
	\begin{equation*}
		\forall ~ U \in \mathscr{B},~ \forall ~ \rho > 0,~ \exists ~ W \in \mathscr{B} ~s.t.~ W \subset \rho U
	\end{equation*}
	Conversely, if $\mathscr{B}$ is a collection of absorbing absolutely convex subsets of a vector space satisfying above condition, it can generate a unique locally convex t.v.s..
\end{thm}

\subsection{Definition by Seminorms}

\begin{defn}
	Let $X$ be a vector space over $\K$. A map $p \colon X \sto \R^{+}$ is called a seminorm if it satisfies:
	\begin{enumerate}[label=\arabic*)]
		\item $p(x+y) \leqslant p(x)+p(y),~ \forall ~ x,~ y \in X$,
		\item $p(\lambda x) = \abs{\lambda} p(x),~ \forall ~ x \in X,~ \forall ~ \lambda \in \K$.
	\end{enumerate}
\end{defn}
\begin{rem}
	In fact, $\ker{p}$ is a linear sbuspace and if $\ker{p} = \{0\}$, $p$ is called a norm.
\end{rem}

By the intuition, the seminorm could construct the continuity of the addition and multiplication since it satisfies above properties. Now, we can build this rigorously.

\begin{defn}
	Let $X$ be a vector space and $A \subset X$ be a nonempty subset. The Minkowski functional of $A$ is the map
	\begin{center}
		\begin{tabular}{l c c l}
			$p_A ~ \colon$ & $X$ & $\longrightarrow$ & $\R$ \\
			~ & $x$ & $\mapsto$ & $\inff\{~ \lambda > 0 \colon x \in \lambda A ~\}$
		\end{tabular}
	\end{center}
\end{defn}

Let $X$ be a vector space and $p$ is a seminorm on $X$, then let $\inte{U}_p = \{~ x \in X \colon p(x) < 1 ~\}$, $U_p = \{~ x \in X \colon p(x) \leqslant 1 ~\}$. Thus, $U$ may be the basis generating the topology. To see it, the following proposition is helpful.

\begin{prop}
	Let $A \subset X$ be a nonempty subset of a vector space, which is absorbing and absolutely convex, then $p_A$ is a seminorm and $\inte{U}_{p_A} \subset A \subset U_{p_A}$. Conversely, if $q$ is a norm on $X$ then $\inte{U}_q$ is an absorbing absolutely convex set and $q = p_{\inte{U}_q}$.
\end{prop}
\begin{proof}
	Since $A$ is balanced, $\xi A \in \lambda A \Leftrightarrow x \in \frac{\lambda}{\abs{\xi}} A$. Thus,
	\begin{equation*}
		p_A(x) = \abs{\xi} \inff{\{~ \frac{\lambda}{\abs{\xi}} \colon x \in \frac{\lambda}{\abs{\xi}} A~\}} = \abs{\xi} p_A(x)
	\end{equation*}
	And $p_A(x) < \infty  ~( \forall ~ x \in X)$ since $A$ is absorbing.\\
	Fixed $x,~ y \in X$, $\forall \varepsilon > 0,~ \exists ~ \lambda,~ \mu > 0$, s.t. $x \in \lambda A ~\&~ y \in \mu A$ and
	\begin{equation*}
		\lambda \leqslant p_A(x) + \varepsilon ~,~ \mu \leqslant p_A(y) + \varepsilon,
	\end{equation*}
	By convexity of $A$, $\lambda A + \mu A \subset (\lambda + \mu) A$. Thus,
	\begin{equation*}
		p_A(x) = \inff{\{~ \delta > 0 \colon x+y \in \delta A ~\}} \leqslant \lambda + \mu \leqslant p_A(x) + p_A(y) + 2\varepsilon
	\end{equation*}
	$\Rightarrow ~ p_A(x)$ is a seminorm.
	\begin{center}
		\begin{tabular}{r c l}
			$x \in \inte{U}_{p_A}$ & $\Rightarrow$ & $\exists ~ \lambda \in [0,1] ~s.t.~ x \in \lambda A \subset A$ \\
			$x \in A$ &  $\Rightarrow$  & $1 \in \inff\{~ \lambda > 0 \colon x \in \lambda A ~\} ~ \Rightarrow ~ p_A(x) \leqslant 1 \Rightarrow x \in U_{p_A}$
		\end{tabular}
	\end{center}
	That means $\inte{U}_{p_A} \subset A \subset U_{p_A}$.\\
	Finally, the statements for $q$ can be obtained easily by the definition.
\end{proof}

Now, we can give the definition of a locally convex t.v.s. by seminorms coinciding with the definition by convex sets.

\begin{thm}
	Let $X$ be a vector space and $\mathscr{P} = \{p_i\}_{i \in I}$ be a family of seminorms. Then the initial topology $\mathscr{T}_P$ generated by $\mathscr{P}$ makes $X$ be a locally convex t.v.s. In fact, the basis of neighbouhoods of the origin in $X$ is like
	\begin{equation*}
		\mathscr{B} = \left\{~ \{x \in X \colon p_{i_1}(x)<\varepsilon,\cdots,p_{i_n}(x)<\varepsilon\} \colon i_1,\cdots,i_n \in I,n \in \N, \varepsilon > 0 ~\right\}
	\end{equation*}
	Conversely, the topology of any locally convex t.v.s. can be generated by a family of seminorms.
\end{thm}
\begin{proof}
	Each element in the subbasis of $\mathscr{T}_P$ is like $\{ x \in X \colon p_i(x) < \varepsilon \} = \varepsilon \inte{U}_{p_i}$, which is clearly absorbing and absolutely convex. Therefore, every element in $\mathscr{B}$ is convex. $(X,\mathscr{T}_P)$ is a locally convex t.v.s..\\
	Convesrly, if $(X, \mathscr{T})$ is a locally convex t.v.s., the basis of neighbourhoods of the origin in $X$ consists of absorbing and absolutely convex stes, which can generate a family of seminorms by above proposition. Therefore, these seminorms can generate a locolly convex topology $\mathscr{T}_P$. In fact, $\mathscr{T}_P = \mathscr{T}$, since $\inte{U}_{p_A} \subset A \subset U_{p_A}$.
\end{proof}
\begin{rem}
	By this theorem, in a vector space $X$, the seminorms on $X$ can coincide with a locally convex topology making $X$ be a t.v.s..
\end{rem}

There some extra properties for the seminorms on a vector space.
\begin{prop} \label{prop1}
	Let $X$ be a vector space and $p$ be a seminorm on $X$. Then,
	\begin{enumerate}[label=\arabic*)]
		\item $\forall~ r>0,~ r \inte{U}_p=\{x \in X \colon p(x) <r\} = \inte{U}_{\frac{1}{r}p}$.
		\item $\forall x \in X,~ x+\inte{U}_p = \{y \in X \colon p(y-x) < 1\}$.
		\item if $q$ is a seminorm on $X$, $p \leqslant q \Leftrightarrow \inte{U}_q \subset \inte{U}_p$.
		\item if $\{s_i\}_{i=1}^{n}$ are seminorms on $X$, then $s(x) = \max_{i=1,\cdots,n}{s_i(x)}$ is also a seminorm and $\inte{U}_s = \bigcap_{i=1}^{n} \inte{U}_{s_i}$
	\end{enumerate}
\end{prop}

\begin{thm}
	Let $\mathscr{P} = \{p_i\}_{i \in I}$ and $\mathscr{Q} = \{q_j\}_{j \in J}$ be two families of seminorms on a vector space $X$ inducing $\mathscr{T}_P$ and $\mathscr{T}_Q$, then
	\begin{equation*}
		\mathscr{T}_Q \subset \mathscr{T}_P \Leftrightarrow \forall~ q \in \mathscr{Q},~ \exists~ \{i_k\}_{k=1}^{n} \subset I,~ \exists~ C>0,~s.t.~ Cq(x) \leqslant \max_{k=1,\cdots,n}{p_{i_k}(x)}
	\end{equation*}
\end{thm}
\begin{proof}
	This right side of above statement is equivalent to that
	\begin{equation*}
		\forall~ q \in \mathscr{Q},~ \exists~ \{i_k\}_{k=1}^{n} \subset I,~ \exists~ C>0,~s.t.~ C \bigcap_{k=1}^{n} \inte{U}_{p_{i_k}} \subset \inte{U}_q
	\end{equation*}
	So, it is clearly equivalent to $\mathscr{T}_Q \subset \mathscr{T}_P$.
\end{proof}
\begin{rem}
	Because of this, we have the definition of equivalent norms. In fact, two norms $p$ and $q$ are said to be equivalent if and only if there exists $C_1,~ C_2 > 0$, s.t. $C_1p(x) \leqslant q(x) \leqslant C_2 p(x)$, for all $x \in X$. This definition means that two equivalent norms can generate one same topology. 
\end{rem}
\begin{cor} \label{cor1}
	The family $\mathscr{P} = \{p_i\}_{i \in I}$ of seminorms and \\$\mathscr{Q} = \{~ \max_{i \in B} p_i \colon \varnothing \neq B \subset I \text{ with } B \text{ finite} ~\}$ can generate one same topology on $X$
\end{cor}

\subsection{Separability and Metrizability}

For a t.v.s., $(T_1)$ is equivalent to Hausdorff and $(T_1)$ is associated with the ability of a topology separating points. Therefore, wheather a t.v.s. is a Hausdorff space or not is completely determined by wheather the topology of it can separate points or not. Then for a locally convex t.v.s., whose topology is induced by a family of seminorms, this separability is related to these seminorms.

\begin{defn}
	A family of seminorms $\mathscr{P} = \{~p_i~\}_{i \in I}$ on a vector space $X$ is said to be speparating, if
	\begin{equation*}
		\forall~ x \in X \backslash \{0\},~ \exists~ i \in I ~s.t.~ p_i(x) \neq 0
	\end{equation*}
\end{defn}
\begin{rem}
	In fact, above condition is equivalent to 
	\begin{equation*}
		\text{if } p_i(x) = 0, ~\forall~ i \in I \Rightarrow x = 0
	\end{equation*}
\end{rem}	

Now, we can give the condition that makes a locally convex t.v.s. be Hausdorff.

\begin{thm}
	A locally convex t.v.s. $X$ is Hausdorff if and only if its topology can be induced by a separating family of seminorms $\mathscr{P} = \{~p_i~\}_{i \in I}$ .
\end{thm}
\begin{proof}
	If $\mathscr{P} = \{~p_i~\}_{i \in I}$ is separating, the fact that $\mathscr{T}_P$ is Hausdorff can be obtained easily by the definition.\\
	Conversely, if $X$ is Hausdorff, for $x \neq 0$, we can find a $U \in \mathscr{F}(0)$, s.t. $U$ can separate $x$ and $0$. But since $X$ is locally convex, $U$ can be chosen as $\inte{U}_p$ for a seminorm $p$. Thus, for this $p$, $p(x) \neq 0$.
\end{proof}

For the metrizability of a locally convex t.v.s., the consequece is also easier than general case.
\begin{thm}
	A locally convex t.v.s. $X$ is metrizable if and only if its topology is determined by a countable separating family of seminorms.
\end{thm}
\begin{proof}
	It can be directly obtained by the Nagata–Smirnov's Metrization Theorem. Also, there is a more explicit proof. If the topology of $X$ is generated by a countable separating family of seminorms $\mathscr{P} = \{~p_n~\}_{n=1}^{\infty}$, we can define the metric $d$ on $X$ by
	\begin{equation*}
		d(x,y) = \sum_{n=1}^{\infty} 2^{-n} \frac{p_n(x-y)}{1+p_n(x-y)}
	\end{equation*} 
	Conversely, if $(X,d)$ is the metric space, the subbasis of the topology generated by this metric is like $U_n = \{x \colon d(x,0) < 1/n\}$. And these $U_n$ can provide a countable separating family of seminorms. In fact, if $\mathscr{Q} = \{~q_i~\}_{i \in I}$ generates the topology of $X$, for each $U_n$, there are $q_1,\cdots,q_k \in \mathscr{Q}$ and $\varepsilon_1,\cdots,\varepsilon_k > 0$, s.t. $\bigcap_{i=1}^{k} \{x \colon q_i(x) < \varepsilon_i\} \subset U_n$. Then let $p_n = \sum_{i=1}^{k} \varepsilon_i^{-1}q_i$. It can check that the family $\{p_n\}_{n=1}^{\infty}$ generate the coincided topology on $X$.
\end{proof}

\subsection{Continuous Linear Maps on LCTVS}

To give the special property of continuous linear maps on LCTVS, we need firstly refine the family of seminorms.
\begin{defn}
	A family $\mathscr{Q} = \{~q_j~\}_{j \in J}$ of seminorms on a vector space $X$ is said to be directed if
	\begin{equation*}
		\forall ~ j_1, \cdots, j_n \in \mathscr{Q},~ \exists ~ j \in J ~\&~ C>0,~ s.t.~ C q_j(x) \geqslant \max_{k=1,\cdots,n} q_{j_k}(x),~ \forall~ x \in X
	\end{equation*}
\end{defn}
\begin{rem}
	By \textbf{Proposition} \ref{prop1} in \textbf{1.2.2}, this definition is equivalent to that
	\begin{equation*}
		\forall~ \inte{U}_{q_{j_1}}, \cdots, \inte{U}_{q_{j_n}},~ \exists~ \inte{U}_{q_j} ~s.t.~ \inte{U}_{q_j} \subset \bigcap_{k=1}^{n} \inte{U}_{j_i}
	\end{equation*}
	And thus the basis of this directed family of seminorms should be like
	\begin{equation*}
		\mathscr{B}_d = \{~ r\inte{U}_q \colon q \in \mathscr{Q},r>0 ~\}
	\end{equation*}
\end{rem}

By this special topology, we can find the condition making linear functional continuous.

\begin{prop}
	Let $\mathscr{T}$ be a locally convex topology on a vector space $X$ generated by a directed family $\mathscr{Q}$ of seminorms on $X$. Then
	\begin{equation*}
		L \colon ~ X \longrightarrow \K
	\end{equation*}
	is a $\mathscr{T}$-continuous linear functional if and only if $\exists ~ q \in \mathscr{Q}$ s.t. $L$ is $q$-continuous, i.e. $\abs{L(x)} \leqslant Cq(x)$ for some $C > 0$.
\end{prop}
\begin{proof}
	In fact, this property of continuous linear functional is because the element in a directed locally convex topology is like $r\inte{U}_q$. In fact, we just need to check the origin point.\\
	If $L$ is continuous, there exists a $r\inte{U}_q$ s.t. $r\inte{U}_q \subset L^{-1}(B_1(0))$, where $B_1(0)$ is the unit ball centered at $0$. This is equivalent to the $q$-continuity of $L$.\\
	Conversely, it is clearly by the fact $\mathscr{T}_q \subset \mathscr{T}$.
\end{proof}

We can easily see that the topology of a locally convex t.v.s. can be always induced by a directed family of seminorms by the \textbf{Corollary} \ref{cor1} in \textbf{1.2.2}. Thus, we have the corollary.

\begin{cor}
	$(X,\mathscr{T})$ is a locally convex t.v.s. and $\mathscr{T}$ is generated by the family $\mathscr{P} = \{p_i\}_{i \in I}$. Then $L \colon ~ X \longrightarrow \K$ is a continuous linear functional if and only if 
	\begin{equation*}
		\exists ~ i_1,\cdots,i_n \in I,~ \exists ~ C>0 ~s.t.~ \abs{L(x)} \leqslant C \max_{k=1,\cdots,n} p_{i_k}(x),~ \forall ~ x \in X
	\end{equation*}
\end{cor}

And this corollary can be easily extended to linear maps. And the proof is similar as above statement In fact, we just need to replace $B_1(0)$ by $\inte{U}_q$
\begin{thm}
	Let $X$ and $Y$ be two locally convex t.v.s.'s generated by $\mathscr{P}$ and $\mathscr{Q}$. Then linear map $f \colon X \sto Y$ is continuous if and only if 
	\begin{equation*}
		\forall ~ q \in \mathscr{Q},~ \exists ~ p_1,\cdots,p_n \in \mathscr{P},~ \exists ~ C>0 ~s.t.~ q(f(x)) \leqslant C \max_{i=1,\cdots,n} p_i(x)
	\end{equation*}
\end{thm}
